%---------------------------------------------------------------------------
\documentclass%%
%---------------------------------------------------------------------------
  [fontsize=10pt,%%          Schriftgroesse
%---------------------------------------------------------------------------
% Satzspiegel
   paper=a4,%%               Papierformat
   enlargefirstpage=on,%%    Erste Seite anders
   pagenumber=headright,%%   Seitenzahl oben mittig
%---------------------------------------------------------------------------
% Layout
   headsepline=off,%%         Linie unter der Seitenzahl
   parskip=off,%%           Abstand zwischen Absaetzen
%---------------------------------------------------------------------------
% Briefkopf und Anschrift
   fromalign=right,%%        Plazierung des Briefkopfs
   fromphone=off,%%           Telefonnummer im Absender
   fromrule=aftername,%%           Linie im Absender (aftername, afteraddress)
   fromfax=off,%%            Faxnummer
   fromemail=on,%%          Emailadresse
   fromurl=off,%%            Homepage
   fromlogo=off,%%           Firmenlogo
   addrfield=off,%%           Adressfeld fuer Fensterkuverts
   backaddress=off,%%          ...und Absender im Fenster
   subject=beforeopening,%%  Plazierung der Betreffzeile
   locfield=wide,%%        (narrow,wide) zusaetzliches Feld fuer Absender
   foldmarks=off,%%           Faltmarken setzen
   numericaldate=off,%%      Datum numerisch ausgeben
   refline=wide,%%         Geschaeftszeile im Satzspiegel
%---------------------------------------------------------------------------
% Formatierung
   draft=on%%                Entwurfsmodus
]{scrlttr2}
\LoadLetterOption{myLetter}
%---------------------------------------------------------------------------
% Weitere Optionen
\KOMAoptions{%%
}
%---------------------------------------------------------------------------
\usepackage{multirow}
\usepackage{graphicx}
\usepackage[ngerman]{babel}
\usepackage[T1]{fontenc}
\usepackage[latin1]{inputenc}
\usepackage{url}
%---------------------------------------------------------------------------
% Fonts
\setkomafont{title}{\huge\rmfamily}
%\setkomafont{fromname}{\large}
\setkomafont{fromaddress}{\small}%% statt \small
\setkomafont{pagenumber}{\sffamily}
\setkomafont{subject}{\mdseries}
\setkomafont{backaddress}{\mdseries}
%\usepackage{mathptmx}%% Schrift Times
%\usepackage{mathpazo}%% Schrift Palatino
%\setkomafont{fromname}{\LARGE}
%---------------------------------------------------------------------------
\addtolength{\textheight}{6\baselineskip}
\begin{document}
%---------------------------------------------------------------------------
% Briefstil und Position des Briefkopfs
%\LoadLetterOption{DIN} %% oder: DINmtext, SN, SNleft, KOMAold.
\makeatletter
\@setplength{locwidth}{\textwidth}
\@addtoplength{locvpos}{30mm}
\@addtoplength{lochpos}{5cm}
\@addtoplength{refvpos}{-2mm}%-\baselineskip}%-.5\baselineskip}
\@setplength{firstheadvpos}{10mm}
\@setplength{firstheadwidth}{\paperwidth}
\ifdim \useplength{toaddrhpos}>\z@
  \@addtoplength[-2]{firstheadwidth}{\useplength{toaddrhpos}}
\else
  \@addtoplength[2]{firstheadwidth}{\useplength{toaddrhpos}}
\fi
\@setplength{foldmarkhpos}{6.5mm}
\makeatother
%-----------------------------------------------------------------------------
%
%????? Absender-----> defined in myLetter.lco for the ``Letters'' folder
%                                               commands there are overridden by this file
%
%------------------DATI PERSONALI (overwrites .lco File) -------------------------------------------------------------------------------
\setkomavar{title}{\bf Lebenslauf}
%\setkomavar{fromname}{Absender Name}
%\setkomavar{fromaddress}{ Stra�e\\12345 Ort.}
%\setkomavar{fromphone}{+49 (0)30 3462 4345}
%\renewcommand{\phonename}{Telefon}
\setkomavar{fromemail}{amantini%
							%.andrea@gmail.com}
							@math.hu-berlin.de}
\setkomavar{backaddressseparator}{, }
\setkomavar{signature}{\raggedright\usekomavar{fromname}\\[+2mm]Berlin, den \today}
%\setkomavar{frombank}{}
%\setkomavar{location}{}
\setkomavar{location}{%\\[8ex]
\raggedright{
%\small{%
%	Address:\usekomavar{fromaddress}\\Staatsangeh\"origkeit: italienisch\\
%      	Telefon:\usekomavar{fromphone}%
\begin{tabular}{r@{\extracolsep{.75em}}p{6cm}@{ }l}
						&&\multirow{10}{\textwidth}{\includegraphics%[height=6cm]%
						%10\baselineskip
						{bewphotos/%__METTI LA FOTO QUI SOTTO !!!!
						%lanAve%
						pissero2-low%
						%bewph1
						%cvPhoto1-alta
						} }\\ 
						&&\\						
%						&&\\
%						&&\\
%Name:
	 					& {\large\bf Andrea Amantini}\\[+\baselineskip]
Geburtsort, Datum: 			& Florenz (Italien), 28.01.1980\\[+1mm]
Staatsb�rgerschaft: 			& italienisch\\[+1mm]
Familienstand:				&ledig\\[+1mm]
Wohnsitz:					& Lausitzerstra{\ss}e 37, 10999 Berlin\\[+1mm]
Tel:						& +49 (0)30 3462 4345\\[+1mm]
Mob:						& +49 (0)176 2829 7792\\[+1mm]
E-Mail:					& \usekomavar{fromemail}%\\[+1mm]
%(alt.~E-Mail:				&amantini.andrea@gmail.com)%
\\[+15mm]
\end{tabular}
}%
%}%
}%% Neben dem Adressfenster
%---------------------------------------------------------------------------
%\firsthead{Frei gestalteter Briefkopf}
%---------------------------------------------------------------------------
\firstfoot{\center\rule{\textwidth}{.5pt}
{\small Andrea Amantini | \usekomavar{fromaddress} | \usekomavar{fromphone} | \usekomavar{fromemail}}}
%---------------------------------------------------------------------------
% Geschaeftszeilenfelder
%\setkomavar{place}{Ort}
%\setkomavar{placeseparator}{, den }
\setkomavar{date}{}
%\setkomavar{yourmail}{1. 1. 2003}%% 'Ihr Schreiben...'
%\setkomavar{yourref} {abcdefg}%%    'Ihr Zeichen...'
%\setkomavar{myref}{}%%      Unser Zeichen
%\setkomavar{invoice}{123}%% Rechnungsnummer
%\setkomavar{phoneseparator}{}
%---------------------------------------------------------------------------
% Versendungsart
%\setkomavar{specialmail}{Einschreiben mit R�ckschein}
%---------------------------------------------------------------------------
% Anlage neu definieren
\renewcommand{\enclname}{Anlage}
\setkomavar{enclseparator}{: }
%---------------------------------------------------------------------------
% Seitenstil
%\pagestyle{plain}%% keine Header in der Kopfzeile
%---------------------------------------------------------------------------
\begin{letter}{~}
%---------------------------------------------------------------------------
%\setkomavar{subject}{Halo}
%---------------------------------------------------------------------------
\opening{~}
\begin{tabular}{r@{\extracolsep{2em}}p{83mm}}
\multicolumn{2}{l}{\large\bf Bildungsgang}\\[+1mm]
\hline\\[+1mm]
Okt 2005 - heute		&Doktorand an der Humboldt Universit�t zu Berlin\\ 	
					&im Fach Mathematik -- Schwerpunkt: Algebra, Mathematische Logik, Modelltheorie -- Dissertationstitel:
					{\em Hrushovski predimensions on nilpotent Lie algebras}, eingereicht am 8.12.2010 --
					Verteidigung: voraussichtlich Fr\"uhjahr 2011\\[+3mm]%({\sl Betreuer: Prof. Dr. A. Baudisch})\\[+3mm]
Sep 2008 - Okt 2008			&Forschungsaufenthalt an der {\em Universit� Lyon 1}, Inst. Camille
						Jordan -- {\sl Lyon, France}\\[+3mm]
Okt 2005 - Sep 2008		&Teilnahme an dem Forschungstrainingsnetzwerk %und -Stipendium
					MODNET (\url{www.logique.jussieu.fr/modnet/Home})
					gef\"ordert d\"urch die Marie Curie-Ma�nahme FP6 der Europ\"aischen Kommission\\[+3mm]
Okt 1999 - Apr 2005			&Studium an der {\em Universit� degli Studi di Firenze} ({\sl Florenz, Italien}) %\\
					-- Studiengang Mathematik, Abschlusspr�fung:
					Diplom ({\sl Laurea in Matematica}) --
					Abschlussnote: 110/110 \emph{cum Laude}\\[+3mm]
1994-1999			&Gymnasium {\sl Liceo Scientifico}, ({\sl Florenz, Italien})%
\\[+10mm]
%\end{tabular}
%\newpage
%\begin{tabular}{r@{\extracolsep{2em}}p{9cm}}
\multicolumn{2}{l}{\large\bf Stellen}\\[+1mm]
\hline\\[+1mm]
Okt 2008 - Dez 2009 	& Wissenschaftlicher Mitarbeiter an der HU Berlin\\[+3mm]
Okt 2005 - Sep 2008 	& EU-Gastwissenschaftler an der HU Berlin im Rahmen des Marie\,Curie-Forschungsstipendiums
					und Netzwerkprojekts MODNET\\[+10mm]
\end{tabular}
\newpage
\begin{center}
\begin{tabular}{r@{\extracolsep{2em}}p{7.3cm}}
\multicolumn{2}{l}{\large\bf Lehrt�tigkeiten an der HUB}\\[+1mm]
\hline\\[+1mm]
WS 08/09	& �bungsleiter im Fach Algebra I -- Dozent: Dr. W.~Kleinert\\[+1mm]
SoSe 09		& �bungsleiter im Fach Gew�hnlichen Differentialgleichungen -- Dozent: Dr.~L.~Recke\\[+1mm]
WS 09/10	& �bungsleiter im Fach Algebra II (BMS Basic Course Commutative Algebra) -- Dozent: Dr. W.~Kleinert\\[+6mm]
\multicolumn{2}{l}{\large\bf Sprachen}\\[+1mm]
\hline\\[+1mm]
Italienisch & Muttersprache\\[+1mm]
English & fliessend\\[+1mm]
Deutsch & fliessend {\sl (Zeugnisse  A1.2, B2.2, C1 Deutsch als Fremdsprache -- HU Sprachenzentrum, Berlin)}\\[+1mm]
Franz\"osisch & gute Kenntnisse%, sehr gutes Verst\"andnis beim Lesen
\\[+6mm]
\multicolumn{2}{l}{\large\bf Betriebssysteme, Programmiersprachen, IDEs}\\[+1mm]
\hline\\[+1mm]
Mac OS X				 	&erweiterte Kenntnisse\\[+1mm]
Unix/Linux, Windows			&gute Kenntnisse\\[+1mm]
\LaTeX \,``Typesetting" 		&erweiterte Kenntnisse\\[+1mm]
{\sl XHTML, CSS, PhP, MySQL}		&Grundkenntnisse\\[+1mm]
Python, Java, (Obj-) %$C$,
$C^{++}$ 	&Grundkenntnisse\\[+1mm]
Netbeans, XCode				&Grundkenntnisse\\[+6mm]
\multicolumn{2}{l}{\large\bf Laufende Projekte}\\[+1mm]
\hline\\[+1mm]
Labirinto Theater &     \\[+7mm]

%\multicolumn{2}{l}{\large\bf Zusammenfassung der Dissertation}\\[+1mm]
%\hline
\end{tabular}

%\bigskip
%\begin{minipage}{.95\textwidth}
%In meiner Doktorarbeit wird das Fra\"iss\'e-Hrushowskis Amalgamationsverfahren in Zusammenhang
%mit nilpotenten graduierten Lie Algebren \"uber einem endlichen K\"orper untersucht.
%Die (Pr\"a-)Dimensionen die in der Konstruktion auftauchen sind mit der gruppentheoretischen Invariante der {\em Defizienz}
%zu vergleichen, welche auf homologische Methoden zur\"uckgef\"uhrt werden kann.
%
%Dar�ber hinaus wird die Magnus-Lazardsche Korrespondenz zwischen den oben genannten Lie Algebren
%und nilpotenten Gruppen von Primzahl-Exponenten beschrieben.
%%Dabei werden solche Gruppen durch die Baker-Haussdorfsche Formel
%%in den entsprechenden Algebren definierbar interpretiert.
%%Es wird eine $\omega$-stabile Lie Algebra von Nilpotenzklasse 2 und Morleyrang $\omega\cdot2$ erhalten, indem
%%man eine {\em unkollabierte} Version der von Baudisch konstruierten {\em new uncountably categorical group} betrachtet.
%%Diese wird genau analysiert. Unter anderem wird die Unabh\"angigkeitsrelation des Nicht-Gabelns durch die
%%Konfiguration des freien Amalgams charakterisiert.
%
%\indent
%Ziel dieser Arbeit ist eine Erweiterung auf h\"oheren Nilpotenzklassen der von Baudisch
%konstruierten nil-2 Gruppe von primen Exponenten. Daf\"ur werden -- mittels eines induktiven Ansatzes
%-- die Grundlagen entwickelt, um neue Pr\"adimensionen f\"ur
%Lie Algebren der Nilpotenzklassen gr\"o\ss er als zwei zu schaffen.
%\end{minipage}
\end{center}


\bigskip
\begin{flushright}
Andrea Amantini

Berlin, den \today
\end{flushright}
\end{letter}
\end{document}