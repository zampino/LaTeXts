%---------------------------------------------------------------------------
\documentclass%%
%---------------------------------------------------------------------------
  [fontsize=10pt,%%          Schriftgroesse
%---------------------------------------------------------------------------
% Satzspiegel
   paper=a4,%%               Papierformat
   enlargefirstpage=on,%%    Erste Seite anders
   pagenumber=headright,%%   Seitenzahl oben mittig
%---------------------------------------------------------------------------
% Layout
   headsepline=off,%%         Linie unter der Seitenzahl
   parskip=off,%%           Abstand zwischen Absaetzen
%---------------------------------------------------------------------------
% Briefkopf und Anschrift
   fromalign=right,%%        Plazierung des Briefkopfs
   fromphone=off,%%           Telefonnummer im Absender
   fromrule=aftername,%%           Linie im Absender (aftername, afteraddress)
   fromfax=off,%%            Faxnummer
   fromemail=on,%%          Emailadresse
   fromurl=off,%%            Homepage
   fromlogo=off,%%           Firmenlogo
   addrfield=off,%%           Adressfeld fuer Fensterkuverts
   backaddress=off,%%          ...und Absender im Fenster
   subject=beforeopening,%%  Plazierung der Betreffzeile
  locfield=narrow,%%        (narrow,wide) zusaetzliches Feld fuer Absender
   foldmarks=off,%%           Faltmarken setzen
   numericaldate=off,%%      Datum numerisch ausgeben
   refline=wide,%%         Geschaeftszeile im Satzspiegel
%---------------------------------------------------------------------------
% Formatierung
   draft=on%%                Entwurfsmodus
]{scrlttr2}
\LoadLetterOption{myLetter}
%---------------------------------------------------------------------------
% Weitere Optionen
\KOMAoptions{%%
}
%---------------------------------------------------------------------------
\usepackage[ngerman]{babel}
\usepackage[T1]{fontenc}
\usepackage[latin1]{inputenc}
\usepackage{url}
%---------------------------------------------------------------------------
% Fonts
\setkomafont{title}{}
%\setkomafont{fromname}{\large}
\setkomafont{fromaddress}{\small}%% statt \small
\setkomafont{pagenumber}{\sffamily}
\setkomafont{subject}{\mdseries}
\setkomafont{backaddress}{\mdseries}
%\usepackage{mathptmx}%% Schrift Times
%\usepackage{mathpazo}%% Schrift Palatino
%\setkomafont{fromname}{\LARGE}
%---------------------------------------------------------------------------
\begin{document}
%---------------------------------------------------------------------------
% Briefstil und Position des Briefkopfs
%\LoadLetterOption{DIN} %% oder: DINmtext, SN, SNleft, KOMAold.
\addtolength{\textheight}{2\baselineskip}
\makeatletter
\@setplength{locwidth}{\textwidth}
%\@addtoplength{locvpos}{2mm}
%\@addtoplength{lochpos}{3.3cm}
\@addtoplength{refvpos}{-7\baselineskip}
\@setplength{firstheadvpos}{10mm}
\@setplength{firstheadwidth}{\paperwidth}
\ifdim \useplength{toaddrhpos}>\z@
  \@addtoplength[-2]{firstheadwidth}{\useplength{toaddrhpos}}
\else
  \@addtoplength[2]{firstheadwidth}{\useplength{toaddrhpos}}
\fi
\@setplength{foldmarkhpos}{6.5mm}
\makeatother
%-----------------------------------------------------------------------------
%
%????? Absender-----> defined in myLetter.lco for the ``Letters'' folder
%                                               commands there are overridden by this file
%
%-------------------------------------------------------------------------------------------------
\setkomavar{title}{\textnormal{ Zusammenfassung der Dissertation}}
%\setkomavar{fromname}{Absender Name}
%\setkomavar{fromaddress}{ Stra�e\\12345 Ort.}
%\setkomavar{fromphone}{+49 (0)30 3462 4345}
%\renewcommand{\phonename}{Telefon}
%\setkomavar{fromemail}{absender.name@provider.de}
\setkomavar{backaddressseparator}{, }
%\setkomavar{signature}{\flushright\usekomavar{fromname}}
%\setkomavar{frombank}{}
\setkomavar{location}{}%% Neben dem Adressfenster
%---------------------------------------------------------------------------
%\firsthead{Frei gestalteter Briefkopf}
%---------------------------------------------------------------------------
%\firstfoot{Fu�zeile}
%---------------------------------------------------------------------------
% Geschaeftszeilenfelder
%\setkomavar{place}{Ort}
%\setkomavar{placeseparator}{, den }
\setkomavar{date}{}
%\setkomavar{yourmail}{1. 1. 2003}%% 'Ihr Schreiben...'
%\setkomavar{yourref} {abcdefg}%%    'Ihr Zeichen...'
%\setkomavar{myref}{}%%      Unser Zeichen
%\setkomavar{invoice}{123}%% Rechnungsnummer
%\setkomavar{phoneseparator}{}
%---------------------------------------------------------------------------
% Versendungsart
%\setkomavar{specialmail}{Einschreiben mit R�ckschein}
%---------------------------------------------------------------------------
% Anlage neu definieren
\renewcommand{\enclname}{Anlage}
\setkomavar{enclseparator}{: }
%---------------------------------------------------------------------------
% Seitenstil
%\pagestyle{plain}%% keine Header in der Kopfzeile
%---------------------------------------------------------------------------
\begin{letter}{~}
%---------------------------------------------------------------------------
%\setkomavar{subject}{Halo}
%---------------------------------------------------------------------------
\opening{~}
%\centering
%\begin{minipage}{.9\textwidth}
In dieser Arbeit wird das Fra\"iss\'e-Hrushowskis Amalgamationsverfahren in Zusammenhang
%mit nilpotenten Gruppen von endlichem Exponent beziehungsweise
mit nilpotenten graduierten Lie Algebren \"uber einem endlichen K\"orper untersucht.

Die Pr\"adimensionen die in der Konstruktion auftauchen sind mit dem gruppentheoretischen Begriff der {\em Defizienz}
zu vergleichen, welche auf homologische Methoden zur\"uckgef\"uhrt werden kann.

Dar�ber hinaus wird die Magnus-Lazardsche Korrespondenz zwischen den oben genannten Lie Algebren
und nilpotenten Gruppen von Primzahl-Exponenten beschrieben.
Dabei werden solche Gruppen durch die Baker-Haussdorfsche Formel
in den entsprechenden Algebren definierbar interpretiert.

\medskip
Es wird eine $\omega$-stabile Lie Algebra von Nilpotenzklasse 2 und Morleyrang $\omega\cdot2$ erhalten, indem
man eine {\em unkollabierte} Version der von Baudisch konstruierten {\em new uncountably categorical group} betrachtet.
Diese wird genau analysiert. Unter anderem wird die Unabh\"angigkeitsrelation des Nicht-Gabelns durch die
Konfiguration des freien Amalgams charakterisiert.
%Ziel dieser Arbeit ist eine Erweiterung auf h\"oheren Nilpotenzklassen der von Baudisch
%konstruierten nil-2 {\sl } von prim Exponent.

\smallskip
Mittels eines induktiven Ansatzes werden die Grundlagen entwickelt, um neue Pr\"adimensionen f\"ur Lie Algebren der Nilpotenzklassen
gr\"o\ss er als zwei zu schaffen.
%F\"ur den nil-3 Fall geben wir eine Notion einer {\em selbstgen\"ugende} Erweiterung;
%damit wird ein erstes Amalgamationslemma bewiesen.

Dies erweist sich als wesentlich schwieriger als im Fall 2.
Wir konzentrieren uns daher auf die Nilpotenzklasse 3, als Induktionsbasis des oben genannten Prozesses.

In diesem Fall wird die Invariante der Defizienz
auf endlich erzeugte Lie Algebren adaptiert. Erstes Hauptergebnis der
Arbeit ist der Nachweis dass diese Definition zu einem vern\"uftigen Begriff selbst-gen\"ugender
Erweiterungen von Lie Algebren f\"uhrt und
sehr nah einer gew\"unschten Pr\"adimension im Hrushovskischen Sinn ist.

Wir zeigen -- als zweites Hauptergebnis -- ein erstes Amalgamationslemma bez\"uglich
selbst-gen\"ugender Einbettungen.
%\end{minipage}

\vfil


\begin{flushright}
%\begin{minipage}{.5\textwidth}
Andrea Amantini\\[+3mm]
Berlin, den \today
%\end{minipage}
\end{flushright}

%---------------------------------------------------------------------------
\end{letter}
\end{document}