\documentclass[10pt,a4paper,english,german]{article}
\usepackage{babel}
\usepackage[latin1]{inputenc}
\usepackage{amsmath,amsfonts,amssymb}
\newcommand{\LN}{\mathcal{LN}}
\linespread{1.3}
%\pagestyle{empty}
%\pagenumbering{1}
\newcommand{\crule}{\rule{13cm}{.4pt}}
%{\begin{center}\rule{13cm}{.4pt}\end{center}}
%
\title{Curriculum Vitae}
\date{}
%____________________________________________________________---
\begin{document}
\maketitle
%\section*{Curriculum Vitae}
%\bigskip
%\subsection*
{\bf Pers\"onliche Daten}\\
\crule\\[+2mm]
%\begin{tabular*}{\textwidth}{@{\bf}r@{\extracolsep{2,5em}}l}%@{\bf}
\begin{tabular}{r@{\extracolsep{2,5em}}l}
%\begin{tabular}{@{\sl}r>{\bf}l}

Name: 					& Andrea Amantini\\
Geburtsort, Datum: 			& Florenz, 28/01/1980\\
Staatsangeh�rigkeit: 		& Italienisch\\
Adresse:					& Lausitzerstra{\ss}e 37, 10999 Berlin\\
%Tel:						& +49 (0)30 2093 5826\\[+3mm]
%Privatanschrift (bis 05/08/2008):& Prenzlauer Allee 196, 10405 Berlin\\
Tel:						& +49 (0)30 44031350\\
Mob:						& +49 (0)176 2829 7792\\
E-Mail:					& amantini@math.hu-berlin.de
\end{tabular}
\vfil
%\bigskip
%\subsection*
{\bf Bildungsgang}\\
\crule\\[+2mm]
\begin{tabular*}{\textwidth}{@{}r@{\extracolsep{2,5em}}p{\textwidth}l@{}}
1994-1999				&Gymnasium (Liceo Scientifico), in Florenz\\
1999-2005				&Studium an der {\sl Universit� degli Studi di Firenze}, 									Studiengang Mathematik\\
						&Abschlusspr\"ufung Diplom 22/04/2005,\quad 110/110 									\emph{cum laude}\\[+3mm]
%August 2004				&attended S.M.I Summer School in Perugia, organized by I.N.D.A.M. (Istituto Nazionale di Alta Matematica)\\
2005-2008				&Doktorand an der Humboldt Universit\"at zu Berlin. ({\sl Betreuer Prof. Dr. A. Baudisch}), Marie Curie FP6 (MODNET) Stipendiat
\end{tabular*}
\vfil
{\bf Stellen}\\
\crule\\[+2mm]
\begin{tabular*}{\textwidth}{@{}r@{\extracolsep{2,5em}}p{\textwidth}l@{}}
2005-2008%&angestellt als 
&EU-Gastwissenschaftler an der Humboldt Universit\"at zu Berlin\\
WS 2008-2009 &Wissenschaftlicher Mitarbeiter an der HU Berlin
\end{tabular*}
\newpage
\subsection*{Mathematische Hochschulbildung}
% ------>  QUESTO LO LASCEREI

%{\sl Grund- u. Hauptstudium. Eine kurze Beschreibung (auf Englisch) folgt}

%A brief description follows those courses which I regarded as relevant
%for the present application.

%\bigskip
%%\begin{tabular}{rl}
%\begin{tabular}{@{}r@{\extracolsep{2em}}p{9cm}l}

%Analisi Matematica I and II & \\
%Geometria I and II & \\
%Algebra & Introduction to the Theory of Groups.\\
%&Elements of Ring and Field Theory.\\
%Fisica Generale I and II & \\
%Meccanica Razionale & \\
%Algebra Superiore & Groups with Operators. Decomposition of groups.
%\\& Modules. Semisimple Rings.\\&Representation of Finite Groups\\
%Istituzioni di Geometria Superiore&Algebraic Topology. Classifications of
%Compact Surfaces.\\& Poincar� Groups. Introduction to Manifolds and\\&Diferential Forms\\
%Istituzioni di Analisi Superiore&\\
%Istituzioni di Fisica Matematica&\\
%Istituzioni di Algebra Superiore I& Permutation Groups. Multiple and Sharp Transitivity.\\
%& Orbitals and Orbital Graphs. Primitive Groups.\\
%&Elements of 1st Order Logic.\\
%&Automorphism Groups of Relational Structures.\\&Countably Categorical Theories.\\
%&Universal and Homogeneous Structures.\\
%&Examples of Back and Forth Techniques.\\
%&Universal Properties of Random Graph.\\
%&Oligomorphic Groups\\
%Istituzioni di Algebra Superiore II& Free Groups and Presentation of Groups.\\
%&Finitely Presented Groups. Word Problem. \\
%&Topology on Groups. Projective Limits of Groups\\&Profinite Groups\\
%Topologia & Differential Topology on Manifolds. Fixed Point Theory.\\
%& Bifurcations. Singular Homology Theory.\\
%&Topological Degree\\
%Geometria Superiore & Simplectic Manifolds. % Stability.\\
%%&Stone-Che\v{c} Cohomology. 
%Deformations of Complex Manifolds\\
%\end{tabular}
%\subsection*{Zusammenfassung der Diplomarbeit}
%\crule

{\bf Zusammenfassung der Diplomarbeit {\rm (auf Englisch)}}\\
%\begin{tabular}{@{}r@{\extracolsep{2em}}p{10cm}l}
\crule
\begin{itemize}
\item[Titel:] on Pseudo-free Locally Nilpotent Groups

%Zusammenfassung:				
In my thesis I studied Locally Nilpotent Pseudo-free Groups due to Shelah et al.{}. These groups are devoted to find epi-universal objects in the class $\LN$ 						of locally nilpotent groups.

%Here a group $G$ is epi-universal in $\LN$ if any group in $\LN$ of cardinality
%at most $\card{G}$ is an epimorphic image of $G$.
							To obtain an epi-universal $\LN$ group was necessary 								to attain a
							universal object within a class of relational structures on 								which Pseudo-free groups are constructed. This was 								achieved by means of a Fra\"iss� limit
							construction. Finally I proved Pseudo-free groups to be								residually finite groups.

\end{itemize}
%\end{tabular}
\vfil
%\bigskip\noindent
{\bf Vorlesungen u. Seminare an der HU w\"ahrend des Promotionstudiums}\\[+2mm]
\crule\\[+2mm]
\begin{tabular*}{\textwidth}{r@{\extracolsep{2,5em}}l@{}}
WS 05/06			&Algebra und Logik. {\sl Einf\"urung in der Modelltheorie}\\
SS 06			&Modelltheorie. {\sl Hrushowski-Fra\"{i}ss\'e Amalgamation}\\  
WS 06/07			&Modelltheorie. {\sl Hrushowski's Fusion}\\
SS 07			&Algebra und Logik. {\sl Einf\"urung in ``Forcing Constructions''}\\
WS 07/08			&Modelltheorie. {\sl NIP und Einfache Theorien}
\end{tabular*}
\vfil
{\bf Thema der Doktorarbeit}\\
\crule\\[+2mm]
\dots



%\bigskip
%\vfil
%\subsection*
\newpage
{\bf Ausgew\"alte Konferenzen, Workshops u. Sommerschulen}\\
\crule\\[+2mm]
\begin{tabular}{@{}r@{\extracolsep{2,5em}}p{10cm}l@{}}

August 2004		&Perugia, S.M.I. Sommer Schule. %Prof. G.Alcober - Euskal Herriko Univ. Bilbao-\\
				{\sl Group Actions. Classification of Finite Groups of some ``simple'' 					order.
				Soluble and Nilpotent Groups}\\
%Complex Analysis& \sl Prof T.Suwa -Hokkaido University-\\
Dezember 2005	&Leeds, MODNET Sommer Schule. {\sl Elements of Stability Theory.
				Intermediate Model Theory}\\
April 2006			&Freiburg, MODNET Sommer Schule. {\sl Advanced Stability. Algebraic 				Geometry}\\
Juni 2006			&Lyon, MODNET Training Workshop. {\sl Hrushowski Amalgamation and 				Fusion. Simple Theories}\\
Juni 2006			&Lyon, Logicum Lugdunensis. {\sl http://math.univ-lyon1.fr/$\sim$logicum/logicumlugdunensis}\\
September 2006	&Oxford, MODNET Workshop in model theory. {\sl www2.maths.ox.ac.uk/logic/wsSept06.shtml}\\
November 2006	&Antalya, MODNET Mid-Term Conference. {\sl www.math.metu.edu.tr/modnet/}\\ 
Januar 2007		&Oberwolfach, MFO Workshop on Model Theory of Groups.\\
Juni 2007			&Camerino, MODNET Sommer Schule. {\sl Model Theory of Modules. 					Introduction to o-minimality. Stable Groups}\\
September 2007	&Berlin, MODNET Training Workshop. {\sl Model Theory of Fields and 				Applications. Construction of o-minimal Structures}\\
April 2008			&La Roche, MODNET Training Workshop. {\sl Motivic Integration. 					Model Theory of Valued Fields. Interaction between Model Theory 					and Number Theory}\\
Juni 2008			&Leeds, Around Classification Theory.\\
%{\sl www.maths.leeds.ac.uk/$\sim$%
%pillay/Classification\%20theory.htm}\\
Juli 2008			&Manchester, MODNET Summer School. {\sl Groups of finite Morley 				Rank. Finite Model Theory}

\end{tabular}

\newpage

%\subsubsection*
{\bf Sprachen}\\
\crule\\[+2mm]
\begin{tabular}{r@{\extracolsep{2,5em}}p{10cm}l@{}}
Italienisch & muttersprache\\
English & fliessend\\ % gute Kenntnis\\
Deutsch & fliessend. {\sl Sprachkurse  A1.2, B2.2, C1 Deutsch als Fremdsprache am Sprachenzentrum der HU.}
%& Technical Vocabulary was tested by an exam\\&during the undergraduate courses.
\end{tabular}
\vfil
\noindent
Berlin, \today\\[+7mm]
%Andrea Amantini\\[+7mm]
\dots\dots\dots\dots\dots\dots\dots
\end{document}
