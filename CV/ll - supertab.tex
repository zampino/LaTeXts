\documentclass[10pt,a4paper,english,german]{article}
\usepackage{babel}
\usepackage[latin1]{inputenc}
%\usepackage[top=3cm,left=3cm]{geometry}
%\usepackage{amsmath,amsfonts,amssymb}
\usepackage{graphicx}
\usepackage{multirow}
\usepackage[pageshow]{supertabular}
%\usepackage{fancyhdr}
%%\setlength{\headheight}{15.2pt}
%\pagestyle{fancy}
%\lhead{Andrea Amantini}
%\rhead{Lebeslauf}
%\renewcommand{\headrulewidth}{0.5pt}
%\newcommand{\LN}{\mathcal{LN}}
%\voffset=-100pt
\addtolength{\textheight}{4.55cm}
\linespread{1.25}
\pagestyle{empty}
%_____________________________________________________
%\title{Curriculum Vitae}
\title{Lebenslauf}
\date{}
\begin{document}
\maketitle
\thispagestyle{empty}
\centering
\begin{supertabular}{r@{\extracolsep{2em}}p{9cm}}
Name: 					& {\large\bf Andrea Amantini}\\
Geburtsort, Datum: 			& Florenz (Italien), 28.01.1980\\
Staatsangeh\"origkeit: 			& Italienisch\\
Adresse:					& Lausitzerstra{\ss}e 37, 10999 Berlin\\
Tel:						& +49 (0)30 3462 4345\\
Mob:						& +49 (0)176 2829 7792\\
E-Mail:					& amantini.andrea@gmail.com\\[+10mm]
\multicolumn{2}{l}{\large\bf Bildungsgang}\\[+1mm]
\hline\\[+1mm]
1994-1999	&Gymnasium ({\em Liceo Scientifico}) -- {\sl Florenz, Italien}\\[+3mm]
1999-2005	&Studium an der {\em Universit� degli Studi di Firenze} -- {\sl Florenz, Italien}\\
		&Studiengang Mathematik, Abschlusspr�fung:
		Diplom ({\sl Laurea in Matematica}) --
		Abschlussnote: 110/110 \emph{cum Laude}\\[+3mm]
Okt 2005 - heute		&Doktorand an der Humboldt Universit�t zu Berlin\\ 	
				&im Fach Mathematik -- Schwerpunkt: Algebra - Mathematische Logik, Modelltheorie,
				({\sl Betreuer: Prof. Dr. A. Baudisch})\\[+3mm]
Sep 2008 - Okt 2008			&Forschungsaufenthalt an der {\em Universit� Lyon 1}, Inst. Camille
Jordan -- {\sl Lyon, France}\\[+10mm]
\multicolumn{2}{l}{\large\bf Stellen}\\[+1mm]
\hline\\[+1mm]
Okt 2005 - Sep 2008 	& EU-Gastwissenschaftler an der HU Berlin im Rahmen des Marie Curie
					FP6 Netzwerksprojekt (MODNET)\\[+3mm]
Okt 2008 - Dez 2009 	& Wissenschaftlicher Mitarbeiter an der HU Berlin\\[+10mm]
%\end{tabular}
%%
%\begin{tabular}{r@{\extracolsep{2em}}p{8.5cm}}
\multicolumn{2}{l}{\large\bf Lehrt�tigkeiten an der HUB}\\[+1mm]
\hline\\[+1mm]
WS 08/09	& �bungsleiter im Fach Algebra I\\
SS 09		& �bungsleiter im Fach Gew�hnlichen Differentialgleichungen\\
WS 09/10	& �bungsleiter im Fach Algebra II\\[+6mm]
\multicolumn{2}{l}{\large\bf Sprachen}\\[+1mm]
\hline\\[+1mm]
Italienisch & Muttersprache\\
English & fliessend\\
Deutsch & fliessend {\sl (Zeugnisse  A1.2, B2.2, C1 Deutsch als Fremdsprache -- HU Sprachenzentrum, Berlin)}\\
Franz\"osisch & Grundkenntnisse\\[+5mm]
\multicolumn{2}{l}{\large\bf Betriebssysteme, Programmiersprachen, IDEs}\\[+1mm]
\hline\\[+1mm]
Mac OS X ($10.4-10.6$) 	&gute Kenntnisse\\
Unix/Linux, Windows			&friedlicher Umgang\\
\LaTeX \,``Typesetting" 	&erweiterte Kenntnisse\\
{\sl HTML, CSS}		&Grundkenntnisse\\
Objective-$C$, $C^{++}$ 	&Grundkenntnisse\\
Eclipse, XCode			&Grundkenntnisse\\
\end{supertabular}
\vfil
\flushright{Andrea Amantini\\Berlin, \today}
\end{document}