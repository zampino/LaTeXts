\documentclass[10pt,a4paper,english]{article}
\usepackage{babel}
\usepackage[latin1]{inputenc}
\usepackage{amsmath,amsfonts,amssymb}
\newcommand{\LN}{\mathcal{LN}}
\linespread{1.3}
%\pagestyle{empty}
%\pagenumbering{1}
\newcommand{\crule}{\begin{center}\rule{13cm}{.4pt}\end{center}}
%
\title{Curriculum Vitae}
\date{}
%____________________________________________________________---
\begin{document}
\maketitle
%\section*{Curriculum Vitae}
%\bigskip
\section*{Personal Data}
\crule
\begin{tabular*}{\textwidth}{@{}r@{\extracolsep{\fill}}l@{}}
Name: &\quad\textbf{Andrea Amantini}\\
Place and Date of Birth: &\quad\textbf{Florence, 28/01/1980}\\
Nationality: &\quad\textbf{Italian}\\
Address:&\quad\textbf{via de' Serragli 111, Florence 50124, Italy}\\
Telephone:&\quad\textbf{+39 055 229378}\\
e-mail:&\quad\texttt{amantini@students.math.unifi.it\quad(aldamant@tin.it)}
\end{tabular*}

\bigskip
\section*{Education}
\crule
\begin{tabular*}{\textwidth}{@{}r@{\extracolsep{\fill}}l@{}}
1994-1999&\quad %\textbf
{High school - Liceo Scientifico, Florence}\\
1999-2005&\quad %\textbf
{Student in Mathematics at Universit� degli Studi di Firenze}\\
&\quad Graduated 22/04/2005,\quad 110/110 \emph{cum laude}\\
August 2004&\quad %\textbf
S.M.I Summer School in Perugia, organized by\\&\quad I.N.D.A.M.
(Istituto Nazionale di Alta Matematica)
%Academic Record to Date&\quad Graduated 22/04/2005\quad 110/110 \emph{cum laude}
\end{tabular*}
\newpage
\section*{Mathematical Training}
\crule
	\subsection*{List of Undergraduate Courses attended}
	A brief description follows those courses which I regarded as relevant
	for the present application.

	\begin{tabular}{rl}
	Analisi Matematica I and II & \\
	Geometria I and II & \\
	Algebra & Introduction to the Theory of Groups.\\
	&Elements of Ring and Field Theory.\\
	Fisica Generale I and II & \\
Meccanica Razionale & \\
Algebra Superiore & Groups with Operators. Decomposition of groups.
\\& Modules. Semisimple Rings.\\&Representation of Finite Groups\\
Istituzioni di Geometria Superiore&Algebraic Topology. Classifications of
Compact Surfaces.\\& Poincar� Groups. Introduction to Manifolds and\\&Diferential Forms\\
Istituzioni di Analisi Superiore&\\
Istituzioni di Fisica Matematica&\\
Istituzioni di Algebra Superiore I& Permutation Groups. Multiple and Sharp Transitivity.\\
& Orbitals and Orbital Graphs. Primitive Groups.\\
&Elements of 1st Order Logic.\\
&Automorphism Groups of Relational Structures.\\&Countably Categorical Theories.\\
&Universal and Homogeneous Structures.\\
&Examples of Back and Forth Techniques.\\
&Universal Properties of Random Graph.\\
&Oligomorphic Groups\\
Istituzioni di Algebra Superiore II& Free Groups and Presentation of Groups.\\
&Finitely Presented Groups. Word Problem. \\
&Topology on Groups. Projective Limits of Groups\\&Profinite Groups\\
Topologia & Differential Topology on Manifolds. Fixed Point Theory.\\
& Bifurcations. Singular Homology Theory.\\
&Topological Degree\\
Geometria Superiore & Simplectic Manifolds. % Stability.\\
%&Stone-Che\v{c} Cohomology. 
Deformations of Complex Manifolds\\
\end{tabular}
\subsubsection*{%
%\multicolumn{2}{l}{
%\textbf{
Courses attended at S.M.I}
%&\\
\begin{tabular}{rl}
Algebra& \sl Prof G.Alcober -Euskal Herriko Univ. Bilbao-\\
&Group Actions. Classification of Finite Groups of
\\& some %\lqq 
simple %\rqq 
order.\\
&Soluble and Nilpotent Groups\\
Complex Analysis& \sl Prof T.Suwa -Hokkaido University-\\
\end{tabular}

\bigskip
%\newpage
\section*{Degree Thesis}
\subsection*{\textnormal{title:}\quad Pseudo-free Locally Nilpotent Groups}
\crule
\begin{quote}
In my thesis I studied Locally Nilpotent Pseudo-free Groups due to Shelah et al..
These groups are devoted to find epi-universal objects in the class $\LN$ of locally
nilpotent groups.

%Here a group $G$ is epi-universal in $\LN$ if any group in $\LN$ of cardinality
%at most $\card{G}$ is an epimorphic image of $G$.
To obtain an epi-universal $\LN$ group was necessary to attain a
universal object within a class of relational structures on which 
Pseudo-free groups are constructed. This was achieved by means of a Fra\"iss� limit
construction.
Finally I proved Pseudo-free groups to be residually finite groups.
\end{quote}

\bigskip
\section*{Languages}
\crule
\begin{tabular}{ll}
Italian & Motherlanguage\\
English & Good Knowledge\\
& Technical Vocabulary was tested by an exam\\&during the undergraduate courses.
\end{tabular}
\end{document}
