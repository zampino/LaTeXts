\documentclass[10pt,a4paper,english,german]{article}
\usepackage{babel}
\usepackage[latin1]{inputenc}
%\usepackage[top=3cm,left=3cm
%]{geometry}
\usepackage{amsmath,amsfonts,amssymb}
\usepackage{graphicx}
\usepackage{multirow}
\newcommand{\LN}{\mathcal{LN}}
\linespread{1.25}
%\pagestyle{empty}
%\pagenumbering{1}
\newcommand{\crule}{\rule{13cm}{.4pt}}
%____________________________________________________
%\title{Curriculum Vitae}
\title{Curriculum Vitae}
\date{}

\begin{document}
\maketitle
{\bf Pers\"onliche Daten}\\
\crule\\[+2mm] %\\[+3mm]
%\begin{tabular*}{\textwidth}{r@{\extracolsep{2em}}@{\bf}lr}
\begin{tabular}{r@{\extracolsep{2em}} l@{}c}
%\begin{tabular}{@{\sl}r>{\bf}l}

Name: 					& Andrea Amantini \\%&\multirow{7}{*}{
%\includegraphics[height=4.75cm]{cvPhoto1-alta} }\\
Geburtsort, Datum: 			& Florenz, 28.01.1980\\
Staatsangeh\"origkeit: 			& Italienisch\\
Adresse:					& Lausitzerstra{\ss}e 37, 10999 Berlin\\
%Tel:						& +49 (0)30 2093 5826\\[+3mm]
%Privatanschrift (bis 05/08/2008):& Prenzlauer Allee 196, 10405 Berlin\\
Tel:						& +49 (0)30 3462 4345\\
Mob:						& +49 (0)176 2829 7792\\
E-Mail:					& amantini.andrea@gmail.com
\end{tabular}
\vfil
%--------------------------------------------------------------------
{\bf %\large 
Bildungsgang}\\
\crule\\[+2mm]
\begin{tabular}{@{   }r@{\extracolsep{2em}}l}
%\multicolumn{2}{l}{\bf Bildungsgang}\\[+2mm] \hline 
%{\bf Bildungsgang}&	\\[+2mm] \hline

Sep 1994 - Jun 1999	&Gymnasium (Liceo Scientifico),\: Florenz\\[+3mm]
Okt 1999 - Feb 2005	&Studium an der {\sl Universit� degli Studi di Firenze}\\
				&Studiengang Mathematik\\
				&Abschlusspr�fung: Diplom (Laurea in Matematica)\\%22/04/2005,\:
				&110/110	\emph{cum Laude}\\[+3mm]
%August 2004				&S.M.I Sommer Schule in Perugia, organized by I.N.D.A.M. (Istituto Nazionale di Alta Matematica)\\
Okt 2005 - heute		&Doktorand an der Humboldt Universit�t zu Berlin\\ 	
				&in Mathematik - Algebra/Mathematische Logik, Modelltheorie\\
				&({\sl Betreuer Prof. Dr. A. Baudisch})\\
Okt 2005 - Sep 2008	&Stipendium Marie Curie FP6 MODNET\\
				
\end{tabular}
\vfil
{\bf %\large
Stellen}\\
\crule\\[+2mm]
\begin{tabular}{r@{\extracolsep{2em}}p{10cm}l}
Okt 2005 - Sep 2008 	& EU-Gastwissenschaftler an der HU Berlin -- Institut f�r Mathematik\\[+2mm]
Okt 2008 - Dez 2009 	& Wissenschaftlicher Mitarbeiter an der HU Berlin -- Institut f�r Mathematik\\
\end{tabular}
\newpage

{\bf Lehrt�tigkeiten an der Humboldt Universit�t}\\
\crule\\[+2mm]
\begin{tabular}{r@{\extracolsep{2em}}l}
WS 08/09	& �bungsleiter im Fach Algebra I\\
SS 09		& �bungsleiter im Fach Gew�hnlichen Differentialgleichungen\\
WS 09/10	& �bungsleiter im Fach Algebra II
\end{tabular}

\vfill\vfill
%{\bf\Large Mathematische Hochschulbildung}\\%[1mm]
% ------>  QUESTO LO LASCEREI

%{\sl Grund- u. Hauptstudium. Eine kurze Beschreibung (auf Englisch) folgt}

%A brief description follows those courses which I regarded as relevant
%for the present application.

%\bigskip
%%\begin{tabular}{rl}
%\begin{tabular}{@{}r@{\extracolsep{2em}}p{9cm}l}

%Analisi Matematica I and II & \\
%Geometria I and II & \\
%Algebra & Introduction to the Theory of Groups.\\
%&Elements of Ring and Field Theory.\\
%Fisica Generale I and II & \\
%Meccanica Razionale & \\
%Algebra Superiore & Groups with Operators. Decomposition of groups.
%\\& Modules. Semisimple Rings.\\&Representation of Finite Groups\\
%Istituzioni di Geometria Superiore&Algebraic Topology. Classifications of
%Compact Surfaces.\\& Poincar� Groups. Introduction to Manifolds and\\&Diferential Forms\\
%Istituzioni di Analisi Superiore&\\
%Istituzioni di Fisica Matematica&\\
%Istituzioni di Algebra Superiore I& Permutation Groups. Multiple and Sharp Transitivity.\\
%& Orbitals and Orbital Graphs. Primitive Groups.\\
%&Elements of 1st Order Logic.\\
%&Automorphism Groups of Relational Structures.\\&Countably Categorical Theories.\\
%&Universal and Homogeneous Structures.\\
%&Examples of Back and Forth Techniques.\\
%&Universal Properties of Random Graph.\\
%&Oligomorphic Groups\\
%Istituzioni di Algebra Superiore II& Free Groups and Presentation of Groups.\\
%&Finitely Presented Groups. Word Problem. \\
%&Topology on Groups. Projective Limits of Groups\\&Profinite Groups\\
%Topologia & Differential Topology on Manifolds. Fixed Point Theory.\\
%& Bifurcations. Singular Homology Theory.\\
%&Topological Degree\\
%Geometria Superiore & Simplectic Manifolds. % Stability.\\
%%&Stone-Che\v{c} Cohomology. 
%Deformations of Complex Manifolds\\
%\end{tabular}
%\subsection*{Zusammenfassung der Diplomarbeit}
%\crule
%----------------------------------------------------------------

{\bf %\large
Zusammenfassung der Diplomarbeit}\\
\crule\\[+2mm]
In meiner Diplomarbeit ``�ber pseudo-freie lokal nilpotente Gruppen"\:
wird die Existenz epi-universeller Objekte untersucht
in der Klasse der lokal nilpotente Gruppen. Zu dieser Ziel
wurden von Shelah et al. pseudo-freie lokal nilpotente Gruppen eingef�hrt,
die sich auf besondere relationale Strukturen st�tzen, wobei universellen Strukturen, epi-universelle Gruppen assoziiert sind.

Wehrend der Analyse der Torsion solcher Gruppen habe ich einen Fehler im originalen Artikel von obigen Autoren gefunden. Letztendlich ist bewiesen, die
betrachtete Gruppen seien residual endlich.

%In my thesis I studied Locally Nilpotent Pseudo-free Groups due to Shelah et al.{}. These groups are devoted to find epi-universal objects in the class $\LN$ 						of locally nilpotent groups.

%%Here a group $G$ is epi-universal in $\LN$ if any group in $\LN$ of cardinality
%%at most $\card{G}$ is an epimorphic image of $G$.
%To obtain an epi-universal $\LN$ group was necessary to attain a
%universal object within a class of relational structures on 
%which Pseudo-free groups are constructed. This was achieved by means of a %Fra\"iss� limit
%construction. Finally I proved Pseudo-free groups to be residually finite groups.
%\end{itemize}
%\end{tabular}
%-----------------------------------------
%\vfill\vfill
%{\bf Seminare an der HU w\"ahrend des Promotionstudiums}\\
%\crule\\[+2mm]
%\begin{tabular*}{\textwidth}{r@{\extracolsep{2em}}l@{}}
%WS 05/06			&Algebra und Logik. {\sl Einf\"urung in der Modelltheorie}\\
%SS 06			&Modelltheorie. {\sl Hrushowski-Fra\"{i}ss\'e Amalgamation}\\  
%WS 06/07			&Modelltheorie. {\sl Hrushowski's Fusion}\\
%SS 07			&Algebra und Logik. {\sl Einf\"urung in ``Forcing Constructions''}\\
%WS 07/08			&Modelltheorie. {\sl NIP und Einfache Theorien}
%\end{tabular*}

\vfill %\vfill
{\bf Aktueller Forschungsbereich bzw. Dissertationsthema}\\
\crule\\[+2mm]
Fra�ss�-Hrushowski Amalgamsverfahren werden untersucht in Zusammenhang mit
nilpotenten Gruppen endlichen Exponent beziehungsweise mit nilpotenten graduierten Lie-Algebren �ber einem endlichen K�rper. Ziel meiner Arbeit ist eine Erweiterung
auf h�heren Nilpotenzklassen der von Andreas Baudisch 1996 konstruierten
nil-$2$ {\sl new uncountably Categorical Group} von prim Exponent.

Diesem Zweck nach neue Pr�dimensionen sind eingef�hrt die Relatoren in h�herem Gewicht isolieren und eine
�ber die Nilpotenzklasse induktiv iterierte Konstruktion erlauben. Unsere Betrachtung
kann auf Homologische Methoden zur�ckgef�hrt werden.

%\bigskip
%\vfil
%\subsection*
\newpage
{\bf Ausgew. Workshops, Sommerschulen u. Konferenzen}\\
\crule\\[+2mm]
\begin{tabular}{r@{\extracolsep{2em}}p{10cm}l}

August 2004		&Perugia, S.M.I. Sommer Schule. %Prof. G.Alcober - Euskal Herriko Univ. Bilbao-\\
				{\sl Group Actions. Classification of Finite Groups of some ``simple'' 					order.
				Soluble and Nilpotent Groups}\\
%Complex Analysis& \sl Prof T.Suwa -Hokkaido University-\\
Dezember 2005	&Leeds, MODNET Sommer Schule. {\sl Elements of Stability Theory.
				Intermediate Model Theory}\\
April 2006			&Freiburg, MODNET Sommer Schule. {\sl Advanced Stability. Algebraic 				Geometry}\\
Juni 2006			&Lyon, MODNET Training Workshop. {\sl Hrushowski Amalgamation and 				Fusion. Simple Theories}\\
Juni 2006			&Lyon, Logicum Lugdunensis. {\sl http://math.univ-lyon1.fr/$\sim$logicum/logicumlugdunensis}\\
September 2006	&Oxford, MODNET Workshop in model theory. {\sl www2.maths.ox.ac.uk/logic/wsSept06.shtml}\\
November 2006	&Antalya, MODNET Mid-Term Conference. {\sl www.math.metu.edu.tr/modnet/}\\ 
Januar 2007		&Oberwolfach, MFO Workshop on Model Theory of Groups.\\
Juni 2007			&Camerino, MODNET Sommer Schule. {\sl Model Theory of Modules. 					Introduction to o-minimality. Stable Groups}\\
September 2007	&Berlin, MODNET Training Workshop. {\sl Model Theory of Fields and 				Applications. Construction of o-minimal Structures}\\
April 2008			&La Roche, MODNET Training Workshop. {\sl Motivic Integration. 					Model Theory of Valued Fields. Interaction between Model Theory 					and Number Theory}\\
Juni 2008			&Leeds, Around Classification Theory.\\
%{\sl www.maths.leeds.ac.uk/$\sim$%
%pillay/Classification\%20theory.htm}\\
Juli 2008			&Manchester, MODNET Summer School. {\sl Groups of finite Morley 				Rank. Finite Model Theory}
\end{tabular}

\vfill
%\subsubsection*
{\bf Sprachen bzw. Programmiersprachen}\\
\crule\\[+2mm]
\begin{tabular}{r@{\extracolsep{2em}}p{8cm}l}
Italienisch & Muttersprache\\
English & fliessend\\ % gute Kenntnis\\
Deutsch & fliessend {\sl (Sprachkurse  A1.2, B2.2, C1 Deutsch als Fremdsprache -- HU Sprachenzentrum, Berlin)}\\
Franz\"osisch & Grundkenntnisse\\[+5mm]
\LaTeX \,``Typesetting" &erweiterte Kenntnisse\\
$C^{++}$ &Grundkenntnisse 
\end{tabular}

\smallskip
%\vfil
%\noindent

\flushright{Berlin, \today}

%Andrea Amantini\\[+7mm]
%\dots\dots\dots\dots\dots\dots\dots
\end{document}
