%---------------------------------------------------------------------------
\documentclass%%
%---------------------------------------------------------------------------
  [fontsize=10pt,german%%          Schriftgroesse
%---------------------------------------------------------------------------
% Satzspiegel
   paper=a4,%%               Papierformat
   enlargefirstpage=on,%%    Erste Seite anders
   pagenumber=headright,%%   Seitenzahl oben mittig
%---------------------------------------------------------------------------
% Layout
   headsepline=on,%%         Linie unter der Seitenzahl
   parskip=off,%half,%%           Abstand zwischen Absaetzen
%---------------------------------------------------------------------------
% Briefkopf und Anschrift
   fromalign=right,%%        Plazierung des Briefkopfs
   fromphone=off,%%           Telefonnummer im Absender
   fromrule=on,%aftername,%%           Linie im Absender (aftername, afteraddress)
   fromfax=off,%%            Faxnummer
   fromemail=on,%%          Emailadresse
   fromurl=off,%%            Homepage
   fromlogo=on,%%           Firmenlogo
   addrfield=on,%%           Adressfeld fuer Fensterkuverts
   backaddress=off,%%          ...und Absender im Fenster
   subject=beforeopening,%%  Plazierung der Betreffzeile
   locfield=narrow,%%        (narrow,wide) zusaetzliches Feld fuer Absender
   foldmarks=off,%%           Faltmarken setzen
   numericaldate=off,%%      Datum numerisch ausgeben
   refline=narrow,%%         Geschaeftszeile im Satzspiegel
%---------------------------------------------------------------------------
% Formatierung
   draft=off%%                Entwurfsmodus
]{scrlttr2}
\LoadLetterOption{myLetter}
%---------------------------------------------------------------------------
% Weitere Optionen
\KOMAoptions{%%
}
%---------------------------------------------------------------------------
\usepackage[english,ngerman]{babel}
\usepackage[T1]{fontenc}
\usepackage[latin1]{inputenc}
\usepackage{url}
%---------------------------------------------------------------------------
% Fonts
%\setkomafont{fromname}{\large}
\setkomafont{fromaddress}{\small}%% statt \small
\setkomafont{pagenumber}{\sffamily}
\setkomafont{subject}{\mdseries}
\setkomafont{backaddress}{\mdseries}
%\usepackage{mathptmx}%% Schrift Times
%\usepackage{mathpazo}%% Schrift Palatino
%\setkomafont{fromname}{\LARGE}
%---------------------------------------------------------------------------
\begin{document}
%---------------------------------------------------------------------------
% Briefstil und Position des Briefkopfs
%\LoadLetterOption{DIN} %% oder: DINmtext, SN, SNleft, KOMAold.
\makeatletter
\@addtoplength{toaddrvpos}{-5mm}
\@addtoplength{refaftervskip}{5mm}
\@addtoplength{refvpos}{-5mm}
\@setplength{firstheadvpos}{10mm}
\@setplength{firstheadwidth}{\paperwidth}
\ifdim \useplength{toaddrhpos}>\z@
  \@addtoplength[-2]{firstheadwidth}{\useplength{toaddrhpos}}
\else
  \@addtoplength[2]{firstheadwidth}{\useplength{toaddrhpos}}
\fi
\@setplength{foldmarkhpos}{6.5mm}
\makeatother
%-----------------------------------------------------------------------------
%
%????? Absender-----> defined in myLetter.lco for the ``Letters'' folder
%                                               commands there are overridden by this file
%
%-------------------------------------------------------------------------------------------------
%\setkomavar{fromname}{Absender Name}
%\setkomavar{fromaddress}{ Stra�e\\12345 Ort.}
%\setkomavar{fromphone}{+49 (0)30 3462 4345}
\renewcommand{\phonename}{Telefon}
%\setkomavar{fromemail}{absender.name@provider.de}
\setkomavar{backaddressseparator}{, }
%\setkomavar{signature}{(Name)}
%\setkomavar{frombank}{}
\setkomavar{location}{}
%\setkomavar{location}{\\[8ex]\raggedleft{\footnotesize{\usekomavar{fromaddress}\\
%      Telefon:\ usekomavar{fromphone}}}}%% Neben dem Adressfenster
%---------------------------------------------------------------------------
%\firsthead{Frei gestalteter Briefkopf}
%---------------------------------------------------------------------------
%\firstfoot{Fu�zeile}
%---------------------------------------------------------------------------
% Geschaeftszeilenfelder
%\setkomavar{place}{Ort}
%\setkomavar{placeseparator}{, den }
\setkomavar{date}{\today}
%\setkomavar{yourmail}{1. 1. 2003}%% 'Ihr Schreiben...'
%\setkomavar{yourref} {abcdefg}%%    'Ihr Zeichen...'
%\setkomavar{myref}{}%%      Unser Zeichen
%\setkomavar{invoice}{123}%% Rechnungsnummer
%\setkomavar{phoneseparator}{}
%---------------------------------------------------------------------------
% Versendungsart
%\setkomavar{specialmail}{Einschreiben mit R�ckschein}
%---------------------------------------------------------------------------
% Anlage neu definieren
\renewcommand{\enclname}{Anlage}
\setkomavar{enclseparator}{: }
%---------------------------------------------------------------------------
% Seitenstil
%\pagestyle{plain}%% keine Header in der Kopfzeile
%---------------------------------------------------------------------------
\begin{letter}{an
Annette Lindemann

Personalabteilung

Fachinformationszentrum Karlsruhe\\
Gesellschaft f�r wissenschaftlich-technische Information mbH\\[+1mm]

Hermann-von-Helmholtz-Platz 1\\
76344 Eggenstein-Leopoldshafen}
%---------------------------------------------------------------------------
\setkomavar{subject}{\bf Bewerbung als Fachredakteur des ZentralBlatt MATH, Abteilung Mathematik u. Informatik, Berlin}
%---------------------------------------------------------------------------
\opening{Sehr geehrte Frau Lindemann}
Mein Name ist Andrea Amantini und ich m�chte mich hiermit f�r die von Ihnen ausgeschriebene Stelle als
Fachredakteur des ZentralBlatt MATH bewerben.

Maine Allgemeine Kenntnisse und Leistungen in der mathematische Forschung sind erst aus meinem
akademischen Bildungsgang zur�ckzuziehen:
\begin{itemize}
\item[>]abgeschlossener Diplomsstudium der Mathematik in Florenz, mit der Note 110/110 {\sl cum laude}.
\item[>]Mathematischer Promotionsstudium an der Humboldt Universit\"at. Schwerpunkt Mathematische Logik/Modelltheorie.
\end{itemize}
Zum letztere, da ich am 8.~Dezember die Dissertation eingereicht habe,
wird die Verteidigung und der Abschluss des Promotionsverfahrens voraussichtlich im Fr\"uhjahr 2011
stattfinden.

\smallskip
Meine bisherigen Studien der Gruppentheorie (MSC 20F05, 20F18, 20F40) als Diplomand und Forschungen in
der Modelltheorie, %(MSC 03Cxx),
sind schon seit mindestens 7 Jahren Jahren der besondere Schwerpunkt meiner akademischen und beruflichen Entwicklung.

Insbesondere habe ich mich mit Fragen aus der algebraischen und angewandte Modelltheorie, sowie aus
der geometrischen Stabilit�tstheorie f�r meine Dissertation besch�ftigt (03C45, 03C60, 03C98, 20F05, 20F11, 20J05).

%Grundlage der -- isbesondere algebraische -- Systeme
Im Allgemeinen ist Modelltheorie eine Branche der Mathematische Logik, die eine Einsicht %-- aus einem axiomatischen Standpunkt --
auf verschiedene Gebiete der Mathematik und deren axiomatischen Grundlagen bietet.
Als Beispiel gelte, der modelltheoretische Beweis\footnote{} von E.~Hrushovski
der %geometrischen
Mordell-Lang Vermutung aus der algebraische Geometrie.

Dar�ber hinaus besch�ftig sich die Modelltheorie der geordnete Strukturen (03C64) mit Fragen aus topologischen und
real-analytischen Felder. Andererseits besch�ftigt  sich die sogenannte {\em endliche} Modelltheorie (03C13) --
grob geschildert -- mit Fragen der Entscheidbarkeit und
aus der Komplexit�tstheorie (68Q15, 03D15).

\medskip
Aus meiner bisherige Erfahrung in der Forschung habe ich gelernt, wie es relevant sei, auf digitale
Ressourcen, sowie traditionelle Zeitschriften, verl�sslich und unkompliziert zugreifen zu k\"onnen.

Man merkt die Wichtigkeit der Indizierung und (Retro-)Digitalisierung der wissenschaftlichen Dokumente, insbesondere beim recherchieren von ``alten'' Originaltexten.
Dank der Database der
Digitalisierungszentrum (GDZ) an der SUB G�ttingen, z.B.~k�nnte ich eine Arbeit\footnote{} von
W.~Magnus von 1940 in der {\em Zeitschrift f�r die reine und
angewandte Mathematik} finden, welche der h�chsten Bedeutung f�r meine Dissertation gewesen ist.
Dasselbe gilt f�r einen Artikel\footnote{} auf Franz�sisch von M.~Lazard des Jahres 1954, den ich auf
dem NUMDAM gefunden habe.

\medskip
Ich interessiere mich an diese Stelle, weil
ich engagiert beim kompetenten Verbreiten des Wissenschaftlichen Wissens dabei sein will. 
Die Motivation daf�r liegt -- neben der Spannung beim Recherchieren --
auch bei meiner Zuneigung zur mathematischen Schrift als eleganter Kommunikationsform,
und der Faszination f�r wissenschaftlichen Publikationswesen als Objekt in sich, seien es B�cher einer Bibliothek sowie
pdf-texte einer elektronischen Verzeichnis.
Ich erkenne dabei, zwischen den beiden Medien eine Kontinuit�t,
die zuk�nftig weiterentwickelt werden muss und einer gegenseitige Unterst�tzung dient.

\bigskip
%Neben die oben genannten direkten Erfahrung mit der mathematischen Forschung
Eine Zusammenfassung der Eigenschaften, die ich mitbringen werde, will ich hier unten auflisten:
\begin{itemize}
\item Expertise mit den verschiedensten {\em front-ends} elektronischer %Mathematischen
Zeitschriften-Daten-banken bzw. Digitalisierter Bibliotheken: die ich f�r die Quelle meiner eigenen Forschung
benutzt habe: u.a.~ZBMath, MR (MathSciNet) der AMS, JStor, arXiv.org, Google Scholar,
Project Euclid, Science Direct usw. und daher meine Interesse,
bei einer der weltweit wichtigen Datenbanken ``hinter den Kulissen'' arbeiten zu k�nnen.

\item Aufmerksame Analyse der wissenschaftlichen Texten. In meiner Forschung habe ich mehrmals
Fehler in ver�ffentlichten Artikeln gefunden:
schon wehrend meiner Diplomarbeit �ber {\em Pseudo-Freien Lokal-nilpotenten Gruppen}\footnote{
In \dots wird behauptet solche Gruppen seien torsionsfrei, was allgemein nicht stimmt.}, habe ich einen wichtigen Fehler
im Originaltext entdeckt, unter dessen Koautoren, gilt S.~Shelah als einer der bedeutendsten zeitgen�ssischen
Logiker. Dasselbe geschah mit einem Artikel\footnote{\dots} von meinem Betreuer Prof.~A.~Baudisch bez�glich eines Beweises
der Amalgamation nilpotenter Lie Algebren.

\item Erfahrung mit Lehre auf Hochschulniveau, die ich als wissenschaftlicher Mitarbeiter der Humboldt Universit�t gef�hrt
habe. Dabei bin ich 
Herrn Dr.~W.~Kleinert sehr dankbar, mir Freihand gegeben zu haben, Spannende nicht-elementare
Aspekte der Algebra, sowie der Kategorientheorie und der Homologie den Studenten beizubringen. 

\item Erfahrungen in der netzwerk-orientierten europ�ischen Forschung durch den {\em Research Training Network}
MODNET im Rahmen der Marie Curie-Ma�nahme FP6 der Europ�ischen Kommission. Wehrend dieses Forschungsstipendiums
habe ich vielf�ltige Aspekte der wissenschaftlichen Kooperation zwischen mathematischen Institute und Dozenten aus
u.a.~Frankreich, England und Spanien kennen gelernt.

\item Erweiterte Kenntnisse der \LaTeX~Auszeichnungssprache f\"ur die Komposition -- nicht unbedingt -- mathematischer Texten,
in Zusammenhang mit dem Bib-\TeX~Literaturverzeichnis-Compiler. Erfahrung mit dem KOMA-script Paket.

\item sicherer Umgang mit den �blichen elektronischen DV- und Internet-Softwares, sowie Erfahrungen mit einige Programmiersprachen.
Die Bereitschaft, schnell und effektiv meine IT-Kenntnisse zu vertiefen, rund um der Optimierung der Datenbank und der
Webauftritt vom ZBMath Archiv. 

\end{itemize}
Ich bin bereit, die Aufgaben die mir Herr Teschke telefonisch kurz geschildert hat
(Auswahl und Indexierung der eingehenden Publikationen, Koordination der Gutachter, usw.)
effizient, verantwortungsbewusst und ausf�hrlich anzunehmen. 

\smallskip
Ich werde mich sehr freuen zum Vorstellungsgespr�ch eingeladen zu werden.
\closing{Mit freundlichen Gr��en}
%---------------------------------------------------------------------------
\ps{PS: Wenn n�tig, w�rde mein Doktorvater Prof.~Dr.~A.~Baudisch\footnote{erreichbar unter \url{baudisch@math.hu-berlin.de}}
sich freuen einen mir betreffenden Empfehlungsschreiben Ihnen zu schicken.}
%\encl{}
%\cc{}
%---------------------------------------------------------------------------
\end{letter}
\end{document}