%---------------------------------------------------------------------------
\documentclass%%
%---------------------------------------------------------------------------
  [fontsize=11pt,german,%%          Schriftgroesse
%---------------------------------------------------------------------------
% Satzspiegel
   DIV=9,
   paper=a4,%%               Papierformat
   enlargefirstpage=on,%%    Erste Seite anders
   pagenumber=headright,%%   Seitenzahl oben mittig
%---------------------------------------------------------------------------
% Layout
   headsepline=on,%%         Linie unter der Seitenzahl
   parskip=off,%half,%%           Abstand zwischen Absaetzen
%---------------------------------------------------------------------------
% Briefkopf und Anschrift
   fromalign=right,%%        Plazierung des Briefkopfs
   fromphone=off,%%           Telefonnummer im Absender
   fromrule=aftername,%%           Linie im Absender (aftername, afteraddress)
   fromfax=off,%%            Faxnummer
   fromemail=on,%%          Emailadresse
   fromurl=off,%%            Homepage
   fromlogo=on,%%           Firmenlogo
   addrfield=on,%%           Adressfeld fuer Fensterkuverts
   backaddress=off,%%          ...und Absender im Fenster
   subject=beforeopening,%%  Plazierung der Betreffzeile
   locfield=narrow,%%        (narrow,wide) zusaetzliches Feld fuer Absender
   foldmarks=off,%%           Faltmarken setzen
   numericaldate=off,%%      Datum numerisch ausgeben
   refline=narrow,%%         Geschaeftszeile im Satzspiegel
%---------------------------------------------------------------------------
% Formatierung
   draft=off%%                Entwurfsmodus
]{scrlttr2}
\LoadLetterOption{myLetter}
%---------------------------------------------------------------------------
% Weitere Optionen
\KOMAoptions{%%
}
%---------------------------------------------------------------------------
\usepackage[ngerman]{babel}
\usepackage[T1]{fontenc}
\usepackage[latin1]{inputenc}
\usepackage{url}
\usepackage{xcolor}
\newcommand{\tilda}{{\raise.17ex\hbox{$\scriptstyle\sim$}}}


%---------------------------------------------------------------------------
% Fonts
%\setkomafont{fromname}{\large}
\setkomafont{fromaddress}{\small}%% statt \small
\setkomafont{pagenumber}{\sffamily}
\setkomafont{subject}{\mdseries}
\setkomafont{backaddress}{\mdseries}
%\usepackage{mathptmx}%% Schrift Times
%\usepackage{mathpazo}%% Schrift Palatino
%\setkomafont{fromname}{\LARGE}
%---------------------------------------------------------------------------
\usepackage[backref,pdftex,hidelinks=true]{hyperref}
\begin{document}
%---------------------------------------------------------------------------
% Briefstil und Position des Briefkopfs
%\LoadLetterOption{DIN} %% oder: DINmtext, SN, SNleft, KOMAold.
\makeatletter
\@addtoplength{toaddrvpos}{-25mm}
\@addtoplength{refaftervskip}{-3mm}
\@addtoplength{refvpos}{-35mm}
\@setplength{firstheadvpos}{7mm}
\@setplength{firstheadwidth}{\paperwidth}
\ifdim \useplength{toaddrhpos}>\z@
  \@addtoplength[-2]{firstheadwidth}{\useplength{toaddrhpos}}
\else
  \@addtoplength[2]{firstheadwidth}{\useplength{toaddrhpos}}
\fi
\@setplength{foldmarkhpos}{6.5mm}
\makeatother
%-----------------------------------------------------------------------------
%
%????? Absender-----> defined in myLetter.lco for the ``Letters'' folder
%                                               commands there are overridden by this file
%
%-------------------------------------------------------------------------------------------------
%\setkomavar{fromname}{Absender Name}
%\setkomavar{fromaddress}{ Stra�e\\12345 Ort.}
%\setkomavar{fromphone}{+49 (0)30 3462 4345}
\renewcommand{\phonename}{Telefon}
\setkomavar{fromemail}{amantini%.andrea@%gmail.com}%
@math.hu-berlin.de}
\setkomavar{backaddressseparator}{, }
%\setkomavar{signature}{(Name)}
%\setkomavar{frombank}{}
\setkomavar{location}{}
%\setkomavar{location}{\\[8ex]\raggedleft{\footnotesize{\usekomavar{fromaddress}\\
%      Telefon:\ usekomavar{fromphone}}}}%% Neben dem Adressfenster
%---------------------------------------------------------------------------
\firsthead{\center\raggedleft\textcolor[HTML]{262626}{Andrea Amantini\\[-2mm]
\rule{\textwidth}{.5pt}
\usekomavar{fromaddress}\\
\usekomavar{fromemail}}
}
%---------------------------------------------------------------------------
\firstfoot{\begin{center}
\textcolor[HTML]{696969}{\rule{\textwidth}{.5pt}
\small{Andrea Amantini | \usekomavar{fromaddress} | \usekomavar{fromphone} | \usekomavar{fromemail}}}
\end{center}}
%---------------------------------------------------------------------------
% Geschaeftszeilenfelder
%\setkomavar{place}{Ort}
%\setkomavar{placeseparator}{, den }
\setkomavar{date}{\today}
%\setkomavar{yourmail}{1. 1. 2003}%% 'Ihr Schreiben...'
%\setkomavar{yourref} {abcdefg}%%    'Ihr Zeichen...'
%\setkomavar{myref}{}%%      Unser Zeichen
%\setkomavar{invoice}{123}%% Rechnungsnummer
%\setkomavar{phoneseparator}{}
%---------------------------------------------------------------------------
% Versendungsart
%\setkomavar{specialmail}{Einschreiben mit R�ckschein}
%---------------------------------------------------------------------------
% Anlage neu definieren
\renewcommand{\enclname}{Anlage}
\setkomavar{enclseparator}{: }
%---------------------------------------------------------------------------
% Seitenstil
%\pagestyle{plain}%% keine Header in der Kopfzeile
%---------------------------------------------------------------------------
\begin{letter}{\bf Konrad-Zuse-Zentrum f�r Informationstechnik Berlin\\Bereich Diskrete Mathematik\\Takustr. 7, 14195 Berlin-Dahlem.}
%---------------------------------------------------------------------------
\setkomavar{subject}{\bf Bewerbung als Wissenschaftlicher Angestellter
im Projekt ``Informationstechnische Werkzeuge f�r Museen'' - Kennziffer: WA 12/11}
%---------------------------------------------------------------------------
\opening{Sehr geehrte Damen und Herren,}
hiermit bewerbe ich mich als Wissenschaftlicher Angestellter
zur Mitarbeit im Projekt ``Informationstechnische Werkzeuge f�r Museen''.

\medskip
Aus meinem Diplom- und Promotionstudium der Mathematik -- mit den
Schwerpunkten Algebra
bzw. Mathematische Logik -- habe ich gelernt, komplex strukturierte Systeme durch rigorose syntaktische Konsistenz sowie kreatives Denken
zu analysieren, verstehen und erweitern.

Die auf der Seite des Projektes geschilderte Aufgaben, Standards und
Institutionen, sowie die informationstechnologische Verwaltung und Vermittlung musealer
Metadaten und Wissensinhalte haben mein Interesse stark geweckt.

Als pers�nliche Weiterentwicklung w�rde ich
die M�glichkeit einer engagierten Zusammenarbeit 
mit den bedeutendsten kulturellen Partnern
Berlins und Deutschlands 
und die Spezialisierung in semantischen Webtechnologien
f�r das museale Umfeld sehr sch�tzen.

\medskip
Neben dem Interesse an einer intensiven Einarbeitung
wird meine Bewerbung durch folgende Erfahrungen und Kompetenzen
unterst�tzt:
\begin{itemize}
\item Erfahrung mit diversen objektorientierten Sprachen
in Zusammenhang mit Web-anwendungen und CMS:
\begin{itemize}
\item Erfahrung mit {\em Buildout} Befehlszeile-gesteuerter Entwicklung unter Python
f�r die Erweiterung des Plone CMS -- zur Zeit bei der Unterst�tzung der Platform
RawNews (siehe \href{}{http://dev.raw-news.net}). 

\item Entwicklung unter dem PhP + MySQL basierten CMS Wordpress.
Design und Pflege der Webseite \href{}{www.labirintotheater.com}.
\end{itemize}

\item Webdesign durch XHTML und CSS zur
Erstellung benutzerfreundlicher Web-applikationen
(siehe meine Webseite \href{http://www.math.hu-berlin.de/~amantini}{www.math.hu-berlin.de/\tilda amantini}).
Erfahrung mit Javascript und der Bibliothek JQuery f�r ein effizientes {\em traversing} des DOMs und eine dynamische
Steuerung des CSS.
Interesse an den neuesten Webstandards wie HTML5 und CSS3.

\item Kenntnisse der elementaren SQL Syntaktische Regeln (mySQL) und Konzepte zum scripting relationaler Datenbanken. 
Ableitung relationaler Tabellen aus ER-Modellen. Anfangskenntnisse objektorientierter Datenbanken, haupts�chlich
in Verbindung mit den pythonischen Zope und Zeo.

\item Kommandozeile-betrieb und essentielle Serververwaltung unter Unix Systeme, insbesondere Debian Linux.
Erweiterte Kenntnisse des Mac OS X und Windows.

\item Ausgepr�gte Kommunikations- und Teamf�higkeit aus einer mehrj�hrigen netzwerkorientierten Forschungserfahrung im
durch die EU-Kommission geforderten Marie Curie {\em Research Training Network} MODNET.\\
Flie\ss ende Sprachkenntnisse in Italienisch, Englisch und Deutsch -- schriftlich und w�rtlich.
\end{itemize}

\closing{Mit freundlichen Gr��en,}

\end{letter}
\end{document}