%---------------------------------------------------------------------------
\documentclass%%
%---------------------------------------------------------------------------
  [fontsize=10pt,german,%%          Schriftgroesse
%---------------------------------------------------------------------------
% Satzspiegel
   DIV=9,
   paper=a4,%%               Papierformat
   enlargefirstpage=on,%%    Erste Seite anders
   pagenumber=headright,%%   Seitenzahl oben mittig
%---------------------------------------------------------------------------
% Layout
   headsepline=on,%%         Linie unter der Seitenzahl
   parskip=off,%half,%%           Abstand zwischen Absaetzen
%---------------------------------------------------------------------------
% Briefkopf und Anschrift
   fromalign=right,%%        Plazierung des Briefkopfs
   fromphone=off,%%           Telefonnummer im Absender
   fromrule=aftername,%%           Linie im Absender (aftername, afteraddress)
   fromfax=off,%%            Faxnummer
   fromemail=on,%%          Emailadresse
   fromurl=off,%%            Homepage
   fromlogo=on,%%           Firmenlogo
   addrfield=on,%%           Adressfeld fuer Fensterkuverts
   backaddress=off,%%          ...und Absender im Fenster
   subject=beforeopening,%%  Plazierung der Betreffzeile
   locfield=narrow,%%        (narrow,wide) zusaetzliches Feld fuer Absender
   foldmarks=off,%%           Faltmarken setzen
   numericaldate=off,%%      Datum numerisch ausgeben
   refline=narrow,%%         Geschaeftszeile im Satzspiegel
%---------------------------------------------------------------------------
% Formatierung
   draft=off%%                Entwurfsmodus
]{scrlttr2}
\LoadLetterOption{myLetter}
%---------------------------------------------------------------------------
% Weitere Optionen
\KOMAoptions{%%
}
%---------------------------------------------------------------------------
\usepackage[ngerman]{babel}
\usepackage[T1]{fontenc}
\usepackage[latin1]{inputenc}
\usepackage{url}
%---------------------------------------------------------------------------
% Fonts
%\setkomafont{fromname}{\large}
\setkomafont{fromaddress}{\small}%% statt \small
\setkomafont{pagenumber}{\sffamily}
\setkomafont{subject}{\mdseries}
\setkomafont{backaddress}{\mdseries}
%\usepackage{mathptmx}%% Schrift Times
%\usepackage{mathpazo}%% Schrift Palatino
%\setkomafont{fromname}{\LARGE}
%---------------------------------------------------------------------------
\begin{document}
%---------------------------------------------------------------------------
% Briefstil und Position des Briefkopfs
%\LoadLetterOption{DIN} %% oder: DINmtext, SN, SNleft, KOMAold.
\makeatletter
\@addtoplength{toaddrvpos}{-14mm}
\@addtoplength{refaftervskip}{1mm}
\@addtoplength{refvpos}{-16mm}
\@setplength{firstheadvpos}{10mm}
\@setplength{firstheadwidth}{\paperwidth}
\ifdim \useplength{toaddrhpos}>\z@
  \@addtoplength[-2]{firstheadwidth}{\useplength{toaddrhpos}}
\else
  \@addtoplength[2]{firstheadwidth}{\useplength{toaddrhpos}}
\fi
\@setplength{foldmarkhpos}{6.5mm}
\makeatother
%-----------------------------------------------------------------------------
%
%????? Absender-----> defined in myLetter.lco for the ``Letters'' folder
%                                               commands there are overridden by this file
%
%-------------------------------------------------------------------------------------------------
%\setkomavar{fromname}{Absender Name}
%\setkomavar{fromaddress}{ Stra�e\\12345 Ort.}
%\setkomavar{fromphone}{+49 (0)30 3462 4345}
\renewcommand{\phonename}{Telefon}
\setkomavar{fromemail}{amantini@math.hu-berlin.de}
\setkomavar{backaddressseparator}{, }
%\setkomavar{signature}{(Name)}
%\setkomavar{frombank}{}
\setkomavar{location}{}
%\setkomavar{location}{\\[8ex]\raggedleft{\footnotesize{\usekomavar{fromaddress}\\
%      Telefon:\ usekomavar{fromphone}}}}%% Neben dem Adressfenster
%---------------------------------------------------------------------------
%\firsthead{Frei gestalteter Briefkopf}
%---------------------------------------------------------------------------
\firstfoot{\center\rule{\textwidth}{.5pt}
{\small Andrea Amantini | \usekomavar{fromaddress} | \usekomavar{fromphone} | \usekomavar{fromemail}}}
%---------------------------------------------------------------------------
% Geschaeftszeilenfelder
%\setkomavar{place}{Ort}
%\setkomavar{placeseparator}{, den }
\setkomavar{date}{\today}
%\setkomavar{yourmail}{1. 1. 2003}%% 'Ihr Schreiben...'
%\setkomavar{yourref} {abcdefg}%%    'Ihr Zeichen...'
%\setkomavar{myref}{}%%      Unser Zeichen
%\setkomavar{invoice}{123}%% Rechnungsnummer
%\setkomavar{phoneseparator}{}
%---------------------------------------------------------------------------
% Versendungsart
%\setkomavar{specialmail}{Einschreiben mit R�ckschein}
%---------------------------------------------------------------------------
% Anlage neu definieren
\renewcommand{\enclname}{Anlage}
\setkomavar{enclseparator}{: }
%---------------------------------------------------------------------------
% Seitenstil
%\pagestyle{plain}%% keine Header in der Kopfzeile
%---------------------------------------------------------------------------
\begin{letter}{an Annette Lindemann{ \\[+2mm]}
Personalabteilung\\
Fachinformationszentrum Karlsruhe}
%Gesellschaft f�r wissenschaftlich-technische Information mbH\\[+1mm]
%---------------------------------------------------------------------------
\setkomavar{subject}{\bf Kennziffer 01/11\\Bewerbung als Fachredakteur f�r mathematische Software, Berlin}
%---------------------------------------------------------------------------
\opening{Sehr geehrte Frau Lindemann}
hiermit m�chte ich mich f�r die von Ihnen ausgeschriebene Stelle als
Fachredakteur.

Aufgrund meines abgeschlossenen Diplomstudiums der Mathematik in Florenz (Note 110/110 cum laude) und des mathematischen Promotionsstudiums an der Humboldt Universit�t in Berlin (Schwerpunkt Mathematische Logik/Modelltheorie),
besitze ich eine �berall solide und umfangreiche mathematische Kenntnis.

Vor kurzem habe ich meine Dissertation eingereicht und kann Fr�hjahr 2011 mit dem Abschluss des Promotionsverfahrens
rechnen.

\smallskip
Dar�ber hinaus verf�ge ich �ber Erfahrungen in der Datenbankbasierte Web-Entwicklung sowie der
Erstellung und Pflege einer Webseite durch PhP und MySQL.

\smallskip
Folgende F�higkeiten und St�rke werde ich mitbringen:
\begin{itemize}
\item[>]Erfahrung mit mathematischen Softwares wie MatLab oder SMath.

\item[>]Erfahrung und Spa� am Programmieren mit PhP und MySQL, neben XHTML, CSS und der Open Source CMS ({\sl hosted})
WordPress\footnote{die Seite \url{www.labirintotheater.com} wird von mir f�r eine Theatergruppe verwalten.}. Grundkenntnisse der
Objektorientierten Entwicklung mit Sprachen wie $C^{++}$ und Python.  

\item[>] Vielf�ltige Fremdsprachenkenntnisse, speziell im Mathematischen Kontext, dabei sehr gute
Deutsch, Englisch und Franz�sisch.

\item[>] Expertise mit den verschiedensten {\em front-ends} elektronischer %Mathematischen
Zeitschriften-Datenbanken bzw. Digitalisierter Bibliotheken:
u.a.~ZBMath, MR (MathSciNet), JStor, arXiv.org, Google Scholar,
Project Euclid, Science Direct, GDZ, NUMDAM.

%\item Aufmerksame  Analyse  wissenschaftlicher  Texte.  In meiner Forschung  habe ich mehrmals
%Fehler in ver�ffentlichten Artikeln gefunden: Unter anderem in dem Originaltext
%{\em On universal and epi-universal locally nilpotent groups} von
%R.\,G�bel, S.\,Shelah et al.~sowie in einem Artikel
%meines Betreuers Prof.~A.\,Baudisch bez�glich eines Beweises der Amalgamation nilpotenter Lie Algebren.
%%Aufmerksame Analyse der wissenschaftlichen Texten. In meiner Forschung habe ich mehrmals
%%Fehler in ver�ffentlichten Artikeln gefunden:
%%schon wehrend meiner Diplomarbeit �ber {\em Pseudo-Freien Lokal-nilpotenten Gruppen}\footnote{
%%In \dots wird behauptet solche Gruppen seien torsionsfrei, was allgemein nicht stimmt.}, habe ich einen wichtigen Fehler
%%im Originaltext entdeckt, unter dessen Koautoren, gilt S.~Shelah als einer der bedeutendsten zeitgen�ssischen
%%Logiker. Dasselbe geschah mit einem Artikel\footnote{\dots} von meinem Betreuer Prof.~A.~Baudisch bez�glich eines Beweises
%%der Amalgamation nilpotenter Lie Algebren.

%\item Erfahrung mit Lehre auf Hochschulniveau (u.a.~der Algebra, Kategorientheorie und Homologie), die ich als wissenschaftlicher Mitarbeiter von Herrn Dr.~W.~Kleinert
%an der Humboldt Universit�t gef�hrt habe.

\item[>] Erfahrungen in der Netzwerk-orientierten europ�ischen Forschung durch den {\em Research Training Network}
MODNET im Rahmen der Marie Curie-Ma�nahme FP6 der Europ�ischen Kommission.

\item[>] Erweiterte Kenntnisse der \LaTeX~Auszeichnungssprache f\"ur die Komposition mathematischer Texte,
in Zusammenhang mit der Literatur-kompilierer Bib-\TeX. Erfahrung mit dem KOMA-Skript Paket.

\end{itemize}
Ich bin bereit, die Aufgaben der Position verantwortungsbewusst anzunehmen und daf�r
neue IT Kenntnisse engagiert zu lernen oder vertiefen.


�ber die Einladung zu einem pers�nlichen Gespr�ch freue ich mich sehr.
\closing{Mit freundlichen Gr��en}
%---------------------------------------------------------------------------
\ps{PS: Wenn n�tig, w�rde mein Doktorvater Prof. Dr. A. Baudisch (\url{baudisch@math.hu-berlin.de})
sich freuen, Ihnen ein mich betreffendes  Empfehlungsschreiben
zu schicken.}
%---------------------------------------------------------------------------
\end{letter}
\end{document}