%---------------------------------------------------------------------------
\documentclass%%
%---------------------------------------------------------------------------
  [fontsize=12pt,%%          Schriftgroesse
%---------------------------------------------------------------------------
% Satzspiegel
   paper=a4,%%               Papierformat
   enlargefirstpage=on,%%    Erste Seite anders
   pagenumber=headright,%%   Seitenzahl oben mittig
%---------------------------------------------------------------------------
% Layout
   headsepline=on,%%         Linie unter der Seitenzahl
   parskip=off,%half,%%           Abstand zwischen Absaetzen
%---------------------------------------------------------------------------
% Briefkopf und Anschrift
   fromalign=right,%%        Plazierung des Briefkopfs
   fromphone=off,%%           Telefonnummer im Absender
   fromrule=aftername,%%           Linie im Absender (aftername, afteraddress)
   fromfax=off,%%            Faxnummer
   fromemail=on,%%          Emailadresse
   fromurl=off,%%            Homepage
   fromlogo=on,%%           Firmenlogo
   addrfield=on,%%           Adressfeld fuer Fensterkuverts
   backaddress=off,%%          ...und Absender im Fenster
   subject=beforeopening,%%  Plazierung der Betreffzeile
   locfield=narrow,%%        (narrow,wide) zusaetzliches Feld fuer Absender
   foldmarks=off,%%           Faltmarken setzen
   numericaldate=off,%%      Datum numerisch ausgeben
   refline=narrow,%%         Geschaeftszeile im Satzspiegel
%---------------------------------------------------------------------------
% Formatierung
   draft=off%%                Entwurfsmodus
]{scrlttr2}
\LoadLetterOption{myLetter}
%---------------------------------------------------------------------------
% Weitere Optionen
\KOMAoptions{%%
}
%---------------------------------------------------------------------------
\usepackage{german}
\usepackage[T1]{fontenc}
\usepackage[latin1]{inputenc}
\usepackage{url}
%---------------------------------------------------------------------------
% Fonts
%\setkomafont{fromname}{\large}
\setkomafont{fromaddress}{\small}%% statt \small
\setkomafont{pagenumber}{\sffamily}
\setkomafont{subject}{\mdseries}
\setkomafont{backaddress}{\mdseries}
%\usepackage{mathptmx}%% Schrift Times
%\usepackage{mathpazo}%% Schrift Palatino
%\setkomafont{fromname}{\LARGE}
%---------------------------------------------------------------------------
\begin{document}
%---------------------------------------------------------------------------
% Briefstil und Position des Briefkopfs
%\LoadLetterOption{DIN} %% oder: DINmtext, SN, SNleft, KOMAold.
\makeatletter
\@addtoplength{toaddrvpos}{-10mm}
\@setplength{firstheadvpos}{10mm}
\@setplength{firstheadwidth}{\paperwidth}
\ifdim \useplength{toaddrhpos}>\z@
  \@addtoplength[-2]{firstheadwidth}{\useplength{toaddrhpos}}
\else
  \@addtoplength[2]{firstheadwidth}{\useplength{toaddrhpos}}
\fi
\@setplength{foldmarkhpos}{6.5mm}
\makeatother
%-----------------------------------------------------------------------------
%
%????? Absender-----> defined in myLetter.lco for the ``Letters'' folder
%                                               commands there are overridden by this file
%
%-------------------------------------------------------------------------------------------------
%\setkomavar{fromname}{Absender Name}
%\setkomavar{fromaddress}{ Stra�\\12345 Ort.}
%\setkomavar{fromphone}{+49 (0)30 3462 4345}
\renewcommand{\phonename}{Telefon}
%\setkomavar{fromemail}{absender.name@provider.de}
\setkomavar{backaddressseparator}{, }
%\setkomavar{signature}{(Name)}
%\setkomavar{frombank}{}
%\setkomavar{location}{}
%\setkomavar{location}{\\[8ex]\raggedleft{\footnotesize{\usekomavar{fromaddress}\\
%      Telefon:\ usekomavar{fromphone}}}}%% Neben dem Adressfenster
%---------------------------------------------------------------------------
%\firsthead{Frei gestalteter Briefkopf}
%---------------------------------------------------------------------------
%\firstfoot{Fu�eile}
%---------------------------------------------------------------------------
% Geschaeftszeilenfelder
%\setkomavar{place}{Ort}
%\setkomavar{placeseparator}{, den }
\setkomavar{date}{\today}
%\setkomavar{yourmail}{1. 1. 2003}%% 'Ihr Schreiben...'
%\setkomavar{yourref} {abcdefg}%%    'Ihr Zeichen...'
%\setkomavar{myref}{}%%      Unser Zeichen
%\setkomavar{invoice}{123}%% Rechnungsnummer
%\setkomavar{phoneseparator}{}
%---------------------------------------------------------------------------
% Versendungsart
%\setkomavar{specialmail}{Einschreiben mit Rckschein}
%---------------------------------------------------------------------------
% Anlage neu definieren
\renewcommand{\enclname}{Anlage}
\setkomavar{enclseparator}{: }
%---------------------------------------------------------------------------
% Seitenstil
%\pagestyle{plain}%% keine Header in der Kopfzeile
%---------------------------------------------------------------------------
\begin{letter}{an Annette Lindemann\\
Personalabteilung

Fachinformationszentrum Karlsruhe\\
Gesellschaft fr wissenschaftlich-technische Information mbH

Hermann-von-Helmholtz-Platz 1\\
76344 Eggenstein-Leopoldshafen}
%---------------------------------------------------------------------------
\setkomavar{subject}{\bf Bewerbung als Fachredakteur vom ZentralBlatt MATH, Abteilung Mathematik u. Informatik, Berlin}
%---------------------------------------------------------------------------
\opening{Sehr geehrte Frau Lindemann}
Mein Name ist Andrea Amantini und ich m\"ochte mich hiermit fr die von Ihnen ausgeschriebene Stelle als
Fachredakteur vom ZentralBlatt MATH.

Maine Allgemeine Kenntnisse und Leistungen in der mathematische Forschung sind aus meinem
Akademischen Bildungsgang zur\"uckzuziehen:
\begin{itemize}
\item[>]abgeschlossener Diplomsstudium der Mathematik in Florenz, mit der Note 110/110 {\sl cum laude}
\item[>]Mathematischer Promotionsstudium an der Humboldt Universit\"at. Schwerpunkt Mathematische Logik/Modelltheorie
\end{itemize}
Zum letztere, da ich am 8.~Dezember die Dissertation eingereicht habe,
kann man im Fr\"uhjahr 2011 mit der Abschlu{\ss} der Promotion rechnen.

\smallskip
Meine bisherigen Studien der Gruppentheorie als Diplomand und Forschungen in
der algebraischen Modelltheorie, sind schon seit viele Jahren der besondere Schwerpunkt meiner akademischen Entwicklung.

Modelltheorie selbst ist eine Branche der Mathematische Logik, die eine Einsicht
auf den verschiedenste Gebiete der Mathematik bietet und die axiomatische Grundlage der -- isbesondere algebraische -- Systeme untersucht.

 wie algebraische
Geometrie, Algebra, Topologie. 

F\"ur beide mein Diplom- wie f\"ur meine Doktorarbeit habe ich mich st\"andig


daher ist mir h\"ochst bekannt wie es relevant sei, ein
verl�slichen und unkomplizierten Zugriff auf digitale Ressourcen, sowie traditionelle Zeitschriften.

von Neuesten Artikeln sowie \"alteste Dokumente.

dank der Database der
Digitalisierungszentrum (GDZ) an der SUB G\"ottingen, k\"onnte ich
eine Arbeit von Wilhelm Magnus 1940 in der {\em Zeitschrift f\"ur die reine und
angewandte Mathematik} finden, welche h\"ocster Bedeutung f\"ur meine Dissertation gewesen ist.

Dasselbe gilt f\"ur einen Artikel von Lazard, den ich auf
den ENDAM auf Franz�isch

Das nur zu 


\vspace{2em}
{\bf warum interessiere ich mich ?}
\begin{itemize}
\item[*] Engagement  beim Verbreiten der Wissenschaftlichen Wissen 

\item[*] Meine Zuneigung zum Schrift der Mathematik als elegante Kommunikationsform
und die Faszination f\"ur wissenschaftliche Publikationswesen als Objekt in sich, seien es Bcher einer Bibliothek sowie auch.

Die Spannung beim Recherchieren -- vom Inhaltsverzeichnis -- direkt im K\"orper 
 

\begin{itemize}
\item 
\end{itemize}
  
\end{itemize}


\vspace{2em}
{\bf Eine Zusammenfassung von dei Eigenschaften, die ich mitbringen werde:}
\begin{itemize}

\item Erfahrung mit Lehre auf Hochschulniveau: wie mein Curriculum Vit{\ae} erweist,
habe 

\item Expertise mit den verschiedensten {\em front-end} der elektronischen Mathematischen
Zeitschrift-Databasen bzw. Digitalisierte Bibliotheken: die ich f\"ur die Quelle meiner eigenen Forschung
benutzt habe: u.a. ZBMAth, MR (MAthSciNet) der AMS,
Project Euclid, JStor, Science Direct usw. und die Interesse,
bei einer der weltweit wichtigen \lq unter den Culissen \rq arbeiten zu
konnen.

\item Erweiterte Kenntnisse der \LaTeX~Auszeichnungssprache f\"ur die Komposition Mathematische Schriften,
sowie des Bib-\TeX~Literaturverzeichnis-Editors.

Die berietschaft 

\item methodische Indexierung der eingehenden Publikationen und Koordination der Gutachter
\item sicherer Umgang mit den \"ubliche elektronischen DV Software, sowie einige Programmiersprachen. Die Bereitschaft,
schnell und effektiv, neue 

\item Bereitschaft zum lernen von weiteren 

	\begin{itemize}
		\item Lokalisierung des Internet Dienstleistungen auf diversen Plattformen f\"ur
		eine Browser-unabh\"angige fruition der Zeitschriften
		\item Optimierung der Databasen und der {\sl interface}
		
	\end{itemize}
\item 


\end{itemize}
wie ich die Gebiete di Herr Teschke telefonisch mir kurz geschildert hat, effizient und
verantwortungsbewusst ausf\"urlich bedecken kann. 


Ich werde mich sehr freuen zum Vorstellungsgespr\"ach eingeladen zu werden.










\closing{Mit freundlichen Gr\"u\ss en}
%---------------------------------------------------------------------------
%\ps{PS:}
%\encl{}
%\cc{}
%---------------------------------------------------------------------------
\end{letter}
\end{document}