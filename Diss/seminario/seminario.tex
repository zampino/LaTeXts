\documentclass[a4paper,11pt,german,english]{report}
\usepackage{babel}
\usepackage[latin1]{inputenc}
\usepackage{amsmath,amsfonts,amssymb,amsthm}
\usepackage{ModNet}
\usepackage{FHL}
\usepackage{pdfsync}
%\usepackage{syntonly}
%\usepackage[all]{xy}
%\linespread{1.1}
%\usepackage[mathscr]{euscript}
\newcommand{\nl}[1]{\mathfrak{L}_{#1,\,p}}
\title{Seminar}
\pagestyle{empty}
%\renewcommand{\thesection}{\arabic{section}} 
%\setcounter{tocdepth}{3}
%\syntaxonly
%\includeonly{}
\begin{document}
%\subsubsection*{Motivations}
%\noindent
{\bf Motivations}
\begin{itemize}
\item[] '88 E.Hrushowski confutes Zil'ber trichotomy, exhibiting a strongly minimal
non locally modular structure with no groups, hence no fields. It was a Fra\"iss\'e amalgamation of relations mediated by a ``deficiency'' function between points and relations
$\delta(A)=\card{A}-\card{R(A)}$
\item[]'96 Baudisch constructs a Morley rank-$2$ Group with a non locally modular geometry with no fields. The $\delta$ argument is adapted to counting {\em generators} minus {\em relators}.
\item[]In both cases ``no field interpreted'' is a consequence of being $CM$-trivial. This property seems to be shared by all known FH-amalgam construction. [Wagner, Blossier, Pizarro]
\end{itemize}

\indent
{\bf Groups vs. Lie algebras}
\begin{itemize}
\item[]If we look for a FMR group with no field interpreted
then it must have a precise structure as the following result prescribes
%\end{itemize}
\begin{teo*}
Let $G$ be a connected group of finite Morley rank, which doesn't interpret a field,
then $G$ is a central product of a definable abelian subgroup $A$ and
a nilpotent one $B$ of bounded exponent.
\end{teo*}
%\begin{itemize}
\item[]Thereafter a reasonable class we may look at is the variety of groups $\mathfrak{N}_{c}
\cap\mathfrak{B}_{p}$: prime exponent $p$ and nilpotent of class $c$.

To be able to {\em count relations} we switch to dimension-friendly Lie algebras
after the standard correspondence
$$\mathfrak{N}_{c}\cap\mathfrak{B}_{p}\leftrightsquigarrow
\mathfrak{L}_{c,\,p}$$
%\nla{c,p}$$ %(\Fp)$$
$\mathfrak{L}_{c,\,p}$ stands for a suitable subcategory of graded Lie algebras of nilpotency class 
$c$ over the field with $p$ elements.

In one direction, for  any group $G$, the lower central filtration $(\gamma_{i}(G))_{i<\omega}$ induces a Lie ring $L_{G}=\bigoplus_{i<\omega}S^{i}$ for $S^{i}=\gamma_{i}(G)\quot\gamma_{i+1}(G)$
and Lie multiplication induced by the group commutator $[\,\,,\,\,]\colon S^{i}\wedge S^{j}\rightarrow S^{i+j}$.
In particular, as a ring $L_{G}$ is generated by the subgroup $S^{1}=G_{ab}$.

On the other hand, the Campbell-Baker-Haussdorf formula, produces a group $G^{L}$ from a given Lie Algebra $L$ (here is crucial $p>c$).

\item[]We require elements $M=M_{1}\oplus\cdots\oplus M_{c}$ of $\mathfrak{L}_{c,\,p}$
to be generated, as rings by the lowest-weight homogeneous $\Fp$-subspace, i.{}e. $M=\gen{M_{1}}$. We call $H$ a subalgebra of $M$ if $H=\gena{H_{1}}{M}$
for some vector subspace $H_{1}$ of $M_{1}$.
\end{itemize}

\indent
{\bf Modo Inductivo}
\begin{itemize}
\item[]{\em If} the nil-$2$ case works fine, and a (pre-) dimension ($\delta_{2}$) $d_{2}$ is constructed, for a nil-$3$ graded structure $M=M_{1}+M_{2}+M_{3}$ we will define a predimension for nil-$3$ algebras as
follows $$\delta_{3}M=d_{2}(M/M_{3})-\text{``{\em new} relators for $M_{3}$''}$$


Along the same line, for an arbitrary nilpotency class $c$,
we aim at obtaining: $\delta_{c-1}\rightsquigarrow d_{c-1}\rightsquigarrow\delta_{c}$. The corresponding geometries should arrange in extensions $\cl_{2}\inn\cl_{3}\inn\dots\inn\cl_{c}%\inn\dots\inn\cl_{p}
$ for a fixed prime $p$.
\end{itemize}
%\indent
%{\bf nil-$2$ to nil-$3$}

\begin{itemize}
\item[]%Recall the construction for the nil-$2$ case so call $\nla{2}(\Fp)$ our bi-graded
For $M\in\mathfrak{L}_{c,\,p}$ and a finite $H_{1}\inn M_{1}$ we define the {\em deficiency}
$$\delta_{2}(H_{1})=\dfp H_{1}-\dfp\left(
\exs H_{1}\cap N^{2}(M)\right)$$
if $M=M_{1}+M_{2}$ is {\em presented} by the free nil-$2$
algebra $\fla{2}{M_{1}}=M_{1}+\exs M_{1}(\in\mathfrak{L}_{2,\,p})$ modulo a {\em relators} ideal $N^{2}(M)$
which is homogeneous of weight $2$:
%$$\delta_{2}M=\dfp M_{1}-\dfp N^{2}(M)$$
$$M=M_{1}\oplus\exs M_{1}\quot N^{2}(M).$$


This is an isomorphism invariant of finitely generated algebras, since also
for $H=\gena{H_{1}}{M}$, $N^{2}(H)$ {\em equals} $\exs H_{1}\cap N^{2}(M)$.

\item[]Recall the definition of {\em strong embedding} $M\zsu L$ ($\delta_{2}$ has minimum
in $M_{1}$ among the subspaces of $L_{1}$ which contain $M_{1}$).

Construct the $\zsu[]{}$-homogeneous {\em rich} monster $\K^{2}$ whose age are the finitely generated objects of $\Kl^{2}$, the class formed essentially by ``locally'' positive-deficiency algebras:
$$\Kl^{2}=\{M\in\nla{2}\mid \triv\zsu M\}$$

{\em Uncollapsed} axiomatisation of $T^{2,\omega}=th(\K^{2})$ is made possible by ``asymmetric'' amalgamation, while
description of types is performed along the ``ab initio'' style evaluating the geometric
dimension $d_{2}$ arising from $\delta_{2}$. The uncollapsed axioms need a notion
of $n$-strongness, this provides an $\omega$-saturated model {\em infinite}
$d_{2}$-dimension since it embeds strongly free $n$-generated algebras for
all $n<\omega$.

\item[]Be careful: all the HF machinery actually takes place on the {\em definable} set $M_{1}$
(in particular $M_{1}$ is the domain for the pregeometry $\cl_{2}$ induced by $\delta_{2}$)
so actually all the relevant definition are referred to $M_{1}$. We choose indeed language
$\La_{ring}%\cup\{\lambda_{k}:k<p\}
\cup\{P_{1},\,P_{2},\,\dots,\,P_{c}\}$ for $\nl{c}$.
\end{itemize}

\indent
{\bf Relevant Relators in $\mathfrak{L}_{3,\,p}$}
\begin{itemize}
\item[]Let now $\mathfrak{L}_{3,\,p}\ni M=M_{1}+M_{2}+M_{3}=\fla{3}{M_{1}}\quot R$.

We want to isolate relations in $R_{3}$ which do really count -- read, don't arise from
relations in weight $2$ trough Lie brackets -- in order to compute their occurrences without
{\em repetitions}.

We define a ``left inverse'' to the natural
projection $\nl{3}\to\nl{2}$ which sends $M\mapsto M_{*}:=M\quot M_{3}$.

We mean a ``free lift'' map $\map{{\sf fl}}{\mathfrak{L}_{2,\,p}}{\mathfrak{L}_{3,\,p}}$ which sends
$$M\mapsto\fla{3}{M_{1}}\quot\J(M)$$ for $\J(M)$
%=N^{2}+[\fla{3}{M_{1}},N^{2}]$
the ideal of $\fla{3}{M_{1}}$ generated by
$N^{2}$ if $M=\fla{2}{M_{1}}\quot N^{2}$ i.{}e. $\J(M)=N^{2}+[\fla{3}{M_{1}},N^{2}]$.
Note that, even though $R$ lays in $\fla{2}{M_{1}}$, it may be
considered as an homogeneous additive subspace of $\fla{3}{M_{1}}$.

$\fl M$ satisfies the {\em universal property}: $(\fl M)_{*}=M$ and for any
$L\in\nla{3}$ with $L_{*}\simeq M$ (hence $L_{1}\simeq M_{1}$) we have
$\fl M\onto L$.

\item[]Define $F_{M}:=\fl {M_{*}}$ and obtain an
{\em induced presentation}
$$N^{3}(M)\hookrightarrow F_{M}\onto M$$
so that we may define,
for finitely generated algebras $M$ a new invariant deficiency
$$\delta_{3}(M)=\delta_{2}(M_{*})-\dfp N^{3}(M).$$

%Therefore the canonical map $\gamma_{\sss B}^{\sss A}$ where $B$ is a subalgebra of $A$,
%is to be understood as $\map{\gamma_{\sss B}^{\sss A}}{\flt{B}}{\flt{A}}$ if
%$A,B\in\nla{2}$ or $\map{\gamma_{\sss B}^{\sss A}}{\ftr{B}}{\ftr{A}}$
%or also $\map{\gam{B}{A}}{\ft{B}}{\ft{A}}$ if $A,B\in\nla{3}$.

\item[]The functor ${\sf fl}$ ``does not preserve embeddings'': if we consider the natural map
$\frl(i)=:\gam{M}{L}\colon F_{M}\to F_{L}$, this is not necessarily injective
for extensions $i\colon M\inn L$.%For subalgebra we mean $M=\gena{M_{1}}{L}$ for some subspace $M_{1}$ of $L_{1}$.

%The map above is natural since ${\gena{M_{1}}{F_{L}}}_{*}=M_{*}$.
Since ${\gena{M_{1}}{F_{L}}}_{*}=M_{*}$, we have $Im(\gam{M}{L})=\gena{M_{1}}{F_{L}}$ and $\ker(\gam{M}{L})=F^{3}(M_{1})\cap
\J(L)\quot\J(M)$.

\item[]It seems indeed more effective to define a {\em relative} deficiency depending
on the ambient structure $M$. For $H_{1}\inn M_{1}$:
$$\ded^{M}(H_{1})=d^{M}_{2}H_{1}-\dfp N^{3}_{M}(H_{1})$$

where by $N^{3}_{M}(H_{1})$ we mean $N^{3}(M)\cap\gena{H_{1}}{F_{M}}=\gam{H}{M}
(N^{3}(H))$ since $\gam{H}{M}\pam{M}=\pam{H}$.
%then $N_{M}^{3}(H_{1})$ is the full image
%under $\gam{H}{M}$ of $N^{3}(H)$ also
Also note $N^{3}(H)\nni\ker\gam{H}{M}$.

\item[]But we have first a good behaviour of $\fl{}$ on strong $\delta_{2}$-embeddings:

\begin{lem*}$M\zsu L\Rightarrow F_{M}\hookrightarrow F_{L}$
%Let $A\quot B$ be a strong extension of $\nla{2}$-algebras, then
%$\gam{B}{A}$ is a Lie monomorphism of $\flt{B}$ into $\flt{A}$ with image
%$\gena{B_{1}}{\flt{A}}$.
\end{lem*}

%   ########################################################################
%  ####### SHORTEN AND RETYPE INTO A NEW FILE ``SKETCH.tex''   ##############
% #######################################################################	

By the embedding lemma above
it follows that $\ded^{M}=\ded^{L}$ for all strong extensions $M\zsu L$.
Moreover $\ded^{M}$ and $\delta_{3}$ do agree on finite $2$-strong algebras of $M$.

%This choice is also motivated by this definition of $H\dsu M$:
We define $3$-strongness of $H$ in $M$, $H\dsu M$ if
\begin{itemize}
\item $H\zsu M$
\item $\ded^{M}(u\quot H)\geq0$ for all finite $u$ in $M_{1}$
\end{itemize}
\item[]Now the class we wish to amalgamate are the finite pieces in
$$\Kl^{3}=\{M\in\nla{3}\mid M_{*}\in\Kl^{2},\;\triv\dsu M\}$$

The following lemma ensures the class above
is first order axiomatisabile
\begin{lem*}
Let $L\in\nla{3}$ with $L_{*}\in\Kl^{2}$, then  $L\in\Kl^{3}$ if and only
if $\delta_{3}(H)\geq0$ for all finite $H\inn L$.
\end{lem*}

And this is essentially because, the unexpected turns out true: $\ded$ and $\delta_{3}$
are actually comparable.
\begin{lem*}
For any $L\in\nla{3}$ such that $L_{*}\in\Kl^{2}$ and any finite subalgebra $M\inn L$ with $M=\gena{M_{1}}{L}$, we have
$\ded^{L}(M)\leq\delta_{3}(M)$.
\end{lem*}
%\begin{proofof}[the lemma]
%Consider the map $\map{\gamma_{\sss M}^{\sss L}}{\ftr{M}}{\ftr{L}}$, since $\pam{M}=
%\gamma_{\sss M}^{\sss L}\,\pam{L}$ we have that $\gamma_{\sss M}^{\sss L}$ maps
%$\ker\pam{M}=N^{3}(M)$ onto $\ker\pam{L}\cap\gam{M}{L}(F_{M})=N_{L}^{3}(M_{1})$ and that $\ker\gamma_{\sss M}^{\sss L}\inn N^{3}(M)$.

%Therefore $\dfp N_{L}^{3}(M_{1})
%=\dfp N^{3}(M)-\ka{M}{L}$ and  $\delta_{3}(M)-\ded^{L}(M)=\delta_{2}(M)-
%d_{2}^{L}(M)-\ka{M}{L}$.
%\end{proofof}

\item[]And this lemma relies essentially on the following,
at a first glance improbable theorem which improves the above
embedding result
\begin{teo*}
Let $A\quot B$ be a finite extension of algebras in the class $\Kl^{2}$ and $A=\ssc_{2}^{A}(B)$, then $k_{B}^{A}:=\dfp\ker\gam{B}{A}\leq-\delta_{2}(A\quot B)$.
\end{teo*}
To use the statement above, first observe
that if $B$ is a finite subalgebra of some $M$ in $\nla{2}$ and if $A$ is the selfsufficient closure  
of $B$ in $M$, then since $\gam{B}{M}
=\gam{B}{A}\gam{A}{M}$ and $A\zsu M$, we have $k_{B}^{M}=k_{B}^{A}$.
Also since $A=\ssc_{2}^{M}(B)=\ssc_{2}^{A}(B)$, we have
\begin{cor*}
For any $M$ in $\nla{2}$ with $\sig{2}{2}$ and any finite subalgebra $B=\gena{B_{1}}{M}$
we have $\ker\gam{B}{M}\leq\delta_{2}(B)-d_{2}^{M}(B)$ for any $M$.
\end{cor*}

The theorem above relies on the following crucial lemma.
\begin{lem*}
Let $A\quot C$ be an extension of $\nla{2}$-algebras with %$\dfp(A_{1}\quot C_{1})=1$ and
$A_{1}=\gen{C_{1},a}$ for some $a$ in $A_{1}$ independent over $C_{1}$.

Assume $\delta_{2}(A\quot C)\leq0$ and $\left(\tpl{\psi}{n}\right)$ is a basis
for $N^{2}(A)$ over $N^{2}(C)$, where $\psi_{i}=[c_{i},a]-w_{i}$ for independent $\tpl{c}{n}$ in
$C_{1}$ and $w_{i}$ in $\exs C_{1}$, then $k_{C}^{A}\leq\dfp N^{2}(\geno{\tpl{c}{n}})$.
\end{lem*}
\end{itemize}
\indent
{\bf Amalgamation Issues}
\begin{itemize}
\item[]Sketch construction of the (asymmetric) free amalgam.
\item[]Force subaddittivity of $N^{3}_{M}(-)$ to get nonnegative local deficiency
of the free amalgam. This is solved for subspaces $H_{1},K_{1}\inn M_{1}$ with
$$\gena{H_{1}}{F_{M}}\cap\gena{K_{1}}{F_{M}}=\gena{H_{1}\cap K_{1}}{F_{M}}.$$

We can force this behaviour at level of $M$:
$$\gena{H_{1}}{{M}}\cap\gena{K_{1}}{{M}}=\gena{H_{1}\cap K_{1}}{{M}}$$
if we assume $H_{1}$ and $K_{1}$ to be ``$\cl_{a}$-closed'' in terms
of solutions of $[a,X]-w$ for $w\in H_{2}$ and a fixed $a\in H_{1}$.

Since $\cl_{a}\inn\cl_{2}$, a to pass from $H_{1}$ to $\cl_{a}(H_{1})$ seems
effective in $\ded$-computations.

\item[]One then try to amalgamate within the class
$$\Kl^{3}_{a}=\{M\in\nla{3}\mid M_{*}\in\Kl^{2},\,M_{1}=\cl_{a}(M_{1}),\,\triv\dsu M\}.$$

\item[]A notion of minimal extension, although seemingly sound, is of no use
if transitivity of $\dsu$ fails.
\end{itemize}

\newpage
\indent
{\bf Finitely Presented Groups}
\begin{itemize}
\item[]A {\em finitely presented} group $G$ is $(n,r)$-presented, if admits a presentation
with $n$ generators and $r$ relators. We define the {\em deficiency} of $G$ as
$$\mathrm{def}(G)=\max(n-r\mid G\,\,\text{is $(n, r)$-presented}).$$

It is possible to estimate the deficiency of $G$ in terms of the {\em Schur multiplicator}
$H_{2}(G)=H_{2}(G,\Z)$ of $G$: $$\text{def}(G)\leq r_{0}(G_{ab})-d(H_{2}(G))$$
here $d$ denotes the minimum cardinality of a set of generators.

The following form for the second integral homology group of $G$ given by {\em Hopf's
formula} is surprisingly close to our definitions:
$$H_{2}(G)=F^{\prime}\cap R\quot [F,R]$$
if $R\to F\to G$ presents $G$. The Schur multiplicator, does not depend on the particular
presentation for $G$.

There exists a notion of {\em variety} Schur multiplicator
$H^{\mathcal{V}}_{2}(G,B)$ for a right $G$-module and $G$ in the group variety
$\mathcal{V}$.

If we also take coefficient in $\Fp$ we get
$$\mathrm{def_{\mathcal{V}}}(G)\leq\dfp(G_{ab})-\dfp H_{2}^{\mathcal{V}}(G,\Fp).$$
Where now Hopf formula for $H_{2}^{\mathcal{V}}(G)$ is computed with respect
to a $\mathcal{V}$-presentation of $G$.

\item[]The above machinery can be used to prove things:
\begin{teo*}
Let $\phi$ be a group morphism of $G$ in $K$, if $\phi$
induces an isomorphism of $G_{ab}$ to $K_{ab}$ and an epimorphism
$\phi_{*}$ of $H_{2}(G)$ onto $H_{2}(K)$, then $\phi$ induces
isomorphisms of $G\quot\gamma_{i}(G)$ to $K\quot\gamma_{i}(K)$ for all
$i<\omega$.
\end{teo*}
Which has the corollary
\begin{teo*}
Let a group $G$ in $\mathcal{V}$ be $(n+r,r;\mathcal{V})$-presented, if $G_{ab}$
is generated by the $n$ element $\tpl{\bar x}{n}$ then $\tpl{x}{n}$ generate a $\mathcal{V}$-free subgroup of $G$.
\end{teo*}

\item[]The above translate, with minor changes, to our
Lie algebras.

In particular for a presentation $\mathfrak{r}\to\mathfrak{f}\to\mathfrak{g}$ one has the
Hopf formula $$H_{2}(\mathfrak{g})=\mathfrak{f}^{\prime}\cap\mathfrak{r}\quot[\mathfrak{f},
\mathfrak{r}].$$

In our graded case, then $\mathfrak{r}\inn\mathfrak{f}^{\prime}$ and
so the Schur multiplicator mods out the full relations with respect to the {\em shifted} term
$[\mathfrak{f},\mathfrak{r}]$ which automatically eliminates repetitions. Hence $H_{2}(\mathfrak{g})$ is a generalised $N^{3}(\mathfrak{g})$.
\end{itemize}
\end{document}