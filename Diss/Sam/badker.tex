\documentclass[a4paper,11pt,german,english]{article}
\usepackage{babel}
\usepackage[latin1]{inputenc}
\usepackage{amsmath,amsfonts,amssymb,amsthm}
\usepackage{ModNet}
\usepackage{FHL}
\usepackage{pdfsync}
%\linespread{1.3}
\title{A Bad Kernel}


\begin{document}
\maketitle
If $A\quot B$ is an extension of graded nil-2 Lie algebras, with $K(A\colon\!B)$ we denote the kernel of the canonical map $\map{\phi^{\,\ssm{A}}_{\tiny B}}{\fr{3}B}{\fr{3}A}$.
Moreover $k(A\colon\!B)=\dfp K(A\colon\!B)$.
We know that $K(A\colon\!B)=(\fla{3}{B_{1}}\cap\,\J(A))\quot\,\J(B)$ where $\J(M)=[N^{2}(M),M_{1}]$
is the ideal of $\fla{3}{M_{1}}$ generated by $N^{2}(M)$ for any graded nil-2 algebra $M$.

Assume now $A\nni A^{'}\nni B$.\\
From the definition above, it follows that $K(A\colon\!B)$ maps into
$K(A\colon\!A^{'})$ with kernel $K(A^{'}\colon\!B)$, hence
$k(A\colon\!B)\leq k(A\colon\!A^{'})+k(A^{'}\colon\!B)$ and, although $\phi_{B}^{\,A}=\phi_{B}^{\,A^{'}}\phi_{A^{'}}^{\,A}$ we cannot conclude equality holds.

Moreover our embedding Lemma\footnote{
Attention! Embedding Lemma \emph{does not} follow from inequality $k(A\colon\!B)
\leq-\delta_{2}(A\quot B)$ \underline{true for one-point extensions} $A\quot B$ such that $\delta_{2}(A\quot B)<0$.}
tells us that whenever $B\zsu A$ then $k(A\colon\!B)=0$.

%On the other hand we have $\phi_{B}^{\,A}=\phi_{B}^{\,A^{'}}\phi_{A^{'}}^{\,A}$ hence we conclude
%that 

\bigskip
We are going to build a non-selfsufficient nil-$2$ algerbas extension $A\quot B$ such that
$\delta_{2}(A\quot B)=-1$ but $k(A:B)\geq2$. Which contradicts a possible statement of the form
$k(A\colon\!B)\leq-\delta_{2}(A\quot B)$ in a general setting\footnote{
In the example which follows we may actually assume that $A$ is the selfsufficient closure
of $B$ is some big structure $M$.}.

\medskip
We split $A\quot B$ in two sections: $B\zsu A^{'}\subsetneq
A$ such that $\delta_{2}(A^{'}\quot B)=6$ and $\delta_{2}(A\quot A^{'})=-7$.
 
The generator spaces look as follows $B_{1}=\gen{b,\,b_{1},\,\dots,\,b_{8},\,e_{1},\,\dots,\,e_{8}}$,
$A^{'}_{1}=\gen{B_{1},\,a_{1}\dots,\, a_{6}}$ and $A_{1}=\gen{A_{1}^{'},\,c}$.

The idea is that $A^{'}$ is a six-point free extension of $B$ while $A\quot A^{'}$ is an one-point extension which carries $8$ independent relators. Therefore we assume
$N^{2}(A^{'})=N^{2}(B)$ and we let
$N^{2}(A\quot A^{'})=N^{2}(A\quot B)$ be generated by $\eta_{i}=[b_{i},c]-[b,e_{i}]$ for $i=1,\,\dots,\,8$.
Then assume $\varphi=[b_{1},b_{2}]-[b_{3},b_{4}]$ and $\psi=[b_{5},b_{6}]-[b_{7},b_{8}]$ generate $N^{2}(B)$.

\medskip
Now we want two independent elements $\Phi$ and $\Psi$ in $K(A\colon\!B)$, so set
$$\Phi=-[\eta_{1},b_{2}]+[\eta_{2},b_{1}]+[\eta_{3},b_{4}]-[\eta_{4},b_{3}]+[\varphi,c]$$
and
$$\Psi=-[\eta_{5},b_{6}]+[\eta_{6},b_{5}]+[\eta_{7},b_{8}]-[\eta_{8},b_{7}]+[\psi,c].$$
After Jacobi simplifications we get
$$\Phi=[b,e_{1},b_{2}]-[b,e_{2},b_{1}]-[b,e_{3},b_{4}]+[b,e_{4},b_{3}]$$
and
$$\Psi=[b,e_{5},b_{6}]-[b,e_{6},b_{5}]-[b,e_{7},b_{8}]+[b,e_{8},b_{7}]$$

Which are therefore elements in $\fla{3}{B}\cap\J(A)$ independent over $\J(B)$ and
hence lay in $K(A\colon\!B)$.

\medskip
Since $k(A^{'}\colon\!B)=0$ our inequality argument only tells us that 
$k(A\colon\!B)\leq k(A\colon\!A^{'})\leq -\delta_{2}(A\quot A^{'})$ but 
$-\delta_{2}(A\quot A^{'})>-\delta_{2}(A\quot B)$.

\medskip
Our main concern is that we lose information each time $\delta_{2}$ increases, moreover,
starting from the example above we may construct a $3$-nil algebra $B$ for which
the local delta $\delta_{3}(B)$ is actually smaller than the relative one $\delta_{3}^{M}(B)$.
%[EXP]--> just impose n^{3}_{A}(B) to be 1 so that n^{3}(B) will be bigger than 1+2=3.
% Then assume A to be the selfsufficient closure of B.

\medskip
To use an induction proof of the desired inequality we should force, for example,
one-point algebraic extension to always appear in those extensions of $B$
we want to calculate the relative delta in.
This cannot be required for the selfsufficient closure in general.

In particular it may very well happen
that $\ssc_{2}(B)\cap\ta B$ intersect in $B$ only. Although I am not sure a different
situation could help.
\end{document}
