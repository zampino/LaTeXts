\documentclass[english,german,11pt]{article}
\usepackage{babel}
\usepackage[latin1]{inputenc}
\usepackage{amsmath,amsfonts,amssymb,amsthm}
\usepackage{ModNet}
\usepackage{FHL}
%\usepackage{syntonly}
%\usepackage[all]{xy}
\title{Ein Beispiel}
\author{}
\date{}
%\abstract{Bubbo}
%\pagestyle{empty}
%\pagenumbering{1}
\begin{document}
\maketitle
Man betrachtet die folgende Algebren $A,B$ und $M$ aus $\nla{3}$.
$$B=\gen{x_{2},x_{3},x_{4},x_{5}:[x_{2},x_{3}]-[x_{4},x_{5}]}$$
$$A=\gen{x_{1},x_{2},\dots,x_{5}:[x_{1},x_{2}]-[x_{3},x_{4}],\,[x_{2},x_{3}]-[x_{4},x_{5}]}$$
$$M=\gen{x_{2},x_{3},\dots,x_{6}:[x_{2},x_{3}]-[x_{4},x_{5}],\,[x_{2},x_{3},x_{4}]-[x_{3},x_{4},x_{6}]}.$$

Wobei die Kommutatoren \emph{einfach} sind (d.h. $[\,\:,\:,\:]=[[\,\:,\:],\:]$) und
$[x_{2},x_{3},x_{4}]-[x_{3},x_{4},x_{6}]$ eine erlaubte Relation, die in $N^{3}(M)$ liegt, ist.

\smallskip
Man sieht leicht, da{\ss} $A\dso B\dsu M$. Dabei ist $A\quot B$ eine \emph{flache} Erweiterung
($\delta_{2}(A\quot B)=\delta_{3}(A\quot B)=0$) und
$M\quot B$ eine \emph{algebraische}.

\smallskip
Falls $L^{\prime}$ das freie Amalgam der auf Stufe $2$ abgeschnittenen Strukturen $A^{\prime},B^{\prime}$ und $M^{\prime}$ bezeichnet, gilt dann also $\fr{3}L^{\prime}=\fla{3}{x_{1},\dots,x_{6}}\quot\jei{L}$ wobei
$$\jei{L}=\genid{[x_{2},x_{3}]-[x_{4},x_{5}],\,[x_{1},x_{2}]-[x_{3},x_{4}]}{F^{3}}.$$

Nun geh\"ort $\alpha=[x_{2},x_{3},x_{4}]-[x_{3},x_{4},x_{6}]$ zu $N^{3}(M)$. Ferner hat man, modulo $\jei{L}$
$$\alpha=[x_{4},x_{5},x_{4}]-[x_{3},x_{4},x_{6}]=[x_{4},x_{5},x_{4}]-[x_{1},x_{2},x_{6}].$$
Deshalb liegt $\alpha$ in $N^{3}(x_{1},x_{2},x_{4},x_{5},x_{6})$ aber nicht in $N^{3}(x_{2},x_{4},x_{5},x_{6})$.

Schliesslich, da $N^{3}(A)=\triv$ ist, gilt zwar $N^{3}(L)\quot N^{3}(M)=\triv$, aber $N^{3}(x_{1},x_{2},x_{4},x_{5},x_{6})\quot N^{3}(x_{2},x_{4},x_{5},x_{6})\neq\triv$.
\pagestyle{empty}
\end{document}