\documentclass[english,german,11pt]{article}
\usepackage{babel}
\usepackage[latin1]{inputenc}
\usepackage{amsmath,amsfonts,amssymb,amsthm}
\usepackage{ModNet}
\usepackage{FHL}
%\usepackage{syntonly}
%\usepackage[all]{xy}

\title{$B$ strong in $A$ but not in $\ta A$}
\author{}
\date{}
\begin{document}
\maketitle
Assume $B\in\Kl^{3}$.

Remember $a$ is in any structure of the class, assume
$$B=\gen{a,e_{1},\,e_{2},\,\dots,\,e_{6},\,\dots\,|\,R(B)}$$
and $A=\gen{B_{1},\,u,\,v\,\mid R(B),\,[u,v,e_{1}]-[e_{1},e_{2},e_{3}],\,[u,v,e_{4}]-[e_{4},e_{5},e_{6}]}$.

Clearly $B\dsu A$. If now we attempt to solve equation $[a,X]-[u,v]$ and we add say a solution $z$ to $A_{1}$,
we get $A^{+}=\gen{A_{1},\,z\mid R(A),\,[a,z]-[u,v]}.$

\medskip
Unfortunately now $B$ is not $\delta_{3}$-strong in $A^{+}$.

We have in fact $d_{2}(z\quot B_{1})=1$ but
both $[a,z,e_{1}]-[e_{1},e_{2},e_{3}]$ and $[a,z,e_{4}]-[e_{4},e_{5},e_{6}]$ belongs to $N^{3}(B_{1},z)\non N^{3}
(B)$.

Observe that $A\quot B$ is a \emph{prealgebraic}�extension which is not \emph{good} in the sense of the definition given in {\tt status.tex}.
\end{document}
