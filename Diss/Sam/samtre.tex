\documentclass[english,german,11pt]{article}
\usepackage{babel}
\usepackage[latin1]{inputenc}
\usepackage{amsmath,amsfonts,amssymb,amsthm}
\usepackage{ModNet}
\usepackage{FHL}
\linespread{1.2}
\title{ein Beispiel gegen Subadditivit\"at}
\pagestyle{empty}
\author{}
\date{}
%\pagestyle{empty}
\begin{document}
\maketitle
\begin{itemize}
\item[] Man betrachtet die folgende Algebren $A,B$ und $M$ aus $\nla{3}$.
$$B=\gen{b_{1},b_{2},b_{3},b_{4}}$$
$$A=\gen{B_{1},\,a\mid[a,b_{1}]-[b_{2},b_{3}]}$$
$$M=\gen{m_{1},m_{2},\,B_{1}\mid\alpha=[b_{2},b_{3},b_{4}]-[m_{1},m_{2},m_{2}],\,\beta=[b_{2},b_{3},m_{1}]-[b_{4},m_{2},m_{2}]}.$$

Man sieht, da{\ss} $M\dso B\dsu A$, wobei $A\quot B$ eine algebraische
(mit $\delta_{2}(A\quot B)=\delta_{3}(A\quot B)=0$) und
$M\quot B$ eine \emph{pr\"aalgebraische}  Erweiterung ist.

Falls $L_{*}$ das freie Amalgam der auf Stufe $2$ abgeschnittenen Strukturen $A_{*}$,
$B_{*}$ und $M_{*}$ bezeichnet, gilt dann also $F_{L}=\fla{3}{m_{1},m_{2},B_{1},a}
\quot\jei{L}$ wobei
$$\jei{L}=\genid{[a,b_{1}]-[b_{2},b_{3}]}{\fla{3}{L_{1}}}.$$

Setze $N^{3}(L)=N^{3}(M)+N^{3}(A)$ and $L=F_{L}\quot N^{3}(L)$.

Sei nun $H_{1}=\geno{m_{1},m_{2},b_{1},b_{4},a}$, so $H_{1}\cap M_{1}=\geno{m_{1},m_{2},b_{1},b_{4}}$.

Darum liegen beide $\alpha\equiv[a,b_{1},b_{4}]-[m_{1},m_{2},m_{2}]$
und $\beta\equiv[a,b_{1},m_{1}]-[b_{4},m_{2},m_{2}]$ in $N_{L}^{3}(H_{1})\non N_{L}^{3}(H_{1}\cap M_{1})$.

Dabei ist $H_{1}$, und daher auch $H_{1}\cap M_{1}$, ein $\delta_{2}$-starker unterraum
mit $\delta_{2}(H\quot M)=\delta_{2}(A\quot M)=\delta_{2}(A\quot B)=0$ wehrend
$\delta_{2}(H\quot H\cap M)=1$.

Deshalb folgt $0=\ded^{L}(H\quot M)>\ded^{L}(H\quot H\cap M)=-1$, welches
der Subadditivit\"at des $\ded$ widerspricht.

\end{itemize}
\end{document}