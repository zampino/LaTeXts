\documentclass[a4paper,11pt,german,english]{article}
\usepackage{babel}
\usepackage[latin1]{inputenc}
\usepackage{amsmath,amsfonts,amssymb,amsthm}
\usepackage{ModNet}
\usepackage{FHL}
\usepackage{pdfsync}
%\usepackage{syntonly}
\usepackage[all]{xy}
%\linespread{1.3}
%\usepackage[mathscr]{euscript}
\title{Asymmetric Amalgam}
%\renewcommand{\thesection}{\arabic{section}} 
%\setcounter{tocdepth}{3}
%\syntaxonly
%\includeonly{}
\begin{document}
\maketitle\noindent
%\tableofcontents
Starting with $M\nni B\dsu A$ nil-$3$ algebras, we use the following

\smallskip
{\bf Notation and Setting:}
\begin{itemize}
\item[-]Asterisks in the subscript substitute the old $\tr{2}$ i.e. $M_{\ast}=\tr{2}M$
\item[-]$L_{*}:=M_{*}\amalg_{B_{*}}A_{*}$ {\em the $\Kl^{2}$ free (graded) amalgam}
\item[-]$\ftr{L}$ {\em is the ``free lift''} of $L$ or of $L_{*}$ i.e. $\fr{3}L$ or $\fr{3}\tr{2}{L}$ in the old notation, according to whether $L$ is in $\nla{2}$ or $\nla{3}$.
\item[-]if $L_{*}^{\sss +}:=\ta L_{*}$ {\em the $\ta$-closure of $L_{*}$} then we set $\ftr{L^{\sss +}}=\fr{3}{(L_{*}^{\sss +})}$
\item[-]$\gam{X}{Y}$ is the canonical map of $F_{X}$ into $F_{Y}$ with image $\gena{X_{1}}
{F_{Y}}$.
\end{itemize}

\medskip
We first prove a result concerning intersection of subalgebras under free lifting
%!TEX encoding = UTF-8 Unicodestrengthen an older result,
which will be crucial in the following.

Assume $L\in\nla{2}$ and $H=\gena{H_{1}}{L}$ and
$K=\gena{K_{1}}{L}$ are subalgebras of $L$,
%such that $\gena{H_{1}}{{L}}\cap\gena{K_{1}}{{L}}=\gena{H_{1}\cap K_{1}}{{L}}$.
then we ask whether
\begin{itemize}
\item[(d)]\quad\quad$\gena{H_{1}}{F_{L}}\cap\gena{K_{1}}{F_{L}}=\gena{H_{1}\cap K_{1}}{F_{L}}.$

A necessary condition to get (d) is first to get this intersection in weight $2$. Observe
that if $M\nni B\inn A$ are $\Kl^{2}$-algebras and if $L$ denotes their amalgam as above,
then for any $E_{1}\nni M_{1}$ and $D_{1}\nni A_{1}$ one has $\gena{E_{1}}{L}\cap\gena{D_{1}}{L}=\gena{E_{1}\cap D_{1}}{L}$. This is for if $w_{E}-w_{D}\in N^{2}(L)=N^{2}(M)+N^{2}(A)$ where $w_{E}$ and $w_{D}$ are homogeneous polynomials of weight $2$, then $w_{E}-v_{M}=w_{D}+v_{A}\in\exs E_{1}\cap D_{1}$ for some $v_{A}$ and $v_{M}$.

%Naturally a symmetric result holds
We now prove (d) in a very special case.
\end{itemize}

\begin{lem*}
Assume $L,M,A,B$ as above and $E_{1}\nni M_{1}$. Then
\begin{itemize}
\item[$(*)$]\quad\quad$\gena{E_{1}}{F_L}\cap\gena{A_{1}}{F_L}=\gena{E_{1}\cap A_{1}}{F_L}$
\end{itemize}
\end{lem*}
\begin{proof}
Suppose $w_{E}-w_{A}\in\J(L)$ where we
may assume $w_{E}$ and $w_{A}$ are homogeneous polynomial of weight $3$ lying
in $\fla{3}{E_{1}}$ and $\fla{3}{A_{1}}$ respectively. Both these algebras are considered
free subalgebras of $\fla{3}{L_{1}}$.

We arrange a basis $X$ for $L_{1}$ as follows $X=\{X^{a}>X^{e}>X^{B}>X^{m}\}$, where
$X^{B}$ is a basis for $B_{1}$, $X^{B}X^{m}$ is a basis for $M_{1}$, $X^{m}X^{B}X^{e}$ is
a basis for $E_{1}$ and $X^{e}X^{B}$ is a basis for $E_{1}\cap A_{1}$. With $X^{a}$ we complete
$X^{e}X^{B}X^{m}$ to a basis for $L_{1}$ and $X^{a}X^{e}X^{B}$ is a basis of $A_{1}$.


Now  since $\J(L)_{3}=\gen{\J(A),\J(M),[N^{2}(A),M_{1}],[N^{2}(M),A_{1}]}$ as an $\Fp$-subspace,
without loss of generality $w_{E}-w_{A}$, may be written as a sum of terms like
$[\nu_{M},x]$ and %$[\nu_{M},y]$ and
$[\nu_{A},y]$. Here and below, we consider $x\in X^{a}$ % $y\in X^{e}$
and $y\in X^{m}$, moreover $\nu_{A}\in N^{2}(A)$ and $\nu_{M}\in N^{2}(M)$.

Once each $\nu_{A}$ and $\nu_{M}$ is expressed as sums of basic monomials over $X$, we obtain an equality in $\fla{3}{L_{1}}$:

$$w_{E}-w_{A}=\mathcal{B}^{M,a}+%\mathcal{B}^{M,e}+
p\mathcal{B}^{A,m}$$

where 
\begin{itemize}
\item[$\mathcal{B}^{M,a}$] is a sum of basic terms $[m_{1},m_{2},x]$ for $m_{i}\in X^{B}X^{m}$
%\item[$\mathcal{B}^{M,e}$] is a sum of basic terms $[m_{1},m_{2},y]$ for $m_{i}\in X^{B}X^{m}$
\item[$p\mathcal{B}^{A,m}$]is a sum of prebasic terms $[a_{1},a_{2},y]$ for $a_{i}\in X^{a}X^{e}X^{B}$.
\end{itemize}
We now transform all prebasic monomials above into $[a_{1},y,a_{2}]-[a_{2},y,a_{1}]$ which are basic
and whose sum we denote by $\mathcal{B}^{A,m}_{*}$.

We obtain a sum of basic commutators
$$w_{E}-w_{A}=\mathcal{B}^{M,a}+
\mathcal{B}^{A,m}_{*}.$$

Since an expression in basic monomials for $w_{E}-w_{A}$ does not involve
{\em mixed} terms (i.{}e.{\,}monomials supported on both $X^{a}$ and $X^{m}$), by uniqueness all mixed terms must cancel each other from the sum $\mathcal{B}^{M,a}+%\mathcal{B}^{M,e}+
\mathcal{B}^{A,m}_{*}$. Cancellations do not arise within the same group, the only possibility
instead is that mixed $\mathcal{B}^{A,m}_{*}$-monomials cancel mixed $\mathcal{B}^{M,a}$-monomials and vice versa.

Consider indeed a term $[m_{1},m_{2},x]$ above with $m_{2}\in X^{m}$, %(note that such a term has to exist in $\mathcal{B}^{M,a}$),
%the corresponding prebasic which is meant to cancel it will be 
this is to be neutralised by the prebasic commutator
$[x,m_{1},m_{2}]$ of $p\mathcal{B}^{A,m}$, hence $m_{1}\in X^{B}$ and $\mathcal{B}^{A,m}_{*}$ contains the basic commutator $[x,m_{2},m_{1}]$ which differs from any $\mathcal{B}^{M,a}$-term. We deduce that no mixed $\mathcal{B}^{M,a}$-term is present in the sum above, and, with much similar arguments no
mixed $p\mathcal{B}^{A,m}$-term shows up as well.

We showed $w_{E}$ and $w_{A}$ lies, modulo $\J(L)$, in $\fla{3}{E_{1}\cap A_{1}}$.
\end{proof}

%\medskip
%Denote $D_{1}=H_{1}\cap K_{1}$, and set $\mathcal{H}:=(\gena{H_{1}}{F_{L}})_{3}$, $\mathcal{K}:=(\gena{K_{1}}{F_{L}})_{3}$, $\mathcal{D}:=(\gena{D_{1}}{F_{L}})_{3}$.

%\smallskip
%We artificially induce a graded Lie algebra structure on
%$$X=L\oplus\left( \mathcal{H}\oplus_{\mathcal{D}}\mathcal{K}\right)$$
%if we define Lie brackets to be nontrivial
%only for products of weight not greater than $3$, which arise from $\gena{H_{1}}{L}$
%or $\gena{K_{1}}{L}$.\footnote{this means $[w,y]\neq\triv$ if both $w$ and $y$
%belong to $\gena{H_{1}}{L}$ or $\gena{K_{1}}{L}$, moreover the value
%of $[w,y]$ is the one in $\mathcal{H}$ or in $\mathcal{K}$. This makes sense since
%$\gena{H_{1}}{F_{L}}=\gena{H_{1}}{L}\oplus\mathcal{H}$, and the same holds for $K$ and $D$}

%Now since $X_{*}=L$, we have an epimorphismus $\pam{X}$ of $\ftr{L}$ onto
%$X$. Moreover it {\em should be clear} that $\pam{X}$ maps $\gena{H_{1}}{\ftr{L}}$
%isomorphic onto $\gena{H_{1}}{X}$, and the same holds with $H_{1}$ and $K_{1}$.

%Observe that, since at the first two levels we have a good intersection, in $X$ we have
%$\gena{H_{1}}{X}\cap\gena{K_{1}}{X}=\gena{D_{1}}{X}$. Now,
%considering preimages through $\pam{X}$ we get $(\ast)$.

\bigskip\noindent
We proceed amalgamating the triple $M_{*}\nni B_{*}\zsu A_{*}$ to get $M_{*}\zsu L_{*}\nni A_{*}$, so that
$\gam{B}{A}$ and $\gam{M}{L}$ are monomorphisms.

Then set $\ke{B}{}=\ker\gam{B}{M}$ and $\ke{A}{}=\ker\gam{A}{L}$. Since $\gam{B}{M}\gam{M}{L}=\gam{B}{A}\gam{A}{L}$, we have $$\gam{B}{A}\ke{B}{}=\ke{A}{}\cap\gena{B_{1}}{\ftr{A}}$$
({\em by a basic commutator argument, similar to those used before, we can actually show that $\ke{A}{}\inn\gena{B_{1}}{\ftr{A}}$ and hence
$\gam{B}{A}\ke{B}{}=\ke{A}{}$ but what we stated should be enough for what follows}).

%\bigskip
%Thes i sthe argument \dots We choose an ordered base $X_{1}^{L}=\{X_{1}^{a}>X_{1}^{B}>X_{1}^{m}\}$ of $L_{1}$ where $X_{1}^{B}$ is a basis for $B_{1}$,
%$X_{1}^{A}=X_{1}^{a}X_{1}^{B}$ is a basis for $A_{1}$ and $X_{1}^{m}$ completes $X_{1}^{B}$ to a basis $X_{1}^{M}$ of $M_{1}$. Each of the three segments of $X_{1}^{L}$ is ordered arbitrarily.

%We consider the surjective map
%$\map{\lambda_{A}}{\fr{3}\tr{2}A}{\gena{A_{1}}{\fr{3}\lda}}$ as in \pref{lollipop}, where
%$\fr{3}\lda$ is presented by $\fla{3}{X_{1}^{L}}\quot\jei{L}$ and $\fr{3}\tr{2}A$ by $\fla{3}{X_{1}^{A}}\quot\jei{A}$.

%%To prove that $\lambda_{A}$ is an injection we have to show that
%We have $$\ker(\lambda_{A})=\frac{\fla{3}{X_{1}^{A}}\cap\jei{L}}{\jei{A}}%=\mathbf{0}
%$$
%moreover $\ker(\lambda_{A})$ is homogeneous of weight $3$.

%We note that
%\begin{labeq}{bulaba}
%\jei{L}_{3}=\gen{\jei{M}_{3},\,\jei{A}_{3},\,[\nu^{A},y],\,[\nu^{M},x]}_{+}^{\fla{3}{X_{1}^{L}}}
%\end{labeq}
%$$\text{where}\quad x\in X_{1}^{a},\,y\in X_{1}^{m},\,\nu^{A}\in N^{2}(A),\,\nu^{M}\in N^{2}(M)$$
%and we can assume both the $\nu^{A}$'s and the $\nu^{M}$'s to be independent over $\exs B_{1}$.

%Following the reasoning which led to expression \pref{basipre}, an element $w$ in $\ker(\lambda_{A})$
%%which is not zero
%may be written modulo $\jei{A}$ as
%\begin{labeq}{baluba}
%B_{A}=w=B^{M}+pB^{M}+B^{Ma}+pB^{aM}
%\end{labeq}
%where $B^{M}$, $pB^{M}$ are respectively sums of basic and prebasic commutators
%over $X_{1}^{B}X_{1}^{m}$ which cover the $\jei{M}$-part of $w$. $B^{Ma}$ is a sum of commutators
%arising from terms of type $[\nu^{M},x]$ in $w$, these are necessarily basic after the order of $X_{1}^{L}$
%we chose. $pB^{aM}$ denotes the sum of prebasic commutators obtained by the terms $[\nu^{A},y]$ of $\jei{L}$.
%$B_{A}$ is a linear combination of basic commutators over $X_{1}^{A}$ which write the word $w$
%as member of $\fla{3}{X_{1}^{A}}$.

%We transform the prebasic commutators in the right term of \pref{baluba} in basic ones,  by means of substitutions \pref{prebi}, thus we have
%$$B_{A}=B^{M}+B^{M}_{*}+B^{Ma}+B^{aM}_{*}$$
%where now both terms of the equality concern basic commutators over $X_{1}^{L}$.

%By uniqueness all
%the terms on the right which do not appear in $B_{A}$ must be cancelled by the sum.

%By the shape of the generators for $\jei{M}$ in \pref{bulaba}, we see that we fall in contradiction if
%some term of type $B^{Ma}$ or $pB^{aM}$ is present in \pref{baluba},
%because, after basic transformations, these cannot cancel each other, nor be cancelled by terms of $B^{M}+B_{*}^{M}$.

%As a consequence $B_{A}=w=B^{M}+pB^{M}\in\jei{M}$, and this forces $w$ to lay in
%$\fla{3}{X_{1}^{A}}\cap\fla{3}{X_{1}^{M}}=\fla{3}{X_{1}^{M}\cap X_{1}^{A}}=\fla{3}{X_{1}^{B}}$.

%We have
%\begin{labeq}{kerlam}
%K:=\ker(\lambda_{A})=\frac{(\jei{M}\cap
%\fla{3}{X_{1}^{B}})+\jei{A}}
%{\jei{A}}\inn\gena{B_{1}}{\fr{3}\tr{2}A}.
%\end{labeq}

%Now because $B_{1}\zsu A_{1}$, on account of lemma \ref{bellemma} we have $\jei{B}=\jei{A}\cap\fla{3}{X_{1}^{B}}$. Hence if we look at the analog of $\lambda_{A}$ for $B$ $\map{\lambda_{B}}{\fr{3}\tr{2}B}{\fr{3}\tr{2}M}$, it holds
%\begin{multline*}
%\ker\lambda_{B}=
%\frac{\jei{M}\cap\fla{3}{X_{1}^{B}}}{\jei{B}}=\\
%=\frac{\jei{M}\cap\fla{3}{X_{1}^{B}}}{\jei{A}\cap\fla{3}{X_{1}^{B}}}
%=\frac{\jei{M}\cap\fla{3}{X_{1}^{B}}}{\jei{A}\cap(\jei{M}\cap\fla{3}{X_{1}^{B}})}
%\simeq K
%\end{multline*}


\medskip
So if we write 
$\overline{F}_{\sss A}$ for $\ftr{A}\quot_{\!\ke{A}{}}$ and
$\overline{F}_{\sss B}$ for $\ftr{B}\quot_{\!\ke{B}{}}$ and we bar the maps
we take the quotients of,
%and call the respective quotient maps $
we obtain the following
{\em injective} commutative arrows
\begin{labeq}{pream}
\xymatrix{
&{\ftr{L}}&\\
{\ftr{M}}\ar%@{^{(}->}
[ur]^{\gam{M}{L}}&&{\overline{F}_{\sss A}%\quot_{\ke{A}{}}
}\ar%@{_{(}->}
[ul]_{\bgam{A}{L}}\\
&{\overline{F}_{\sss B}%\quot_{\ke{B}{}}
}\ar%@{_{(}->}
[ul]^{\bgam{B}{M}}\ar%@{^{(}->}
[ur]_{\bgam{B}{A}}&}
\end{labeq}
Recall we have $\gena{M_{1}}{\ftr{L}}=\gam{M}{L}\ftr{M}$ and $\gena{A_{1}}{\ftr{L}}=
\gam{A}{L}F_{\sss A}=\bgam{A}{L} \overline{F}_{\sss A}$.

Now since $L_{*}$ is the free amalgam of $M_{*}$ and $A_{*}$ over $B_{*}$, we have
 $\gena{M_{1}}{{L}_{*}}\cap\gena{A_{1}}{{L}_{*}}=\gena{M_{1}\cap A_{1}}{{L}_{*}}$
hence by $(\ast)$ above, $\gena{M_{1}}{\ftr{L}}\cap\gena{A_{1}}{\ftr{L}}=\gena{M_{1}\cap A_{1}}{\ftr{L}}=\gena{B_{1}}{\ftr{L}}$.

%\begin{center}
%\begin{itemize}
%\item[($\ast$)] we artificially induce a graded Lie algebra structure on
%$$X=L_{*}\oplus\left( (F_{\sss M})_{3} \oplus_{(\overline{F}_{B})_{3}}
%(\overline{F}_{\sss A})_{3} \right)$$
%if we define Lie brackets to be nontrivial
%for products of weight not greater than $3$, which arises from $\gena{M_{1}}{L_{*}}$
%or $\gena{A_{1}}{L_{*}}$ only\footnote{this means $[w,y]\neq\triv$ if both $w$ and $y$
%belong to $\gena{M_{1}}{L_{*}}$ or $\gena{A_{1}}{L_{*}}$, moreover the value
%of $[w,y]$ is the one in $\ftr{M}$ or in $\overline{F}_{\sss A}$. This makes sense since,
%$F_{\sss M}=\gena{M_{1}}{L_{*}}\oplus(F_{\sss M})_{3}$, $\overline{F}_{\sss A}=\gena{A_{1}}
%{L_{\ast}}\oplus(\overline{F}_{A})_{3}$ and $\overline{F}_{\sss B}=\gena{B_{1}}
%{L_{*}}\oplus(\overline{F}_{\sss B})_{3}$.
%}.

%Now since $X_{*}=L_{*}$, we have an epimorphismus $\pam{X}$ of $\ftr{L}$ onto
%$X$. Moreover $\pam{X}$ is injective on $\gena{M_{1}}{\ftr{L}}\simeq\ftr{M}$ and maps $\gena{B_{1}}{\ftr{L}}$
%onto $\gena{B_{1}}{X}$.

%Observe that since at the first two levels we are in a free amalgam we have
%$\gena{M_{1}}{X}\cap\gena{A_{1}}{X}=\gena{B_{1}%M_{1}\cap A_{1}
%}{X}$, now
%considering preimages through $\pam{X}$ we get the desired intersection.
%\end{itemize}
%\end{center}

\bigskip 
We now define $N_{A}:=\gam{A}{L}N^{3}(A)$
%=\bgam{A}{L}\overline{N}^{3}(A)$.
and $N_{M}:=\gam{M}{L}N^{3}(M)\simeq
N^{3}(M)$ and also set
%\begin{multline*}
$N_{B}:=\gam{A}{L}N^{3}_{A}(B_{1})=
%[EXP]
%\bgam{A}{L}( ( N^{3}_{A}(B_{1}) + \ke{A}{} )\quot \ke{A}{} )
\bgam{A}{L}(\bgam{B}{A} (N^{3}(B)\quot\ke{B}{}) )
=\gam{M}{L}(\bgam{B}{M} (N^{3}(B)\quot\ke{B}{}) )
=\gam{M}{L}N^{3}_{M}(B_{1})=N_{M}\cap\gena{B_{1}}{F_{L}}$.
%\end{multline*}

\medskip
Moreover all inclusions and arrows above preserve in $\ftr{L^{\sss +}}$ since
$\gam{L}{L^{\sss +}}$ is a monomorphism ($L_{1}\zsu L^{+}_{1}$), this allows us to define:
$$L:=\frac{\ftr{L}}{N_{M}+N_{A}}\quad\text{and}\quad L^{\sss +}:=\frac{\ftr{L^{\sss +}}}{N_{M}+N_{A}}$$
where $L\simeq\gena{L_{1}}{L^{+}}$ for $L_{1}=M_{1}\oplus_{B_{1}}A_{1}$
and a weight argument yields first that $N_{M}+N_{A}$ is an ideal of $\ftr{L^{\sss +}}$ and then
shows $N^{3}(L)=N^{3}(L^{+})$ is actually $N_{M}+N_{A}$.

We have now to show $N^{3}_{L^{+}}(A_{1})=N^{3}_{L}(A_{1})=N_{A}$, and this holds since
\begin{multline*}
N^{3}_{L^{+}}(A_{1})=
N^{3}(L^{+})
\cap\gena{A_{1}}{\ftr{ L^{\sss +} } }=(N_{M}+N_{A})\cap\gena{A_{1}}{\ftr{ L}}=\\
=N_{A}+( N_{M} \cap \gena{A_{1}}{\ftr{ L}} )
=N_{A}+( N_{M}\cap\gena{M_{1}}{\ftr{L}} \cap \gena{A_{1}}{\ftr{ L}} )=\\
=N_{A}+( N_{M}\cap\gena{B_{1}}{\ftr{L}})=N_{A}+N_{B}=N_{A}
\end{multline*}

Since $\ke{A}{}\inn N^{3}(A)$, this also implies $A\simeq\ftr{A}\quot_{\!N^{3}(A)} \simeq\overline{F}_{\sss A}\quot_{\!N_{A}}\simeq\gena{A_{1}}{L^{\sss +}}(\simeq\gena{A_{1}}{L})$ and the same results hold for $M$ of course.

\bigskip\noindent
Now $M\dsu L$.

\smallskip
Assume $M_{1}\inn E_{1}\zsu L_{1}$.
%If we assume $A$ to be $\ta$-closed as well\footnote{
%or if we have $\gena{E_{1}}{L_{*}}\cap\gena{A_{1}}{L_{*}}=\gena{E_{1}\cap A_{1}}{L_{*}}$
%considering $L$ and {\em not} $L^{+}$. This is however
%to be discussed with the proof of rich/model theorem},
By $(\ast)$ we have $\gena{E_{1}}{\ftr{L}}\cap\gena{A_{1}}{\ftr{L}}=\gena{E_{1}\cap A_{1}}{\ftr{L}}$, hence if
%and to get this we essentially repeat the argument $(\ast)$ above,\footnote{{\bf Note:} To make this work here we have to assume that
%$\gena{E_{1}}{L^{+}_{*}}\cap\gena{A_{1}}{L^{+}_{*}}=
%\gena{E_{1}\cap A_{1}}{L^{+}_{*}}$ and this is achieved if we assume, say,
%$A$ to be $\ta$-closed as well.}
%this time with
%$$
%X:=L_{*}^{+}\oplus\left( (F_{\sss E})_{3} \oplus_{(\overline{F}_{D})_{3}}
%(\overline{F}_{\sss A})_{3} \right)
%$$
%where $D=\gena{E_{1}\cap A_{1}}{F_{L^{+}}}$.
we compute $N^{3}_{L^{+}}(E_{1})$ we get
$$
(N_{M}+N_{A})\cap\gena{E_{1}}{F_{L}}=N_{M}+(N_{A}\cap
\gena{E_{1}}{F_{L}})=N_{M}+N_{L}^{3}(E_{1}\cap A_{1}).
$$

%On the other hand $N_{M}\cap N_{L^{+}}^{3}(E_{1}\cap A_{1})=
%N^{3}_{L^{+}}(M_{1})
Therefore $N^{3}_{L}(E_{1})\quot N^{3}_{L}(M_{1})$ is isomorphic to
$N^{3}_{L}(E_{1}\cap A_{1})\quot N^{3}_{L}(B_{1})$ which is epimorphic image
of  $N^{3}_{A}(E_{1}\cap A_{1})\quot N^{3}_{A}(B_{1})$ through $\gam{A}{L}$.

\medskip
Since on the other hand we have $d_{2}^{L}(E_{1}\quot M_{1})\geq d_{2}^{A}(E_{1}\cap
A_{1}\quot B_{1})$,\footnote{follow arguments already used in the symmetric amalgam}
we get in the end $\ded^{L}(E_{1}\quot M_{1})\geq\ded^{A}(E_{1}\cap A_{1}\quot
B_{1})\geq0$ since $B\dsu A$.

\bigskip
If we wish $M\dsu L^{+}$ as well, we should prove (d) for
$$\gena{E_{1}}{\ftr{L^{+}}}\cap\gena{A_{1}}{\ftr{L^{+}}}=\gena{E_{1}\cap A_{1}}{\ftr{L^{+}}}$$
where now $M_{1}\inn E_{1}\inn L_{1}^{+}$ and conclude with the above arguments.

Since we do not have (d) in the case above, even for a couple of $2$-strong subalgebras,
we actually cannot prove $M\dsu L^{+}$ even if $L$ is a {\em symmetric} amalgam.
\end{document}
