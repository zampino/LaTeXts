\documentclass[11pt,english]{article}
\usepackage{babel}
\usepackage[latin1]{inputenc}
\usepackage{amsmath,amsfonts,amssymb,amsthm}
\usepackage{ModNet}
\usepackage{FHL}
\usepackage[normalem]{ulem}
\usepackage{mathrsfs}
\usepackage[leftbars,color]{changebar}
\cbcolor{black}
\setlength{\changebarwidth}{1pt}
\setlength{\changebarsep}{0.8cm}
%-----------MARGIN NOTES--------------------------------
\usepackage{marginnote}
\setlength\marginparsep{1cm}
\renewcommand*{\marginfont}{\sffamily\small} %\mdseries} %{\sffamily}
%\reversemarginpar
\begin{document}
\title{}
\author{}
\date{}
\maketitle
For any strong subalgebra $H$ of $\mathbb{M}$, we call a tuple $\bar a\inn\mathbb{M}_{1}$ a {\em strong tuple over $H$},
if $\bar a$ is linearly independent of $H_{1}$ and if $\gen{H_{1},\bar a}\zsu{}\mathbb{M}_{1}$.
\begin{teo}\label{teowei}
We fix a constant $c$ to the language $\Lan{2}$.

For any small model $M$ of $T^{2}_{c}$, and any type $p$ in $\ssp{n}{M}$ with $P_{1}(\bar x)$,
there is a finite strong $\nla{2}$-subalgebra $C$ of $M$,
%which is left invariant by those automorphisms of $M$ which fix the type $p$.
such that
%$C^{\sigma}=C$ for any automorphism $\phi$ of $M$ which fixes $p$
$C\inn\acl(Cb(p))$ and that $Cb(p)\inn\dcl^{eq}(C)$.
\end{teo}
\begin{proof} We divide the proof by cases.\\[+0.7mm]\noindent
{\bf Case 1:}\quad{\sl $p=\tp{\bar a}{M}$ for some
%Assume $p$ is $\tp{\bar m}{M}$, for some tuple $\bar m$ in $\mathbb{M}_{1}$.
{\em strong tuple} $\bar a=(\tupl{a}{0}{n-1})$ over $M$.}
%be a tuple in $\mathbb{M}_{1}$ linearly independent of $M_{1}$, such that
%$\ssc(M,\bar m)=\gena{M,\bar a}{\mathbb{M}}$.
%Let further $\bar b$ be a subtuple of $\bar a$ of length $d(\bar m/M)=d(\bar a/M)$ such that
%$d(\bar b/M)=\delta(\bar b/M)=d(\bar a/M)$. Note that $\bar b$ collects the first transcendental steps
%of any minimal decomposition of $\ssc(M,\bar m)$ over $M$.

\medskip
Let $\aut_{\{\!M\!\}}(\mathbb{M})$ denote the group of all automorphism of the monster $\mathbb{M}$,
which leave $M$ invariant.

If $\sigma$ is an automorphism of $\mathbb{M}$, whose restriction to ${M}$ fixes the type $p$,
then $\bar a^{\sigma}\equiv_{M}\bar a$.
%This means, by Proposition \ref{bafo}, that the application fixing $M$ and sending $\bar m$ to $\bar m^{\sigma}$
%may be extended to an isomorphism $\phi$ of $\gena{M,\bar a}{\mathbb{M}}$ onto $\ssc(M,\bar m^{\sigma})=
%%:\gena{M,\bar b}{\mathbb{M}}$.
%\gena{M,\bar a^{\sigma}}{\mathbb{M}}=:\gena{M,\bar b}{\mathbb{M}}$\mn{Careful! $\bar a^{\phi}$ {\bf may not} be $\bar a^{\sigma}$!!!}
%for some $\bar b\inn\mathbb{M}_{1}$ with $\phi\colon \bar a\stackrel[M]{\simeq}{\longmapsto}\bar b$.
%
%Proposition \ref{bafo} again now implies that $\phi$ is actually elementary, that is $\phi\colon\bar a
%\stackrel[M]{\equiv}{\longmapsto}
%\bar b$.
Since $M$ is small, the strong homogeneity of the monster implies, that
the action {\em on} $M$ of the stabiliser of the type $p$ %with respect of the action of
in $\aut(M)$ %on $\ssp{}{M}$
coincides with the action on $M$ %orbit of $\bar c$ %$C$
under the pointwise stabiliser of %$\bar m$ {\em and } %$\bar a$ (and 
$\bar a$ in $\aut_{\{\!M\!\}}(\mathbb{M})$. % which we denote by $\aut_{\{\!M\!\}}(\mathbb{M})$.%_{\bar m}$.

%On the other hand if $\sigma\in\aut_{\{\!M\!\}}(\mathbb{M})$ and fixes $\bar m$, then $\gen{M_{1},\bar a}^{\sigma}=
%\ssc(M_{1},\bar m)^{\sigma}=\ssc(M_{1},\bar m^{\sigma})=\gen{M_{1},\bar a}$ and hence $\gen{\bar a}^{\sigma}=\gen{\bar a}$.

%Since the self-sufficient closure commutes with automorphisms,
%one has
%$$\gena{M,\bar a^{\sigma}}{\mathbb{M}}=\ssc(M,\bar m)^{\sigma}=\ssc(M^{\sigma},\bar m^{\sigma})%{\gena{M_{1},\bar a}{\mathbb{M}}}^{\sigma}=\gena{M_{1},\bar a^{\sigma_{1}}}{\mathbb{M}}
%=\ssc(M,\bar m)=\gena{M,\bar a}{\mathbb{M}}$$
%and in particular $\gen{\bar a^{\sigma}}=\gen{\bar a}$ in $\mathbb{M}_{1}$.

%For any automorphism $\sigma$ of $\aut_{\{\!M\!\}}(\mathbb{M})$ %fixes $\bar a$ pointwise and let
%let $\sigma_{1}$ denote the restriction of $\sigma$ to $M_{1}$: $\sigma_{1}$ is a linear isomorphism in
%$\mathit{GL}(\mathbb{M}_{1})$ which leaves the subspace $M_{1}$ invariant.
%Let also $\sigma_{*}$ denote the graded Lie isomorphism induced by $\sigma_{1}$ on the free graded algebra
%$\fla{2}{\mathbb{M}_{1}}=\mathbb{M}_{1}\oplus\exs\mathbb{M}_{1}$.
%Remark that $\sigma_{*}$ leave the
%subspace $\exs\gen{M_{1},\bar a}$ of $\exs\mathbb{M}_{1}$ invariant.

\medskip
Let %$\mathcal{B}$ an ordered basis of $M_{1}$ and
$(\rho_{i})_{i<m}$ be a set in $\exs\gen{M_{1},\bar a}$ linearly independent over $\exs M_{1}$ which is a basis
of $\rd_{\mathbb{M}}(M,\bar a)$ over $\rd_{\mathbb{M}}(M)$. Since $M$ is self-sufficient, then $m\leq n$.%\dfp(\bar a/M_{1})$.
%After having expressed each $\rho_{i}$ as linear combinations of basic monomials
%over the ordered basis $\mathcal{A}=\{\bar a>\mathcal{B}\}$ of $\gen{M_{1},\bar a}$, we may assume that

We may find for all $i<m$ a tuple $\bar b_{i}=(\utpl{b_{i}}{0}{n})$ of length $n$, of not necessarily linearly independent elements of $M_{1}$
such that 
\begin{labeq}{rhodi}
\rho_{i}=\alpha_{i}+\beta_{i}+\gamma_{i}
%\rho_{i}=\alpha_{i}(\bar a)+\beta_{i}(\bar a,\bar c)+\gamma_{i}(\bar c)
\end{labeq}
%where $\alpha_{i}\in\exs\gen{\bar a}$, $\beta_{i}\in\exs\gen{\bar a,\bar c}\non\exs\gen{\bar a}+\exs\gen{\bar c}$ and
%$\gamma_{i}\in\exs\gen{\bar c}\non\rd(M)$ are supposed to be
%where the terms
%$\alpha_{i}$, $\beta_{i}$ and $\gamma_{i}$ are linear combinations of basic monomials
%over the ordered set $\{\bar a>\bar c\}$ and the ordering on $\bar c$ is the one inherited from $\mathcal{B}$. We can also
%require
where each $\alpha_{i}=\alpha_{i}(\bar a)$ is in $\exs\gen{\bar a}$
and every $\gamma_{i}$ %(\bar c)\in
-- when nontrivial -- lays in $\exs M_{1}\non\rd(M)$ for all $i<m$.
\sout{This yields\mn{Try: get the $c$'s right now}, since $M$ is a model,
that we can find a $c_{i}\in M_{1}$, such that $[c,c_{i}]=\bar\gamma_{i}$ for all $i<m$. Here $\bar\gamma_{i}$ denotes
the element of $M$ $\gamma_{i}+\rd(M)$.}

Moreover we can assume that
$$\beta_{i}=\beta(\bar a,\bar b_{i})=\sum_{k<n}[a_{k},{b_{i}}^{k}]\in\exs\gen{M_{1},\bar a}\non\exs\gen{\bar a}+\exs M_{1}.$$

\smallskip
%Now take a tuple $\bar b$ of $\mathcal{B}$ %be a tuple of $M$,
%minimal with the properties:
%\begin{itemize}
%\item[-]$\{\bar a,\bar b\}\nni\supp_{\mathcal{A}}(\beta_{i})$ for all $i$ and
%\item[-]$\gen{\bar a,\bar b}\nni\bar m$.\mn{or, embed this feature somewherelse!!!}
%\end{itemize}


\smallskip
Let now $\sigma$ be an automorphism  in
$\aut_{\{\!M\!\}}(\mathbb{M})$ which fixes %$\bar a$ and
$\bar a$ pointwise. 

\uwave{Denote by $\sigma_{1}$ the restriction of $\sigma$ to $M_{1}$: $\sigma_{1}$ is a linear isomorphism in
$\mathit{GL}(\mathbb{M}_{1})$ which leaves the subspace $M_{1}$ invariant.}
Let also $\sigma_{*}$ be the graded Lie isomorphism induced by $\sigma_{1}$ on the free graded algebra
$\fla{2}{\mathbb{M}_{1}}=\mathbb{M}_{1}\oplus\exs\mathbb{M}_{1}$.

With this notation, as $\sigma(\gena{M,\bar a}{\mathbb{M}})=\gena{M,\bar a}{\mathbb{M}}$, it follows $\sigma_{*}\left(\rd(M_{1},\bar a)\right)=\rd(M_{1},\bar a)$.
Moreover since $(\rho_{i})_{i<m}$ is a basis of $\rd(\bar a/M)$, for all $i$ we have %,${\rho_{i}}^{\sigma_{*}}\in\rd(M,\bar a)$, it follows
\begin{labeq}{rhosig}
{\rho_{i}}^{\sigma_{*}}-\sum_{j<m} s_{j}\rho_{j}=\mu
\end{labeq}
%for all $i$ and
for some $\mu$ in $\exs M_{1}$ and $s_{j}$ in $\Fp$. On the other side
$$
{\rho_{i}}^{\sigma_{*}}=\alpha_{i}(\bar a)%^{\sigma_{*}}
+\beta_{i}(\bar a,{\bar b_{i}})^{\sigma_{*}}+\gamma_{i}^{\sigma_{*}}
=
\alpha_{i}(\bar a)+\beta_{i}(\bar a,{\bar b_{i}}^{\,\,\sigma})+\gamma_{i}^{\,\sigma_{*}}
$$
where
\begin{labeq}{betasig}
\beta(\bar a,{\bar b_{i}}^{\,\sigma})=\sum_{k<n}[a_{k},\sigma({b_{i}}^{k})]
\end{labeq}

Now \pref{rhosig} becomes
\begin{labeq}{rhomuc}
\alpha_{i}-\sum_{j<m}s_{j}\alpha_{j}+\beta_{i}%(\bar a,{\bar b_{i}}^{\,\sigma})
^{\,\sigma_{*}}-\sum_{j<m}s_{j}\beta_{j}%(\bar a,{\bar b_{j}})
=\mu-\gamma_{i}^{\,\sigma_{*}}+\sum_{j<m}s_{j}\gamma_{j}\:\in\exs M_{1}
\end{labeq}
and %, by the choice of the terms in \pref{rhodi},
since $\gen{\alpha_{i},\beta_{i},\beta_{i}^{\,\sigma_{*}}\mid i<m}\cap\exs M_{1}=\triv$, one has
$(\alpha_{i}+\beta_{i}(\bar a,{\bar b_{i}}))^{\sigma_{*}}=\sum_{j<m}
s_{j}(\alpha_{j}+\beta_{j}(\bar a,{\bar b_{j}}))$ and by the same arguments, one gets
$$\beta(\bar a,{\bar b_{i}}^{\,\sigma})=\sum_{j<m}s_{j}\beta(\bar a,{\bar b_{j}}).$$

Hence in particular
%On the other hand, since $\gen{\bar a}=\gen{\bar a^{\,\sigma}}$, we have $a_{k}=\sum_{l<n}r_{k}^{l}\sigma(a_{l})$ for
%some $r_{k}^{l}$ in $\Fp$ and
\begin{labeq}{beta}
%\beta_{i}(\bar a^{\,\sigma},{\bar b_{i}}^{\,\sigma})
\sum_{k<n}[a_{k},\sigma({b_{i}}^{k})]=
\sum_{j<m}s_{j}\beta(\bar a,{\bar b_{j}})=\sum_{k<n}[a_{k},\sum_{j<m}s_{j}{b_{j}}^{k}].
%=\sum_{l<n}[\sigma(a_{l}),\sum_{\substack{j<m\\k<n}}r_{k}^{l}s_{j}{b_{j}}^{k}].
\end{labeq}

Now by Hall's Theorem \ref{},
in $\exs\mathbb{M}_{1}$ we have $[a_{k},{M}_{1}]\cap\sum_{j\neq k}[a_{j},{M}_{1}]=\triv$.
%and hence $\exs\gen{M_{1},\bar a}=\bigoplus_{k<n}[a_{k},{M}_{1}]$.
Therefore \pref{betasig} and \pref{beta} imply for all $k$, $[a_{k},\sigma({b_{i}}^{k})]=
\sum_{k<n}[a_{k},\sum_{j<m}s_{j}{b_{j}}^{k}]$
in $\exs\mathbb{M}_{1}$.

This yields $\sigma({b_{i}}^{k})=\sum_{j<m}s_{j}{b_{j}}^{k}$ and hence
$\gen{\bar b_{i}\mid i<m}^{\sigma}=\gen{\bar b_{i}\mid i<m}$.

\medskip
If we denote by $\bar b$ the tuple $\tupl{\bar b}{0}{m}$,
%since $\sigma$ fixes $\bar m$, we may also assume -- by making $\bar b$ bigger if necessary --
%that $\bar m\inn\gen{\bar a,\bar b}$ and again
then $\bar b\inn M_{1}$ and $\gen{\bar b}^{\sigma}=\gen{\bar b}$.

\medskip
Let now $\Gamma_{i}$ denote the image of $\gamma_{i}$ in $M_{2}$ modulo $\rd(M)$, that is
$\Gamma_{i}=\gamma_{i}+\rd(M)$ for all $i$. Since, by \pref{rhosig} and \pref{rhomuc} we have
$$
\rd(M)\ni{\rho_{i}}^{\sigma_{*}}-\sum_{j<m} s_{j}\rho_{j}=\gamma_{i}^{\,\sigma_{*}}-\sum_{j<m}s_{j}\gamma_{j},$$
we deduce
$${\Gamma_{i}}^{\sigma}=\gamma_{i}^{\,\sigma_{*}}+\rd(M)=\sum s_{j}\gamma_{j}
+\rd(M)=\sum s_{j}\Gamma_{j}$$
and hence, if $\overline{\Gamma}$ denotes the tuple $(\tupl{\Gamma}{0}{m})$,
we get $\gen{\overline{\Gamma}}^{\sigma}=\gen{\overline{\Gamma}}$.

\smallskip
Till now we have shown, that for any automorphism $\sigma$ of $M$,
if $\sigma$ fixes the type $p$, then $\sigma$ leaves both spaces $\gen{\bar b}$ and $\gen{\overline{\Gamma}}$
invariant. We will now prove that -- as a consequence $\gen{\bar b}$ and $\gen{\overline{\Gamma}}$ lay in $\acl(Cb(p))$.

\cbstart
To see this take an $\omega$-saturated elementary extension $M^{\prime}$ %\ess\mathbb{M}$
of $M$ which is forking independent of $\bar m$ over $M$.

Since $M$ is a model, $\fin{\bar m}{M}{M^{\prime}}$ implies both $\ssc(M_{1},\bar m)\cap M^{\prime}_{1}=M_{1}$
and $\ssc(M^{\prime},\bar m)\simeq\am{M^{\prime}}{M}{\ssc(M,\bar m)}=\gena{M^{\prime},\bar a}{\mathbb{M}}$
by lemma \ref{forkingchar}.

Then in particular $\rd(M^{\prime},\bar a)=\rd(M^{\prime})+\rd(M,\bar a)$
and hence %$\delta(\bar a/M^{\prime})=\delta(\bar a/M)$ and this means,
$\rd(\bar a/M^{\prime})=\rd(\bar a/M)$. It follows $(\rho_{i}=\alpha_{i}+\beta_{i}+\gamma_{i})_{i<n}$ is
a basis for $\rd(M^{\prime},\bar a)$ over $\rd(M^{\prime})$.
This means, the spaces $\gen{\bar b}$ and $\gen{\overline{\Gamma}}$
%_{i}\mid i<n}$ with $\Gamma_{i}=\gamma_{i}+\rd(M^{\prime})$
%can play the same role also for $\tp{\bar m}{M^{\prime}}$, and hence
%It follows both {\em finite sets} $\gen{\bar b}$ and $\gen{\overline{\Gamma}}$
are setwise fixed by the automorphism of
$M^{\prime}$ which fix $\tp{\bar m}{M^{\prime}}$ as well.

Moreover, as we are in a totally transcendental theory, $Cb(p)=Cb(\bar m/M)\\
=Cb(\bar m/M^{\prime})$ is
the $eq$-definable closure of a {\em finite} imaginary.

We may conclude that, since $M^{\prime}$ is enough saturated with respect to $\gen{\bar b}\cup\gen{\overline{\Gamma}}$ and $Cb(p)$,
both sets $\gen{\bar b}$ and $\gen{\overline{\Gamma}}$ must be permuted also under automorphism {\em of} $\mathbb{M}$ which
fix $Cb(p)$ pointwise. Since these are finite sets, this yields $\gen{\bar b, %}\cup\gen{
\overline{\Gamma}}\inn\acl(Cb(p))$.
\cbend

%\nni\bar b$ and $\ssc(\bar a,\bar b,\bar c)=\gen{C_{1},\bar a}$.
%Then
%$$\gen{{C_{1}}^{\sigma},\bar a}=\ssc(\bar b,\bar a)^{\sigma}=\ssc(\bar b^{\,\sigma},\bar a)=\gen{C_{1},\bar a}$$
%and hence $C^{\sigma}=C$.

\medskip
%If now $\bar b$ is the trivial tuple, that is $\rho_{i}=\alpha_{i}+\gamma_{i}$ for all $i$,
%then by $\pref{rhomuc}$ 

%\uwave{Since we may assume},\mn{{\bf risiko}!!!}
%that $\bar b$ is not constantly equal to $\triv$,
%let $b_{0}$ be an element of $\bar b$ and -- with 

Now since $M$ is a model, by means of $\sig{2}{4}$ we can find $c_{i}$ in $M_{1}$ such that
$[c,c_{i}]=\Gamma_{i}$ in $M$ for all $i<m$. This means $[c,c_{i}]$ can play the role of $\gamma_{i}$ in $\exs M_{1}$.
Moreover if the tuple $\bar c$ collects all the $c_{i}$, %by above remarks we have
we have $\bar c\inn\acl(%\bar b,
\bar\Gamma)$. % and $\overline{\Gamma}\inn\dcl(\bar b,\bar c)$.

Take $C\zsu{}M$ with $C_{1}=\ssc(\bar b,\bar c,c)$,
then on one side $C\inn\acl(\bar b,\bar c)$ and hence $C\inn\acl(Cb(p))$.

\medskip
On the other hand we have $\delta(\bar a/C)=\delta(\bar a/M)$. % and $\gen{C_{1},\bar a}\nni\bar m$,
Therefore, as $C$ is strong in $M$ and $\bar a$ is a strong tuple over $M$, Lemma \ref{fincharssc} imply
$$
d(\bar a/M)=\delta(\bar a/M)=\delta(\bar a/C)\geq d(\bar a/C)\geq d(\bar a/M)
$$
and hence $d(\bar a/C)=d(\bar a/M)$.

This also yields -- with Lemma \ref{fincharssc} again --  $M+\gen{C,\bar a}\zsu{}\mathbb{M}$ and hence $\ssc(C_{1},\bar a)=\gen{C_{1},\bar a}$. That is $\fin{\bar a}{C}{M}$.

\medskip
Now since $M$ is a model $\gen{C_{1},\bar a}$ meets
$\acl(C_{1})$ necessarily in $C_{1}$ and this gives with Corollary \ref{stationary}, that $\tp{\bar a}{C}$
is stationary. Now fact \ref{ziecb}\,(2.) implies $Cb(p)\inn\dcl^{eq}(C)$.

For such strong tuples $\bar a$ over $M$, denote $C$ with $WCb(\bar a/M)$. We have shown
\begin{gather*}
\begin{cases}C\inn\acl(Cb(\bar a/M))\\
Cb(\bar a/M)\inn\dcl^{eq}(C)\end{cases}
\end{gather*}\\[+1mm]\noindent
%%--------------------------CASE 2---------------------------
{\bf Case 2:}\quad{\sl $p=\tp{\bar e}{M}$ for $\bar e=\tupl{e}{0}{r-1}$ a linearly independent tuple in $\mathbb{M}_{1}$
over $M_{1}$}

\medskip
Take a strong tuple $\bar a$ over $M$ such that $\gena{M,\bar a}{\mathbb{M}}$ is $\ssc(M,\bar e)$.

We assume $\bar a$ is of length $n>r$ and that $a_{k}=e_{k}$ for all $i<r$.

Take a {\em Morley sequence} $(\bar a^{i})_{i<\omega}$ in $q=\tp{\bar a}{M}$ and let $C$ be a {\em weak canonical base}
for $q$ as constructed in Case 1 above.

With this we mean $\fin{\bar a^{i}}{M}{\bar a^{<i}}$ for all $i<\omega$ and since $\fin{\bar a}{C}{M}$, it follows
that $(\bar a^{i})_{i<\omega}$ is a Morley sequence also in $\tp{\bar a}{C}$.

\smallskip
By indiscernibility, if we let $\bar e^{i}$ denote the tuple $\tupl{a^{i}}{0}{r-1}$, we have in particular $\ssc(M,\bar e^{i})=\gen{M,\bar a^{i}}$
and also $\fin{\bar e^{i}}{M}{\bar e^{<i}}$ for all $i<\omega$. That menas, $(\bar e^{i})$ is a Morley sequence in $p$.
We have $C_{1}=\ssc(\bar b,\bar c,c)$ for tuples $\bar b$ and $\bar c$ constructed in the first case above and $\gen{C,\bar a^{i}}=\ssc(C,\bar e^{i})$ for all $i$.

\medskip
We claim that that the tuple $\bar b$ is contained in $\acl(Cb(p))$.

\smallskip\noindent
Let $m>\dfp(C_{1})$ and $Y$ denote $\gena{C,\bar a^{i}\mid i<m}{\mathbb{M}}=\underset{i<m}{{\circledast_{C}}}\gen{C,\bar a^{i}}$.
We denote by $Z$ the self-sufficient closure $\ssc(\bar e^{i}\mid i<m)$. We first show that $\genp{\bar b}\inn Z$.

%is strictly contained inside $Y$, then
Assume on the contrary $\bar b\nsubseteq Z_{1}$ not.
%does not contain the tuple $\bar b\bar c$ etirely. And not even, entirely the tuple $\bar b$ alone.

\medskip
Observe that $\fin{\bar e^{i}}{C}{\bar a^{<i}}$ implies - with Proposition \ref{forkingchar} -- for all $i<m$
\begin{labeq}{sscmolge}
\ssc(C,\bar a^{<i},\bar e^{i})=\gen{C,\bar a^{\leq i}}
\end{labeq}

Now define for all $i<m$
$$X^{i}:=\gen{C_{1},\bar a^{<i}}+\left(Z_{1}\cap\gen{C_{1},\bar a^{\leq i}}\right)=\gen{C_{1},\bar a^{\leq i}}\cap\left( Z_{1}+\gen{C_{1},\bar a^{<i}}\right)$$
hence $\gen{C,\bar a^{<i},\bar e^{i}}\inn X^{i}\inn\gen{C,\bar a^{\leq i}}$.

\medskip
Let for all $i<m$ the (possibly empty) tuple $\bar y^{i}$ be a basis
%\mn{maybe restrict the $y$'s just for the right $i$'s}
of $\gen{{C,\bar a^{\leq i}}}$ over $X^{i}$.

It follows that the set $\{\bar y^{i}\mid i<m\}$ generates $Y_{1}$ over $C_{1}+Z_{1}$, and in particular we have

$$
\dfp(Y_{1}/Z_{1})=\sum_{i<m}\card{\bar y^{i}}+\dfp(C_{1}/Z_{1})
$$
On the other hand let $\bar\rho^{i}$ be a basis of $\rd(C,\bar a^{\leq i})$ over $\exs X^{i}$, allowing $\bar\rho^{i}$ to be the
empty tuple whenever $\rd(X^{i})$ equals $\rd(C,\bar a^{\leq i})$. This yields, that the subset
$\{\bar\rho^{i}\mid i<m\}$ of $\rd(Y)$ is linearly independent over $\exs Z_{1}$.

To prove the last claim, we adopt an inductive argument to show the set $\{\bar\rho^{k}\mid k<i\}$ is linearly independent of $\exs Z_{1}$.
For assume, there exist scalar tuples $\underline\lambda_{k}\inn\Fp$ for $k\leq i$, which yield the following dependecy
$$\underline\lambda_{i}\boldsymbol{\cdot}\bar\rho^{i}+\underline\lambda_{i-1}\boldsymbol{\cdot}\bar\rho^{i-1}+\cdots+\underline\lambda_{0}\boldsymbol{\cdot}\bar\rho^{0}\in\exs Z_{1}.$$
But then
\begin{multline}
%\begin{split}
\underline\lambda_{i}\boldsymbol{\cdot}\bar\rho^{i}\in\left(\exs Z_{1}\cap\exs\gen{C_{1},\bar a^{\leq i}}\right)+\exs\gen{C_{1},\bar a^{<i}}=\\
=\exs\gen{C_{1},\bar a^{\leq i}}\cap\left(\exs Z_{1}+\exs\gen{C_{1},\bar a^{<i}}\right)\inn\exs X^{i}
%\end{split}
\end{multline}
and hence $\underline\lambda_{i}\equiv\triv$.

\medskip
Denote by $I$ the set of all $i<m$ for which $X^{i}\subsetneq\gen{C_{1},\bar a^{\leq i}}$. Then by \pref{sscmolge}
for all $i$ of $I$ we have $\card{\bar\rho^{i}}>\card{\bar y^{i}}$.


On the other hand -- since $\genp{\bar b}$ do not lay completely in $Z_{1}$, to all $j\notin I$ there is a $\tau^{j}$ in $\rd(C,\bar a^{j})$ which is not in $\rd(Z)$.

{\bf Moreover we care to choose $\tau^{j_{1}},\dots,\tau^{j_{k}}$ to be independent of $\gen{\bar\rho^{i}\mid i<j_{k}}$ over $\exs Z_{1}$.
(Can we do this?)}

It follows, the set $\{\bar\rho^{i},\tau^{j}\mid i\in I, j\notin I\}$ is linearly independent over $\exs Z_{1}$.
This implies
$$\sum_{i<m}\card{\bar y^{i}}-\dfp(\rd(Y/Z))\leq-m.$$
and hence
$$\delta(Y/Z)=\sum_{i<m}\card{\bar y^{i}}-\dfp(\rd(Y/Z))+\dfp(C_{1}/Z_{1})<0$$
contradicting $Z\zsu{}Y$.\\[+1mm]\noindent
%%--------------------------CASE---3---------------------------
{\bf Case 3:}\quad{\sl $p$ is the type of an arbitrary tuple $\bar d$ over $M$}

\smallskip
We may assume that

\end{proof}

\rule{\textwidth}{1pt}


\bigskip
The following result from \cite{pilcm} will also be used
\begin{fact}\label{pilcb}
Assume $M\ess\mon$ is a model of a stable theory, $\mon$ its monster model and let $c,d$ be tuples in $\mon^{eq}$.

If any of the following two conditions
%and denote by $\mathscr{C}$, $Cb(c/M)$ and by $\mathscr{D}$, $Cb(d/M)$.T
%then the following holds:
\begin{itemize}
\punto{i}$c\in\acl(d)$ %\quad\Rightarrow\quad Cb(c/M)\inn\acl^{eq}(Cb(d/M))$
\punto{ii}$\ffin{c}{d}{M}$ %\quad\Rightarrow\quad Cb(d/M)\inn\acl^{eq}(Cb(c/M))$
\end{itemize}
holds, then $Cb(c/M)\inn\acl^{eq}(Cb(d/M))$.
\end{fact}

We can now prove 
\begin{prop}
$T^{2}$ is a $CM$-trivial thory.
\end{prop}
\begin{proof}
%As pointed out before, it is enough prove the statement for $T^{2}_{m}$.
Let $\bar c$, $M\ess N\ess\mathbb{M}$ be a triple satisfying the the assumptions of Fact \ref{cmt}, that is
$\acl(M,\bar c)\cap N=M$. We have to show $Cb(\bar c/M)\inn\acl^{eq}(Cb(\bar c/N))$.
 
\medskip
We first claim that we actually may consider tuples from $\mathbb{M}_{1}$ only. To see this, if $\bar c$ is not
entirely contained in $\mathbb{M}_{1}$, take
$t$ in $\mathbb{M}_{1}$ such that $\ffin{t}{\bar c}{\,N}$.
With $\sig{2}{4}$ it is possible to find a tuple $\bar b$ in $\mathbb{M}_{1}$ such that
for all $c\in\bar c$, $[t,b]=c_{2}$ for some $b\in\bar b$. Let also $\bar b$ contain every $P_{1}$-component\mn{{\bf define it!}}
of the elements in $\bar c$. Assume further that all of $\bar b$ is involved in such tasks.
%We denote by $\bar c_{1}$ the (possibly empty) tuple consisting of all $P_{1}$-components of elements in $\bar c$.

As already pointed out often before, by $\sig{2}{2}$ it follows $\bar c\inn\dcl(t,%\bar c_{1},
\bar b)$ and $%\bar c_{1},
\bar b\inn\acl(t,\bar c)$.
Thus, as $\ffin{t}{\bar c}{\,N}$ implies $\ffin{t\,%\bar c_{1}
\bar b}{\bar c}{\,N}$, Fact \ref{pilcb} now yields $Cb(\bar c/M)\inn\acl^{eq}(Cb(t,%\bar c_{1},
\bar b/M))$ and $Cb(t,%\bar c_{1},
\bar b/N)\inn\acl^{eq}(Cb(\bar c/N))$.
 
On the other hand we still have $\acl(M,t,%\bar c_{1},
\bar b)\cap N=M$, for if $e\in\acl(M,t,%\bar c_{1},
\bar b)$ then
$$\ffin{t\,%\bar c_{1}
\bar b}{\bar c}{\,N}\:\Rightarrow\:\ffin{t\,%\bar c_{1}
\bar b}{M\bar c}{\,N}\:\Rightarrow\:\ffin{e}{M\bar c}{\,N}.$$
Hence if $e\in N$ then by irreflexivity $e\in\acl(M,\bar c)$ and hence $e\in M$. 

We have shown with this, that if the statement of the proposition is true of tuples in $\mathbb{M}_{1}$, then it is true of
arbitrary ones.
 
%Using the map $\mu_{m}$ defined
%in the proof of Theorem \ref{teowei} above, we can claim that any tuple in $\mathbb{M}$ is
%interalgebraic with some tuple in $\mathbb{M}_{1}^{eq}$ and hence -- by (WEI) for $\mathbb{M}_{1}$ -- %Theorem \ref{teowei} --
%with some tuple of $\mathbb{M}_{1}$. By the remark above
%about canonical bases of algebraic tuples, we can restrict ourself
%to consider just tuples of elements in $\mathbb{M}_{1}$ in Fact \ref{cmt}.



\medskip
Let therefore $\bar m$ be a tuple in $\mathbb{M}_{1}$ and $M\ess N\ess\mathbb{M}$ with
$\acl(M,\bar m)\cap N=M$. 
By \pref{acluno} follows in particular $\ssc(M_{1},\bar m)\cap N_{1}=M_{1}$.
As a consequence, if $\bar a$ is
a tuple of $\mathbb{M}_{1}$, linearly independent over $M_{1}$ such that $\ssc(M_{1},\bar m)=\gen{M_{1},\bar a}$,
then $\bar a$ is linearly independent over $N_{1}$ as well.

Now if $\ssc(N_{1},\bar m)$ is $\gen{N_{1},\bar c}$ with $\bar c$ linearly independent of $N_{1}$,
then we may assume that $\bar a$ is a subtuple of $\bar c$. Since $\gen{M_{1},\bar a}\cap N_{1}=M_{1}$ we have
by \pref{exsmod} $\rd(M,\bar a)\cap\rd(N)=\rd(M)$ and hence
$$\rd_{\mathbb{M}}(\bar a/M)\inn\rd_{\mathbb{M}}(\bar c/N).$$
That means, any basis of $\rd_{\mathbb{M}}(\bar a/M)$ can be extended to a basis of $\rd_{\mathbb{M}}(\bar c/N)$.

Therefore, by the construction of a geometrical base $C$ for $\tp{\bar m}{M}$ in the proof of Theorem \ref{teowei},
we may extend any such $C$ to a finite strong subalgebra $D$ of $N$ which is a geometrical base for $\tp{\bar m}{N}$.

%Let $D$ be a geometrical base for $\bar m$ over $N$ as constructed in Definition \ref{gbase}, then in particular
%$\rd(N_{1},\bar b)\non\rd(N)\inn\exs\gen{D_{1},\bar b}$ and by \pref{exsmod}
%$$\rd(M_{1},\bar b)\non\rd(M)\inn(\rd(N_{1},\bar b)\non\rd(N))\cap\exs\gen{M_{1},\bar b}\inn\exs\left(\gen{D_{1},\bar b}\cap\gen{M_{1},\bar b}\right).$$
%
%Since $\gen{D_{1},\bar b}\cap\gen{M_{1},\bar b}=\gen{D_{1}\cap M_{1},\bar b}$, by minimality
%any geometrical base $C$ for $\bar m$ over $M$ is contained in $D$.

We may now conclude with Theorem \ref{teowei}, that
$Cb(\bar m/M)\inn\dcl^{eq}(C)$ and $D\inn\acl(Cb(\bar m/N))$.

%Take an $\omega$-saturated elementary extension $N^{\prime}$ %\ess\mathbb{M}$
%of $N$ which is forking independent of $\bar m$ over $N$.
%
%Since $N$ is a model, $\fin{\bar m}{N}{N^{\prime}}$ implies both $\ssc(N_{1},\bar m)\cap N^{\prime}_{1}=N_{1}$
%and, by lemma \ref{forkingchar}, $\ssc(N^{\prime},\bar m)=N^{\prime}+\ssc(N,\bar m)=\gena{N^{\prime},\bar c}{\mathbb{M}}$.
%
%Then in particular $\delta(\bar c/N^{\prime})=\delta(\bar c/N)=\delta(\bar c/D)$ and this means, by the above remarks, that
%$D$ is also a geometrical base for $\tp{\bar m}{N^{\prime}}$ and of course $Cb(\bar m/N)=Cb(\bar m/N^{\prime})$.
%
%Since $D$ is setwise fixed by all automorphism of $N^{\prime}$ which fix $\tp{\bar m}{N^{\prime}}$ and $N^{\prime}$
%is enough saturated with respect of $D$ and $Cb(\bar m/N)$,
%%if we arrange $D$ in a tuple $\bar d$, then $\bar d$ must have finitely many conjugates
%$D$ must be permuted also under automorphism {\em of} $\mathbb{M}$ which
%fix $Cb(\bar m/N)$ pointwise. Since $D$ is finite, this gives $D\inn\acl(Cb(\bar m/N))$.

Since $C\inn D$, we get $Cb(\bar m/M)\inn\dcl^{eq}(\acl(Cb(\bar m/N)))=\acl^{eq}(Cb(\bar m/N))$ as desired.
\end{proof}

By \cite[Proposition 3.2]{pilcm}, we may conclude
\begin{cor}
No infinite field is interpretable in $T^{2}$.
\end{cor}

\end{document}