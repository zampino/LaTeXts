\section{$\delta_{2}$-Calculus}
%\begin{citation}
In the praxis\mn{put in the introduction} Hrushovski constructions,
the two steps referred as to be the usual ones,
exhibition of the infinite rank geometry and
successive collapse to finite rank, are tightly
combined. The collapsed theory  is often directly axiomatised,
so that there is no mention 

In this chapter we propose a new version of
what might be essentially found in \cite{BAD96}. The
first part of the construction is given here with
particular attention to \dots. The two presentation decisively differ in the axiomatisation of
the theory we get, this one being of infinte rank. There will be principally no mention of groups here,
the homogeneous procedure to get groups from algebras has been discussed in \ref{graal}.
%\end{citation}

\bigskip
The structures considered in this chapter belong to $\nla{2}$. As described in section \ref{graal}
this category contains graded Lie algebras $2$-nilpotent $M=M_{1}\oplus M_{2}$ for which $M$ is always generated
as a ring by its {\em unilinear} subspace $M_{1}$.
Any object $M$ of $\nla{2}$ may thus be associated to a {\em presentation}
$$R\into \fla{2}{M_{1}}\onto M$$
a short exact sequence of $\nla{2}$-morphisms, which gives $M$ isomorphic to $\fla{2}{M_{1}}\quot R$ where
$R$ an homogeneous ideal of $\fla{2}{M_{1}}$ with $R_{1}=\triv$. 

It is customary to describe free nilpotent Lie algebras of class $2$, which are objects of $\nla{2}$, in terms of the
external square of a vector space (see among others \cite{}), that is $\fla{2}{M_{1}}\simeq M_{1}\oplus\exs M_{1}$.

By $\exs M_{1}$ we could mean by definition $(\fla{2}{M_{1}})_{2}$.

Hence if $M$ is given by the presentation above, since we have $R_{1}=\triv$, $M\simeq M_{1}\oplus\,\exs M_{1}\quot R$.
We denote $R$ by $\rd(M)$ or ambiguously $\rd(M_{1})$.%$\rd(M)$ or also $\rd(M_{1})$.

An object $M$ of $\nla{2}$ is therefore univocally determined by the couple $\left(M_{1},\rd(M_{1})\right)$, the approach of \cite{BAD96} relies on these pairs.

\medskip
If $X=\{x_{\alpha}:\alpha<\lambda\}$ is a basis of $M_{1}$, we denote by $\flat_{2}(X)=\{[x_{\alpha},x_{\beta}]\mid\lambda>\alpha>\beta\}$ the set of \emph{basic commutator} of weight $2$ over $X$. According to Hall's theorem \ref{} $X\cup\flat_{2}(X)$
is a basis for $\fla{2}{M_{1}}$.

\medskip
If $M\in\nla{2}$ and $H_{1}$ a subspace of $M_{1}$ we consider $\exs H_{1}$ as a natural subspace of $\exs M_{1}$.
To any such $H_{1}$ or equivalently, to any $\nla{2}$-subalgebra $H=\gena{H_{1}}{M}$ of $M$ we set
$$\rd(H_{1}):=\rd(M)\cap\exs{H_{1}}.$$

Observe that if $H$ is $\gena{H_{1}}{M}$, then $H\simeq \fla{2}{H_{1}}+\rd(M)\quot\rd(M)\simeq\fla{2}{H_{1}}\quot \rd(H_{1})$, 
therefore we will write $\rd(H_{1})$ or ambiguously $\rd(H)$.

\bigskip
We now introduce an integer valued %\emph{predimension}
function $\delta_{2}$ with entries on the finite $\Fp$-subspaces of $M_{1}$ which measure,
in terms of $\Fp$-dimension, {\sl how much a finitely generated structure differs from a free one.}

For a finite subspace $A_{1}$ of $M_{1}$ set
\[
\delta_{2}A_{1}=\dfp(A_{1})-\dfp\left(\rd(A_{1})\right).
\]
we call $\delta_{2}(A_{1})$ the {\em deficiency} of the $\nla{2}$-subalgebra $A=\gena{A_{1}}{M}$ and
write indifferently  $\delta_{2}(A_{1})$ or $\delta_{2}(A)$. It is an invariant of the isomorphism type
of the structure $A$.

Note how close this definition resembles the relational
$\delta$ function cited in \ref{} and in \ref{} as a difference between the size
of generators compared to relations.

For arbitrary subspaces $H_{1}$ of $M_{1}$, and any finite $C_{1}$, we introduce a {\em relative} deficiency\footnote{with values in $\Z\cup\{-\infty\}$.}
by means of
\[
\delta_{2}(C_{1}\quot H_{1})=\dim_{\Fp}(C_{1}\quot H_{1})-\dfp\left(\rd(C_{1}\quot H_{1})\right)
\]
provided we define $\rd(C_{1}\quot H_{1})$ to be the quotient space $\rd(H_{1}+C_{1})\quot \rd(H_{1})$. This 
$\delta(C\quot H)$ is what Poizat calls ``{\it le solde}'' of the extension, first appeared about \dots in \cite{}.

\medskip
Fixed an algebra $M$ of $\nla{2}$, we observe a modular behaviour of the operator
$\rd$ on the subspaces of $M_{1}$, that is
\begin{labeq}{2modker}
\rd(H_{1}\cap K_{1})=\rd(H_{1})\cap \rd(K_{1})
\end{labeq}
for all subspaces $H_{1}$, $K_{1}$ of $M_{1}$. This follows directly by the same condition
on exterior squares, which actually holds for all $H_{1}$ and $K_{1}$ in $M_{1}$,
namely $\exs(H_{1}\cap K_{1})=\exs H_{1}\cap\exs K_{1}$\mn{need some more expl?}.

This simple result turns out to be strongly decisive in the whole further discussion, we anticipate
here that such a straightforward conclusion cannot be drawn, in a suitable setting, for higher nilpotency class.

As a first consequence of modularity in $\rd$, we obtain that $\delta$ is actually a predimension. We have in fact
a stronger version of the subadditivity property as defined in section \ref{qdim}:

\begin{lem}\label{2transmogrifer}
Let $H_{1},\,V_{1}$ and $C_{1}$ be subspaces of an $\nla{2}$-structure $M$ with $C_{1}$ finite and such that $C_{1}\cap H_{1}\inn V_{1}\inn H_{1}$,
then $\delta_{2}(C_{1}\quot H_{1})\leq\delta_{2}(C_{1}\quot V_{1})$.
\end{lem}
\begin{proof}
Just observe,\mn{expand} for the negative part of $\delta_{2}$, that $\rd(C_{1}\quot V_{1})$ embeds into
$\rd(C_{1}\quot H_{1})$. This follows from $\rd(H_{1})\cap \rd(V_{1}+C_{1})\inn \rd(V_{1})$.
\end{proof}
As an extremal case, we get \emph{subadditivity} for $\delta_{2}$ as defined in \ref{clpred}: for
finite subspaces $A_{1}$ and $B_{1}$, condition $\delta_{2}(A_{1}\quot B_{1})\leq\delta_{2}(A_{1}\quot A_{1}\cap B_{1})$
is by definition of $\rd(A_{1}\quot B_{1})$ {\em exactly} $\delta_{2}(\gen{A_{1},B_{1}})+\delta_{2}(A_{1}\cap B_{1})\leq\delta_{2}(A_{1})+\delta_{2}(B_{1})$.

The next step is to force $\delta_{2}$ to be non-negative. To do this we fix
for the rest of this section an algebra $M$ of $\nla{2}$ that satisfies
the following property:
\begin{itemize}
\punto{$\sig{2}{2}$}for any finite
$A_{1}\inn M_{1}$, it
holds $\delta_{2}A_{1}\geq\min(2,\,\dfp A_{1})$.
\end{itemize}
Since $\delta_{2}$ is an invariant of
the isomorphism type of finite $\nla{2}$-structures, property $\sig{2}{2}$ is first order expressible in
$\La_{2}$ by a denumerable axiom system. Just negate the diagrams of those which \emph{do not}
have the desired property.\mn{comment the 2}

The above considerations imply the following:
\begin{prop}\label{preg2}
Assume $M$ has $\sig{2}{2}$, if $\cl_{1}$ denotes the $\Fp$-linear closure in $M_{1}$, then
$\delta_{2}$ defines a $\cl_{1}$-predimension on $M_{1}$ according to definition \ref{clpred}.
\end{prop}
We denote by $d_{2}^{M}$ or simply $d_{2}$ if $M$ is clear from the context, the dimension function associated to $\delta_{2}$ and and by $\cl_{2}$ the resulting closure.
Lemma \ref{preg} ensures that $(M_{1},\,\cl_{2})$ is a pregeometry extending the linear closure on $M_{1}$.

Recall how $d_{2}$ was defined: for a finite subspace $A_{1}$ of $M_{1}$ we set 
$d_{2}(A_{1})$ as the minimal value among all $\delta_{2}(C_{1})$, where the $C_{1}$ range over
all finite subspaces of $M_{1}$ containing $H_{1}$,
this is a sound definition by virtue of property $\sig{2}{2}$ of $M$. On the other hand, an element $b$ of $M$
is in $\cl_{2}(A_{1})$ if $d_{2}(A_{1}a)=d_{2}(A_{1})$.

The crucial notion of {\em selfsufficiency} which now follows,
allows us to make a choice\mn{this is false} for such a subspace of minimal predimension.
\begin{dfn}
Let $H_{1}$ be a subspace of $M_{1}$, for $M\in\nla{2}$.
We say that $H_{1}$ is ($\delta_{2}$-)\emph{strong} or 
\emph{self-sufficient} in $M_{1}$ if
for any finite subspace $C_{1}\inn M_{1}$, we have $\delta_{2}(C_{1}\quot H_{1})\geq0$.\\
We say that an $\nla{2}$-substructure $H$ of $M$ is self-sufficient, if this is true of $H_{1}$ in $M_{1}$. We write
indifferently $H_{1}\zsu{}M_{1}$ or $H\zsu{}M$. 
\end{dfn}
Evidently, for a finite subspace $A_{1}$\mn{use {\bf always} A, B for finite spaces and H,K for the rest}
of $M_{1}$ to be self-sufficient means that
$\delta_{2}B_{1}$ is not smaller than $\delta_{2}A_{1}$ for any finite $B_{1}\nni A_{1}$ and in this case
$d_{2}(A_{1})=\delta_{2}(A_{1})$. Conversely this condition for a finite subspace $A_{1}$ implies the
self-sufficiency of $A$.

\smallskip
Let $B_{1}$ be a finite subspace, define a \emph{self-sufficient closure} of $B_{1}$ in $M_{1}$ to be a 
$\inn$-minimal subspace $A_{1}$ of $M_{1}$ containing $B_{1}$ with $\delta_{2}(A)=d_{2}^{M}(B)$.
%???????PLACE THESE???????????
%as the intersection in $M_{1}$ of all the
%finite strong $F_{1}$ containing $H_{1}$.
%
%Such a minimal strong
%overset exists because, as $\delta$ is positive, an $F_{1}$
%of minimal $\delta$ above $H_{1}$ can always be found, and such
%a subspace is obviously strong.
%Now set $d_{2}H_{1}=\delta_{2}\ssc_{2}H_{1}$ for any finite subspace of $M_{1}$ and
We will see with corollary \ref{interstrong} below, that the family of strong subspaces of $M_{1}$ is closed under
intersection, as a consequence the notion of self-sufficient closure of a finite space $A_{1}$ depends only
by $A$ and $M$ and it is univocally determined: we can {\em define} it as the intersection
of all strong finite subspaces containing $A_{1}$. It will be denoted by $\ssc^{M}(A_{1})$ or simply $\ssc(A_{1})$:
$$\ssc(A_{1})=\bigcap\{C_{1}\zsu{}M_{1}\mid C_{1}\, \text{finite and}\, C_{1}\nni A_{1} \}$$

Note that this definition implies the finitary operator $\ssc$ is actually a {\em closure}:
it is monotone, and has properties (cl1) and (cl2) of definition $\ref{pregdef}$.
%the intersection above may be considered  to involve finitely many subspaces only.
%In the next section it will be clear how the {\sl ambient structure} $M$ influences the nature of $\ssc$.
By the fact $d_{2}(A_{1})=\delta_{2}(\ssc(A_{1}))=d_{2}(\ssc(A_{1}))$, it follows now 
$A_{1}\inn\ssc(A_{1})\inn\cl_{2}(A_{1})$ for all $A_{1}$ and by lemma \ref{preg2}
$d_{2}(A_{1})\leq\dfp(A_{1})$. 

We also define the {\em subalgebra} $\ssc^{M}\!(H)$ of $M$ to be $\gena{\ssc(H_{1})}{M}$,
this will be called the self-sufficient closure of $H$ in $M$.

It is fundamental to remark that although each notion we introduced so far, like {\em }, {\em } or {\em },
actually takes place within $M_{1}$, of course the whole structure $M=M_{1}\oplus M_{2}$ plays
an active role in

\medskip
A finer analysis of $\delta_{2}$ computations and of self-sufficiencies now follows.
For the rest of the section, we assume that an algebra
$M$ of $\nla{2}$ has been  fixed, with property $\sig{2}{2}$. All the spaces mentioned will be subspaces of $M_{1}$ or $\nla{2}$-subalgebras of
$M$.

Lemma \ref{2transmogrifer} implies that, for any subspace $H_{1}$ and finite $A_{1}$,
the notion of relative deficiency $\delta_{2}(A_{1}\quot H_{1})$ retains a finitary character
with respect to the {\em base} $H_{1}$. We have indeed
$$\delta_{2}(A_{1}\quot H_{1})=\mn{expand (2 lines more)}
\inf\left(\delta_{2}(A_{1}\quot C_{1})\mid A_{1}\cap H_{1}\inn C_{1}\inn H_{1},\, C_{1}\text{ finite}\right).$$
Moreover if $\delta_{2}(A_{1}\quot H_{1})$  is finite, or in particular if $H$ is self-sufficient, then
the lowest bound above is attained by some big enough $C_{1}$ in $H_{1}$. This also follows by the
definition of $\delta_{2}(A_{1}\quot H_{1})$ above: since $\rd(A_{1}\quot H_{1})$ has finite dimension,
choose $C_{1}$ in $H_{1}$ such that $\exs (C_{1}+A_{1})$ supports each element of a basis of
$\rd(H_{1}+A_{1})$ over $\rd(H)$.

As a  consequence of this, self-sufficiency is closed under increasing chains of subspaces: assume $B_{1}^{i}\zsu{} M_{1}$ for all $i<\omega$ is an increasing family of finite subspaces
whoose union subspace is $B_{1}\inn M_{1}$. Then by the previous remarks, for any finite $A_{1}$, we can prove $\delta_{2}(A_{1}\quot B_{1}^{i})\searrow\delta_{2}(A_{1}\quot B_{1})$ for $i\rightarrow\infty$.
%[EXT] -->  \dfp K_{1}\quot V_{1}^{i} is def.ly constant = \dfp K_{1}\quot V_{1} while \rdK_{1}\quot V_{1}^{i} converges increasing to \rdK_{1}\quot V_{1}.
This gives
$V_{1}\zsu{} M_{1}$.

\medskip
We prove in the next lemma {\em transitivity} of strong embedings
\begin{lem}\label{2trans}
If $H\zsu{} K$ and $K\zsu{} M$, then $H\zsu{}M$.
\end{lem}
\begin{proof}
Let $C_{1}$ be a finite subspace, we prove the statement of the lemma once we show $\delta_{2}(C_{1}\quot H_{1})-\delta_{2}(C_{1}\cap K_{1}\quot H_{1})\geq\delta_{2}(C_{1}\quot K_{1})$. This clearly holds for the positive part $\dfp$ of $\delta_{2}$.
For the negative one observes that $\rd\left(H_{1}+C_{1}\quot
H_{1}+(C_{1}\!\cap\!K_{1})\right)$ embeds\mn{expand!} into $\rd(C_{1}\quot K_{1})$.
\end{proof}


Another straightforward application of lemma \ref{2transmogrifer} is the following:
\begin{lem}[Cut Lemma]\label{2cut}
If $H$ is self-sufficient in $K$, then for any subspace $V_{1}$ of $M_{1}$, we have
$H_{1}\cap V_{1}\zsu{} K_{1}\cap V_{1}$.
\end{lem}
\begin{cor}\label{interstrong}
If $H$ and $K$ are self-sufficient,
then the intersection $H_{1}\cap K_{1}$ is also strong.
\end{cor}
\begin{proof}
By lemma \ref{2cut} we have $H_{1}\cap K_{1}\zsu{} K_{1}$.
Then conclude by transitivity of $\zsu{}$ (lemma \ref{2trans}).
\end{proof}


\subsubsection*{A finer $\delta$-analysis}
\begin{prop}\label{2strnchar}
Let $H_{1}$  be an arbitrary subspace of $M_{1}$ then
$H_{1}$ is strong if and only if for any finite subspace
$C_{1}$ of $H_{1}$ there exists a finite
subspace of $H_{1}$, $C_{1}^{*}$ containing $C_{1}$ such that
$C_{1}^{*}\zsu{} M_{1}$.
\end{prop}
\begin{proof}
In one direction we use the fact that $\delta_{2}$ is positive.
Given any finite $C_{1}$ in $H_{1}$, the subspace we
may look at, among others, $\ssc_{2}C_{1}\cap H_{1}$. We conclude
with transitivity.

\smallskip
For the converse, let $A_{1}$ be finite in $M_{1}$.
We want $\delta_{2}(A_{1}\quot H_{1})$ to be non negative.
If we consider an arbitrary finite $B_{1}$ in $H_{1}$ containing
$A_{1}\cap H_{1}$ and take a $C_{1}^{*}\zsu{} M_{1}$
with $B_{1}\inn C_{1}^{*}\inn H_{1}$, then $\delta_{2}(
A_{1}\quot B_{1})\geq\delta_{2}(A_{1}\quot C_{1}^{*})\geq0$.
It follows the lowest bound of $\delta_{2}(A_{1}\quot B_{1})$ as $B_{1}$ range over
suitable these finite sets of $H_{1}$, which is $\delta_{2}(A_{1}\quot H_{1})$,
cannot be negative.
\end{proof}
By the proof above we realise that for a self-sufficient $H_{1}$ and a finite $A_{1}$, there exists
a minimal finite strong subspace of $H_{1}$, such that $\delta_{2}(A_{1}\quot H_{1})=\delta_{2}(A_{1}\quot B_{1})$.

Assume then we have $H_{1}\zsu{}M_{1}$, we {\em define} 


\medskip
Given two algebras $M\inn L$ of $\nla{2}$ the self-sufficient
closure of a finite subspace $A_{1}$ as computed in $M$ may differ from the one computed in $L$. But, as we might expect, we have
\begin{cor}\label{d2coerente}
Let $M$ be an $\nla{2}$-subalgebra of $L$. Then $M$ is $\delta_{2}$ strong in $L$ if and only if for any finite subspace
$A_{1}$ of $M_{1}$ the dimension $d_{2}^{M}(A_{1})$ coincides with $d_{2}^{L}(A_{1})$.
\end{cor}
\begin{proof}
We have to show that the family of all self-sufficient $B_{1}$ in $M_{1}$ coincides.

\end{proof}

\begin{lem}\label{samedelta2}
Assume $H_{1}\zsu{} M_{1}$ and $\delta_{2}(K_{1}\quot H_{1})=0$ for some finite subspace $K_{1}$, then it follows
$H_{1}+K_{1}$ is $\delta_{2}$-self-sufficient as well.
The same is true of a strong $H_{1}$ with an element $e$ of $M_{1}$ whenever $d_{2}(e\quot H_{1})=1$.
\end{lem}
\begin{proof}
As $H_{1}$ is strong, and by Lemma \ref{2strnchar},
we can assume $\delta_{2}(K_{1}\quot H_{1})=\delta_{2}(K_{1}
\quot C_{1}^{*})$ for a finite $C_{1}^{*}\zsu{} M_{1}$.
Now the statement is trivially true for finite strong subspaces.

The second statement is proven in a similar way.
\end{proof}

For an arbitrary space $H_{1}$ we
define the self-sufficient closure $\ssc_{2}H_{1}$ as the union
of the self-sufficient closures of all its finite subspaces.
It remains true, by Lemma \ref{2strnchar}, that a subspace of $M_{1}$ is strong if and only if it coincides with its self-sufficient closure. This is again the intersection of all the strong subspaces of $M_{1}$ containing $H_{1}$.\mn{Is it so in general? cfr. below}

%It is clear that if $d_{2}H_{1}$ is finite, its self-sufficient closure\footnote{
%{\bf defined} as the union of the self-sufficient closure of all the finite parts of $H_{1}$},
%is the intersection of all the $\delta_{2}$-strong subspaces of $M_{1}$ containing $H_{1}$ and of finite $d_{2}$, will retain such a finite $d_{2}$-dimension.
%?????????????????????????????????????????????????
%{\em Just discompose $H_{1}$ in a tower of $H^{i}$ each of which has $d_{2}$ fixed
%equal to $d_{2}H_{1}$ then take the union
%$\ssc_{2}(H_{1}^{i})$ this is strong and of the same $d_{2}$!}

\begin{prop}[Finite character of $\ssc$]
Assume we have a strong subspace $H_{1}$ of $M_{1}$ and $B_{1}$ is a finite space,
then
\begin{itemize}
\punto{i}$\ssc(H_{1}+B_{1})$ is a {\em finite} extension of $H_{1}$,
\punto{ii}$d_{2}(B_{1}\quot H_{1})=\delta_{2}(\ssc(H_{1}+B_{1})\quot H_{1})$.
\end{itemize}
%Moreover 
\end{prop}
\begin{proof}
%It suffices to prove the case with $H_{1}=B_{1}$ finite.\mn{is it?}
Since $\delta_{2}(A_{1}\quot H_{1})$ is non-negative for all finite $A_{1}$ in $M_{1}$, take
such an $A_{1}$ containing $B_{1}$ with minimal value of $\delta_{2}(A_{1}\quot H_{1})$.
It follows that for an arbitrary finite $C_{1}$ one has
$$\delta_{2}(C_{1}\quot H_{1}+A_{1})=\delta_{2}(C_{1}+A_{1}\quot H_{1})-\delta_{2}(A_{1}\quot H_{1})\geq0.$$
This means $H_{1}+A_{1}$ is self-sufficient in $M_{1}$ and contains $\ssc(H_{1}+B_{1})$.

\smallskip
By (i) we can assume $\ssc(H_{1}+B_{1})=H_{1}+A_{1}$ for a finite $A_{1}$. Since
$A_{1}\inn\cl_{2}(B_{1}\quot H_{1})$ we also have $d_{2}(B_{1}\quot H_{1})=d_{2}(A_{1}\quot H_{1})$.

Now $\delta_{2}(\ssc(H_{1}+B_{1})\quot H_{1})=\delta_{2}(A_{1}\quot H_{1})=\delta_{2}(A_{1}\quot D_{1})$
for a big enough finite  subspace $D_{1}\nni A_{1}\cap H_{1}$ of $H_{1}$. On the other hand, as $d_{2}$ is a dimension, there
is also a finite $E_{1}\nni A_{1}\cap H_{1}$ in $H_{1}$ such that $d_{2}(A_{1}\quot H_{1})=d_{2}(A_{1}\quot E_{1})$.

If we set $A_{1}^{\prime}$ to be $\ssc(A_{1}+D_{1}+E_{1})$, since $H_{1}+A_{1}$ is self-sufficient,
then $H_{1}+A_{1}^{\prime}=H_{1}+A_{1}$ and we have 
\begin{multline*}
\delta_{2}(A_{1}\quot H_{1})=\delta_{2}(A_{1}^{\prime}\quot H_{1}\cap A_{1}^{\prime})=\\
=\delta_{2}(A_{1}^{\prime})-
\delta_{2}(A_{1}^{\prime}\cap H_{1})=d_{2}(A_{1}^{\prime})-d_{2}(A_{1}^{\prime}\cap H_{1})=\\
=d_{2}(A_{1}^{\prime}\quot H_{1}\cap A_{1}^{\prime})=d_{2}(A_{1}\quot H_{1}).
\end{multline*}
\end{proof}
From the last proposition it follows, for $H$ and $B$ as above, that $\ssc(H_{1}+B_{1})$ is
the intersection of all strong subspaces of $M_{1}$ containing $H_{1}+B_{1}$ and finite over $H_{1}$.
Furthermore for any finite $A_{1}$ and  strong $H_{1}$ %with finite $\dfp(D_{1}\quot H_{1})$,
we have in general $d_{2}(A_{1}\quot H_{1})\leq\delta_{2}(A_{1}\quot H_{1})$.
%{\em [EXT.] Pass to finite $K_{1}$ with  $H_{1}+K_{1}=D_{1}$
%		and such that $d_{2}K_{1}\quot H_{1}=d_{2}D_{1}\quot H_{1}$
%		then $d_{2}K_{1}\quot H_{1}\leq d_{2}(K_{1}\quot
%		H_{1}^{'}$ for any finite big enough $H^{'}$ strong in $H$ so
%		$\leq\delta_{2}K_{1}\quot H_{1}^{'}$. Use char of $\delta{2}$
%		in terms of $\inf$ to conclude that $\delta_{2}K_{1}=
%		\delta_{2}D_{1}$ over $H_{1}$.}

\medskip
\begin{lem}
Let $B$ be a finitely generated $\nla{2}$-subalgebra of $M$, then
$$\cl_{2}(B_{1})=\gen{\bigcup\{C_{1} \text{finite}\mid\delta_{2}(C_{1}\quot\ssc(B_{1}))=0\}}.$$
\end{lem}




