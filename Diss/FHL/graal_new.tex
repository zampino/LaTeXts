Once a prime number $p$ is fixed, we define $\nla{n,p}$, or
short $\nla{n}$, to be the class of nilpotent Lie Algebras of nilpotency class $n$,
over the finite field $\Fp$,
with the additional property that each algebra $M$ of $\nla{n}$,
is generated by some $\Fp$-vector space $M_{1}$, that is $M=\gen{M_{1}}$.
In this way we obtain a natural graduation of $M$ according to the \emph{commutator weight} with respect to
$M_{1}$, that is
$M=\oplus_{i\leq n}M_{i}$, where $M_{i}$ denote the $i^{\mathrm{th}}$-\emph{homogeneous component}
of $M$, the $\Fp$-vector subspace of $M$ generated by all Lie products of $M_{1}$-weight $i$. By convention $M_{0}=\triv$.

If $M$ belongs to $\nla{n}$, we say that $H$ is an $\nla{n}$-subalgebra
of $M$ if $H=\gena{H_{1}}{M}$ where $H_{1}$ is a vector
subspace of $M_{1}$. Of course $H\in\nla{n}$ as well.
In the future for a subalgebra of $M\in\nla{n}$ we will
always mean a subalgebra of this special kind.

For a morphism $\map{\phi}{L}{M}$ of $\nla{n}$-algebras we mean a \emph{graded} Lie morphism of $L$ to $M$, that
is $\phi(L_{i})\inn M_{i}$ for all $i$.

With $M^k$ we name the $k$-th term of the \emph{lower central chain} of $M$, namely the ideal
$M^k=\sum_{k\leq i}M_i$.
We define as $\map{\tr{n}}{\nla{n+1}}{\nla{n}}$ the map defined by quotienting out the last component:
$\tr{n}A=A\quot A_{n+1}$, note that here $A_{n+1}=A^{n+1}$, in particular $A_{n+1}$ is an ideal.
Moreover if $A=\oplus_{i\leq n+1}A_{i}$ then $\tr{n}A\simeq\oplus_{i\leq n}A_{i}$, therefore, since $\nla{n}\inn\nla{n+1}$, we can regard $\tr{n}$ as an $\nla{n+1}$ morphism which maps identically the homogeneous components up to the $n+1^{th}$,
which is mapped to $\triv$. 

\medskip
$\fla{n}{X}$ will denote the \emph{free $n$-nilpotent Lie Algebra} over $\Fp$ with set of generators $X$.
We know, see for example [Bh], that if $V$ is an $\Fp$-vector space with basis $X$, then $\fla{n}{V}=\fla{n}{X}$. $\fla{n}{X}$ is the $\mathfrak{N}_{n}$-free algebra with free generator set $X$, where $\mathfrak{N}_{n}^{p}$ denotes the variety\footnote{
whose defining word is $[x_{1},\,\dots,\,x_{n+1}]$}
of nilpotent Lie algebras of class $\leq n$ over the field $\Fp$. One sees easily that $L^{n}(X)$ lies in $\nla{n}$ as well. Note that $\nla{n}$ is not a subvariety of $\mathfrak{N}_{n}$.

\bigskip
If $A\in\nla{n}$ we can assume $A=\fla{n}{X}\quot R$, where $X$ is an $\Fp$ basis of the vector space $A_1$,
and $R$ is an ideal of $\fla{n}{X}$ which contains what we call the \emph{relators} of $A$.
According to the previous observation we have, for $\nla{n}$-algebras, a canonical \emph{presentation} $A=\fla{n}{A_{1}}\quot
R$. The notation $A=\gen{x\in A_{1}\mid \rho\in R}$ or $A=\gen{A_{1}\mid\mathcal{R}}$ will also be used, where $\mathcal{R}$ is a set of words generating the verbal ideal $R$.
By our assumpion on $\nla{n}$, we can assume that $R$ is
a \emph{homogeneous} ideal\footnote{\dots be sure.},
that is $R=\sum_{i\leq n}R\cap{\fla{n}{A_{1}}}_{i}=:\sum_{i\leq n}R_{i}$. Moreover, in this case we have $R_{1}=\triv$.

If now $M$ is in $\nla{n}$ presented as $M=\gen{M_{1}\mid\mathcal{R}}$, where
the set of words $\mathcal{R}$ are, by definition, objects of the (absolutely) free Lie algebra $L_{p}(M_{1})$ over the field $\Fp$. We define the subvariety $\mathcal{V}^{n+1}(M)$ of $\mathfrak{N}
_{n+1}^{p}$ as follows
$$\mathcal{V}^{n+1}(M)=\left\{L\in\mathfrak{N}_{n+1}^{p}\mid\mathcal{R}(L)=\triv\right\}=
\mathfrak{N}_{n+1}^{p}\cap\var^{p}(M).$$

%Define now $\fr{n+1}M$ to be the free object in the variety $\mathcal{V}^{n+1}(M)$
%with set of free generators $M_{1}$. We have then
%$\fr{n+1}M=\fla{n+1}{M_{1}}\quot\J(M)$ where $\J(M)$ is the verbal ideal of $\fla{n+1}{M_{1}}$
%generated by the words $\mathcal{R}$. Also $\fr{n+1}M\in\nla{n+1}$.
%
%\medskip
%From the universal property of free objects in varieties we deduce the following.
%Let an algebra $L$ of $\nla{n+1}$ belong to $\mathcal{V}^{n+1}(M)$ with $L_{1}$ isomorphic to $M_{1}$,
%in other words $\tr{n}L\simeq M$, then there exists a unique Lie morphism $\pi$ of
%$\fr{n+1}M$ onto $L$, such that (up to $M$-isomorphic translates of $\tr{n}$) the following diagram commutes.
%\begin{labeq}{communo}
%\xymatrix{
%{\fr{n+1}M}\ar[dr]_{\tr{n}}\ar[rr]^{\pi}&&N\ar[dl]^{\tr{n}}\\
%&M&}
%\end{labeq}
%
%The epimorphism $\pi$ may be constructed as follows. Assume for brevity $\tr{n}N=M$ and
%assume $\map{\chi}{\fla{n+1}{N_1}}{N}$ is a presentation of $N$.
%We may further assume that $N_{1}=M_{1}$, thus $\chi$ induces a presentation $\chi^M$
%for $M$ to the quotient $\fla{n}{M_1}$ such that
%$$\xymatrix{
%&\fla{n+1}{M_1}\ar[dl]_{\tr{n}}
%\ar[dr]^{\chi}&\\\fla{n}{M_1}%\ar[ur]^{i}
%\ar[dr]^{\chi^M}&&N\ar[dl]_{\tr{n}}\\&M&
%}$$
%commutes.
%
%Since in the actual setting $\jei{M}$
%is the verbal ideal $\mathcal{R}(\fla{n+1}{M_{1}})$, where the set of words $\mathcal{R}$
%generates $\ker\chi^{M}$ in $\fla{n}{M_{1}}$, it follows $\J(M)\inn\ker\chi$.
%%=\genid{i(\ker(\chi^{M}))}{\fla{n+1}{M_{1}}}\inn\ker\chi$, 
%Therefore the identical map of $\fla{n+1}{M_{1}}$ induces a morphism $\pi$ of $\fla{n+1}{M_{1}}\quot\J(M)$ onto $\fla{n+1}{M_{1}}\quot\ker\chi$, which
%satisfies \pref{communo}.
%Note also that $\ker\pi\inn(\fr{n+1}M)_{n+1}$.