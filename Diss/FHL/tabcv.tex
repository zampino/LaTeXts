\section*{Lebenslauf}
\medskip
\thispagestyle{empty}
\selectlanguage{ngerman}
%\centering
\begin{tabular}{r@{\extracolsep{2em}}p{8.5cm}}
\multicolumn{2}{l}{\bf Pers\"onliche Daten}\\[+1mm]
\hline\\[+1mm]
Name: 					& Andrea Amantini\\
Geburtsort, Datum: 			& Florenz (Italien), 28.01.1980\\
Staatsangeh\"origkeit: 		& italienisch\\
Anschrift:					& Lausitzerstra{\ss}e 37, 10999 Berlin\\
Tel:						& +49 (0)30 3462 4345\\
%Mob:						& +49 (0)176 2829 7792\\
E-Mail:					& amantini@math.hu-berlin.de\\[+4mm]
\multicolumn{2}{l}{\bf Bildungsgang}\\[+1mm]
\hline\\[+1mm]
Okt 2005 - heute		&Doktorand an der Humboldt Universit�t zu Berlin\\ 	
				&im Fach Mathematik, Schwerpunkt Mathematische Logik -- Modelltheorie\\
				&({\sl Betreuer Prof. Dr. A. Baudisch})\\[+3mm]
%Okt 2005 - Sep 2008	&Stipendium Marie Curie FP6 -- MODNET\\
Sep 2008 - Okt 2008			&Forschungsaufenthalt an der {\em Universit� Lyon 1}, Inst. Camille Jordan -- {\sl Lyon, France}\\[+3mm]
1999-2005	&Studium an der {\em Universit� degli Studi di Firenze} -- {\sl Florenz, Italien}\\
		&Studiengang Mathematik, Abschlusspr�fung:
		Diplom ({\sl Laurea in Matematica}) -- Titel der Abschlu{\ss}arbeit ({\sl Tesi di Laurea)}: {\em Gruppi pseudo-liberi
		localmente nilpotenti}
		Abschlussnote: 110/110 \emph{cum Laude}\\[+3mm]
1994-1999	&Gymnasium ({\em Liceo Scientifico}) -- {\sl Florenz, Italien}\\[+6mm]
\multicolumn{2}{l}{\bf Stellen}\\[+1mm]
\hline\\[+1mm]
Okt 2008 - Dez 2009 	& Wissenschaftlicher Mitarbeiter an der
					HU Berlin\\[+3mm]
Okt 2005 - Sep 2008 	& EU-Gastwissenschaftler an der
					HU Berlin -- im Rahmen des {\sl Marie Curie FP6 research training network} MODNET\\[+6mm]
\multicolumn{2}{l}{\bf Lehrt�tigkeiten an der HUB}\\[+1mm]
\hline\\[+1mm]
WS 08/09	& �bungsleiter im Fach Algebra I\\
SS 09		& �bungsleiter im Fach Gew�hnlichen Differentialgleichungen\\
WS 09/10	& �bungsleiter im Fach Algebra II
\end{tabular}
\newpage
\thispagestyle{empty}
%\centering
\begin{center}
\begin{tabular}{r@{\extracolsep{2em}}p{9cm}}
\multicolumn{2}{l}{\bf Ausgew\"alte Konferenzen, Workshops u. Sommerschulen}\\[+1mm]
\hline\\[+1mm]


August 2004		&Perugia, S.M.I. Sommer Schule. %Prof. G.Alcober - Euskal Herriko Univ. Bilbao-\\
				{\sl Group Actions. Classification of Finite Groups of some ``simple'' order.
				Soluble and Nilpotent Groups}\\
Dezember 2005	&Leeds, MODNET Sommer Schule. {\sl Elements of Stability Theory.
				Intermediate Model Theory}\\
April 2006			&Freiburg, MODNET Sommer Schule. {\sl Advanced Stability. Algebraic Geometry and Model Theory}\\
Juni 2006			&Lyon, MODNET Training Workshop. {\sl Hrushowski Amalgamation and Fusion. Simple Theories}\\
%Juni 2006			&Lyon, Logicum Lugdunensis. \url http://math.univ-lyon1.fr/logicum/logicumlugdunensis\\
September 2006	&Oxford, MODNET Workshop in model theory\\
November 2006	&Antalya, MODNET Mid-Term Conference.\\
Januar 2007		&Oberwolfach, MFO Workshop on Model Theory of Groups.\\
Juni 2007			&Camerino, MODNET Sommer Schule. {\sl Model Theory of Modules. Introduction to o-minimality. Stable Groups}\\
September 2007	&Berlin, MODNET Training Workshop. {\sl Model Theory of Fields and Applications. Construction of o-minimal
				structures}\\
April 2008			&La Roche, MODNET Training Workshop. {\sl Motivic Integration. Model Theory of Valued Fields.
				Interaction between Model Theory and Number Theory}\\
Juni 2008			&Leeds, Around Classification Theory.\\
Juli 2008			&Manchester, MODNET Summer School. {\sl Groups of finite Morley Rank. Finite Model Theory}
\end{tabular}
\vfill
%\vspace{4ex}
%\bigskip
\end{center}
\begin{flushright}
Berlin, den~\today\\[+7mm]
\dots\dots\dots\dots\dots\dots\dots\dots\dots
\end{flushright}
\newpage
\thispagestyle{empty}
\cleardoublepage

