The facts from stability theory we use are quite basic.
In the case of the uncollapsed nil-$2$ Lie algebra we construct in Section \ref{t2axioms}, the theory obtained is $\omega$-stable.
Only properties of such theories will therefore be needed. For the concepts and the definitions of this section,
we essentially follow \cite{ziebous} or \cite{zietent}.

\medskip
By a {\em totally transcendental theory} we mean a theory in which each formula $\varphi(\bar x)$
in $n$ variables has ordinal Morley rank, for all $n<\omega$.
By Fact \ref{ziemr} below this is equivalent to require every $1$-formula to have ordinal Morley rank or
to require the formula $x=x$ to have such a rank.
\begin{fact*}
An $\omega$-stable theory $T$ is totally transcendental (short t.t.). Moreover the two notion coincide
if the language of $T$ is countable.
\end{fact*}
The facts recalled in the rest of the section, if not otherwise specified, concern a fixed large saturated monster $\mon$ of
a totally transcendental theory $T$.
{\em Small} sets are subsets of $\mon$ whose cardinality is less than $\card{\mon}$ and {\em models} are
small elementary substructures of $\mon$.

We assume Morley rank and degree are defined on partial types $p$ in $T$ over parameters from $\mon$
and write respectively $\mr(p)$ and $\md(p)$. We also denote by $\mrd(p)$ the ordered pair $(\mr(p),\md(p))$.

For a formula $\varphi(\bar x)$, $\mrd(\varphi(\bar x))$ stands for $\mrd(\{\varphi(\bar x)\})$ while $\mr(\bar a/B)$ will be the Morley rank of $\tp{\bar a}{B}$, for any tuple $\bar a$ and small subset set $B$ of $\mon$.

\begin{rem}\label{belrango}
Morley rank is {\em continuous}, that is for any complete type $p$, $\mr(p)$
is the rank of a formula $\phi$ in $p$, and for any complete type $q\ni\phi$, $\mr(q)\leq\mr(\phi)$.
Moreover for any formula $\psi(\bar x)$ we have dually
$$\mr(\psi(\bar x))=\max(\mr(p)\mid p\in\ssp{\bar x}{A},\psi\text{ is over $A$ and }\psi\in p).$$
\end{rem}

\smallskip
Both statements of the following fact will be used in Section \ref{rango} further below.
The first is an easy exercise on rank computation under
algebraicity, the second is due to a result of Erimbetov (\cite{eri}), in the formulation of which we follow \cite{ziemr}.
The {\em product} of two ordinals $\alpha$ and $\beta$, denoted by $\alpha\cdot\beta$, is defined as
the order type of the lexicographic order on $\alpha\times\beta$.

\begin{fact}\label{ziemr}
Let $\map{f}{\mathscr{D}}{\mathscr{E}}$ be a definable map between definable (possibly with parameters) classes of the monster
model of an arbitrary theory.

\begin{itemize}
\item[1.]If $f$ is finite-to-one and onto $\mathscr{E}$, then $\mr(\mathscr{D})=\mr(\mathscr{E})$.
\item[2.]If $\mathscr{E}$ has Morley rank $\beta$ and the Morley rank of all fibres $f^{-1}(e)$ is bounded by an ordinal $\alpha>0$, the Morley rank of $\mathscr{D}$ is bounded by $\alpha\cdot(\beta+1)$.
\end{itemize}
\end{fact}

\medskip
We assume a notion of {\em non-forking} extension of types is given (through {\em dividing}).
In a totally transcendental setting, non-forking is expressed in terms of Morley rank.
For tuples $\bar a$ and small subsets $A,B$ in $\mon$ we write
\begin{labeq}{forkrank}
\ffin{\bar a}{B}{A}%\stackrel{def}{
\iff\mr(\bar a/B)=\mr(\bar a/AB).
\end{labeq}
to say that $\tp{\bar a}{AB}$ {\em does not fork over} $B$.
 
The {\em Lascar rank} (\cite{las}) on complete types $p$ of a stable theory, denoted by $U(p)$ is the smallest {\em connected} notion
of rank on complete types, whose gap on type extensions witness forking (see also \cite{haha} or and \cite[\S6]{bue}).
This means, if $q$ extends $p$, $q$ is a forking extension of $p$ iff $U(q)<U(p)$.

Moreover connectedness means, that $U(p)=\alpha$ and $\beta\leq\alpha$ implies the existence of a
complete type $q\nni p$ with $U(q)=\beta$.\footnote{Morley rank is connected with respect to {\em formulas} but
not on complete types.}

%is  or $U$-rank as emerges, for instance
%in \cite{las} and \cite{haha} or %. This is the $\mathit{L}$-rank defined by Shelah in
%\cite[V.7.5]{sh90}.
%which is a priori defined in an unstable context\mn{clarify the setting!}: $U$ is the unique {\em connected} (and hence
%{\em minimal}) notion
%of rank on complete types (cfr.\,\cite{la}); for an extension $p\inn q$ of
%and hence, in a stable theory, witness the {\em non-forking} relation on complete type extensions:
%for $q$ to be a {\em forking extension} of $p$ is equivalent to $U(p)<\infty$ and $U(q)<U(p)$.
%It is in fact the smallest connected ordinal rank to isolate the properties of forking, 
In particular we have $U(p)\leq\mr(p)$ on all complete types $p$ of $T$.

\smallskip
The strength of Lascar rank lays in its additive property. We refer to the book
of Buechler cited, for the definition of the {\em connected sum} $\alpha\oplus\beta$ of two ordinal numbers
$\alpha,\beta$.
\begin{fact}[{\cite[Theorem 8]{las}}]\label{uddi}
In a superstable theory $T$, for all tuples $\bar a,\bar b$ and sets $B$, we have
$$
U(\bar a/B\bar b)+U(\bar b/B)\leq U(\bar a\bar b/B)\leq U(\bar a/B\bar b)\oplus U(\bar b/B).$$
Moreover since the ordinal sum $+$ and $\oplus$ coincide on finite ordinals, when
$U(\bar a/B\bar b)$ and $U(\bar b/B)$ are both finite, we have
$$
U(\bar a\bar b/B)=U(\bar a/B\bar b)+U(\bar b/B).
$$
\end{fact}
This additive behaviour resembles additivity of Morley rank in strongly minimal sets and will turn out
very useful when computing the rank of the theory $T_{2}$ in Section \ref{t2axioms}.

Unfortunately the two notions of rank introduced so far do not in general coincide on complete types even in
an $\omega$-stable context.  In \cite[\S6 and \S7]{bue} one finds an extensive account of 
examples and conditions under which these ranks do or do not coincide. Among the affirmative cases we find, for instance, the
uncountably categorical theories.

Rank computations in Section \ref{rango} involve the following very special instance, which we prove below
\begin{lem}\label{RMU}
Let $\mathfrak{X}$ be a family of complete isolated types in $T$, %each type in $\mathfrak{X}$ is
over finite sets of parameters.

Suppose further that for any type $p\in\mathfrak{X}$ and each finite set $C$ containing the parameters of $p$,
any complete extension of $p$ over $C$ lays again in ${\mathfrak X}$.

Then Morley rank and $U$-rank agree on $\mathfrak{X}$.
\end{lem}
\begin{proof}
Let $p\in\ssp{\bar x}{A}$ be a type in $\mathfrak{X}$ for a finite set $A$, and assume that $M\!R(p)\geq\alpha$ for some ordinal number $\alpha$, we show
by induction on $\alpha$, that $U(p)\geq\alpha$. Let the statement by true of types from ${\mathfrak X}$
for all ordinals $\alpha<\kappa$. If $\kappa$ is a limit
ordinal, then by the definition of ranks it follows $U(p)\geq\kappa$.

Let $\kappa$ be $\alpha+1$ for some ordinal $\alpha$, and $\mr(p)\geq\alpha+1$. Since $p$ is isolated, there is a formula
$\varphi(\bar x)$ over $A$ which implies $p$, hence $\mr(\varphi)\geq\alpha+1$. Since $T$ is t.t., let $\psi(\bar x)$ be a formula
over some finite $C\nni A$ implying $\varphi$ with $\mr(\psi)=\alpha$. Choose a type $q$ in $\ssp{\bar x}{C}$ generic
in $\psi$, then we have $\mr(q)=\alpha$, $q$ implies $\varphi$ and hence $q$ is a forking extension of $p$.
This yelds $q\in\mathfrak{X}$ and
since $\mr(q)\geq\alpha$, by induction, $U(q)\geq\alpha$, this means exactly $U(p)\geq\alpha+1$.

We actually showed that the assumptions force Morley rank to be connected on ${\mathfrak X}$.
\end{proof}

\medskip
We will also need the characterisation of forking in terms of a {\em notion of independence}:
a ``stable version'' of Kim-Pillay results for simple theories.

With this respect, we follow the approach of \cite[Theorem 5.8]{haha} and Ziegler and Tent in (\cite[Theorem 36.10]{zietent}).

The last authors seem to exhibit an overall {\em shortest} list of properties for a distinguished class of type extensions to
coincide with the non-forking relation. We stick however to the equivalent formulation in terms of an independence relation
among {\em sets} rather than types.
\begin{fact}\label{stableforking}
Assume a complete theory $T$ is endowed with a ternary relation $\fin{\bar x}{X}{Y}$ between tuples $\bar x$ and pairs of
(small) sets $X,Y$ of $T$,
which is invariant under $\aut(\mon)$.
Then $T$ is stable if an only if $\ind$ satisfies:
%\begin{itemize}
%\punto{\small{\sc Local Character}}there is a cardinal $\kappa$ such that for all tuple $\bar a$ and set $C$, there is $C_{0}\inn C$ of
%cardinality at most $\kappa$ such that $\fin{\bar x}{C_{0}}{C}$.
%\punto{\small{\sc Boundedness}}There is a cardinal $\mu$ such that for all $A\nni B$ and any tuple $\bar a$, there are at most
%$\mu$ $\aut_{A}(\mon)$-orbits among tuples $\bar a^{\prime}$ with $\fin{\bar a^{\prime}}{B}{A}$ and $\bar a\equiv_{B}\bar a^{\prime}$.
%\end{itemize}
\begin{description}
\item{\small{\sc (Local Character)}} there is a cardinal $\kappa$ such that for all tuple $\bar a$ and set $C$, there is $C_{0}\inn C$ of
cardinality at most $\kappa$ such that $\fin{\bar x}{C_{0}}{C}$.
\item{\small{\sc (Boundedness)}} There is a cardinal $\mu$ such that for all $A\nni B$ and any tuple $\bar a$, there are at most
$\mu$ $\aut_{A}(\mon)$-orbits among tuples $\bar a^{\prime}$ with $\fin{\bar a^{\prime}}{B}{A}$ and $\bar a\equiv_{B}\bar a^{\prime}$.
\end{description}
\smallskip
If in addition $\ind$ satisfies, for all sets $A\nni B\nni C$:
\begin{description}
\item{\small\sc(Transitivity)} for any tuple $\bar a$, %and $A\nni B\nni C$,
from $\fin{\bar a}{C}{B}$ and $\fin{\bar a}{B}{A}$, follows $\fin{\bar a}{C}{A}$.
\item{\small\sc(%Weak
Monotony)} For all %$A\nni B\nni C$ and
$\bar a$, $\fin{\bar a}{C}{A}$ implies $\fin{\bar a}{C}{B}$.
\item{\small\sc(Existence)} %For all sets $A\nni B$ 
There always exists a tuple $\bar a$ with $\fin{\bar a}{B}{A}$.
\end{description}
then $\ind$ coincides with non-forking independence,
that is $\fin{\bar a}{B}{A}$ holds, exactly when $\tp{\bar a}{AB}$ {\em does not fork over} $B$.
\end{fact}
Of course properties above specialise to the case t.t.\,theories, i.{}e. Local Character becomes
{\em finite} Local Character and a {\em finite} instance of Boundedness property is satisfied. 

On the contrary, finite local character and finite boundedness of a notion of independence in a
small theory imply $\omega$-stability.
\begin{rem}\label{re:extrafking}
Stable forking independence satisfies in addition:
\begin{align}
\tag{Symmetry}\forall\bar a,\bar b,B,\quad\ffin{\bar a}{B}{\bar b}&\iff\ffin{\bar b}{B}{\bar a}\\
\tag{Irreflexivity}\forall\bar a,B,\quad\ffin{\bar a}{B}{\bar a}&\Rightarrow \bar a\in\acl(B)\\
\tag{Algebraicity}\forall\bar a,\; C\inn\acl(A,B)\text{ and }\ffin{\bar a}{B}{A}\quad&\Rightarrow\ffin{\bar a}{B}{C}\\
\tag{Base Monotonicity}\forall \bar a,A\nni B\nni C,\quad\ffin{\bar a}{C}{A}&\Rightarrow\ffin{\bar a}{B}{A}
\end{align}
\end{rem}
%\begin{itemize}
%\punto{Irreflexivity}$\ffin{\bar a}{B}{\bar a}\Rightarrow \bar a\in\acl(B)$
%\punto{Algebraicity}If $C\in\acl(A,B)$ then $\ffin{\bar a}{B}{A}\Rightarrow\ffin{\bar a}{B}{C}$
%\punto{Base Monotonicity}For $A\nni B\nni C$, $\ffin{\bar a}{C}{A}\Rightarrow\ffin{\bar a}{B}{A}$.
%\punto{Symmetry}For all $\bar a,\bar b$ and $B$, $\ffin{\bar a}{B}{\bar b}\iff\ffin{\bar b}{B}{\bar a}$.
%\end{itemize}
For a comprehensive account on the possible axiomatic choices for a notion of independence
we refer to \cite{ad}.

\bigskip
In the last section of Chapter \ref{due}, we prove some results around
weak elimination of imaginaries and also draw a strategy toward a proof of $CM$-triviality for our uncollapsed structure.

We recall below some essential facts about these notions, following
\cite{ziebous} and \cite{cafa}.
\begin{dfn}\label{wei}
A theory $T$ has {\em weak elimination of imaginaries} (WEI) if for every imaginary element $e$,
in $M^{eq}$ for any model $M$ of $T$, there is
a real tuple $\bar c$ such that $e$ is definable over $\bar c$ and $\bar c$ is algebraic over $e$, that is
$$e\in\dcl^{eq}(\bar c)\quad\text{and}\quad\bar c\in\acl^{eq}(e).$$
\end{dfn}
Imaginary elements are used essentially to deal with canonical bases of types and definable sets.

In our t.t.{\,}theory $T$ for a complete stationary type $p=p(\bar x)$ ($\md(p)=1$) over a set $A$,
the {\em canonical base} of $p$ is the definable closure $Cb(p)$ of the -- at most $\card{T}$-many -- canonical
parameters of the $p$-definition formulas $d_{p}x\varphi(x,y)$ (\cite[p.29]{ziebous}) as $\varphi(x,y)$ ranges over the language of $T$.
In our context $Cb(p)$ is the definable closure of a finite sequence of imaginaries.

$Cb(p)$ lays a priori in $\mon^{eq}$ and is point-wise fixed by exactly those automorphism $\sigma$ of $\mon$
for which $p$ and $p^{\sigma}$ have the same global non-forking extension.
Therefore if ${\bf p}$ is a global type, ${\bf p}$ is fixed by exactly the automorphisms which fixes $Cb({\bf p})$ point-wise.
For a global type ${\bf p}$ and a set $A$
of parameters we will also need the following renown property of canonical bases:
\begin{fact}[{\cite[Theorem 4.2]{ziebous}}]\label{ziecb}{\ }
\begin{itemize}
\punto{1.}{\bf p} does not fork over $A$ iff $Cb({\bf p})\inn\acl^{eq}(A)$
\punto{2.}{\bf p} is the unique non-forking extension of the (stationary) type $\res{{\bf p}}{A}$ iff $Cb({\bf p})\inn\dcl^{eq}(A)$.
\end{itemize}
\end{fact}
%I refer in general to \cite{ziebous,cafa} for the aforementioned definitions and facts about canonical parameters and imaginaries.
We will write $Cb(\bar a/B)$ to denote $Cb(\tp{\bar a}{B})$, provided $\tp{\bar a}{B}$ is stationary.

The following result which may be derived from \cite[Proposition 2.5]{cafa} will be also mentioned in Section \ref{cmt}. It is a statement
about the existence of {\em weak} canonical bases for types over models.
For ease of reference, we adapt the proof to the total transcendental setting.
\begin{lem}\label{weimodels}
%Let $\mon$ the monster model of a totally transcendental theory $T$, then
$T$ has {\em (WEI)} if and only if for any (small) model $M\ess\mon$, and any type $p\in\ssp{}{M}$, there exists a {\em real} tuple $\bar c$
in $M$ such that:
\begin{itemize}
\punto{{\it i\,}}the pointwise stabiliser of $\bar c$ in $\aut(M)$ fixes the type $p$,
\punto{{\it ii\,}}$\bar c$ has finitely many conjugates under the automorphisms of $M$ which leave $p$ fixed.
\end{itemize}
\end{lem}
\begin{proof}
Let then $e$ be an imaginary of $T$ such that $e=\bar a/\epsilon$ where $\epsilon(\bar x,\bar y)$ is a $0$-definable
equivalence relation, and $\bar a$ is in $\mon$. Let $\mathbf{p}$ a global generic type in $\epsilon(\bar x,\bar a)$.

By taking a small but sufficiently saturated model $M$ ($\omega$-saturation will do), containing $\bar a$ and such that ${\bf p}$ does not
fork over $M$, we obtain a real tuple $\bar c$ with properties (i) and (ii) related to $\aut(\mon)$ and ${\bf p}$.
%For a totally transcendental theory, $\omega$-saturation of $M$ is enough.

But then we have $e\in\dcl^{eq}(\bar c)$, for if $\sigma\in\aut_{\bar c}(\mon)$, then ${\bf p}^{\sigma}={\bf p}$ and
this implies that $\epsilon(\bar x,\bar a)\wedge\epsilon(\bar x,\bar a^{\sigma})$ must be consistent, thus $\sigma$ fixes $e$.

On the other side, the group $\aut_{e}(\mon)$ {\em transitively} permutes the generic global types of the formula
$\epsilon(\bar x,\bar a)$. Since these are but in a finite number,
if $\aut(\mon)_{\bf p}$ denotes the stabiliser of the type ${\bf p}$ under the action of $\aut_{e}(\mon)$,
then the index of $\aut(\mon)_{\bf p}$ in $\aut_{e}(\mon)$ is finite. By the hypothesis, $\bar c$ has a finite orbit under $\aut(\mon)_{\bf p}$, then it has
necessarily a finite orbit under $\aut_{e}(\mon)$. This gives $\bar c\in\acl^{eq}(e)$.

\smallskip
For the converse statement, (WEI) implies that for any type $p$ over a model $M$, we can find a real tuple $\bar c$ such that
\begin{gather}
\begin{cases}\label{ciBi}
\bar c\in\acl^{eq}(Cb(p))\\Cb(p)\in\dcl^{eq}(\bar c).
\end{cases}
\end{gather}
These properties imply (i) and (ii) above.
\end{proof}
A real finite set with property \pref{ciBi} above will be found -- for types of self-sufficient tuples -- in Lemma \ref{teowei} of
Chapter \ref{due}.

The following result from \cite{pilcm} will also be useful
\begin{fact}\label{pilcb}
Assume $M\ess\mon$ is a model of a stable theory, $\mon$ its monster model and let $c,d$ be tuples in $\mon^{eq}$.

If any of the following two conditions
%and denote by $\mathscr{C}$, $Cb(c/M)$ and by $\mathscr{D}$, $Cb(d/M)$.T
%then the following holds:
\begin{itemize}
\punto{i}$c\in\acl(d)$ %\quad\Rightarrow\quad Cb(c/M)\inn\acl^{eq}(Cb(d/M))$
\punto{ii}$\ffin{c}{d}{M}$ %\quad\Rightarrow\quad Cb(d/M)\inn\acl^{eq}(Cb(c/M))$
\end{itemize}
holds, then $Cb(c/M)\inn\acl^{eq}(Cb(d/M))$.
\end{fact}

\smallskip
We recall next the definition of $CM$-triviality.
%which actually 
%This first appears in \cite{hruabi} together with two other equivalent conditions.
%The formulation below is particularly suitable to work with the notion of weak canonical base
%for self-sufficient tuples we will give after Theorem \ref{teowei}.
\begin{dfn}\label{cmtdef}
A theory $T$ is said to be {\sl CM}-trivial, if for any algebraically closed sets $B\inn A$ of the monster $\mathbb{C}^{eq}$ of $T$,
and all tuple $c$ in $\mathbb{C}^{eq}$ with $\acl^{eq}(B,c)\cap A=B$ we always have $Cb(c/B)\inn\acl^{eq}\left(Cb(c/A)\right)$.
\end{dfn}

With Pillay's \cite[Corollary 2.5]{pilcm}, we can rephrase the definition above in terms of models of $T$ and {\em real} tuples:
\begin{fact}\label{pilcmt}
A theory $T$ is {\sl CM}-trivial iff for all small models $M\ess N$ and (real) tuples $\bar c$ from $\mathbb{C}$ with
$acl(M,\bar c)\cap N=M$ we have $Cb(\bar c/M)\inn\acl^{eq}\left(Cb(\bar c/N)\right)$.

Moreover $T$ is $CM$-trivial iff it is such after adding some set of parameters to $T$.
\end{fact}
Such a property for $T$ prevent the theory from interpreting fields:
\begin{fact}[{\cite[Proposition 3.2]{pilcm}}]
No infinite field is interpretable in a $CM$-trivial theory $T$.
\end{fact}