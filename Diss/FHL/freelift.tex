%\medskip
Fix a prime $p$ grater than $c$ and
assume $M=M_{1}\oplus\cdots\oplus M_{c}$ is a graded Lie algebra of $\nla{c}$ as defined
in Definition \ref{lcp} of Section \ref{nilgral}, in particular $M=\gen{M_{1}}$.

The $c^{\textsl{th}}$-homogeneous component $M_{c}$ of $M$ coincides
with the ideal $\gamma_{c}(M)$ of $M$ -- the $c^{\textsl{th}}$ term of the lower central series.
If we denote by $M_{*}$ the quotient $M/M_{c}$, then $M_{*}$ is
again generated by $M_{1}$ (modulo $M_{c}$) and $M_{*}\simeq M_{1}\oplus\cdots\oplus M_{c-1}$. That is $M_{*}\in\nla{c-1}$ and we can refer to $M_{*}$ as the {\em truncation} of $M$ to $\nla{c-1}$. We denote by $\ast$ the resulting functor of $\nla{c}$ into $\nla{c-1}$.

%Since $M_{*}=M$ for all $M\in\nla{c-1}\inn\nla{c}$, $*$ is a retraction of $\nla{c}$ onto $\nla{c-1}$ and
For a given $M$ in $\nla{c}$, with
abuse of the above notation, we denote again by $\ast$ the canonical epimorphism of $M$ onto $M_{*}$.
Moreover, for any $m$ in $M$, we set $m_{*}=m\ast=m+M_{c}$.

\medskip
In Section \ref{nilgral}, we denoted by $\fla{n}{X}$ the free $n$-nilpotent Lie $\Fp$-algebra over the set $X$. This is
in particular a free object of $\nla{n}$.
Moreover any $A$ in $\nla{n}$, is an epimorphic image of $\fla{n}{A_{1}}$.

\smallskip
Now for an algebra $M$ in $\nla{c-1}$, let $R$ be the homogeneous ideal of $\fla{c-1}{M_{1}}$ which gives
the presentation $\gen{M_{1}\mid R}$ of $M$, as defined in section \ref{nilgral}.

We identify $\fla{c-1}{M_{1}}$ with the subspace $L_{1}\oplus\cdots\oplus L_{c-1}$ of $L=\fla{c}{M_{1}}$, and denote by
$\iota$ the $\Fp$-linear embedding of $\fla{c-1}{M_{1}}$ into $\fla{c}{M_{1}}$.

Let $\J(M)$ denote the ideal of $\fla{c}{M_{1}}$ generated by $\iota(R)$.
Since $R$ is an ideal of $\fla{c-1}{M_{1}}$, in the notation of section \ref{nilgral}, we have
$\J(M)=\genid{\iota(R)}{}=\genp{\iota(R),[\iota(R),\fla{c}{M_{1}}]}$. %in $\fla{c}{M_{1}}$.
Notice that $\J(M)$ is homogeneous with
$\J(M)_{1}=\triv$ (cfr.\,\pref{homojei} below).
\begin{dfn}\label{freelifting}
For any $M$ of $\nla{c}$, define $\fl M$ to be %${\rm F}(M)$ or by $F^{n}_{M}$
the quotient $\fla{c}{M_{1}}/\J(M)$ and call {\em free lift}, the map
\begin{align*}\tag{\sf fl}
\lmap{\frl}{\nla{c-1}&}{\nla{c}}\\
M&\longmapsto F_{M}=\gen{M_{1}\mid\J(M)}.
\end{align*}
\end{dfn}
\begin{prop}\label{morphifreelift}
For any algebra $M=\gen{M_{1}\mid R}$ of $\nla{c-1}$ one has $(\fl M)_{*}\simeq M$ and we adopt the convention to identify $(\fl M)_{*}$ with $M$,
coherently with the choice for $(\fl M)_{1}$ to be $M_{1}$.

\smallskip
For any $N$ in $\nla{c}$, if $\varphi$ is an $\nla{c-1}$-morphism
of $M$ into $N_{*}$, then there exists a unique $\nla{c}$-morphism $\widetilde\varphi$ of $\fl M$ into $N$
such that $\widetilde\varphi{}{*}={*}\varphi$.
\begin{labeq}{commufree}
\begin{split}
\xymatrix@C+4mm{
\fl{M}\ar[r]^{\widetilde\varphi}\ar[d]^{*}&N\ar[d]^{*}\\
M\ar[r]^{\varphi}&N_{*}
.}
\end{split}
\end{labeq}

Moreover, the map $\varphi\mapsto\widetilde\varphi$ yelds a bijection
\begin{labeq}{freeliftadjoint}
\Hom_{\nla{c-1}}(M,N_{*})\to\Hom_{\nla{c}}(\frl( M),N)
\end{labeq}
for any $M$ in $\nla{c-1}$ and $N$ in $\nla{c}$.
%then $\fl{M}$ has the following features:
%\begin{itemize}
%\item[-]$(\fl M)_{*}\simeq M$
%\item[-]for any $N\in\nla{c}$ with $N_{*}\simeq M$ (hence $N_{1}\simeq M_{1}$) we have a \sout{unique} canonical
%$\nla{c}$-epimorphism $\epi{\epsilon}{\fl M}{N}$ up to \dots.
%\end{itemize}
\end{prop}

%\vfill
%Once a prime number $p$ is fixed, we define $\nla{n,p}$, or
%short $\nla{c}$, to be the class of nilpotent Lie Algebras of nilpotency class $n$,
%over the finite field $\Fp$,
%with the additional property that each algebra $M$ of $\nla{c}$,
%is generated by some $\Fp$-vector space $M_{1}$, that is $M=\gen{M_{1}}$.
%In this way we obtain a natural graduation of $M$ according to the \emph{commutator weight} with respect to
%$M_{1}$, that is
%$M=\oplus_{i\leq n}M_{i}$, where $M_{i}$ denote the $i^{\mathrm{th}}$-\emph{homogeneous component}
%of $M$, the $\Fp$-vector subspace of $M$ generated by all Lie products of $M_{1}$-weight $i$. By convention $M_{0}=\triv$.
%
%If $M$ belongs to $\nla{c}$, we say that $H$ is an $\nla{c}$-subalgebra
%of $M$ if $H=\gena{H_{1}}{M}$ where $H_{1}$ is a vector
%subspace of $M_{1}$. Of course $H\in\nla{c}$ as well.
%In the future for a subalgebra of $M\in\nla{c}$ we will
%always mean a subalgebra of this special kind.
%
%For a morphism $\map{\phi}{L}{M}$ of $\nla{c}$-algebras we mean a \emph{graded} Lie morphism of $L$ to $M$, that
%is $\phi(L_{i})\inn M_{i}$ for all $i$.
%
%With $M^k$ we name the $k$-th term of the \emph{lower central chain} of $M$, namely the ideal
%$M^k=\sum_{k\leq i}M_i$.
%We define as $\map{\tr{n}}{\nla{c+1}}{\nla{c}}$ the map defined by quotienting out the last component:
%$\tr{n}A=A\quot A_{n+1}$, note that here $A_{n+1}=A^{n+1}$, in particular $A_{n+1}$ is an ideal.
%Moreover if $A=\oplus_{i\leq n+1}A_{i}$ then $\tr{n}A\simeq\oplus_{i\leq n}A_{i}$, therefore, since $\nla{c}\inn\nla{c+1}$, we can regard $\tr{n}$ as an $\nla{c+1}$ morphism which maps identically the homogeneous components up to the $n+1^{th}$,
%which is mapped to $\triv$. 
%
%\medskip
%$\fla{c}{X}$ will denote the \emph{free $n$-nilpotent Lie Algebra} over $\Fp$ with set of generators $X$.
%We know, see for example [Bh], that if $V$ is an $\Fp$-vector space with basis $X$, then $\fla{c}{V}=\fla{c}{X}$. $\fla{c}{X}$ is the $\mathfrak{N}_{n}$-free algebra with free generator set $X$, where $\mathfrak{N}_{n}^{p}$ denotes the variety\footnote{
%whose defining word is $[x_{1},\,\dots,\,x_{n+1}]$}
%of nilpotent Lie algebras of class $\leq n$ over the field $\Fp$. One sees easily that $L^{n}(X)$ lies in $\nla{c}$ as well. Note that $\nla{c}$ is not a subvariety of $\mathfrak{N}_{n}$.
%
%\bigskip
%If $A\in\nla{c}$ we can assume $A=\fla{c}{X}\quot R$, where $X$ is an $\Fp$ basis of the vector space $A_1$,
%and $R$ is an ideal of $\fla{c}{X}$ which contains what we call the \emph{relators} of $A$.
%According to the previous observation we have, for $\nla{c}$-algebras, a canonical \emph{presentation} $A=\fla{c}{A_{1}}\quot
%R$. The notation $A=\gen{x\in A_{1}\mid \rho\in R}$ or $A=\gen{A_{1}\mid\mathcal{R}}$ will also be used, where $\mathcal{R}$ is a set of words generating the verbal ideal $R$.
%By our assumpion on $\nla{c}$, we can assume that $R$ is
%a \emph{homogeneous} ideal\footnote{\dots be sure.},
%that is $R=\sum_{i\leq n}R\cap{\fla{c}{A_{1}}}_{i}=:\sum_{i\leq n}R_{i}$. Moreover, in this case we have $R_{1}=\triv$.
%
%If now $M$ is in $\nla{c}$ presented as $M=\gen{M_{1}\mid\mathcal{R}}$, where
%the set of words $\mathcal{R}$ are, by definition, objects of the (absolutely) free Lie algebra $L_{p}(M_{1})$ over the field $\Fp$. We define the subvariety $\mathcal{V}^{n+1}(M)$ of $\mathfrak{N}
%_{n+1}^{p}$ as follows
%$$\mathcal{V}^{n+1}(M)=\left\{L\in\mathfrak{N}_{n+1}^{p}\mid\mathcal{R}(L)=\triv\right\}=
%\mathfrak{N}_{n+1}^{p}\cap\var^{p}(M).$$
%Define now $\fl{M}$ to be the free object in the variety $\mathcal{V}^{n+1}(M)$
%with set of free generators $M_{1}$. We have then
%$\fl{M}=\fla{c+1}{M_{1}}\quot\J(M)$ where $\J(M)$ is the verbal ideal of $\fla{c+1}{M_{1}}$
%generated by the words $\mathcal{R}$. Also $\fl{M}\in\nla{c+1}$.

\medskip
\begin{proof}
Let $L$ denote $\fla{c}{M_{1}}$. Then $L_{c}=\gamma_{c}(L)$ and $\J(M)$ are ideals of $L$ such that
$\fl{M}=L/\J(M)$ and $L/L_{c}\simeq\fla{c-1}{M_{1}}$. We have the following isomorphisms of Lie algebras\footnote{
In the row below $+$ is the ordinary {\em sum} between subalgebras and ideals of a Lie algebra.}
$$
(\fl{M})_{*}\simeq\frac{ L}{L_{c}+\J(M)}\simeq\frac{L/L_{c}}{(L_{c}+\J(M))/L_{c}}%\simeq\frac{\fla{c-1}{M_{1}}}{\J(M)/(\J(M)\cap L_{c})}.
$$
Moreover, as $R$ is homogeneous and equals $R_{2}+\cdots+R_{c-1}$, we have
\begin{labeq}{homojei}
\J(M)=\iota(R)\oplus\sum_{i=2}^{c-1}[R_{i},L_{c-i}]%\gamma_{c-i}(L)]
\end{labeq}
where the right direct summand is contained in $L_{c}$.

Hence -- as algebras -- $(L_{c}+\J(M))/L_{c}\simeq\J(M)/(\J(M)\cap L_{c})\simeq R$ and therefore
$(\fl{M})_{*}\simeq_{\nla{c-1}}\fla{c-1}{M_{1}}/R=M$.
The first assertion is proved.

\medskip
Now by interpreting $M$ and $N_{*}$ as objects of $\nla{c}$, we obtain the presentations $\map{\mu}{\fla{c}{M_{1}}}{M}$ and
$\map{\eta}{\fla{c}{N_{1}}}{N_{*}}$. For any morphism $\varphi$ of $M$ into $N_{*}$, we obtain
a unique $\nla{c}$-morphism $\widehat\varphi$ of $\fla{c}{M_{1}}$ into $\fla{c}{N_{1}}$ by means of  Lemma \ref{commufreeno}, with
the property $\widehat\varphi\eta=\mu\varphi$.

Through the canonical $\Fp$-embedding $\iota$, we identify $\fla{c-1}{M_{1}}$ with a subspace of $\fla{c}{M_{1}}$ as described above.
If we denote by $R$ the %subspace of $\fla{c}{M_{1}}$ which is the 
kernel of the restriction of $\mu$ to $\fla{c-1}{M_{1}}$, we 
obtain $\ker(\mu)=R\oplus\gamma_{c}(\fla{c}{M_{1}})$ and $\fla{c-1}{M_{1}}$ presents $M$ modulo $R$.

It follows $\J(M)$ is the ideal of $\fla{c}{M_{1}}$ generated by $R$. In particular $\J(M)\inn\ker(\mu)$ and, if $\pi_{\J}$ presents
$\fl M$ as a quotient of $\fla{c}{M_{1}}$, then $\pi_{\J}{}*=\mu$.

On the other hand if $\epsilon_{N}$ presents $N$ from $\fla{c}{N_{1}}$, then $\epsilon_{N}{}\ast=\eta$.

\smallskip
If now $w\in R$, %\inn\fla{c}{M_{1}}$
then $(w\,\widehat\varphi\epsilon_{N})_{*}=w\,\widehat\varphi\eta=w\,\mu\varphi=\triv$. But this means $w\,\widehat\varphi\epsilon_{N}$
lays inside $N_{c}\cap R^{\,\widehat\varphi\epsilon_{N}}$ which is trivial by a matter of weight
and hence $w\,\widehat\varphi\epsilon_{N}=\triv$.

This yields that $\J(M)\inn\ker(\widehat\varphi\epsilon_{N})$ and hence we {\em define} $\widetilde\varphi$ as
the quotient of $\widehat\varphi\epsilon_{N}$ modulo $\J(M)$, that is $\bar w\mapsto w\,\widehat\varphi\epsilon_{N}$ for
$w\in\fla{c}{M_{1}}$.

Any other map $\varphi^{\prime}$ of $\fl M$ to $N$ with $\varphi^{\prime}\ast=\ast\varphi$ fits in the diagram below
in the place of $\widetilde\varphi$. In particular $\widehat\varphi\epsilon_{N}=\pi_{\J}\varphi^{\prime}$ and hence
$\varphi^{\prime}=\widetilde\varphi$.

$$\xymatrix@R-5mm@C+2mm{
&\fla{c}{M_{1}}\ar^(0.4){\widehat\varphi}[drr]\ar_{\mu}[dddl]\ar^{{\pi}_{\J}}[dd]&&\\
&&&\fla{c}{N_{1}}\ar_(0.45){\eta}[dddl]\ar^{\epsilon_{N}}[dd]\\
&\fl M\ar^(0.4){*}[dl]\ar^(0.4){\widetilde\varphi}[drr]|!{[rru];[rdd]}\hole&&\\
M\ar^(0.52){\varphi}[drr]&&&N\ar^(0.4){*}[dl]\\
&&N_{*}&
}$$
\end{proof}


Consider now a morphism $\map{\phi}{M}{N}$ of $\nla{c-1}$-algebras $M$ and $N$, by identifying $N$ with $(\fl N)_{*}$
we define the {\em free lift} of $\phi$ as the morphism
$\frl(\phi)\defeq\widetilde\phi$ of $\fl M$ into $\fl N$ given by Proposition \ref{morphifreelift}. In particular \pref{commufree} holds for $\phi$
and hence we have
\begin{cor}
The free lift mapping $\frl$ is a functor of the category $\nla{c-1}$ into $\nla{c}$, adjoint to $\ast$.
Moreover for any $\nla{c-1}$-morphism $\map{\phi}{M}{N}$, the square below
\begin{labeq}{morphistar}
\begin{split}
\xymatrix@C+5mm{
\fl{M}\ar[r]^{\frl(\phi)}\ar[d]^{*}&\fl{N}\ar[d]^{*}\\
M\ar[r]^{\phi}&N
}
\end{split}
\end{labeq}
commutes. Also by \pref{commufresco} and the construction of $\frl(\phi)$ we obtain.
\begin{labeq}{liftbild}
\im(\frl(\phi))=\gena{\phi(M_{1})}{\fl N}
\end{labeq}
\end{cor}