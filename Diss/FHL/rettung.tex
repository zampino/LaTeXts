\begin{teo}\label{teowei}
For any small model $M$ of $T^{2}$, and any type $p$ in $\ssp{n}{M}$ with $P_{1}(\bar x)$,
there is a finite strong $\nla{2}$-subalgebra $C$ of $M$,
%which is left invariant by those automorphisms of $M$ which fix the type $p$.
such that
%$C^{\sigma}=C$ for any automorphism $\phi$ of $M$ which fixes $p$
$C\inn\acl(Cb(p))$ and that $Cb(p)\inn\dcl^{eq}(C)$.
\end{teo}
\begin{proof}
%Let $p=\tp{\bar a}{M}$ and let $\bar b$ a finite tuple in $\mathbb{M}_{1}$ linear independent over $M_{1}$
%such that $\ssc(M_{1},\bar a)=\gen{M_{1},\bar b}$.
%
%With Lemma \ref{fincharssc}.(v) we can find a finite subspace $H_{1}^{\sss 0}$ of $M_{1}$ such that $d_{2}(\bar a/M)=\delta(\bar b/M)=\delta(\bar b/H^{\sss 0})$ and that is enough to imply $\ssc(M_{1},\bar a)=M_{1}+\gen{H_{1}^{\sss 0},\bar b}$ where $\gen{H_{1}^{\sss 0},\bar b}$
%is $\ssc(H_{1}^{\sss 0},\bar a)$.
%
%Choose a finite strong subspace $C_{1}$ in $M_{1}$,
%%such that $\delta(\bar a/C)=\delta(\bar a/M)$ and that $\gen{C_{1},\bar a}\zsu{}\mathbb{M}_{1}$. We also require that $C_{1}$ is
%minimal among all finite $H_{1}^{\sss 0}$ with the properties mentioned above.
%Then $C$ is characterised by being minimal among the self-sufficient subalgebras $C$ of $M$ which share the condition
%$\delta(\bar b/M)=\delta(\bar b/C)$. By Lemma \ref{freecomp}, %Since $\bar b$ is linearly independent of $M_{1}$,
%the last equality is equivalent to
%$$\rd_{\mathbb{M}}(M_{1},\bar b)\non\rd_{\mathbb{M}}(M)%:=\rd(\mathbb{M})\cap\exs\gen{M_{1},\bar b}
%\inn\exs\gen{C_{1},\bar b}$$
%%$$\rd_{\mathbb{M}}(M_{1},\bar b)/\rd_{\mathbb{M}}(M)%:=\rd(\mathbb{M})\cap\exs\gen{M_{1},\bar b}
%%\inn\exs\gen{C_{1},\bar b}/\exs C_{1}$$
%
%Conversely, for a finite subspace $D_{1}\inn M_{1}$ to have $\delta(\bar b/D)=\delta(\bar b/M)$
%implies $\delta(\bar b/D)=d_{2}(\bar a/M)$.
%
%If $C=\triv$ then we are done, since in this case $\ssc(M_{1},\bar a)=M_{1}+\ssc(\bar a)$ and $\ssc(\bar a)$ is
%linear independent of $M_{1}$,
%hence $\tp{\bar a}{M}$ is fixed by the whole group $\aut(M)$. We assume $C\neq\triv$ in the
%rest of the proof.
%
%\smallskip
%Now %on one hand, by Lemmas \ref{fincharssc}, \ref{freecomp} and \ref{forkingchar}.(ii),
%since $M_{1}\cap\ssc(C_{1},\bar a)=C_{1}$ and $d(\bar a/C)=d(\bar a/M)$,
%we have $\fin{\bar a}{C}{M}$. As $M$ is algebraically closed, $\ssc(C_{1},\bar a)=\gen{C_{1},\bar b}$ meets $\acl(C_{1})$ exactly in $C_{1}$.
%As remarked in Corollary \ref{stationary} above, this means $\tp{\bar a}{C}$ is stationary.
%If $Cb(p)=Cb(\res{p}{C})$ denotes the canonical base (of the unique non-forking global heir) of $p$, then
%%by renown properties of canonical bases in stable contexts (cfr.\,for instance \cite{ziebous}), 
%by Fact \ref{ziecb}.(2.), we have $Cb(p)\inn\dcl^{eq}(C)$.
%%In particular condition (ii) of Lemma \ref{weimodels} is satisfied by $C$.
%
%\medskip
%On the other hand, let $\aut_{\{\!M\!\}}(\mathbb{M})$ denote the group of all automorphism of the monster $\mathbb{M}$,
%which leaves $M$ invariant. By lemma \ref{bafo} and since $M$ is
%a small set, the orbit of any tuple $\bar c$
%of $M$ under the stabiliser of the type $p$ %with respect of the action of
%in $\aut(M)$ %on $\ssp{}{M}$
%coincides with the orbit of $\bar c$ %$C$
%under the action of the pointwise stabiliser of $\bar b$ in $\aut_{\{\!M\!\}}(\mathbb{M})$.
%
%Hence assume an automorphism $\sigma$ of $\aut_{\{\!M\!\}}(\mathbb{M})$ fixes $\bar b$ pointwise and let
%Since $\gena{M,\bar b}{\mathbb{M}}=\ssc(M,\bar a)=\ssc(\sigma(M),\bar a)$,
%%we may assume that leaves $\bar b$ fixed.
%$\sigma$ fixes $\gena{M_{1},\bar b}{\mathbb{M}}$ and permutes $M$ and $\bar b$ in disjoint (linear independent) orbits. 
%$\sigma_{1}$ denote the restriction of $\sigma$ to $M_{1}$: $\sigma_{1}$ is a linear isomorphism in
%$\mathit{GL}(\mathbb{M}_{1})$ which leaves the subspace $M_{1}$ invariant.
%
%Let $\sigma_{*}$ denote the graded Lie isomorphism induced by $\sigma_{1}$ on the free graded algebra
%$\fla{2}{\mathbb{M}_{1}}=\mathbb{M}_{1}\oplus\exs\mathbb{M}_{1}$.
%
%It follows
%%that $\exs\mathbb{M}_{1}$ which fixes $\exs M_{1}$ setwise. It folows
%that $\sigma_{*}$ fixes setwise $\exs\gen{M_{1},\bar b}$ in $\exs\mathbb{M}_{1}$ %over $\exs M_{1}$
%and also, since $\sigma$ is a Lie
%morphism and %$\gena{M_{1},\bar b}{\mathbb{M}}=\fla{2}{M_{1},\bar b}/\rd_{\mathbb{M}}(M_{1},\bar b)$
%$\gena{M_{1},\bar b}{\mathbb{M}}_{2}=\exs\gen{M_{1},\bar b}/\rd_{\mathbb{M}}(M_{1},\bar b)$, the subspace
%$\rd_{\mathbb{M}}(M_{1},\bar b)$ %/\rd_{\mathbb{M}}(M)$
%is also invariant under the action of $\sigma_{*}$. % on $\exs\gen{M_{1},\bar b}$%/\exs M_{1}$.
%
%In particular since $\rd_{\mathbb{M}}(M_{1},\bar b)\non\rd_{\mathbb{M}}(M)\inn\exs\gen{C_{1},\bar b}$, then
%$$\rd_{\mathbb{M}}(M_{1},\bar b)\non\rd_{\mathbb{M}}(M)=\sigma_{*}(\rd_{\mathbb{M}}(M_{1},\bar b)\non\rd_{\mathbb{M}}(M))\inn
%\sigma_{*}\left(\exs\gen{C_{1},\bar b}\right)=\exs\gen{\sigma_{1}(C_{1}),\bar b}.$$
%
%It follows by \pref{exsmod}, $\rd_{\mathbb{M}}(M_{1},\bar b)\inn\exs\gen{C_{1}\cap\sigma(C_{1}),\bar b}$ and
%by minimality of $C_{1}$, this yields $\sigma_{1}(C_{1})=C_{1}$ %But then $\sigma_{1}(C_{1})=C_{1}$ and
%and of course $\sigma(C)=C$.
%Since $C$ is a finite subalgebra, if we interpret the elements of $C$ in a tuple $\bar c$,
%then necessarily $\bar c$ has a finite -- permutation -- orbit under $\aut_{\{\!M\!\}}(\mathbb{M})$.
%
%This is condition (ii) of Proposition \ref{weimodels}.
Assume $p$ is $\tp{\bar m}{M}$, for some tuple $\bar m$ in $\mathbb{M}_{1}$.

Let $\bar a=(\tupl{a}{0}{n-1})$ be a tuple in $\mathbb{M}_{1}$ linearly independent of $M_{1}$, such that
$\ssc(M,\bar m)=\gena{M,\bar a}{\mathbb{M}}$.
%Let further $\bar b$ be a subtuple of $\bar a$ of length $d(\bar m/M)=d(\bar a/M)$ such that
%$d(\bar b/M)=\delta(\bar b/M)=d(\bar a/M)$. Note that $\bar b$ collects the first transcendental steps
%of any minimal decomposition of $\ssc(M,\bar m)$ over $M$.

\medskip
Let $\aut_{\{\!M\!\}}(\mathbb{M})$ denote the group of all automorphism of the monster $\mathbb{M}$,
which leave $M$ invariant.
%By lemma \ref{bafo}, %and since $M$ is a small set,
%if a tuple $\bar m^{\prime}$ has the same type of $\bar m$ over $M$, then also
%$\ssc(M,\bar m^{\prime})$ has the same type of $\ssc(M,\bar m)$ over $M$.
%Hence %the orbit of any tuple $\bar c$
%Since $M$ is small, the strong homogeneity of the monster implies, that
%the action on $M$ of the stabiliser of the type $p$ %with respect of the action of
%in $\aut(M)$ %on $\ssp{}{M}$
%coincides with the action on $M$ %orbit of $\bar c$ %$C$
%under the pointwise stabiliser of $\bar m$ %$\bar a$
%in $\aut_{\{\!M\!\}}(\mathbb{M})$.

If $\sigma$ is an automorphism of $\mathbb{M}$, whose restriction to ${M}$ fixes the type $p$,
then $\bar m^{\sigma}\equiv_{M}\bar m$.\mn{if also $\bar a\equiv_{M}\bar a^{\sigma}$ we can spare us calctions below \dots}
%This means, by Proposition \ref{bafo}, that the application fixing $M$ and sending $\bar m$ to $\bar m^{\sigma}$
%may be extended to an isomorphism $\phi$ of $\gena{M,\bar a}{\mathbb{M}}$ onto $\ssc(M,\bar m^{\sigma})=
%%:\gena{M,\bar b}{\mathbb{M}}$.
%\gena{M,\bar a^{\sigma}}{\mathbb{M}}=:\gena{M,\bar b}{\mathbb{M}}$\mn{Careful! $\bar a^{\phi}$ {\bf may not} be $\bar a^{\sigma}$!!!}
%for some $\bar b\inn\mathbb{M}_{1}$ with $\phi\colon \bar a\stackrel[M]{\simeq}{\longmapsto}\bar b$.
%
%Proposition \ref{bafo} again now implies that $\phi$ is actually elementary, that is $\phi\colon\bar a
%\stackrel[M]{\equiv}{\longmapsto}
%\bar b$.
Since $M$ is small, the strong homogeneity of the monster implies, that
the action {\em on} $M$ of the stabiliser of the type $p$ %with respect of the action of
in $\aut(M)$ %on $\ssp{}{M}$
coincides with the action on $M$ %orbit of $\bar c$ %$C$
under the pointwise stabiliser of %$\bar m$ {\em and } %$\bar a$ (and 
$\bar m$ in $\aut_{\{\!M\!\}}(\mathbb{M})$. % which we denote by $\aut_{\{\!M\!\}}(\mathbb{M})$.%_{\bar m}$.

On the other hand if $\sigma\in\aut_{\{\!M\!\}}(\mathbb{M})$ and fixes $\bar m$, then $\gen{M_{1},\bar a}^{\sigma}=
\ssc(M_{1},\bar m)^{\sigma}=\ssc(M_{1},\bar m^{\sigma})=\gen{M_{1},\bar a}$ and hence $\gen{\bar a}^{\sigma}=\gen{\bar a}$.

%Since the self-sufficient closure commutes with automorphisms,
%one has
%$$\gena{M,\bar a^{\sigma}}{\mathbb{M}}=\ssc(M,\bar m)^{\sigma}=\ssc(M^{\sigma},\bar m^{\sigma})%{\gena{M_{1},\bar a}{\mathbb{M}}}^{\sigma}=\gena{M_{1},\bar a^{\sigma_{1}}}{\mathbb{M}}
%=\ssc(M,\bar m)=\gena{M,\bar a}{\mathbb{M}}$$
%and in particular $\gen{\bar a^{\sigma}}=\gen{\bar a}$ in $\mathbb{M}_{1}$.

%For any automorphism $\sigma$ of $\aut_{\{\!M\!\}}(\mathbb{M})$ %fixes $\bar a$ pointwise and let
%let $\sigma_{1}$ denote the restriction of $\sigma$ to $M_{1}$: $\sigma_{1}$ is a linear isomorphism in
%$\mathit{GL}(\mathbb{M}_{1})$ which leaves the subspace $M_{1}$ invariant.
%Let also $\sigma_{*}$ denote the graded Lie isomorphism induced by $\sigma_{1}$ on the free graded algebra
%$\fla{2}{\mathbb{M}_{1}}=\mathbb{M}_{1}\oplus\exs\mathbb{M}_{1}$.
%Remark that $\sigma_{*}$ leave the
%subspace $\exs\gen{M_{1},\bar a}$ of $\exs\mathbb{M}_{1}$ invariant.

\medskip
Let %$\mathcal{B}$ an ordered basis of $M_{1}$ and
$(\rho_{i})_{i<m}$ be a set in $\exs\gen{M_{1},\bar a}$ linearly independent over $\exs M_{1}$ which is a basis
of $\rd_{\mathbb{M}}(M,\bar a)$ over $\rd_{\mathbb{M}}(M)$. Since $M$ is self-sufficient, then $m\leq n$.%\dfp(\bar a/M_{1})$.
%After having expressed each $\rho_{i}$ as linear combinations of basic monomials
%over the ordered basis $\mathcal{A}=\{\bar a>\mathcal{B}\}$ of $\gen{M_{1},\bar a}$, we may assume that

We may find for all $i<m$ a tuple $\bar b_{i}=(\utpl{b_{i}}{0}{n})$ of length $n$, of not necessarily linearly independent elements of $M_{1}$
such that 
\begin{labeq}{rhodi}
\rho_{i}=\alpha_{i}+\beta_{i}+\gamma_{i}
%\rho_{i}=\alpha_{i}(\bar a)+\beta_{i}(\bar a,\bar c)+\gamma_{i}(\bar c)
\end{labeq}
%where $\alpha_{i}\in\exs\gen{\bar a}$, $\beta_{i}\in\exs\gen{\bar a,\bar c}\non\exs\gen{\bar a}+\exs\gen{\bar c}$ and
%$\gamma_{i}\in\exs\gen{\bar c}\non\rd(M)$ are supposed to be
%where the terms
%$\alpha_{i}$, $\beta_{i}$ and $\gamma_{i}$ are linear combinations of basic monomials
%over the ordered set $\{\bar a>\bar c\}$ and the ordering on $\bar c$ is the one inherited from $\mathcal{B}$. We can also
%require
where each $\alpha_{i}=\alpha_{i}(\bar a)$ is in $\exs\gen{\bar a}$
and every $\gamma_{i}$ %(\bar c)\in
lays in $\exs M_{1}\non\rd(M)$ for all $i<m$ and %{\em each basic monomial} appearing in
$$\beta_{i}=\beta_{i}(\bar a,\bar b_{i})=\sum_{k<n}[a_{k},{b_{i}}^{k}]\in\exs\gen{M_{1},\bar a}\non\exs\gen{\bar a}+\exs M_{1}.$$

\smallskip
%Now take a tuple $\bar b$ of $\mathcal{B}$ %be a tuple of $M$,
%minimal with the properties:
%\begin{itemize}
%\item[-]$\{\bar a,\bar b\}\nni\supp_{\mathcal{A}}(\beta_{i})$ for all $i$ and
%\item[-]$\gen{\bar a,\bar b}\nni\bar m$.\mn{or, embed this feature somewherelse!!!}
%\end{itemize}


\smallskip
Let now $\sigma$ be an automorphism  in
$\aut_{\{\!M\!\}}(\mathbb{M})$ which fixes %$\bar a$ and
$\bar m$ pointwise. 
Denote by $\sigma_{1}$ the restriction of $\sigma$ to $M_{1}$: $\sigma_{1}$ is a linear isomorphism in
$\mathit{GL}(\mathbb{M}_{1})$ which leaves the subspace $M_{1}$ invariant.
Let also $\sigma_{*}$ be the graded Lie isomorphism induced by $\sigma_{1}$ on the free graded algebra
$\fla{2}{\mathbb{M}_{1}}=\mathbb{M}_{1}\oplus\exs\mathbb{M}_{1}$.

With this notation, as $\sigma(\gena{M,\bar a}{\mathbb{M}})=\gena{M,\bar a}{\mathbb{M}}$, it follows $\sigma_{*}\left(\rd(M_{1},\bar a)\right)=\rd(M_{1},\bar a)$.
Moreover since $(\rho_{i})_{i<m}$ is a basis of $\rd(\bar a/M)$, for all $i$ we have %,${\rho_{i}}^{\sigma_{*}}\in\rd(M,\bar a)$, it follows
\begin{labeq}{rhosig}
{\rho_{i}}^{\sigma_{*}}-\sum_{j<m} s_{j}\rho_{j}=\mu
\end{labeq}
%for all $i$ and
for some $\mu$ in $\exs M_{1}$ and $s_{j}$ in $\Fp$. On the other side
$$
{\rho_{i}}^{\sigma_{*}}=\alpha_{i}(\bar a)^{\sigma_{*}}+\beta_{i}(\bar a,{\bar b_{i}})^{\sigma_{*}}+\gamma_{i}^{\sigma_{*}}=
\alpha_{i}(\bar a^{\,\sigma})+\beta_{i}(\bar a^{\,\sigma},{\bar b_{i}}^{\,\,\sigma})+\gamma_{i}^{\,\sigma_{*}}
$$
where
\begin{labeq}{betasig}
\beta_{i}(\bar a^{\,\sigma},{\bar b_{i}}^{\,\sigma})=\sum_{l<n}[\sigma(a_{l}),\sigma({b_{i}}^{l})]
\end{labeq}

Now \pref{rhosig} becomes
\begin{labeq}{rhomuc}
\alpha_{i}^{\,\sigma_{*}}-\sum_{j<m}s_{j}\alpha_{j}+\beta_{i}%(\bar a,{\bar b_{i}}^{\,\sigma})
^{\,\sigma_{*}}-\sum_{j<m}s_{j}\beta_{j}%(\bar a,{\bar b_{j}})
=\mu-\gamma_{i}^{\,\sigma_{*}}+\sum_{j<m}s_{j}\gamma_{j}\:\in\exs M_{1}
\end{labeq}
and %, by the choice of the terms in \pref{rhodi},
since $\gen{\alpha_{i},\alpha_{i}^{\,\sigma_{*}},\beta_{i},\beta_{i}^{\,\sigma_{*}}\mid i<m}\cap\exs M_{1}=\triv$, one has
$(\alpha_{i}+\beta_{i}(\bar a,{\bar b_{i}}))^{\sigma_{*}}=\sum_{j<m}
s_{j}(\alpha_{j}+\beta_{j}(\bar a,{\bar b_{j}}))$ and by the same arguments, one gets
$\beta_{i}(\bar a^{\,\sigma},{\bar b_{i}}^{\,\sigma})=\sum_{j<m}s_{j}\beta_{j}(\bar a,{\bar b_{j}})$.

On the other hand, since $\gen{\bar a}=\gen{\bar a^{\,\sigma}}$, we have $a_{k}=\sum_{l<n}r_{k}^{l}\sigma(a_{l})$ for
some $r_{k}^{l}$ in $\Fp$ and

\begin{labeq}{beta}
\beta_{i}(\bar a^{\,\sigma},{\bar b_{i}}^{\,\sigma})=\sum_{j<m}s_{j}\beta_{j}(\bar a,{\bar b_{j}})=\sum_{k<n}[a_{k},\sum_{j<m}s_{j}{b_{j}}^{k}]
=\sum_{l<n}[\sigma(a_{l}),\sum_{\substack{j<m\\k<n}}r_{k}^{l}s_{j}{b_{j}}^{k}].
\end{labeq}

Now by Hall's (\dots completing $\bar a^{\,\sigma}$ to a basis of $\gen{M_{1},\bar a}$\dots) in $\exs\mathbb{M}_{1}$ we have $[\sigma(a_{k}),{M}_{1}]\cap\sum_{j\neq k}[\sigma(a_{j}),{M}_{1}]=\triv$. %and hence $\exs\gen{M_{1},\bar a}=\bigoplus_{k<n}[a_{k},{M}_{1}]$.

Therefore \pref{betasig} and \pref{beta} imply for all $l$, $[\sigma(a_{l}),\sigma({b_{i}}^{l})]=[\sigma(a_{l}),
\sum_{j,k}%_{\substack{j<m\\k<n}}
r_{k}^{l}s_{j}{b_{j}}^{k}]$
%and hence by $\sig{2}{2}$ we have
in $\exs\mathbb{M}_{1}$. This yields $\sigma({b_{i}}^{l})=\sum_{j,k}%_{\substack{j<m\\k<n}}
r_{k}^{l}s_{j}{b_{j}}^{k}$ and hence
$\gen{\bar b_{i}\mid i<m}^{\sigma}=\gen{\bar b_{i}\mid i<m}$.

If we denote by $\bar b$ the tuple $\tupl{\bar b}{0}{m}$, since $\sigma$ fixes $\bar m$, we may also assume -- by making $\bar b$ bigger if necessary -- that $\bar m\inn\gen{\bar a,\bar b}$ and again $\bar b\inn M_{1}$ with still $\gen{\bar b}^{\sigma}=\gen{\bar b}$.

\medskip
Let now $\Gamma_{i}$ denote the image of $\gamma_{i}$ in $M_{2}$ modulo $\rd(M)$, that is
$\Gamma_{i}=\gamma_{i}+\rd(M)$ for all $i$. Since, by \pref{rhosig} and \pref{rhomuc} we have
$$
\rd(M,\bar a)\ni{\rho_{i}}^{\sigma_{*}}-\sum_{j<m} s_{j}\rho_{j}=\gamma_{i}^{\,\sigma_{*}}-\sum_{j<m}s_{j}\gamma_{j},$$
we deduce
$${\Gamma_{i}}^{\sigma}=\gamma_{i}^{\,\sigma_{*}}+\rd(M,\bar a)=\sum s_{j}\gamma_{j}
+\rd(M,\bar a)=\sum s_{j}\Gamma_{j}$$
and hence, if $\overline{\Gamma}$ denotes the tuple $(\tupl{\Gamma}{0}{m})$,
we get $\gen{\overline{\Gamma}}^{\sigma}=\gen{\overline{\Gamma}}$.

\smallskip
Till now we have shown, that for any automorphism $\sigma$ of $M$,
if $\sigma$ fixes the type $p$, then $\sigma$ leaves both spaces $\gen{\bar b}$ and $\gen{\overline{\Gamma}}$
invariant. We will now prove that -- as a consequence $\gen{\bar b}$ and $\gen{\overline{\Gamma}}$ lay in $\acl(Cb(p))$.

To see this take an $\omega$-saturated elementary extension $M^{\prime}$ %\ess\mathbb{M}$
of $M$ which is forking independent of $\bar m$ over $M$.

Since $M$ is a model, $\fin{\bar m}{M}{M^{\prime}}$ implies both $\ssc(M_{1},\bar m)\cap M^{\prime}_{1}=M_{1}$
and $\ssc(M^{\prime},\bar m)\simeq\am{M^{\prime}}{M}{\ssc(M,\bar m)}=\gena{M^{\prime},\bar a}{\mathbb{M}}$
by lemma \ref{forkingchar}.

Then in particular $\rd(M^{\prime},\bar a)=\rd(M^{\prime})+\rd(M,\bar a)$
and hence %$\delta(\bar a/M^{\prime})=\delta(\bar a/M)$ and this means,
$\rd(\bar a/M^{\prime})=\rd(\bar a/M)$. It follows $(\rho_{i}=\alpha_{i}+\beta_{i}+\gamma_{i})_{i<n}$ is
a basis for $\rd(M^{\prime},\bar a)$ over $\rd(M^{\prime})$.
This means, the spaces $\gen{\bar b}$ and $\gen{\overline{\Gamma}}$
%_{i}\mid i<n}$ with $\Gamma_{i}=\gamma_{i}+\rd(M^{\prime})$
%can play the same role also for $\tp{\bar m}{M^{\prime}}$, and hence
%It follows both {\em finite sets} $\gen{\bar b}$ and $\gen{\overline{\Gamma}}$
are setwise fixed by the automorphism of
$M^{\prime}$ which fix $\tp{\bar m}{M^{\prime}}$ as well.

Moreover, as we are in a totally transcendental theory, $Cb(p)=Cb(\bar m/M)\\
=Cb(\bar m/M^{\prime})$ is
the $eq$-definable closure of a {\em finite} imaginary.

We may conclude that, since $M^{\prime}$ is enough saturated with respect to $\gen{\bar b}\cup\gen{\overline{\Gamma}}$ and $Cb(p)$,
both sets $\gen{\bar b}$ and $\gen{\overline{\Gamma}}$ must be permuted also under automorphism {\em of} $\mathbb{M}$ which
fix $Cb(p)$ pointwise. Since these are finite sets, this yields $\gen{\bar b, %}\cup\gen{
\overline{\Gamma}}\inn\acl(Cb(p))$.




%\nni\bar b$ and $\ssc(\bar a,\bar b,\bar c)=\gen{C_{1},\bar a}$.
%Then
%$$\gen{{C_{1}}^{\sigma},\bar a}=\ssc(\bar b,\bar a)^{\sigma}=\ssc(\bar b^{\,\sigma},\bar a)=\gen{C_{1},\bar a}$$
%and hence $C^{\sigma}=C$.

\medskip
%If now $\bar b$ is the trivial tuple, that is $\rho_{i}=\alpha_{i}+\gamma_{i}$ for all $i$,
%then by $\pref{rhomuc}$ 

\uwave{Since we may assume},\mn{{\bf risiko}!!!}
that $\bar b$ is not constantly equal to $\triv$,
let $b_{0}$ be an element of $\bar b$ and -- with $\sig{2}{4}$ -- take $c_{i}$ in $M_{1}$ such that
$[b_{0},c_{i}]=\Gamma_{i}$ in $M$ for $i<m$. This means $[b_{0},c_{i}]$ can play the role of $\gamma_{i}$ in $\exs M_{1}$.
Then if the tuple $\bar c$ collects all the $c_{i}$, %by above remarks we have
we have $\bar c\inn\acl(\bar b,\bar\Gamma)$. % and $\overline{\Gamma}\inn\dcl(\bar b,\bar c)$.

Take $C\zsu{}M$ with $C_{1}=\ssc(\bar b,\bar c)$,
then on one side $C\inn\acl(\bar b,\bar c)$ and hence $C\inn\acl(Cb(p))$.

On the other hand we have $\delta(\bar a/C)=\delta(\bar a/M)$ and $\gen{C_{1},\bar a}\nni\bar m$, therefore, as
$C$ is strong in $M$, Lemma \ref{fincharssc} imply
$$
d(\bar m/C)\geq d(\bar m/M)=\delta(\bar a/M)=\delta(\bar a/C)\geq d(\bar a/C)\geq d(\bar m/C)
$$
and hence $d(\bar m/C)=d(\bar m/M)$.

This also yields -- with Lemma \ref{fincharssc} again --  $M+\ssc(C,\bar m)\zsu{}\mathbb{M}$ and hence $\ssc(C_{1},\bar m)=\gen{C_{1},\bar a}$. That is $\fin{\bar m}{C}{M}$.

On the other hand since $M$ is a model $\gen{C_{1},\bar a}$ meets
$\acl(C_{1})$ necessarily in $C_{1}$ and this gives with Corollary \ref{stationary}, that $\tp{\bar m}{C}$
is stationary. Now fact \ref{ziecb}\,(2.) implies $Cb(p)\inn\dcl^{eq}(C)$.
%Now since $C\inn\acl(\bar b,\bar c)$, it follows $Cb(\bar m/C)\inn\acl(\bar b,\bar c)$.
\end{proof}

\begin{dfn}\label{gbase}
For a model $M\ess\mathbb{M}$ and a tuple $\bar m$ of $\mathbb{M}_{1}$ we call a finite strong subalgebra $C$ of $M$ in the 
statement of Theorem \ref{teowei} a {\em geometrical base} for $\bar m$ over $M$, or equivalently,
for the type $\tp{\bar m}{M}$.
\end{dfn}
Remark that for any $\bar m$ and $M$ as above,
if $\ssc(M,\bar m)=\gena{M,\bar a}{\mathbb{M}}$, then for any strong finite subalgebra $D$ of $M$ with
$\delta(\bar a/M)=\delta(\bar a/D)$ and $\gen{D_{1},\bar a}\nni\bar m$,
a geometrical base $C$ for $\tp{\bar m}{M}$ might always be found inside $D$.

\medskip
By Lemma \ref{weimodels} of Section \ref{stab}, we have the following
\begin{cor}\label{weiuno}
With a distinguished constant $c$, $\mathbb{M}_{c}$ has weak elimination of imaginaries.
\end{cor}
\begin{proof}
%The corollary follows from Proposition \ref{weimodels}.
By Theorem \ref{teowei} and the aforementioned lemma follows, that the structure on $\mathbb{M}_{1}$
induced by $T_{2}$ has (WEI).

This is for any model $M$
of $T^{2}$ and all types $p$ over $M$ implying $P_{1}(\bar x)$, we arrange a geometrical base
$C$ for $p$ over $M$ in a {\em finite} tuple $\bar c$. Then $\bar c$ is permuted -- with a {\em finite} orbit --
by the automorphisms of $M$ which fix $p$. Conversely, since $Cb(p)\inn\dcl^{eq}(C)$, then
in particular $p$ is fixed by automorphisms of $M$ which fix $C$ pointwise.

we name a distinguished non-trivial element $m$ of $\mathbb{M}_{1}$, the theory $T^{2}_{\,m}=\mathit{th}(\mathbb{M},m)$ has {\em (WEI)} as well.

The statement of the theorem follows from Corollary \ref{weiuno}, {\bf if we can show} that $\mathbb{M}_{1}$ weakly
eliminates imaginaries also after adding an element $m$ of $\mathbb{M}_{1}$ to $T^{2}$ as a parameter.

Since then, every imaginary of $\mathbb{M}$ is {\em interdefinable} with an imaginary
of $\mathbb{M}_{1}\times\mathbb{M}_{1}/\gen{m}$, which is in particular an imaginary of $\mathbb{M}_{1}$.
\end{proof}

\medskip
In \cite{bad}, Baudisch proves that, after adding a suitable finite set of parameters to the collapsed
theory $T^{2,\mu}$, $\acl(\vac)$ becomes
infinite. With strong minimality, weak elimination of imaginaries follows by Lascar-Pillay.

\medskip
With the help of geometrical bases, we can now prove {\sl CM}-triviality for $T^{2}$.

%The following result from \cite{pilcm} will also be used
%\begin{fact}\label{pilcb}
%Assume $M\ess\mon$ is a model of a stable theory, $\mon$ its monster model and let $c,d$ be tuples in $\mon^{eq}$.
%
%If any of the following two conditions
%%and denote by $\mathscr{C}$, $Cb(c/M)$ and by $\mathscr{D}$, $Cb(d/M)$.T
%%then the following holds:
%\begin{itemize}
%\punto{i}$c\in\acl(d)$ %\quad\Rightarrow\quad Cb(c/M)\inn\acl^{eq}(Cb(d/M))$
%\punto{ii}$\ffin{c}{d}{M}$ %\quad\Rightarrow\quad Cb(d/M)\inn\acl^{eq}(Cb(c/M))$
%\end{itemize}
%holds, then $Cb(c/M)\inn\acl^{eq}(Cb(d/M))$.
%\end{fact}

We can now prove 
\begin{prop}
$T^{2}$ is a $CM$-trivial thory.
\end{prop}
\begin{proof}
As pointed out in Fact \ref{cmt}, it is enough prove the statement for $T^{2}_{m}$.
Let $\bar c$, $M\ess N\ess\mathbb{M}$ be a triple satisfying the the assumptions of Fact \ref{cmt}. That is
$\acl(M,\bar c)\cap N=M$.
 
We first claim that we actually may consider tuples from $\mathbb{M}_{1}$ only. To see this, if $\bar c$ is not
entirely contained in $\mathbb{M}_{1}$, take
$t$ in $\mathbb{M}_{1}$ such that $\ffin{t}{\bar c}{\,N}$.
With $\sig{2}{4}$ it is possible to find a tuple $\bar b$ in $\mathbb{M}_{1}$ such that
for all $c\in\bar c$, $[t,b]=c_{2}$ for some $b\in\bar b$. Let also $\bar b$ contain every $P_{1}$-component\mn{{\bf define it!}}
of the elements in $\bar c$. Assume further that all of $\bar b$ is involved in such tasks.
%We denote by $\bar c_{1}$ the (possibly empty) tuple consisting of all $P_{1}$-components of elements in $\bar c$.

As already pointed out often before, by $\sig{2}{2}$ it follows $\bar c\inn\dcl(t,%\bar c_{1},
\bar b)$ and $%\bar c_{1},
\bar b\inn\acl(t,\bar c)$.
Thus, as $\ffin{t}{\bar c}{\,N}$ implies $\ffin{t\,%\bar c_{1}
\bar b}{\bar c}{\,N}$, Fact \ref{pilcb} now yields $Cb(\bar c/M)\inn\acl^{eq}(Cb(t,%\bar c_{1},
\bar b/M))$ and $Cb(t,%\bar c_{1},
\bar b/N)\inn\acl^{eq}(Cb(\bar c/N))$.
 
On the other hand we still have $\acl(M,t,%\bar c_{1},
\bar b)\cap N=M$, for if $e\in\acl(M,t,%\bar c_{1},
\bar b)$ then
$$\ffin{t\,%\bar c_{1}
\bar b}{\bar c}{\,N}\:\Rightarrow\:\ffin{t\,%\bar c_{1}
\bar b}{M\bar c}{\,N}\:\Rightarrow\:\ffin{e}{M\bar c}{\,N}.$$
Hence if $e\in N$ then by irreflexivity $e\in\acl(M,\bar c)$ and hence $e\in M$. 

We have shown with this, that if the statement of the proposition is true of tuples in $\mathbb{M}_{1}$, then it is true of
arbitrary ones.
 
%Using the map $\mu_{m}$ defined
%in the proof of Theorem \ref{teowei} above, we can claim that any tuple in $\mathbb{M}$ is
%interalgebraic with some tuple in $\mathbb{M}_{1}^{eq}$ and hence -- by (WEI) for $\mathbb{M}_{1}$ -- %Theorem \ref{teowei} --
%with some tuple of $\mathbb{M}_{1}$. By the remark above
%about canonical bases of algebraic tuples, we can restrict ourself
%to consider just tuples of elements in $\mathbb{M}_{1}$ in Fact \ref{cmt}.



\medskip
Let therefore $\bar m$ be a tuple in $\mathbb{M}_{1}$ and $M\ess N\ess\mathbb{M}$ with
$\acl(M,\bar m)\cap N=M$. 
By \pref{acluno} follows in particular $\ssc(M_{1},\bar m)\cap N_{1}=M_{1}$.
As a consequence, if $\bar a$ is
a tuple of $\mathbb{M}_{1}$, linearly independent over $M_{1}$ such that $\ssc(M_{1},\bar m)=\gen{M_{1},\bar a}$,
then $\bar a$ is linearly independent over $N_{1}$ as well.

Now if $\ssc(N_{1},\bar m)$ is $\gen{N_{1},\bar c}$ with $\bar c$ linearly independent of $N_{1}$,
then we may assume that $\bar a$ is a subtuple of $\bar c$. Since $\gen{M_{1},\bar a}\cap N_{1}=M_{1}$ we have
by \pref{exsmod} $\rd(M,\bar a)\cap\rd(N)=\rd(M)$ and hence
$$\rd_{\mathbb{M}}(\bar a/M)\inn\rd_{\mathbb{M}}(\bar c/N).$$
That means, any basis of $\rd_{\mathbb{M}}(\bar a/M)$ can be extended to a basis of $\rd_{\mathbb{M}}(\bar c/N)$.

Therefore, by the construction of a geometrical base $C$ for $\tp{\bar m}{M}$ in the proof of Theorem \ref{teowei},
we may extend any such $C$ to a finite strong subalgebra $D$ of $N$ which is a geometrical base for $\tp{\bar m}{N}$.

%Let $D$ be a geometrical base for $\bar m$ over $N$ as constructed in Definition \ref{gbase}, then in particular
%$\rd(N_{1},\bar b)\non\rd(N)\inn\exs\gen{D_{1},\bar b}$ and by \pref{exsmod}
%$$\rd(M_{1},\bar b)\non\rd(M)\inn(\rd(N_{1},\bar b)\non\rd(N))\cap\exs\gen{M_{1},\bar b}\inn\exs\left(\gen{D_{1},\bar b}\cap\gen{M_{1},\bar b}\right).$$
%
%Since $\gen{D_{1},\bar b}\cap\gen{M_{1},\bar b}=\gen{D_{1}\cap M_{1},\bar b}$, by minimality
%any geometrical base $C$ for $\bar m$ over $M$ is contained in $D$.

We may now conclude with Theorem \ref{teowei}, that
$Cb(\bar m/M)\inn\dcl^{eq}(C)$ and $D\inn\acl(Cb(\bar m/N))$.

%Take an $\omega$-saturated elementary extension $N^{\prime}$ %\ess\mathbb{M}$
%of $N$ which is forking independent of $\bar m$ over $N$.
%
%Since $N$ is a model, $\fin{\bar m}{N}{N^{\prime}}$ implies both $\ssc(N_{1},\bar m)\cap N^{\prime}_{1}=N_{1}$
%and, by lemma \ref{forkingchar}, $\ssc(N^{\prime},\bar m)=N^{\prime}+\ssc(N,\bar m)=\gena{N^{\prime},\bar c}{\mathbb{M}}$.
%
%Then in particular $\delta(\bar c/N^{\prime})=\delta(\bar c/N)=\delta(\bar c/D)$ and this means, by the above remarks, that
%$D$ is also a geometrical base for $\tp{\bar m}{N^{\prime}}$ and of course $Cb(\bar m/N)=Cb(\bar m/N^{\prime})$.
%
%Since $D$ is setwise fixed by all automorphism of $N^{\prime}$ which fix $\tp{\bar m}{N^{\prime}}$ and $N^{\prime}$
%is enough saturated with respect of $D$ and $Cb(\bar m/N)$,
%%if we arrange $D$ in a tuple $\bar d$, then $\bar d$ must have finitely many conjugates
%$D$ must be permuted also under automorphism {\em of} $\mathbb{M}$ which
%fix $Cb(\bar m/N)$ pointwise. Since $D$ is finite, this gives $D\inn\acl(Cb(\bar m/N))$.

Since $C\inn D$, we get $Cb(\bar m/M)\inn\dcl^{eq}(\acl(Cb(\bar m/N)))=\acl^{eq}(Cb(\bar m/N))$ as desired.
\end{proof}

As mentioned in Section \ref{stab}, by \cite[Proposition 3.2]{pilcm}, we may conclude
\begin{cor}
No infinite field is interpretable in $T^{2}$.
\end{cor}
%\end{document}