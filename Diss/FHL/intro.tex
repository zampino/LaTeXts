The purpose of this work is twofold: on one side we propose a new treatment of the structures which led to Baudisch' {\em new
$\aleph_{1}$-categorical group} of nilpotency class $2$ constructed in \cite{bad}. On the other hand we settle
a new framework to possibly achieve Groups with similar properties but in higher nilpotency classes. The main efforts
involve the nilpotent-$3$ case.

For what concerns both aspects, the deep contiguity between nilpotent groups of prime exponent and
graded Lie algebras over finite fields, let us work within the second kind of structures, which support in addition
a linear-algebraic approach. %, suitable for a {\em counting} argument.
This correspondence is explained in detail in Section \ref{nilgral}.

\smallskip
The aforemensioned \lqq Baudisch group\rqq arises from a direct translation in combinatorial group-theoretic terms, of the restyled
Fra\"iss\'e amalgamation technique, which led Hrushovski in \cite{hruabi} to confute Zilber's {\em structural conjecture}
(\cite{zil}). % on uncountably categorical theories.
We briefly review these facts below, as they form in part the guidelines of the
present work.

\medskip
A definable set of a complete first-order theory is called strongly minimal if its Morley rank and degree are both equal to one.
%This translates to the fact that no definable set
%can be infinite and co-infinite, in any elementary extension.
%The essential tool to prove Morley's categoricity results (rephrased in a geometric flavour
%by Baldwin and Lachlan in \cite{blsms}) is the fact that, in strongly minimal structures, the
%algebraic closure is a pregeometry and hence allow a dimension theory.
In a strongly minimal structure, the (model-theoretic) algebraic closure yields a {\em pregeometry}. This allowed for instance
Baldwin and Lachlan in \cite{blsms} to reprove Morley's categoricity results by means of a {\em dimensional} approach, derived
by such pregeometries.
Strongly minimal structures are in particular $\aleph_{1}$-categorical
and on the contrary, uncountably categorical structures do always ``contain'' strongly minimal sets as
-- we might say -- building blocks.

For the definition of a (pre)geometry and related notions, the reader is referred to Section \ref{qdim}.

%In \cite{hruabi} and \cite{jbg} it is recalled that t
The pregeometries attached to the strongly minimal sets definable in a
$\aleph_{1}$-categorical structure, have (after localisation) all isomorphic associated geometries. This local isomorphism type
constitutes therefore an invariant of such structures.

Zilber conjectured indeed that each $\aleph_{1}$-categorical theory $T$ is assigned a geometry according to the following trichotomy
(cfr.\,\cite{hruabi,jbg}).
\begin{itemize}
\punto{1}A disintegrated geometry. No infinite %almost strongly minimal
group is definable in $T$.
\punto{2}A nontrivial modular geometry of a vector space. An infinite % rank-$1$
group is definable in $T$, but no infinite field does.
\punto{3}A non locally modular geometry. $T$ is not one-based and an infinite field is interpretable
in $T$.
\end{itemize}

The conjecture was disproved by Hrushovski in \cite{hruabi} by means of {\em new strongly minimal sets},
which have a non-locally modular geometry, but nevertheless do not interpret an infinite field.
%\xsubsection{The {\em ab Initio} Construction}
These counterexamples rely on a Fra\"{\i}ss\'e amalgamation procedure (described in Section \ref{fraisse}), together with
a pregeometric machinery, which modifies ordinary embeddings. This allows in particular
to control the types of the Fra\"iss\'e limit by means of a dimension function: the
structures obtained are stable, which is not in general the case for Fra\"iss\'e constructions.

\smallskip
To summarise the above process, start -- say -- from a ternary, order-invariant relation $(M,R)$
and define an integer valued function of the finite parts of the domain $M$:
\begin{labeq}{urdelta}
\delta(A)=\card{A}-\card{R(A)}
\end{labeq}
where $R(A)$ describes the set of all ternary {\em links} $(a,b,c)$ with $R(a,b,c)$ -- up to permutation -- which insist among
points of $A$.

This $\delta$ turns out to be a {\em predimension function} in the sense of Section
\ref{pregextsec}; there we explain how to derive a pregeometry from any predimension.
This yields a dimension function $d_{M}$ on each $\{R\}$-structure $M$.

The crucial steps -- rather informally -- are given below and summarise the approach of \cite{jbg}. In this paper, Poizat
divides the construction into two distinct subsequent steps:
\begin{itemize}
\punto{Phase One}Define the class $\Kl{}$ of all finite $\{R\}$-structures with non-negative predimension.
Give a notion of {\em strong} %or {\em self-sufficient}
extensions $A\geqslant B$ in terms of $\delta$ and prove $\Kl{}$ has the properties %(HP), (JEP) and (AP)
of {\em Hereditarity, Joint Embedding} and {\em Amalgamation}
described in Section \ref{fraisse}, with respect to $\leqslant$.
%\end{itemize}

The Fra\"iss\'e limit $K$ of $(\Kl{},\leqslant)$ obtained is $\omega$-saturated and $\omega$-stable of Morley rank $\omega$ and is
ultrahomogeneous with respect to $\leqslant$.
Types of elements over a set $B$ are discerned in base of their dimension $d_{K}$: points
which are dependent over $B$ have all finite (unbounded) Morley rank, while transcendent points have all the same type and
rank $\omega$.
The (forking) geometry of the generic type is the $d_{K}$-pregeometry,
this is not locally modular.
No group diagram is allowed by dimension arguments.

If we restrict the class $\Kl{}$ by changing the initial lower bound of $\delta$ to a fixed positive integer $k$, one obtains a Fra\"iss\'e
limit with a $k$-transitive, non $k-1$-transitive automorphism group.

\punto{Phase Two} A proper subclass $\Kl{}^{\mu}$ of $\Kl{}$ is defined, for which an $\N$-valued function $\mu$
bounds the length of realisations of a family of distinguished {\em minimal pre-algebraic} extensions. With a more difficult proof,
the amalgamation property is true of $\Kl{}^{\mu}$ as well.
The theory $T^{\mu}$ of the Fra\"iss\'e limit  $K^{\mu}$ of $(\Kl{}^{\mu},\leqslant)$ is strongly minimal.

This second phase is referred to in the literature as the {\em collapse}, because the
finite-rank pre-algebraic types in Phase one, are collapsed to algebraic ones, while as a consequence,
the infinite rank type is forced to assume Morley rank $1$. The strongly
minimal geometry on $K^{\mu}$ {\em coincides} with the $d$-pregeometry of $K$ above.
\footnote{In his PhD thesis \cite{fer}, Marco Ferreira proves that the geometries of the collapsed structures are isomorphic to the geometry of the regular type in the uncollapsed construction.}
\end{itemize}

\smallskip
In the original paper \cite{hruabi}, this bipartite analysis is not present and the amalgamation
is carried out directly in the collapsed case. Hrushovski proves the non-interpretability of an infinite group in $T^{\mu}$
as a consequence of {\em flatness}, a property attributed to the geometry of $K^{\mu}$. On the other hand Pillay shows in \cite{pilcm},
that $CM$-{\em trivial} structures do not allow the interpretation of an infinite field:
in \cite{hruabi} it is also proved that that the collapsed structure is $CM$-trivial and that flatness implies $CM$-triviality.

\smallskip
F.{\ }Wagner in \cite{wag}, provides
an axiomatic approach to the above constructions which replaces an explicit predimension argument.
\crule
%\xsubsection{Baudisch Group and the Red Collapse Frame}
\bigskip
In \cite{bad} Baudisch starts from a predimension $\delta$ which is very much alike \pref{urdelta}: it computes
the gap between the number of {\em generators} and {\em relators} of a suitably {\em linearised} presentation of groups.

In the perspective of Zilber's trichotomy, he obtains a pure uncountably categorical group of Morley rank $2$ with no infinite field
interpretable: the associated pregeometry is not locally modular -- because the group obtained is connected and non-abelian
(cfr.{\,}\cite{hp}) -- and its theory is shown to be $CM$-trivial.

The following result indicates which classes of groups may allow such feature.
\begin{fact*}[{\cite[Theorem 2.1]{bad}}]
Assume a connected group $G$ of finite Morley rank does not interpret an infinite field. Then {\em either}
a definable section of $G$ contradicts the Cherlin-Zilber algebraicity Conjecture\footnote{
Infinite simple groups of finite Morley rank are conjectured to be algebraic groups over
an algebraically closed field.}, {\em or}
%its connected component $G^{o}$
$G$ is nilpotent.

In the last case $G$ is the central product of a definable divisible abelian subgroup $A$
and a definable nilpotent subgroup $B$ of $G$ of bounded exponent.
\end{fact*}

To eventually place ourselves on the ``bright side'' of Cherlin-Zilber Conjecture,
the objects considered in \cite{bad} are $2$-nilpotent groups of exponent a fixed prime $p$ bigger than $2$.
Such groups can be reconstructed from the pair of $\Fp$-vector spaces $(G_{ab},G^{\prime})$ -- the sections
of the lower central series -- by means
of the %bilinear map $\map{[\,\,,\,]}{G_{ab}\times G_{ab}}{G^{\prime}}$. 
linear map $\map{c_{G}}{\exs G_{ab}%\wedge G_{ab}
}{G^{\prime}}$, induced by the group commutator in $G$ on the exterior square algebra of $G_{ab}$.
This draws our attention to the pair $(G_{ab},\ker(c_{G}))$: step-$2$ nilpotency
yields a 1-1 correspondence of these groups with the
%The Fra\"isse procedure are in fact conducted along embeddings of
structures $(M,N(M))$, where $M$ is a $\Fp$-vector space and $N(M)$ is a subspace of $\exs M$. %(cfr.\,Remark \ref{baucond}).

In case of a finitely generated $M$, one considers
\begin{labeq}{deltabau}
\delta(M)=\dfp(M)-\dfp(N(M))
\end{labeq}
The Hrushovski amalgamation program described above is carried out in \cite{bad} with this $\delta$ directly for the Collapsed case,
once a suitable native function $\mu$ is implicitly given. 

\smallskip
In Chapter \ref{due} we recast all the steps leading to (Phase One) of the Hrushovski-Baudisch construction in terms
of nilpotent Lie algebras over $\Fp$.

\crule
In Section \ref{nilgral} we present a well-known uniform method to associate a group
with a Lie ring, this uses the sections of the lower central series and the group commutator.
As a consequence, this procedure becomes particularly effective when dealing with nilpotent groups.
%of prime exponent $p$.
If we denote by $\ngb{c}{p}$ the variety %In this case in fact the above correspondence supplies 
of $c$-nilpotent and exponent $p$ groups, we isolate a class of $c$-nilpotent graded Lie algebras $\nla{c}$ over the field $\Fp$
in order to obtain a {\em grading} functor $\gr$ of $\ngb{c}{p}$ into $\nla{c}$, which is surjective
at the level of objects.

The literature about this subject is founded on the work of Lazard, Magnus \cite{laz,mag,mag37} and
Witt's \cite{witt}. In a {\em torsion-free} context this phenomenon is
also called {\em Mal'cev Correspondence}: it establishes an equivalence between the categories of
torsion-free divisible nilpotent groups and nilpotent Lie $\Q$-algebras (see \cite[\S6]{bah}). 

We give two different methods to associate a given Lie algebra $L$ of $\nla{c}$, a group $G$ of $\ngb{c}{p}$ with $\gr(G)=L$:
a group theoretical one, which employes a torsion version of the relationship between free groups and free Lie
rings (this is Witt's {\em Treue Darstellung}) and
a more analytical procedure, which uses the Baker-Hausdorff formula.
%(Theorem \ref{faikaha} and Corollary \ref{co:grupphausdorff}).
This last approach, although less transparent for higher classes $c$, has the advantage of establishing a multiplicative group
structure $(L,\circ)$ directly on the Lie algebra domain $L$.
This group law will be in fact first-order definable in terms of the ring signature.

\smallskip
The additional requirement $G^{\prime}=Z(G)$ for the groups $G$ considered in \cite{bad},
is discussed in Remark \ref{baucond}. This property, which is preserved by the algebra-group correspondence,
will be obtained for $\nla{2}$-algebras as a consequence of the positive lower bound chosen for the predimension.

\smallskip
In Section \ref{schur} we are concerned with an existing notion of group theoretical {\em deficiency}, which computes the difference
between the generators and the relators of a finitely presented group $G$. The second integral homology group of $G$ is involved
in such a measurement.
More precisely, the deficiency of $G$ is always bounded from above, by the difference between the $\Z$-rank of $G_{\it ab}$
and the minimal number of generators for $H_{2}(G,\Z)$.
Following Stammbach and Stallings we derive the correspondent notion of deficiency
for groups in the variety $\ngb{c}{p}$ and homology will be taken with coefficients over $\Fp$.

If we consider a presentation $R\to F\to G$, the so called {\em Hopf formula} returns $H_{2}(G)$ as
the quotient $R\cap F^{\prime}/[R,F]$. This term filters in fact the {\em essential} relators in $R$,
those which actually cause the deficiency to drop.

This filter is basically the same adopted in Chapter \ref{tre} for $\nla{c}$-algebras in order to obtain new kinds of
presentations.
Despite the strong similarity between the above notions and the relators space we constructed,
we encountered this group-homological interpretation only in a very late phase of this work.
We decided to include this section as a sort of {\em a posteriori} motivation.

\bigskip
In the first section of Chapter \ref{due}, we start by adapting the deficiency predimension \pref{deltabau} to finite objects of $\nla{2}$.

Any $M=M_{1}\oplus M_{2}$ in $\nla{2}$, is given by a presentation $R\to\fla{2}{M_{1}}\to M$ from the free nil-2 Lie algebra
$\fla{2}{M_{1}}$ over $M_{1}$.
For a subspace $A$ of $M_{1}$, the integer $\delta(A)$ (or $\delta_{2}(A)$ to distinguish from other nilpotency classes) will be defined as
$\dfp(A)-\dfp(R(A))$. The
relators ideal $R(A)\inn\fla{2}{A}$ depends by the ambient relators $R$ and the subspace $A$.

This function is proved to be a predimension {\em over} the $\Fp$-linear closure, as defined in Section \ref{qdim}.
As a consequence, $\delta$ gives rise to a pregeometry on the vector space $M_{1}$ whose closure operator extends the
linear span. We show directly that this pregeometry is actually a non locally-modular geometry {\em over} $\Fp$.

The notion of {\em self-sufficient} extensions $M\zsu[2]{}N$ of $\nla{2}$-algebras will be given in terms of $\delta$: as
usual $\delta(C)$ cannot drop below $\delta(M)$ on spaces $C$ between $M$ and $N$.

\medskip
Section \ref{amalga2} describes the subclass $\Kl{2}$ of $\nla{2}$, for which
an {\em asymmetric} amalgamation lemma is shown: we define a free amalgam in $\nla{2}$, which preserves a positive lower bound
of the deficiency, provided a kind of one-point algebraic extensions are suitably avoided.
Compared to the correspondent statements in \cite{bad}, the proofs here are overall simplified, left aside some
technicalities (Lemmas \ref{basE} and \ref{reldue})%\mn{\bf maybe we can avoid these, cfr. Lemma \ref{amalsigma2}}
, which we have to borrow with minor changes from the original text.

As part of this section we find the treatment of {\em minimal} strong extensions, these will be
fundamental for the rank computations in the uncollapsed theory. To this end we prove that chains of minimal extensions
commute with free amalgamation.

\medskip
Asymmetric amalgamation yields a first-order axiom system $T^{2}$
for the countable Fra\"iss\'e limit of $\Kl{2}$. As it is meant to happen the $\omega$-saturated models
of $T_{2}$ are exactly the rich structures whose age is $\Kl{2}$. This is Theorem \ref{Cazzuola} of Section \ref{t2axioms},
where we also prove $\omega$-stability of $T_{2}$ and give a description of the algebraic closure in $T_{2}$.

In Section \ref{rango}, we explicitly compute the Morley rank of the %theory $T_{2}$, the Morley rank of the
countable rich model $\mathbb{M}$, which is -- as expected\footnote{It is sort of
by chance that this value coincides with the rank of the uncollapsed {\em black field} of Poizat. In that case this factor
is artificially obtained by the shape of the predimension, while in ours it closely reflects the structural nil-$2$ constraint.}
 -- $\omega\cdot2$.
 
The reason for this number comes from the $\nla{2}$-grading $\mathbb{M}=\mathbb{M}_{1}\oplus\mathbb{M}_{2}$ and the
locally-free behaviour imposed by the axioms.
As our predimension takes its entries among the finite parts of $\mathbb{M}_{1}$, we first obtain Morley rank $\omega$ for this
set, by a geometric type analysis \`a la John B. Goode (cfr.\,Phase One above).
On the other hand, to require a positive deficiency, forces the
homogeneous subspaces $\mathbb{M}_{1}$ and $\mathbb{M}_{2}$ to be definably $\Fp$-isomorphic. This doubles the rank.
The same happens in the collapsed case and explains the rank $2$, there in fact the corresponding set
$\mathbb{M}_{1}$ is strongly minimal.

By applying the aforementioned correspondance
we reconstruct a nil-2 group $\mathbb{G}$ which has Morley rank $\omega\cdot2$. Indeed the whole local construction (amalgamation,
self-sufficient embeddings, richness, etc.)
can be traced back at the level of groups; cfr.~Remark \ref{baucond}.

\smallskip
A complete description of forking in $T^{2}$ follows. This is done in Section \ref{forking} by exhibiting a suitable ternary
{\em independence relation} among sets of the monster model $\mathbb{M}$ which satisfies the
axioms of forking in stable theories. This notion of independence reflects both the {\em geometric} information of
the predimension and the {\em structural} condition imposed by free amalgamation.

\smallskip
In the last section of Chapter \ref{due}, we propose a notion of {\em weak canonical base} for
types of self-sufficient tuples over models.
This is compared with the properties of {\em weak elimination of imaginaries} and $CM$-triviality
for the {\em uncollapsed} theory. On this purpose one may also check the notion
of {\em relative} $CM$-triviality proposed in \cite{cmtr}.

\bigskip
In the third Chapter we study a possible construction of deficiency predimensions in the case of nilpotent
Lie algebras from $\nla{c}$ of class $c$ greater than $2$.

The guiding principle here is an inductive approach\label{indunil} over the nilpotency class, suggested by the graded shape of
a (saturated-homogeneous say) object $\mathbb{M}=\mathbb{M}_{1}\oplus\dots\oplus\mathbb{M}_{c}$ of $\nla{c}$.

This corresponds to a presentation $R\to\fla{c}{\mathbb{M}_{1}}\to \mathbb{M}$ from the free Lie nil-$c$ algebra
$\fla{c}{\mathbb{M}_{1}}$, where the homogeneous ideal $R$ equals $R_{2}+\dots+R_{c}$ (cfr.\,Section \ref{nilgral}).
On the other hand, denote by $\mathbb{M}_{*}$ the {\em truncation}
to $\nla{c-1}$, that is $\mathbb{M}_{*}=\mathbb{M}/\mathbb{M}_{c}\simeq\mathbb{M}_{1}\oplus\dots\oplus\mathbb{M}_{c-1}$.

Now assume we have a notion of deficiency $\delta_{c-1}$
which locally measures the gap among linear dimensions in $\mathbb{M}_{1}$
and the numbers of independent relators from $\mathbb{M}_{*}$ in all possible weights $<c$.
Suppose further, such a function behaves like a predimension and yields a dimension function $d_{c-1}$ on $\mathbb{M}_{1}$.
Then we ideally define $\delta_{c}(A)$ for $A\inn \mathbb{M}_{1}$, as the difference between $d_{c-1}(A)$
and the linear dimension of a new {\em relators space} $\rc(A)$.

$\rc(A)$ is able to isolate elements of $R_{c}$,
from Lie products $[\rho,x_{1},\dots,x_{c-k}]\in R_{c}$, involving relations $\rho\in R_{k}$ of a lesser weight $k<c$.
The definition of $\rc(M)$ is found in Section \ref{maxrels}.

\smallskip
For a fixed prime $p$ and $c$ with\footnote{
The constraint $c<p$ lays in the nature of the Hausdorff series development described in Section \ref{nilgral}.
%the rational coefficient of the homogeneous term of degree $i$ is in fact divided by primes not greater than $i$. 
} $c<p$, in its entirety, this recursive program should produce a sequence of pregeometries
$(\mathbb{M}_{1},\cl_{i})_{i\leq c}$, each one extending the previous ($\cl_{i}\inn\cl_{i+1}$) and all insisting upon the 
same domain set $\mathbb{M}_{1}$. Here $\cl_{1}$ is the $\Fp$-linear closure and $\cl_{2}$ is the pregeometry obtained
from the deficiency $\delta_{2}$, associated to $\nla{2}$-algebras.

This aspect motivates the study of extensions among pregeometries and the notion of predimentions {\em over} a given pregeometry
given in Section \ref{pregextsec}.

\medskip
The above operator $\rc$ relies on a {\em free-lift} functor $\map{\frl}{\nla{c-1}}{\nla{c}}$
defined in Section \ref{freelift}. This is such that $\frl(M)_{*}=M$ for all $M$ in $\nla{c-1}$ and obey the following universal property:
for any other $N\in\nla{c}$ with $N_{*}\simeq_{\nla{c-1}}M$, $\frl(M)$ maps uniquely onto $N$. In other words $\frl(M)$ is the freest possible
object in $\nla{c}$ to have a truncation in $\nla{c-1}$ which is $M$. We prove in fact that $\frl$ is left-adjoint to $\map{_{*}}{\nla{c}}
{\nla{c-1}}$ in Proposition \ref{morphifreelift}.

Composed the other way around, the universal property of $\frl$ yields, for any algebra $M$ of $\nla{c}$, the desired {\em shifted presentation} $\rc(M)\to\frl(M_{*})\to M$. The kernel $\rc(M)$ has the properties mentioned above.

\medskip
This formal strategy is applied, in Section \ref{preditre}, in the step from $\nla{2}$ to $\nla{3}$. Already in this {\em induction
basis}, major difficulties are encountered in the reproduction of both the Fra\"iss\'e procedure and the pregeometric approach.

We define a first deficiency for finitely generated $\nla{3}$-algebras $A$, as the difference between $\delta_{2}(A_{*})$
-- the $\nla{2}$-predimension defined in Chapter \ref{due} -- and the $\Fp$-dimension of the space $\rt(A)$ given above.

So defined, this function is unreliable to control deficiencies within a fixed ambient structure $M$ of $\nla{3}$. That is
is because $\rt(A)$ is not in general contained into $\rt(B)$ for extensions $A\inn B$ inside $M$.

\smallskip
This is due to a structural issue intrinsic to the free-lift functor:
for extensions $M\inn N$ of $\nla{2}$-structures, the lifted algebra $\frl(M)$ {\em does not} always embed into $\frl(N)$.
In Section \ref{embiss} we prove however that if $M$ is a {\em self-sufficient} $\nla{2}$-subalgebra of $N$, then
we have a corresponding extension of the lifted $\nla{3}$-algebras, i.e.\,$\frl(M)\inn\frl(N)$.
This crucial result, which influences the whole subsequent construction,
is proved by using the so called {\em Hall's bases} (Definition \ref{basicommutators}) of {\em basic} commutators
for free Lie algebras. In fact a similar approach to Hall's {\em collecting process} in \cite{mhalll} is employed.

\smallskip
Now fixed an $\nla{3}$-algebra $M$, we define a more adaptive deficiency $\ded^{M}(A)$, which reads subspaces $A$ of
$M_{1}$. This is built in terms of the {\em dimension function} $d_{2}^{M}$ -- induced by the pregeometry from $M_{*}$ -- and a
suitable {\em monotone} operator $\rt_{M}(A)$, which returns subspaces of $\rt(M)$ and depends on $\frl(\gen{A})$. %(\gena{A}{M})$.

As a consequence of the above embedding result, the functions $\delta_{3}$ and $\ded^{M}$ do agree
on $\delta_{2}$-strong subalgebras $A$ of $M$.

This behaviour also suggests the following definition of {\em strong} $\nla{3}$-extensions: to write $A\dsu{} M$ and
say $A$ is self-sufficient in $M\in\nla{3}$, we require in fact that the truncated
structures are
self-sufficient with respect to $\delta_{2}$ ($A_{*}\zsu[2]{}M_{*}$) and that the auxiliary deficiency $\ded^{M}$ assumes values
bigger than $\delta_{3}(A)$ on all $C$ between $A$ and $M_{1}$.

\smallskip
Consequently, we exhibits in Section \ref{amalgatre} a strong amalgam of $\nla{3}$-algebras.
This is obtained as follows: start with a strong configuration like $A\dso{}B\dsu{}C$, then
take the truncated preamalgam $A_{*}\zso[2]{}B_{*}\zsu[2]{}C_{*}$ and obtain, with the results in Chapter \ref{due},
a free $\nla{2}$-amalgam $D_{*}$ of $A_{*}$ and $B_{*}$ over $C_{*}$.

This yields strong $\nla{2}$-inclusions $A_{*}\zsu[2]{}D_{*}\zso[2]{}C_{*}$. Now take the free-lift $\frl(D_{*})$ and by virtue of the aforementioned fact, obtain the embeddings $\frl(A_{*})\inn \frl(D_{*}) \nni \frl(C_{*})$.

Since $A$ and $C$ are isomorphic to the quotients $\frl(A)/\rt(A)$ and $\frl(C)/\rt(C)$, the $\nla{3}$-algebra
$D\defeq\frl(D_{*})/(\rt(A)+\rt(C))$, amalgamates $A$ and $C$ over $B$ and we show $A\dsu{} D\dso{}C$ in
Lemma \ref{amalgatrestrong}.

With a modified procedure we were actually able to prove the {\em asymmetric} version of the above result: from
$A\nni B\dsu{} C$, we obtain $A\dsu{} D\nni C$.
As shown in Chapter \ref{due} in fact, asymmetric amalgamation is indispensable
to approximate richness in a possible axiomatisation of the Fra\"iss\'e limit.


\medskip
A further remark, independent of previous issues, settle at this point the following -- and more critical -- problem:
{\sl to decide whether $\rt_{M}(A)\cap\rt_{M}(B)$ equals $\rt_{M}(A\cap B)$, for given subspaces $A$ and $B$ of $M_{1}$.}
%given subspaces $A$ and $B$ of $M_{1}$, $\rt_{M}(A)\cap\rt_{M}(B)$ does not equal $\rt_{M}(A\cap B)$ in general.

The answer is negative in general and two main obstructions follow thereafter:
\begin{itemize}
\item[-]we prove with examples, that $\ded^{M}$ (and $\delta_{3}$) is not in general submodular.
\item[-]We cannot prove the strong $\nla{3}$-embedding $\dsu{}$is transitive, nor find a transitive notion related to
$\dsu{}$\footnote{
there is a standard way to {\em force} transitivity via a local ``cut lemma'' (cfr. Lemma \ref{2cut}) definition of strongness: in our case
one should define \lqq $A$ is strong in $M$\rqq if for any finite part $U$ of $M_{1}$, $\delta_{3}(A_{1}\cap U)\leq\delta_{3}(U)$.
This definition however does not comply with the amalgamation in $\nla{3}$ described in Lemma \ref{amalgatrestrong}.
}.
\end{itemize}
The first makes void the proof-strategies adopted in Chapter \ref{due}. Submodularity %\pref{summo}
is in fact on one hand the key property to turn a deficiency-like function into a predimension,
on the other, it ensures that free amalgamation preserves the same lower bound
for the deficiency, of the amalgamated structures. 

\smallskip
The efforts of Section \ref{classetre}
goes in the direction of finding {\em local} conditions to force a modular behaviour of $\rt_{M}$ and hence be able to
use submodularity of $\ded$ {\sl just where we need it}.

This is strongly connected to the relationship between $\delta_{3}$ and $\ded$. In this
section we prove indeed that they are uniformly comparable, namely in the direction $\ded^{M}(A)\leq\delta_{3}(A)$ for
any finite algebra $A$ of $\nla{3}$.

In accordance to this and the above amalgamation process,
we define a class $\Kl{3}$ of $\nla{3}$-algebras $M$ with $M_{*}$ in $\Kl{2}$ for which
$\delta_{3}$ is non-negative on the finite subalgebras of $M$. By the above, we can use
indifferently $\delta_{3}$ or $\ded^{M}$ to test whether $M$ is in $\Kl{3}$.

We indicate $\Kl{3}$ as a possible candidate to represent the age of the desired rich $\nla{3}$-algebra, although we couldn't
prove the amalgamation property for $\Kl{3}$. 

%\smallskip
%Despite a very long series of attempts, the whole Fra\"iss\'e construction was not accomplished.
%The long time employed, eventually with very slow and rare improvements, forced us to prematurely put an end to this work.
\crule
The exclusive treatment of the uncollapsed case in this work is also motivated by a later project of Baudisch',
{\em The Additive Collapse} (\cite{addcoll}).
Here an $\omega$-stable theory $T$ is considered, which expands the theory of vector spaces over the finite field $\Fp$.
A pregeometry is assigned on the models of $T$ and a notion of {\em strong embedding}
between subspaces is given, which both influence the elementary type of the saturated monster $\K$ of
$T$. Further properties are required of $T$, which capture the essential features of the {\em uncollapsed} infinite rank versions of the known amalgamation examples. 

After {\em prealgebraic codes} and the aforementioned bound-function $\mu$ are chosen,
the collapsed structure $\K^{\mu}$ of finite rank, is constructed directly {\em inside} $\K$.

This new procedure is meant to unify under a common frame, the Red fields \cite{rf}, the new uncountably categorical group and
the fusion over a vector space \cite{fu}.

Should suitable stable rich $\nla{c}$-algebras ($c>2$) be constructed with the methods described in the present work,
then the additive collapse process would give finite rank nilpotent Lie algebras or groups, with underlying Hrushovski
geometries.