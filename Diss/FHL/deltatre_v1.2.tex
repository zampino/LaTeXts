\section{A Construction of $\delta_{3}$}
Remember that for each $n<\omega$, any $A\in\nla{n}$ is of the form
$A=\gena{A_{1}}{A}=\oplus_{i\leq n}A_{i}$.
Now for all $M\in\nla{n}$ we define with abuse $\tr{2}M=M\quot M^{3}\simeq{M}_{1}\oplus{M}_{2}$.
The monolinear part $M_{1}=({\tr{2}M})_{1}$ can be once more endowed with a quasidimension function
$\map{\delta_{2}}{\fpe{M}}{\Z}$
considering relations from $N^{2}(\tr{2}M)\inn_{\sm{id}}{M}_{2}$. As usual
$\fpe{M}$ denotes the collection of all finite dimensional subspaces of $M_{1}$.
%, coincides with $\fpe{\tr{2}M}$.

Again the notion of \emph{strong spaces} with respect to $\delta _{2}$ is well defined.
The dimension function $d_{2}$, obtained from $\delta_{2}$, gives rise to the pr\"{a}geometry
$(M_{1},\cl_{2})$, where $\cl_{2}=\cl_{d_{2}}$.

%For any $M\in\nla{n}$, we define
%\tojoris{$\fsz{M}$ as the family of all the vector $\Fp$-subspaces $H_{1}$ of $M_{1}$
%for which $d_{1}(H_{1})<\omega$ and such that $H_{1}\zsu M_{1}$.}

\medskip
%Let $T_{2}$ be the elementary theory of the limit structure $\mathfrak{C}_{2}$ of the class $\mathcal{K}_{2}$.
%Let $\fhlim{3}$ be the class of structures $M$ of $\nla{3}$ such that $\tr{2}M\models T_{2}$.
%\smallskip

For each algebra $M$ in $\nla{3}$ let $\map{\pi}{\fr{3}\tr{2}M}{M}$ be the canonical map as in \pref{communo} above,
$\pi$ is defined by
$\bar w\;(\mathit{mod}\:\jei{\tr{2}M})\longmapsto\bar w \;({mod}\:R)$. Where $M=\fla{3}{X_{1}}\quot R$.
Then define $N^{3}(M)=\ker\pi$, we have $N^{3}(M)\inn(\fr{3}\tr{2}M)_{3}$.

Let now $H=\gena{H_{1}}{M}$ be a subalgebra of $M\in\nla{3}$ and assume
$H_{1}\zsu M_{1}$ as defined above.
%$\tr{2}H$ is a selfsufficient substructure in $\tr{2}M$.
%we will confuse equality with isomorphismand write 
we have $\fr{3}\tr{2}H\simeq\gena{H_{1}}{\fr{3}\tr{2}M}$.

For $H\inn M\in\nla{3}$ such that $H=\gena{H_{1}}{M}$ and
$H_{1}\zsu M_{1}$, we define
$$N^{3}(H_{1})=N^{3}(M)\cap\gena{H_{1}}{\fr{3}\tr{2}M}.$$%=N^{3}(M)\cap\fr{3}\tr{2}H

Because $H_{1}$ is selfsufficient in $M_{1}$ and $\tr{2}\gena{H_{1}}{M}=\gena{H_{1}}{\tr{2}M}$,
on account of Lemma \ref{bellemma},
$N^{3}(H_{1})$ will be isomorphic with the kernel of the
canonical projection $\map{\pi^{H}}{\fr{3}\tr{2}H}{H}$. If no confusion arises
this ideal will be also denoted $N^{3}(H)$.
%in this situation our definition does not depend on $M$, but only on the linear part $H_{1}$.
%\medskip
%is a $\cl_{2}$-closed space.}
%If $H_{1}\in\fpz{M}$ then it is also $d_{2}H_{1}<\omega$.

\smallskip
For any $M$ in $\nla{3}$ we define a new map
$$\map{\delta_{3}}{\fpe{M}}{\Z}$$
by means of
\begin{labeq}{deltatre}
\delta_{3}H_{1}=d_{2}H_{1}-\dim_{\Fp}(N^{3}(\ssc_{2}H_{1}))\quad\forall H_{1}\in\fpe{M}
\end{labeq}
%Here of course $H=\gena{H_{1}}{M}$, %when this is clear from the context
%as this will be almost everywhere the case
If $H=\gena{H_{1}}{M}$ we will often write simply $\delta_{3}H$ instead of $\delta_{3}H_{1}$.

%The definition of $\delta_{3}$ does make sense because a $\clz$-space is a $\delta_{2}$-strong space, as we proved in \ref{Kapzwei}, moreover
As the structures in argument of $\delta_{3}$ are of finite character, $\dfp(N^{3}(H))$ is an integer.

The definition of $\delta_{3}$ depends only on $H_{1}$. This means that if $M\inn L\in\nla{3}$ and $M_{1}\zsu L_{1}$ and we define respectively 
$\delta_{3}^{M}$ and $\delta_{3}^{L}$ as above,
%its values on $\fpe{M}\inn\fpe{L}$ will coincide with those
%of $\map{\delta_{3}}{\fpe{M}}{\Z}$
then ${\delta_{3}^{L}}_{|_{\fpe{M}}}\!=\delta_{3}^{M}$
because $\fr{3}\tr{2}M\simeq\gena{M_{1}}{\fr{3}\tr{2}L}$ and
$N^{3}(M)\simeq\gena{M_{1}}{\fr{3}\tr{2}L}\cap N^{3}(L)$.

Therefore working with strong $\delta_{2}$-extensions there won't be need to specify
each time which $\delta_{3}$ we are referring to.

\medskip
For general $M\in\nla{3}$, $\delta_{3}$ will \emph{not} be a $\cl_{1}$ (nor a $\cl_{2}$) quasidimension function
on $M_{1}$.
We have however $\delta_{3}(\ssc_{2}(H_{1},m))\leq\delta_{3}(H)+1.$

Subadditivity holds only in special cases, as the next Lemma shows.
%As done in the $2$-nilpotent case, when $U$ and $V$ are subalgebras of $M\in\nla{3}$,
%with $U+V$ we refer to$\gena{U_{1}+V_{1}}{M}$.
\begin{lem}\label{presubatre}
Let $M\in\nla{3}$ and $U_{1}$, $V_{1}$ are $\delta_{2}$-strong spaces in $M_{1}$ such that
%$\tr{2}U\cap\tr{2}V=\gena{U_{1}\cap V_{1}}{\tr{2}M}$, then
$\gena{U_{1}}{\tr{2}M}\cap\gena{V_{1}}{\tr{2}M}=\gena{U_{1}\cap V_{1}}{\tr{2}M}$.
We have
$$\delta_3\left(\ssc_{2}(U_{1}+V_{1})\right)+\delta_{3}(U_{1}\cap V_{1})\leq\delta_{3}U+\delta_{3}V.$$
\end{lem}
\begin{proof}
%As $\delta_{2}$ is a $\cle$-quasidimension, on one hand we have
%$$\delta_2(U+V)+\delta_{2}(U\cap V)\leq\delta_{2}U+\delta_{2}V.$$
As $d_{2}$ is a dimensionfunction and $\clz$ extends $\cle$, on one hand we have
$$d_{2}(U_{1}+V_{1})=d_{2}(\cle(U_{1}\cup V_{1}))=
d_{2}(U_{1}\cup V_{1})\leq
d_{2}U+d_{2}V-d_{2}(U\cap V).$$
Since $\delta_{3}(U+V)=d_{2}(U+V)-\dfp(N^{3}(U+V))$ we only have to show
$$\dfp(N^{3}(\ssc_{2}(U_{1}+V_{1}))\geq\dfp(N^{3}(U))+\dfp(N^{3}(V))-\dfp(N^{3}(U\cap V)).$$
Let $W=\fr{3}\tr{2}M$ and $N^{3}=N^{3}(M)$, we have
%\begin{multline*}
%N^{3}(\ssc_{2}(U+V))\nni
%N^{3}((U+V)=
$$
N^{3}\cap\gena{\ssc_{2}(U_{1}+V_{1})}{W}\supseteq
%N^{3}\cap\left({\gena{U_{1}+V_{1}}{W}}\right)_{3}\supseteq\\
%\supseteqN^{3}\cap\left({\gena{U_{1}}{W}}_{3}+{\gena{V_{1}}{W}}_{3}\right)\supseteq
N^{3}\cap\left(\gena{U_{1}}{W}\right)+N^{3}\cap\left(\gena{V_{1}}{W}\right).
$$
%\end{multline*}
The previous inclusion, has to be regarded as
between vector spaces in the weight $3$ part of $W$.
Our assumptions and Lemma \pref{bellemmino} now imply
$\gena{U_{1}}{W}\cap\gena{V_{1}}{W}=\gena{U_{1}\cap V_{1}}{W}$, therefore
\begin{multline*}
\dfp(N^{3}(\ssc_{2}(U_{1}+V_{1})))\geq\dfp(N^{3}\cap\gena{U_{1}}{W}+N^{3}\cap\gena{V_{1}}{W})=\\
=\dfp(N^{3}\cap\gena{U_{1}}{W})+\dfp(N^{3}\cap\gena{V_{1}}{W})-\dfp(\gena{U_{1}}{W}\cap\gena{V_{1}}{W}
\cap N^{3})\geq\\
\geq\dfp(N^{3}(U))+\dfp(N^{3}(V))-\dfp(N^{3}(U\cap V)).
\end{multline*}
This concludes the proof.
\end{proof}
%In what follows we make the groundhypothesis that \tojoris{$\fpz{\fhlim{2}}$ is (upward) directed,}
%although it seems not too sound.
%\bigskip
 %Define now $\mathcal{K}_{3}=\{A\in\nla{3}\mid\tr{2}A\in\fpz{\mathfrak{C}_{2}},\,\delta_{3}H\geq0\;\forall H\in
%\fpz{A}\}$. In particular for each $M\in\Ktre$, we have $\tr{2}M\in\Kdue$.
%We observe that $\mathcal{K}_{3}^{\prime}$ and $\mathcal{K}_{3}$ are related by the following
%\begin{lem}
%Assume $M\in\mathcal{K}_{3}^{\prime}$ then $\fpz{M}\inn\mathcal{K}_{3}$.
%\end{lem}
%We use now the properties of the 2-nilpotent limit to prove that 2-spaces in $\fhlim{2}$ have
%nice intersections.
%\begin{lem}
%Assume $U_{1},V_{1}\in\fpz{\fhlim{2}}$
%and $U=\gena{U_{1}}{\fhlim{2}}$, $V=\gena{V_{1}}{\fhlim{2}}$ then $U\cap V=\gena{U_{1}\cap V_{1}}{\fhlim{2}}$.
%\end{lem}
% 
%Consider now $M\in\Ktre$ and $\map{\delta_{3}}{\fpz{M}}{\mathbb{N}}$.
%We want to prove that $\delta_{3}$ is a $\clz$-quasidimension, to do so
%we have to show (qd'1), \dots , (qd'3) are true on such structures $M$.
%So we start proving $\delta_{3}\clz(\vac)=0$. We have that $\clz(\vac)=(0)$ because if $a$ is a nontrivial
%element of $M_{1}$ which lays in $\clz(\vac)\nni\cle(\vac)=(0)$ then we'd have $d_{2}(0)=
%d_{2}(0,a)=d_{2}(a)=1$, which is nonsense as obviously $\delta_{2}(0)=0$.
%%$\tr{2}M\models T_{2}$ and so each space of the form
%%$\gen{a}_{1}$ is $2$-strong for each element $a\in M_{1}$; in particular $d_{2}(0)=\delta_{2}(0)=0$.
%Then by computation $\delta_{3}(0)=0$ and so we get (qd'1).
%\smallskip
%To prove (qd'2) we must find $\delta_{3}\clz(a)\leq1$ for each $a\in M_{1}$.
%In general for an arbitrary subset $X$ of $M_{1}$,
%if $\dim_{2}X=\dim_{2}\clz(X)<\omega$, then $d_{2}X=d_{2}\clz(X)$ as $d_{2}$ and $\dim_{2}$ do agree on finite
%dimensional sets in the sense of the $\clz$-praegeometry.
%In our case $\dim_{2}(a)=1$ so $d_{2}\clz(a)=d_{2}(a)$.
%%As $\tr{2}M\models T_{2}$ it follows that each space of the form $\gen{a}_{1}$ is $2$-strong,
%%for each element
%%$a\in M_{1}$, but this implies $d_{2}(a)=\delta_{2}(a)\leq1$.
%We conclude $\delta_{3}\clz(a)=d_{2}(a)-\dfp(N^{3}(a))=\delta_{2}(a)-\dfp(N^{3}(a))\leq1$.
%\smallskip
%Finally, also (qd'3) holds. Let $U,V\in\fpz{M}$ .
%Since $M\in\Ktre$, on account of the previous lemma, we first have $\tr{2}U\cap\tr{2}V=\gena{U_{1}\cap V_{1}}{\tr{2}M}$.
%We then observe that $\omega>\dim_{2}(U\cup V)=\dim_{2}(\clz(U\cup V))$,
%%a consequence the values of $d_{2}$ and $\dim_{2}$ agrees on both the sets $U\cup V$ and $\clz(U\cup V)$.
%therefore $d_{2}\clz(U\cup V)=d_{2}(U\cup V)=d_{2}\cle(U\cup V)=d_{2}(U+V)$.
%%$d_{2}\clz(U\cup V)=\dim_{2}(\clz(U\cup V))=\dim_{2}(U\cup V)=d_{2}(U\cup V)$. Therefore we have
%On the other hand $N^{3}(U+V)=N^{3}(\cle(U_{1}\cup V_{1}))\inn N^{3}(\clz(U_{1}\cup V_{1}))$.
%At the end we have
%\begin{multline*}
%\delta_{3}\clz(U\cup V)=d_{2}\clz(U_{1}\cup V_{1})-\dfp(N^{3}(\clz(U_{1}\cup V_{1})))\leq\\
%\leq d_{2}(U+V)-\dfp(N^{3}(U+V))=\delta_{3}(U+V).
%\end{multline*}
%Now applying Lemma \pref{presubatre} we get
%$\delta_{3}\cl_{2}(U\cup V)+\delta_{3}(U\cap V)\leq\delta_{3}U+\delta_{3}V$ as we wanted.
%We have indeed proved the following
%\begin{cor}
%$\delta_{3}$ is a $\clz$-quasidimensionfunction, provided $M$ belongs to $\mathcal{K}_{3}$.
%\end{cor}

\begin{dfn}
Let $M\in\nla{3}$ a finite dimensional $H_{1}\inn M_{1}$ is a $\delta_{3}$-strong substructure of $M_{1}$
if
\begin{itemize}
\item[-]$H_{1}\zsu M_{1}$
\item[-]for any $C_{1}\in\fpe{M}$ such that $C_{1}\nni H_{1}$ we have $\delta_{3}H_{1}\leq\delta_{3}C_{1}$.
\end{itemize}
We write ambiguously $H_{1}\dsu M_{1}$ or $H\dsu M$, provided $H=\gena{H_{1}}{M}$.
\end{dfn}
%and $\delta_{3}$-strong embedding.

We define minimal $\delta_{3}$-extensions.
N\"{a}mlich, $B\dsu A$ is a minimal extension if for each $H_{1}\zsu A_{1}$ such that
$A_{1}\supsetneq H_{1}\supsetneq B_{1}$, we have $\delta_{3}A<\delta_{3}H$.

We write $\delta_{3}(a\quot B)$ to denote $\delta_{3}\gen{B_{1},a}-\delta_{3}B$ where $a,B_{1}\inn M_{1}$.

\medskip
In what follows we amalgamate objects of a subclass $\Ktre$ of $\nla{3}$ with respect to $\delta_{3}$
strong embeddings.
Let $$\Ktre=\left\{A=\gen{A_{1}}\in\nla{3}\mid\tr{2}A\in\Kdue,\;(\triv)\dsu A,\;A\sat(\iota)_{3}\right\}$$
with
%Objects of $\Ktre$ will be algebras $A=\gen{A_{1}}\in\nla{3}$ such that
%$A_{1}$ can be identified with a strong, finite dimensional subspace of the limit $\fhlim{2}$, moreover we require
%$\tr{2}A\simeq\gena{A_{1}}{\fhlim{2}}$ with $A_{1}\zsu{\fhlim{2}}_{1}$ and $(\triv)\dsu A$ and $A\sat(\iota)_{3}$.
$$(\iota)_{3}\colon\quad(\forall x,\,P_{1}(x))(\forall y,\,P_{2}(y))([x,y]\neq 0).$$
As $\delta_{3}(\triv)=0$, with $(\triv)\dsu A$ we mean that $\delta_{3}H_{1}\geq0$ for each $H_{1}\in\fpe{A}$.

%Note that for each $M\in\Ktre$, we have $\tr{2}M\in\Kdue$.
Remember that $\Kdue$ is the class of $2$-nilpotent finitely generated
Lie Algebras with a non negative $\delta_{2}$
on the subspaces and which satisfy
$(\iota)_{2}$ where
$$(\iota)_{2}\colon\quad(\forall x,\,P_{1}(x))(\forall y,\,P_{1}(y))([x,y]= 0)\rightarrow
(\textsl{$x$ lin.{}depends on $y$}).$$

\begin{lem}
Let $A,B,C$ be $\Ktre$-structures, such that $A\dso B\dsu C$, where $A\quot B$ is a minimal extension,
then there exists $D\in\Ktre$ da{\ss} alles amalgamiert.
\end{lem}
\begin{proof}
%From the axioms for $\Ktre$ we have $B=\gena{B_{1}}{A}$, in particular $\tr{2}B=\gena{B_{1}}{\tr{2}A}$.
%So we can assume that $B_{1}\zsu A_{1}\zsu{\fhlim{2}}_{1}$ and $\tr{2}B\simeq\gena{B_{1}}{\fhlim{2}}$.
%The same is done with $C$, so $B=\gena{B_{1}}{\fhlim{2}}$ for some $B_{1}\zsu C_{1}$.
We start describing a \emph{na\"{i}ve amalgamation} procedure which first freely amalgamates at level 2, and then raises the structure to a free amalgam in weight 3.
We call it {\bf (NAM)} and it will be recalled in the rest of the proof when suitable ground hypotheses are attained.

If we denote with $A^{\prime}$, $B^{\prime}$ and $C^{\prime}$, respectively $\tr{2}A$, $\tr{2}B$ and
$\tr{2}C$ then
%$A^{\prime}$ and $C^{\prime}$ two algebras
%in $\Kdue$ isomorphic to $\gena{A_{1}}{\fhlim{2}}$ and $\gena{C_{1}}{\fhlim{2}}$ respectively.
%We can say 
we have $A^{\prime}\zso B^{\prime}\zsu C^{\prime}$.
We build the free amalgam $D^{\prime}=\fram{A^{\prime}}{B^{\prime}}{C^{\prime}}$ of $A^{\prime}$ and $C^{\prime}$ over $B^{\prime}$ following the construction in Chapter $2$.
\begin{description}
\item[(NAM)]Assume that $D^{\prime}$ is in $\Kdue$.
%Now, because $\fhlim{2}\zso B^{\prime}\zsu D^{\prime}\in\Kdue$ and
%$\fhlim{2}$ is \emph{reich} with respect to strong embeddings,
%we can strong embed $D^{\prime}$ in $\fhlim{2}$ over $B^{\prime}$.
%With abusive notation we name $\abu$, $\cbu$ and $\dbu$, the monolinear parts in the embedded
%amalgam $D^{\prime}$, no new names for the images in $\fhlim{2}$.
%So far we have $B_{1}$ strong inside both $\abu$ and $\cbu$, these two strong in $\dbu$ and this last
%strong in ${\fhlim{2}}_{1}$ (here strong is strong substructure).
As $B$ coincides both with $\gena{B_{1}}{A}$ and $\gena{B_{1}}{C}$, then $\gena{B_{1}}{A^{\prime}}=B^{\prime}=
\gena{B_{1}}{C^{\prime}}$;
therefore this setting implies $B_{1}=\abu\cap\cbu$ in $D^{\prime}$ and
\begin{labeq}{inter}
B^{\prime}=\gena{B_{1}}{D^{\prime}}=\gena{\abu}{D^{\prime}}\cap\gena{\cbu}{D^{\prime}}.
\end{labeq}
Now because both $A^{\prime}$ and $C^{\prime}$ are selfsufficient in $D^{\prime}$,
on account of Lemma \pref{bellemma},
there exist two embeddings $j_{A}$ and $j_{C}$ of $\fr{3}\tr{2}A$ and $\fr{3}\tr{2}C$
into $\fr{3}D^{\prime}$ their images being $\gena{\abu}{\fr{3}D^{\prime}}$ and $\gena{\cbu}
{\fr{3}D^{\prime}}$ respectively. From \pref{inter} and lemma \ref{bellemmino} instead, we get
\begin{labeq}{freeint}
\gena{B_{1}}{\fr{3}D^{\prime}}=\gena{\abu}{\fr{3}D^{\prime}}\cap\gena{\cbu}{\fr{3}D^{\prime}}
\end{labeq}
here $B_{1}$, $\abu$ and $\cbu$ are identified with its images modulo $j_{A}$ and $j_{C}$. 

We set $N^{3}(A)^{\bullet}=j_{A}(N^{3}(A))$ and $N^{3}(C)^{\bullet}=j_{C}(N^{3}(C))$.
And we build
$$D=\fr{3}D^{\prime}\quot N^{3}(A)^{\bullet}+N^{3}(C)^{\bullet}.$$
As $N^{3}(A)^{\bullet}+N^{3}(C)^{\bullet}$ is an ideal consisting only of weight $3$ elements,
$\fr{3}\tr{2}D=\fr{3}(\tr{2}\fr{3}D^{\prime})%\simeq
=\fr{3}D^{\prime}$, hence we obtain $N^{3}(D)=N^{3}(A)^{\bullet}+N^{3}(C)^{\bullet}$.

From \pref{freeint} we have that
$$N^{3}(B)^{\bullet}:=j_{A}(N^{3}(B))=N^{3}(A)^{\bullet}\cap N^{3}(C)^{\bullet}=j_{C}(N^{3}(B))$$
and the following compatibility equations
\begin{labeq}{compa}
N^{3}(A)^{\bullet}=N^{3}(D)\cap\gena{\abu}{\fr{3}D^{\prime}}
\end{labeq}and
$$
N^{3}(C)^{\bullet}=N^{3}(D)\cap\gena{\cbu}{\fr{3}D^{\prime}}.
$$

So far, are $A$ and $C$ embeddable in $D$ via Lie Algebra monomorphisms,
just regard $A\simeq\fr{3}\tr{2}A\quot N^{3}(A)$ and take the quotient of $j_{A}$:
\begin{eqnarray}
\map{\bar j _{A}}{&\fr{3}\tr{2}A\quot N^{3}(A)}{D}\\
&\bar w\longmapsto \bar w
\end{eqnarray}
on account of \pref{compa} is $\bar j_{A}$ one-to-one, in particular its image is again
$\gena{\abu}{D}=(\gena{\abu}{\fr{3}D^{\prime}}+N^{3}(D))\quot N^{3}(D)$.\quad Do the same for $C$.

Next we show that these are $\delta_{3}$-strong embeddings.

We test it for $C$. From the definition of $\delta_{3}$ it is sufficient to prove $\delta_{3}C_{1}\leq\delta_{3}E_{1}$
for each $E_{1}\zsu\dbu$ such that $E_{1}\nni\cbu$.
As $E_{1}=C_{1}+(E_{1}\cap \abu)$ we finish once we show
$\delta_{3}(E\quot C)=\delta_{3}(E_{1}\cap\abu\quot B)$ because
%$E_{1}\cap \abu$ is a $\delta_{2}$ strong subspace of $\abu$ and 
$B\dsu A$ and $\delta_{3}(E_{1}\cap \abu\quot B)\geq 0$.

We have
$$\delta_{3}(E\quot C)=\delta_{3}(E_{1})-\delta_{3}(C_{1})=d_{2}(E_{1}\quot C_{1})
-\big(\dfp(N^{3}(E))-\dfp(N^{3}(C))\big).$$
Now as $\tr{2}D=D^{\prime}$, in the $2$ amalgam everything behaves good for $\delta_{2}$,
and we have $d_{2}(E_{1})-d_{2}(C_{1})=\delta_{2}(E_{1})-\delta_{2}(C_{1})=\delta_{2}(E_{1}\cap\abu\quot
B_{1})$, we've already seen this in chapter $2$.

To conclude we have to show
$$N^{3}(E_{1}\cap \abu)\quot N^{3}(B)^{\bullet}\simeq
N^{3}(C_{1}+(E_{1}\cap \abu))\quot N^{3}(C)^{\bullet}.$$
Consider the map
\begin{eqnarray*}
N^{3}(E_{1}\cap \abu)\quot N^{3}(B)^{\bullet}&\longrightarrow&N^{3}(E_{1})\quot N^{3}(C)^{\bullet}\\
\bar\eta&\longmapsto&\bar\eta\quad\quad\forall\eta\in N^{3}(E_{1}\cap\abu).
\end{eqnarray*}
Soundness is immediate. Moreover our map is injective because
\begin{multline*}
N^{3}(C)^{\bullet}\cap\gena{E_{1}\cap \abu}{\fr{3}D^{\prime}}=\\
=N^{3}(D)\cap\gena{\cbu}{\fr{3}D^{\prime}}\cap\gena{E_{1}\cap \abu}{\fr{3}D^{\prime}}=\\
=N^{3}(D)\cap\gena{\abu\cap C_{1}}{\fr{3}D^{\prime}}=N^{3}(B)^{\bullet}.
\end{multline*}
Here we used $\gena{\cbu}{\fr{3}D^{\prime}}\cap\gena{E_{1}\cap\abu}{\fr{3}D^{\prime}}=
\gena{\cbu\cap\abu}{\fr{3}D^{\prime}}$ because both $\cbu$ and $E_{1}\cap\abu$ are $\delta_{2}$ strong in $D_{1}$ and because $\gena{\cbu}{D^{\prime}}\cap\gena{E_{1}\cap\abu}{D^{\prime}}=\gena{\bbu}{D^{\prime}}$.

On the other hand, since again $\gena{\abu}{D^{\prime}}\cap\gena{E_{1}}{D^{\prime}}=\gena{E_{1}\cap\abu}{D^{\prime}}$ we have
\begin{multline*}
N^{3}(E_{1})=N^{3}(D)\cap\gena{E_{1}}{\fr{3}D^{\prime}}=\\
=(N^{3}(A)^{\bullet}+N^{3}(C)^{\bullet})\cap\gena{E_{1}}{\fr{3}D^{\prime}}=\\
=(N^{3}(A)^{\bullet}\cap\gena{E_{1}}{\fr{3}D^{\prime}})+N^{3}(C)^{\bullet}=\\
=(N^{3}(D)\cap\gena{E_{1}\cap A_{1}}{\fr{3}D^{\prime}})+N^{3}(C)^{\bullet}
\end{multline*}
and this gives that our map is onto.

We see now that $\dfp(N^{3}(E))-\dfp(N^{3}(C))=\dfp(N^{3}(E_{1}\cap\abu))-\dfp(N^{3}(B)^{\bullet})$.
We have shown that $C$ is $\delta_{3}$-strong embeddable in $D$.

A completely analogous argument proves that $A$ can be found in $D$ as a $\delta_{3}$ strong structure.

Now to prove $(\triv)\dsu D_{1}$ we pick an arbitrary $E_{1}\zsu D_{1}$, since

\center{\textbf{Achtung!\quad Achtung!\quad Achtung!.}}

\smallskip
$\gena{\abu}{D^{\prime}}\cap\gena{E_{1}}{D^{\prime}}=\gena{E_{1}\cap\abu}{D^{\prime}}$,
in this case we can apply Lemma \ref{presubatre} and find
\begin{multline*}
\delta_{3}\ssc_{2}(A_{1}+E_{1})+\delta_{3}(A_{1}\cap E_{1})\leq\\
\leq\delta_{3}A+\delta_{3}E\leq\delta_{3}\ssc_{2}(A_{1}+E_{1})+\delta_{3}E.
\end{multline*}
We used $\delta_{3}A\leq\delta_{3}\ssc_{2}(A_{1}+E_{1})$ because $A\dsu D$.

As $E_{1}\cap A_{1}\zsu A_{1}$ und $A\in\Ktre$, it follows $0\leq\delta_{3}(E_{1}\cap A_{1})\leq\delta_{3}E$.

So far we have constructed a $D\in\nla{3}$ such that $A\dsu D\dso C$ (up to $\delta_{3}$-strong
embedding) with $\tr{2}D\in\Kdue$ and
$\delta_{3}$ is non negative on the $\delta_{2}$-strong subspaces of $D_{1}$.
\end{description}

We proceed with a discussion of minimal $\delta_{3}$ extensions, in order to decide whether or not a {\bf (NAM)}
construction is possible, and whether it leads to a structure $D$ which lays in $\Ktre$.
We will actually provide axioms $(\iota)_{2}$ and $(\iota)_{3}$ for the amalgam $D$.

Assume $A_{1}\quot B_{1}$ is minimal with respect to $\delta_{3}$,
then only the following cases can occur:
\begin{itemize}
\item$\delta_{3}(A\quot B)>0$ (\emph{free minimal extension})
\item%$\delta_{3}(A\quot B)=0$ and 
$A\quot B$ is a \emph{divisor extension}:\\
there exists an element $a\in A_{1}\non B_{1}$ such that\end{itemize}
$$ %\begin{labeq}{dreidiv}
\text{($3$-div.)}\quad
(\exists\beta,\,P_{2}(\beta))\:\beta\in\gena{a,B_{1}}{\fr{3}\tr{2}A}\;\exists\Phi\in\gena{B_{1}}{\fr{3}\tr{2}A}\:\left([a,\beta]-\Phi\,\in N^{3}(A)\right).
%\end{labeq}
$$\begin{itemize}

\item$\delta_{3}(A\quot B)=0$ and no ($3$-div.) divisor (\emph{pr\"{a}algebraic extension}).
\end{itemize}

In the rest of the proof we suppose that our extension is not realized in $C$,
otherwise we take $C$ itself as the desired amalgam.

\medskip
Assume first $A_{1}\quot B_{1}$ is free minimal. %and $\delta_{3}(A\quot B)>0$.
Then we claim $d_{2}A=d_{2}B+1$ and $A_{1}\quot B_{1}$ is also $\delta_{2}$ minimal.

We must however have $d_{2}A>d_{2}B$, assume $d_{2}A>d_{2}B+1$, then there exist an element $a\in A_{1}$,
such that $A_{1}\supsetneq\ssc_{2}(B_{1},a)\supsetneq B_{1}$. Now from minimality it follows
$\delta_{3}\ssc_{2}(B_{1},a)>\delta_{3}A>\delta_{3}B$. But for each $a\in A_{1}$ we have $\delta_{3}\ssc_{2}
(B_{1},a)\leq\delta_{3}B+1$. And this cannot happen.
In particular it follows $\delta_{3}A=\delta_{3}B+1$ and $\dfp(N^{3}(A))=\dfp(N^{3}(B))$.

Now if there exists $A_{1}\supsetneq H_{1}\supsetneq B_{1}$ with $A_{1}\zso H_{1}$, since
$N^{3}(A)=N^{3}(B)$, then $\delta_{3}H$ is either equal to $\delta_{3}A$ or to $\delta_{3}B$. In both cases
that is against minimality. This proves minimality of $A_{1}$ over $B_{1}$ with respect to $\delta_{2}$,
therefore $A_{1}=\gen{B_{1},a}$ for some $a$ in $A_{1}$ free from $B_{1}$.

In this case the free amalgam of $\tr{2}A$ and $\tr{2}C$ over $\tr{2}B$ has $(\iota)_{2}$, and hence gives an element of $\Kdue$. Applying {\bf (NAM)} we obtain a structure $D$ which has still $(\iota)_{3}$ since $N^{3}(D)=N^{3}(C)^{\bullet}$.
Thus $D$ is in $\Ktre$ and amalgamates $A$ and $C$ over $B$.

\medskip
Assume now $A_{1}\quot B_{1}$ is a divisor $\delta_{3}$-extension and suppose ($3$-div.) holds for an
element $a\in A_{1}\non B_{1}$.
By definition of ($3$-div.) there exists a weight $2$ element $\beta\in\gena{B_{1},a}{\fr{3}\tr{2}A}$ and a linear combination of weight $3$ commutators
$\Phi\in\gena{B_{1}}{\fr{3}\tr{2}A}$ such that $[a,\beta]-\Phi\,\in N^{3}(A)$.

Since $B\dsu A$ and $\dfp(N^{3}(\ssc_{2}(a,B_{1})))\geq\dfp(N^{3}(B))+1$,
we obtain $\delta_{3}(a\quot B_{1})=0$ and $d_{2}(a\quot B_{1})=1$.

By $\delta_{3}$-minimality we conclude
that $A_{1}=\ssc_{2}(B_{1},a)$ and $N^{3}(A)$ is generated
in $\fr{3}\tr{2}A$ by $N^{3}(B)$ and $[a,\beta]-\Phi$.

Since $B_{1}$ is $\delta_{2}$-strong and $d_{2}(a,B_{1})=d_{2}B_{1}+1$, we have
that $\gen{B_{1},a}$ is $\delta_{2}$-strong too.
%We conclude $A_{1}=gen{B_{1},a}$.
%We prove next $\gen{B_{1},a}\zsu A_{1}$.
%Let then $H_{1}$ be an arbitrary subspace of $A_{1}$ containing both $B_{1}$ and $a$,
%if we compare $\delta_{3}H_{1}$ and $\delta_{3}A$,
%since $N^{3}(\ssc_{2}H_{1})=N^{3}(A)$, then
%$d_{2}H=d_{2}A$.
%So $\delta_{2}\gen{B_{1},a}\leq\delta_{2}B_{1}+1=d_{2}A_{1}=d_{2}H_{1}\leq\delta_{2}H_{1}$.

As a result $A_{1}=\gen{B_{1},a}$ is $\delta_{2}$-free (minimal) over $B_{1}$, therefore it can be with $C_{1}$
freely amalgamated in $\Kdue$ over $B_{1}$. We can now apply {\bf (NAM)} and conclude.

%Otherwise, there exists a $c\in C_{1}\non B_{1}$ such that $[c,\beta^{\prime}]-\Phi\,\in N^{3}(C)$
%where $\beta^{\prime}$ is obtained replacing $c$ in all occurrences of $a$ appearing in the
%commutators which form $\beta$. Here $\Phi$ must be regarded as an element of $\gena{B_{1}}{\fr{3}\tr{2}C}$.
%Since $\delta_{3}(c\quot B_{1})=0$ again, we have $d_{2}(c\quot B_{1})=1$ and
%$c$ results free from $B_{1}$ with respect to weight $2$ relations.
%It follows, we can map $a$ to $c$ fixing $B_{1}$, and obtain a Lie morphism of $A$ into $C$.
%We take $C$ therefore, as the desired amalgam, in order to keep axiom $(\iota)_{3}$.

We can also prove that if there is $a$ in $A_{1}\non B_{1}$ such that $\delta_{3}(a\quot B)=0$
and $d_{2}(a\quot B)=1$ then $a$ satisfies ($3$-div.).

\medskip
Assume now $A\quot B$ is a minimal prealgebric $\delta_{3}$ extension.

%we have only to assure that the level $2$ admits a free amalgamation which leads to
%an object in $\Kdue$, afterwards we proceed applying {\bf (NAM)} to get a structure
%$D$ which of course has $(\iota)_{3}$, and therefore belongs to $\Ktre$.
Assume first $d_{2}A=d_{2}B$. Since $\delta_{3}A=\delta_{3}B$, we have $N^{3}(A)$=$N^{3}(B)$.
It follows $A_{1}$ is a minimal $\delta_{2}$ extension of $B_{1}$, because a strong proper refinement of it
would produce a $\delta_{3}$ strong refinement strictly between $B_{1}$ and $A_{1}$.

As $A$ is not realized in $C$ over $B$ as a $3$ nilpotent algebra, $A_{1}$ is not realized
in $C_{1}$ over $B_{1}$ considering the induced weight-$2$ structure.
In particular the free amalgam of the tree truncated structures is in $\Kdue$ and we can
%Is $A_{1}\quot B_{1}$ a prealgebraic $\delta_{2}$ extension or a divisor extension ($2$-div.{}),
proceed applying {\bf (NAM)}, the structure $D$ we find has $(\iota)_{3}$.
%If it is a $\delta_{2}$ divisor extension, $A_{1}=\gen{B_{1},a}$ for some $a\in A_{1}$ and there is a morphism over $B_{1}$ mapping $a$ to some $c\in C_{1}$, then, because $A=\gena{B_{1},a}{A}$, the
%weight $3$ structure of $A$ can be reconstructed in $C_{3}$, by means of commutators
%with occurrences of $c$. So $A$ is realized in $C$ over $B$.

Assume then $d_{2}A>d_{2}B$. As no $3$-divisor is present, $\delta_{3}(a\quot B)>0$ for all $a\in A_{1}\non B_{1}$ and a fortiori $d_{2}(a\quot B)>0$ for all such $a$. 
This is equivalent to
\begin{labeq}{cloclo}
B_{1}=\cl_{2}(B_{1})\cap A_{1}.
\end{labeq}
%in fact this is equivalent to $d_{2}H>d_{2}B$, for each $\delta_{2}$-strong proper refinement $H$ of $A_{1}\quot B_{1}$. If we consider such an $H$ then by minimality $\delta_{3}H>\delta_{3}A=\delta_{3}B$, so $d_{2}(H\quot B)>\dfp(N^{3}(H))-\dfp(N^{3}(B))\geq 0$.

On account of \pref{cloclo}, we have in particular
$\delta_{2}(a\quot B_{1})>0$ for each $a\in A_{1}\non B_{1}$, therefore
no element $a$ of $A_{1}\non B_{1}$ can be a divisor in $B$. We remind that this means
there's no $\beta\in\exs B_{1}$ such that $[a,b]-\beta\,\in N^{2}(B)$ for some $b\in B_{1}$.
%In particular no element of $a$ can be realised in $C_{1}$ over $B_{1}$.
%$A_{1}$ may be joint to $F_{1}$ via a chain of minimal extensions none of which is frei.
It follows that even if $A_{1}\quot B_{1}$ is not a minimal extension, the free amalgam $\fram{\tr{2}A}{\tr{2}B}{\tr{2}C}$ is in $\Kdue$. We can now apply {\bf (NAM)} as desired.
%({\bf Note}: if it is somehow too ambitious to consider the entire $A_{1}\quot B_{1}$ we can
%adopt an inductive argument on $m=d_{2}(A_{1}\quot B_{1})$, so we can split the extension in
%$A\quot B^{*}$ and $B^{*}\quot B$, wobei $B^{*}=\clz(F_{1})\cap A_{1}$ where $F_{1}$ is
%a minimal (frei!) extension of $B_{1}$ in $A_{1}$. Naturally $d_{2}(B\quot B^{*})=1$).

This concludes the proof, as there's no other case to be considered.
\end{proof}
