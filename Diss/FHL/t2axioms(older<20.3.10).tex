\section{The infinite-rank Theory $T_{2}^{\omega}$}
Let $M$ be an $\La_{2}$ structure we axiomatise the theory
$T_{2,\omega}$ by means of the following first order schema
\begin{itemize}
\punto{$\sig{2}{1}$}$M$ is an algebra of $\nla{2}$
\punto{$\sig{2}{2}$}For any finite subspace $H_{1}$ of $M_{1}$ with $\dfp(H_{1})\geq2$
holds $\delta_{2}(H)\geq2$. %IN FORMULAE: $\delta_{2}(H)\geq\min(\dfp(H_{1}),\,2)$. 
\punto{$\sig{2}{3}$}for any $n<\omega$ and finite  \mn{minimal?} %minimal
strong extension
$A\quot B$ of $\Kl{2}$-algebras, %with $\delta_{2}(A\quot B)=0$,
if $B$ is $(\dfp(A_{1}\quot B_{1})+n)$-selfsufficient in $M$,
then there exists an isomorphic copy of $A$ in $M$ over $B$, which is $n$-self-sufficient.
\end{itemize}
\begin{teo}
%The rich %countable structures in 
An $\La$-structure $K$ is a rich algebra of $\Kl{2}$
%are exactly the countable $\omega$-saturated models of $T^{2}$.
if and only if $K$ is an $\omega$-saturated model of $T^{2,\omega}$.
\end{teo}
\begin{proof}
We start proving that a rich algebra $K$ of $\Klt{2}$ is also a model of $T^{2}$, which is
henceforth consistent. The second part of the proof shows that an $\omega$-saturated model
of $T^{2}$ is rich, now since rich %countable\mn{be careful!}
structures are back-and-forth equivalent %and hence isomorphic,
we can deduce that rich $\Klt{2}$-structures are $\omega$-saturated.

\smallskip
So let $K$ be such a rich algebra in $\Kl{2}$, it satisfies then automatically
axioms $\sig{2}{1}$ and $\sig{2}{2}$.

Now assume $A$ is a finite strong extension of a subalgebra $B$ of $\mathbb{K}$ which
is $(\dfp(A_{1}\quot B_{1})+n)$-selfsufficient in $\mathbb{K}$. Take a finite strong subalgebra $\tilde{B}$ of $\mathbb{K}$ containing $B$ (the selfsufficient closure of $B$ for instance). Now
use the asymmetric amalgamation lemma \ref{asymalgadue} to obtain a strong extension $\tilde{A}$ in $\Kl{2}$ of $\tilde{B}$,
such that  $A$ is an $n$-selfsufficient substructure of $\tilde{A}$. %and $\tilde{B}_{1}\cap A_{1}=B_{1}$.

Now since $\mathbb{K}$ is rich, $\tilde{A}$ strongly embeds into $\mathbb{K}$ over $\tilde{B}$.
As a consequence of transitivity (lemma \ref{}), $A$ embeds into $\mathbb{K}$ $n$-selfsufficiently over $B$.

\dots\bigskip

{\tt direction $\Leftarrow$: use of bdd-ssuffncy}
Suppose $M$ is a countable $\omega$-saturated model of $T^{2}$ and $A\quot B$
a finite strong extension of $\Kl{2}$-algebras, where $B$ is a strong
substructure of $M$. We may assume, without loss
of generality, that $A\quot B$ is a minimal extension (otherwise decompose it as usual
in minimal sections which are embedded stepwise in $M$).

So assume first $A\quot B$ is a free extension and $B$ is a finite strong substructure in $M$,
we are done if we find $a^{\prime}\in M_{1}$ which is $\cl_{2}$-independent over $B_{1}$.

We actually prove that $d_{2}(M_{1})$ is infinite, and the desired condition will follow at once.
We show, with an induction argument, that we can {\em strongly} embed $F^{2,n}$ in $M$ for any $n<\omega$. We ambiguously denote
as $\F^{2,n}(\bar x)$ both the diagram of {\em and} the $n$-generated free nil-$2$ Lie algebra, with the convention
that $\F^{2,n}(\bar a)$ means $\gen{\tpl{a}{n}}\simeq F^{2,n}$.

Axiom $\sig{2}{2}$ ensure  that for any nontrivial $m_{1}, m_{2}$ which are not linear dependent in $M_{1}$,
$\gena{m_{1},m_{2}}{M}$ is a selfsufficient subalgebra of $M$ and isomorphic to $F^{2,2}(m_{1},m_{2})$.
For the same reason, also $2$-generated subalgebras share the same property;
this will be our inductive base.

Assume now $M\sat F^{2,n}(\bar b)$ and $\gena{\bar b}{M}$ is strong in $M$. Consider
the following collection of formulae $\Phi^{n+1}(x,\bar b)=\left(\phi^{n+1}_{k}(x,\bar b)\mid k<\omega\right)$ where an $\La^{2}$-structure $L$ satisfies $\phi^{l}_{k}(\bar y)$ in $\bar m \inn L_{1}$ exactly when $L\sat\F^{2,l}(\bar m)$ and $\gena{\bar m}{L}$ is $k$-strong in $L$.

Now a finite portion of $\Phi^{n+1}$ is implied by $\phi^{n+1}_{k}(x,\bar b)$ with a sufficiently
large $k$. %, here $k$ may also be assumed to be bigger than. Now use axiom
Now since $\gena{\bar b}{M}$ is strong, we have $M\sat\phi^{n}_{k+1}(\bar b)$ and hence by $\sig{2}{3}$ there exists
$a$ in $M_{1}$ such that $M\sat\phi^{n+1}_{k}(a,\bar b)$.

We showed $\Phi^{n+1}(x,\bar b)$ is consistent with $T^{2}$ and hence realized in $M_{1}$
by $\omega$-saturation for some $m\in M_{1}$. It follows $\gena{m,\bar b}{M}$ is
selfsufficient and hence $d_{2}^{M}(m,\bar b)=\delta_{2}(m,\bar b)=n+1$.

By induction, $M$ has infinite $d_{2}$-dimension.
\medskip

Assume now $A\quot B$ is a minimal strong extension with $\delta_{2}(A\quot B)=0$ (hence
algebraic or prealgebraic). Since $B$ is strong in $M$, (anyone among) axioms $\sig{2}{3}$ ensure
the existence of an isomorphic copy $A^{\prime}$ of $A$ in $M$ over $B$. Because
of $\delta_{2}(A^{\prime}\quot B)=0$, we have that $A^{\prime}$ is selfsufficient in $M$ as
well. This proves that $M$ is a rich Lie algebra in $\Kl{2}$.
\end{proof}

Note that the Fra�ss� limit of $\Kl{2}$ that we constructed in \ref{} is ``the'' countable saturated model of $T^{2,\omega}$.
Also note that $T$ is not $\aleph_{0}$-categorical, because the self-sufficient closure, and  hence
the algebraic closure, of a finite set can be infinite.
\begin{lem}\label{strongelm}
Elementary extensions are strong
and conversely\mn{make any sense?}, strong model extensions are elementary.
\end{lem}


\begin{lem}\label{alg-ex-cls}
Define
\begin{itemize}
\punto{$\sig{2}{4}$} for all $w\in M$ with $P_{2}(w)$ and all $z\in M_{1}$ there is $x\in M_{1}$ such that $[z,x]=w$.
\end{itemize}
then $\sig{2}{4}$ is a consequence of the other axioms in $T^{2}$.
\end{lem}
\begin{proof}
Pick an element $w\in M_{2}$ and $m\in M_{1}$ for some $\La^{2}$-structure $M$\mn{maybe switch to
$\omega$-sat and abovelemma} which satisfies $\sig{2}{i}$ for $i=1,2,3$. Now $w$ belongs to $\gena{B_{1}}{M}$ for some finite
$B_{1}\inn M_{1}$ with also $m\in B_{1}$. Set $C_{1}:=\ssc_{2}(B_{1})$. If there exists $c$ in $C_{1}$ such that $[m,c]=w$ we are done, so assume not. Define the strong extension 
$A$ of $B$ as follows: $A_{1}:=B_{1}\oplus\gen{a}$ and $N^{2}(A):=N^{2}(B)\oplus
\gen{[m,a]-w}$ as a subspace of $\exs A_{1}$. Since $A$ is not realised in $C$,
amalgamate\mn{We need $\sig{2}{2}$ of $A$ if $B$ does!}
$A$ and $C$ over $B$ with lemma \ref{amalsigma2} to get $D=\gena{C_{1},a}{D}$ in $\Kl{2}$
with $C\zsu D$. Now being $C$ strong in $M$, in particular axioms $\sig{2}{3}$ apply to
the finite strong extension $D\quot C$,
ensuring the existence of an element $x$ in $M_{1}$, for which $M\models [m,x]-w=\triv$. 
\end{proof}

In the proof above/below we actually show how any two $\omega$-saturated models of $T^{2}$ are back-and-forth
equivalent

\begin{prop} %\mn{see Zio:Primordial, use \ref{strongelm} in one direction}
Assume $M$ and $M^{\prime}$ are two models of $T^{2,\omega}$. Let $a$ and $a^{\prime}$
be tuples of $M_{1}$ and $M^{\prime}_{1}$ respectively. Then $a\equiv a^{\prime}$\mn{\"aqv:$\gena{a}{M}\equiv \gena{a^{\prime}}{M^{\prime}}$?} if and only
if the selfsufficient closure $\gena{\ssc_{2}(a)}{M}$ is $\nla{2}$-isomorphic to
$\gena{\ssc_{2}(a^{\prime})}{M^{\prime}}$
via a Lie morphism mapping $a$ bijectively onto $a^{\prime}$. Since $\triv$ is self-sufficient
in every model, the theory $T^{2,\omega}$ is complete.
\end{prop}

{\bf (explain in detail:)}{\sl Any two (finite) strong tuples $\bar a ^{1}$ and
$\bar a ^{2}$ from two rich $\Kl{2}$-algebras $K^{1}$ and $K^{2}$ respectively,
such that $\gena{\bar a ^{1}}{K^{1}}\simeq\gena{\bar a ^{2}}{K^{2}}$ can be
matched up by an Ehrenfeucht-Fra\"iss\'e game of lenght $\omega$. Without
loss of generality we may assume both $\bar a ^{i}$ to be contained in $K^{i}_{1}$,
for assume for instance, $w=w_{1}+w_{2}$ is in $\bar a ^{1}$ for $w_{2}\neq\triv$.
If $w_{2}$ is an homogeneous sum of Lie products from $K^{1}_{1}\cap\gena{\bar a ^{1}}{K^{1}}$ then replace $w$ with $w_{1}$ (and do the same for the corresponding
element of $\bar a ^{2}$), otherwise pick $m$ in $K_{1}^{1}$ with $[m,e^{1}]=w_{2}$ for
some $e^{1}\in K^{1}_{1}\cap\gena{\bar a ^{1}}{K^{1}}$ (this space may be also
assumed to be non trivial). We have $\bar a ^{1} m$ is still strong in $K^{1}$.

{\bf FIX: for non $\nla{2}$-subalgebras, it is not clear, what does it mean to be
``strong'' (check on [Bad96] analogous condition for groups)}}.

We say that a tuple $\bar a$ in $M\in\Kl{2}$ is a {\em strong tuple}\mn{{?}see ``constructions'' in AddColl} if there exists
a strong finite subspace $H_{1}$ in $M_{1}$ such that $\bar a\inn \gena{H_{1}}{M}$
and such that the set of elements from $H_{1}$ which are not in $\bar a$ are $d_{2}$-independent over $M_{1}\cap\bar a$.

\bigskip
A description of Types {\em � la John B. Goode} follows. The guiding principle here
is that the type of a finite subspace $B_{1}$ in $T^{2,\omega}$ %of $M_{1}$ where
is actually determined by the quantifier-free diagram of the self-sufficient closure
of $B_{1}$.

Recall that $M_{1}$ is equipped with the pregeometry
coming from $d_{2}$. We will essentially use this dimension
to classify types of elements.

\begin{cor}\mn{or somewhat similar}\label{isola}
Let $B_{1}$ be a strong space of $M_{1}$ for some $\omega$-saturated
model $M$ of $T^{2,\omega}$.

Assume $a$ is a tuple in $M_{1}$ such that $d_{2}(a\quot B_{1})=0$.
Write the quantifier-free diagram of $\ssc(B_{1},a)$ as a $\La_{B_{1}}$ formula $\Delta(x,y)$, where $x$ is a placeholder for $a$
and for any tuple $c$ of $M_{1}$, for $M$ to satisfy $\Delta(a,c)$ means that
$\gena{B_{1},a,c}{M}\simeq_{\Fp}\ssc(B,a)$.

Then $\exists y\Delta(x,y)$ isolates $\tp{a}{B_{1}}$.
\end{cor}
We need also the following (easy) general lemma
\begin{lem}\label{RMU}
Let $T$ be a complete theory with
infinite models.\mn{do we need $T$ to be t.t.} Fix some parameter set $A$
and let $\mathfrak{X}$ be a family of isolated types
in $\ssp{}{A}$ which is closed under extensions.

Then Morley rank and $U$-rank agree on $\mathfrak{X}$.
\end{lem}
\begin{proof}
Since the inequality $U(p)\leq RM(p)$ for all $p$ in $\ssp{}{A}$, we have to show that for any
ordinal number $\kappa$ and all $p$ in $\mathfrak{X}$, if $RM(p)\geq\kappa$ then also $U(p)\geq\kappa$.

We proceed by induction on $\kappa$, and consider just the successor case. Assume the inequality has
then been proved for all cardinals less than $\kappa+1$ and
$RM(p)\geq\kappa+1$. Let $\varphi$ be a formula isolating $p$ and $\psi$ %\models\varphi$ a
a rank $\kappa$ formula which implies $\phi$. Let $q$ be a generic type of $\psi$ which extends $p$, now
since $q$ belongs to $\mathfrak{X}$ and $MR(q)\geq\kappa$ then $U(q)\geq\kappa$ by induction.
We can conclude $U(p)\geq\kappa$.
\end{proof}

We can now prove
\begin{teo}
$T^{2,\omega}$ is totally transcendental\mn{is it true?} of Morley Rank
$\omega2$ and Morley degree $1$.
%Let $M$ be a model of $T^{2,\omega}$, then the definable set
%$M_{1}$ is connected and has Morley rank $\omega$.
\end{teo}
\begin{proof}
Let $M$ be an $\omega$-saturated model, that is,
an elementary strong subalgebra of $\mathbb{M}$ by lemma \ref{}.

Since $M$ is generated as an algebra by $M_{1}$, 
we can restrict our analysis to $\ssp{P_{1}}{B}=\{p\in\ssp{1}{B}\mid p\models P_{1}\}$
for finite sets of parameters $B$ of $M$.\mn{or only countably
many ``$\delta_{2}$-geometric'' constructions � la {\sl Add.ve Collpse}}


\smallskip
Note that for types {\em over models} $\sig{2}{4}$ implies that a type of an element
$a$ of $\mathbb{M}$ is {\em interalgebraic} with the type of an element $m$ of $\mathbb{M}_{1}$.

\smallskip
So consider $\tp{m}{B}$ where $m$ is in $M_{1}$.
Since the self-sufficient closure is always contained in the algebraic closure of $B_{1}$,
during rank computations,
we might always assume that $B=B_{1}$ is a selfsufficient subspace of $M_{1}$.

Assume first $d_{2}(m\quot B_{1})=0$. Take $A_{1}$ to be the self-sufficient closure
$\ssc(B_{1},m)$ of $m$ over $B$. And discompose $A_{1}$ into a finite chain of finite
subspaces $A^{i}$ such that
$$B=A^{0}\zsu{}A^{1}\zsu{}\,\cdots\,\zsu{} A^{n}=A$$
where each $A^{i+1}\quot A^{i}$ is a minimal extension.

Since $d_{2}(m\quot B_{1})=\delta_{2}(A_{1}\quot B_{1})=d_{2}(A\quot B)=0$,
in particular for each $i$ $d_{2}(A^{i+1}\quot A^{i})=0$
hence if we apply lemmas \ref{isola} and \ref{RMU} to this situation,
we can use additivity of $U$-rank as folows
\begin{align*}
RM(m\quot B_{1})=RM(A_{1}\quot B_{1})=U(A_{1}\quot B_{1})=\\
=U(A^{n}\quot A^{n-1})+\,\dots\,+U(A^{1}\quot A^{0}) %=\\
%=RM(A^{n}\quot A^{n-1})+\,\dots\,+MR(A^{1}\quot A^{0})
\end{align*}

We must now show that a minimal non-algebraic $\delta_{2}$-strong extension has
rank and degree $1$. To do this we prove that such a type
is {\em minimal} in the sense that it admits just one non-algebraic extension to
every $C_{1}\nni B_{1}$.

Assume then $A_{1}\quot B_{1}$ is a minimal non-algebraic strong extension
and $C_{1}$ some subspace of $M_{1}$
containing $B_{1}$ which can again be chosen self-sufficient.

Since $A_{1}$ is a prealgebraic extension of $B_{1}$ and the intersection $A_{1}\cap C_{1}$ is again
strong in $M_{1}$, we have that either $A_{1}$ {\em is contained} in $C_{1}$ or $A_{1}\cap C_{1}=B_{1}$
and $0\leq\delta(A_{1}\quot C_{1})\leq\delta(A_{1}\quot B_{1})=0$, which implies that $A_{1}$ and $C_{1}$
are in free composition over $B_{1}$. Since $A_{1}+C_{1}$ is self-sufficient, the isomorphism type
of $\gena{A_{1}+C_{1}}{M}$ fully determines the type of $A_{1}$ over $C_{1}$, this gives but only one type
over $C_{1}$.

We conclude that $\tp{m}{B_{1}}$ has finite rank and degree $1$. Anyway, since in a rich model
we can find arbitrarily many successive prealgebraic extensions nested one on top of the other,
there is no a priori bound to such rank (which does not occur in the corresponding
collapsed theory).


\medskip
If on the other hand $d_{2}(m\quot B_{1})=1$ then $\gen{m,B_{1}}$ is self sufficient\mn{prove} in $M_{1}$
and, hence for this kind of elements there is only one type.

The above dimensional dichotomy describes the picture of $\ssp{P_{1}}{B_{1}}$, this being partitioned
into two subsets: on one side, a collection of types of finite, unbounded rank, on the other one single type, which
has then necessarily rank $\omega$.

\medskip
To see that $M$ has rank $\omega2$
we will essentially map definably onto $M_{1}$ onto $M_{2}$.
and use the fact that $M=M_{1}\oplus M_{2}$.

To do this consider an element $c$ of $M_{1}$, since ranks are preserved up to
a finite choice of fixed parameters, we can actually work in $T^{2,\omega}_{c}$.

Consider the linear map $\map{\lambda_{c}}{M_{1}}{M_{2}}$ defined by $m\mapsto[c,m]$.
Then $\lambda_{c}$ is pure definable, onto by property $\sig{2}{4}$, and by axiom $\sig{2}{2}$,
has kernel $\gen{c}$, which is a $p$-element set. It follows that $RM(M_{2})=RM(M_{1})$.

%-------------------UNCOMMENT TO KEEP CALCULATIONS SEPARATED----------------------------------
%\end{proof}

%As a corollary
%\begin{teo}
%$T^{2,\omega}$ is $\omega$-stable\mn{follows from rank arguments?} connected of Morley Rank $\omega2$.
%\end{teo}
%\begin{proof}
%Let $M$ be  an $\omega$-saturated model of $T^{2,\omega}$, to see that $M$ has rank $\omega2$
%we will essentially map definably onto $M_{1}$ onto $M_{2}$.
%and use the fact that $M=M_{1}\oplus M_{2}$.
%
%To do this consider an element $c$ of $M_{1}$, since ranks are preserved up to
%a finite choice of fixed parameters, we can actually work in $T^{2,\omega}_{c}$.
%
%Consider the linear map $\map{\lambda_{c}}{M_{1}}{M_{2}}$ defined by $m\mapsto[c,m]$.
%Then $\lambda_{c}$ is pure definable, onto by property $\sig{2}{4}$, and by axiom $\sig{2}{2}$,
%has kernel $\gen{c}$, which is a $p$-element set. It follows that $RM(M_{2})=RM(M_{1})$.

%\bigskip
%If we wish to count types in $\ssp{n}{M_{1}}$ where $M$ is a countable model
%we have to look at types of finite tuples $b$ inside some big saturated model $\mathbb{M}$.
%Again take the self-sufficient closure $A_{1}$ of $\gen{M_{1},b}$
\end{proof}

\begin{prop}
Forking in $T^{2,\omega}$ is geometric $d_{2}$-dependence in the following sense:
for a type $\tp{m}{A}$ not to fork
over $B$ exactly means that %$\gen{m}$ is in free composition with $A$ over $B$.
$d_{2}(m\quot A)=d_{2}(m\quot B)$.\mn{necessarily tuples?}

\end{prop}
\begin{proof}
%Assume $d_{2}(m\quot B)=d_{2}(m\quot A)$, 
We will show $U(m\quot B)=U(m\quot A)$ and, by transitivity of forking, assume $A\quot
B$ a minimal self-sufficient extension, with $A$ strong in $\mathbb{M}$ .

Note that, unlike in $T^{2}$, we don't have $d_{2}(m\quot B)=RM(m\quot B)=U(m\quot B)$ here,
which trivially implies the statement.

Our hypothesis implies $\delta_{2}(\ssc_{2}(B_{1},m)\quot B_{1})=\delta_{2}(\ssc_{2}(B_{1},m)\quot A_{1})$,
since
\begin{multline*}
\delta_{2}(\ssc_{2}(B_{1},m)\quot B_{1})=\\
=d_{2}(m\quot B)=d_{2}(m\quot A)\leq\\
\leq d_{2}(\ssc_{2}(B_{1},m)\quot A)\leq\delta_{2}(\ssc_{2}(B_{1},m)\quot A_{1})
\end{multline*}
Since also $B_{1}=\ssc_{2}(B_{1},m)\cap A_{1}=B_{1}$, we have that
$\ssc_{2}(B,m)$ is in free composition with $A$ over $B$.

%Moreover since $A_{1}+ssc(B_{1},m)$ is self-sufficient, we deduce that
%$\ssc_{2}(A,m)\simeq\am{\ssc_{2}(B,m)}{B}{A}$\mn{sci-fi?}
On the other hand we have $\ssc_{2}(A,m)=\ssc(\ssc(B_{1},m)+A_{1})$


\end{proof}
\begin{cor}
$T^{2,\omega}$ has weak elimination of imaginaries.
\end{cor}
\begin{proof}
We show that types over models have real canonical bases.
Consider a finite tuple $a$ in $\mathbb{M}_{1}$, a model $M$ and
the type $\tp{a}{M_{1}}$.
As usual we consider $\ssc(M_{1},a)$ which is $\gen{M_{1},b}$ for some
finite tuple $b$. Inside $\gen{M_{1},b}$ consider the self-sufficient closure $B_{1}$
of $b$ and set $c=B_{1}\cap M_{1}$. $c$ is strong in $\mathbb{M}$. Now with some $\delta_{2}$-calculous
we can show that $c$ is the minimal strong space in $M_{1}$ such that
$d_{2}(a\quot c)=d_{2}(a\quot M)$ and $B_{1}$ is in free composition with $M_{1}$ over $c$. 

Together with the characterisation of forking in $T^{2,\omega}$,
we have that the real tuple $c$ can, up to interdefinability, play the role of a canonical base.
 \end{proof}
 
 Note that in the collapsed case of $T^{2}$, the above result follows immediately by Lascar-Pillay, because
 we have strong minimality and the algebraic (geometric) closure of $\triv$ is infinite.