We refer to a {\em Fra\"iss\'e amalgamation construction} in general as the technique introduced by Roland Fra\"iss\'e in \cite{fra}
to recover universal-homogeneous structures from a prescribed class of finite ones. We follow the treatment
of Ziegler-Tent \cite{zietent} for the countable setting. One may also check \cite{bs} for similar constructions in arbitrary cardinality.

The original results of \cite{fra} are stated in a relational language, but they remain true in the following wider context.

\medskip
Let $\La$ be a countable language; if we say embedding below, we mean $\La$-embedding.
Given a class $\mathcal{K}$ of finitely generated $\La$-structures which is closed under isomorphisms,
we define the following properties for $\Kl{}$:
\begin{itemize}
\punto{HP} For each object $A$ in $\Kl{}$ and substructure $B\inn A$, we have $B\in\Kl{}$ (Hereditary Property).
\punto{JEP} Given any two objects $B$ and $C$, there exists
$A$ in $\Kl{}$ which embeds $B$ and $C$ (Joint Embedding Property).
\punto{AP} For all $D\in\Kl{}$ and embeddings $\emb{\beta}{D}{B}$ and
$\emb{\gamma}{D}{C}$ there exists an object $A$ of $\Kl{}$ and embeddings $\emb{b}{B}{A}$ and $\emb{c}{C}{A}$
such that $\beta b=\gamma c$ (Amalgamation Property).
\end{itemize}
Remark that (JEP) does not follow in general by Amalgamation, as provided by the class of (finite) fields without a
specified characteristic.

Given an $\La$-structure $M$ we define {\em the age of} %l'\^{a}ge} of
$M$, denoted $\age(M)$ to be the class of all finitely generated $\La$-structures which are isomorphic to a substructure of $M$.
Define $\widetilde{\Kl{}}$ to be the class of all $\La$-structures whose age is contained in $\Kl{}$.

For a given $M$, $\age(M)$ satisfies of course (HP) and (JEP), while for $\age(M)$ to have (AP)
it is necessary to require $M$ is {\em strongly homogeneous}. How much, is explained by the next definition and facts.
\begin{dfn}\label{ricca}
An $\La$-structure $M$ is said {\em $\Kl{}$-rich}
if $\age(M)=\Kl{}$ and 
for any embedding $\emb{\beta}{B}{A}$ of $\Kl{}$-objects $A$ and $B$, if $b$ is an embedding of
$B$ into $M$ then there exists an embedding $\emb{a}{A}{M}$ such that $\beta a=b$.
\end{dfn}

For the proof of the following result we refer to \cite[Theorem 13.4]{zietent}.
\begin{fact}[Fra\"iss\'e Limits]\label{fraissteo}
Let $\Kl{}$ be a denumerable class of $\La$-structures for a countable language $\La$ which is closed under isomorphism, then
\begin{itemize}
\punto{i}there exists a countable $\Kl{}$-rich $\La$-structure $M$ in $\widetilde{\Kl{}}$ iff
$\Kl{}$ satisfies \rm{(HP), (JEP)} and {\rm (AP)}.

\punto{ii}Any two countable $\Kl{}$-rich structures are isomorphic. More generally, any two $\Kl{}$-rich structures
are $\La_{\infty,\omega}$-equivalent, that is, they can be matched up by an infinite back and forth correspondence.
\end{itemize}
\end{fact}
The isomorphism type of the countable rich structure is called the {\em Fra\"iss\'e limit} of the class $\Kl{}$.

The class $\widetilde{\Kl{}}$ is not in general elementary.
The classical first examples of this construction are
the class of finite linear orders, which has $(\Q,<)$ as Fra\"iss\'e limit and
the class of finite undirected graphs, of which the Random Graph is the limit.

\medskip
Let now a denumerable class $\Kl{}$ be given, of finitely generated $\La$-structures, for a countable language $\La$.
Assume $\zsu{}$ is a binary relation among objects of $\Kl{}$, which is contained in the $\La$-embedding relation
and which is invariant under $\La$-isomorphisms.

\begin{rem}\label{ModiFra}
Suppose $(\Kl{},\zsu{})$ is a partial order and the properties (JEP) and (AP) are true of $\Kl{}$ with $\zsu{}$
replacing $\La$-embeddings, while (HP) holds in the original fashion.
Then Fact \ref{fraissteo} applies to this situation: there exists a countable structure $\K$ in $\Klt{}$, which is rich with respect to $\zsu{}$.
With this we mean just $\beta$, $b$ and $a$ are to be replaced with $\zsu{}$-embeddings in
Definition \ref{ricca}.
%Moreover we have to redefine $\age(\K)$ as the class of all finitely generated $\Kl{}$-structures, which $\zsu{}$-embeds into $\K$.

We may write in this case that $\K$ is the Fra\"iss\'e limit {\em of} $(\Kl{},\zsu{})$.
\end{rem}

Hrushovski's construction relies on the above modification.
As described in the Introduction, the ``ab Initio'' example substitutes embeddings among relations
with {\em pre-dimensionally strong} embeddings.

In Secton \ref{amalga2}, we describe a similar approach: Lie algebra embeddings are replaced by a suitable
stronger notion.