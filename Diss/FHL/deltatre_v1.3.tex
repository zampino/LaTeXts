\section{A Construction of $\delta_{3}$}
So far we have constructed $\K^{2}$ the countable  infinite-rank limit of the class $\Kl^{2}_{0}$ consisting of
finite dimensional $2$-nilpotent algebras of $\nla{2}$.
$\K^{2}$ is a saturated model of the theory
$T^{2}$ axiomatised by $\sig{2}{1},\,\dots\,,\sig{2}{5}$.

We are on the way to construct an analog generic structure $\K^{3}$ rich in a suitable (pseudo) elementary
class $\Kl^{3}$ of $\nla{3}$-algebras which will be an $\omega$-saturated model of a theory $T^{3}$
axiomatised by properties $\sig{3}{1},\,\dots\,,\sig{3}{5}$.

Objects of $\K^{3}$ will be generated by $\delta_{2}$-strong subspaces of $\K^{2}_{1}$.
A new predimension function $\delta_{3}$ wird introduced on the monolinear component of the
$3$-nilpotent structures, its positive part consisting of the $d_{2}$ dimension obtained
on $\K^{2}_{1}$, the super modular negative part will be counting relation of commutator wieght $3$.

\medskip
We need a definition concerning subspaces of $\K^{2}_{1}$.
Recall that our language contains constant names $a,\,b$, the corresponding elements
will have a central role in the rest of this work.

%\begin{dfn}
Let $H_{1}$ be an $\Fp$ subspace of $\K^{2}_{1}$ containing
$a$ and $H=\gena{H_{1}}{\K^{2}}$.
We say that $H_{1}$ is $\ta$-closed (in $\K^{2}_{1}$) if for each $w\in H$, such that
$P_{2}(w)$ then there is $x\in H_{1}$ such that $[a,x]=w$. $H$ will be also called
$\ta$-closed (in $\K^{2}$).
In what follows it will be assumed that \emph{subspaces} of $\K^{2}_{1}$
always contain the element $a$.
%\end{dfn}

We recall here

\begin{itemize}
\punto{$\sig{2}{3}$}$\quad(\forall x,y)([x,y]=0\rightarrow x\,\text{``lin. depends on''}\,y)$
\punto{$\sig{2}{4}$}$\quad(\forall y,\,P_{2}(y))(\forall z,\,P_{1}(z))(\exists x,\,P_{1}(x))\,[z,x]=y$
\end{itemize}

There's a canonical way of constructing a $\ta$-closure of a subspace $H_{1}$ of $\K^{2}_{1}$.
Consider the linear map $\map{\gamma_{a}}{\K^{2}_{1}}{\K^{2}_{2}}$ defined by $x\mapsto[a,x]$.
On account of axioms $\sig{2}{3}$ and $\sig{2}{4}$ is $\gamma_{a}$ surjective and of finite fiber.
As usual $H$ is the algebra generated in $\K^{2}$ by $H_{1}$, define $H^{1}_{1}=H_{1}$ and
$H^{2}_{1}=\gamma_{a}^{-1}(H_{2})$ and recursively
$H^{n+1}_{1}=\gamma_{a}^{-1}(H^{n}_{2})$,
where $H^{n}=\gena{H^{n}_{1}}{\K^{2}}$.
We have $H^{n}_{1}\inn H^{n+1}_{1}$, so take $\ta H_{1}=\cup_{n<\omega}H^{n}_{1}$ and define $\ta H=\gena{\ta H_{1}}{\K^{2}}$.

If we start with an $H_{1}$ of finite linear dimension such that $H_{1}\zsu\K^{2}_{1}$ then we can see that at each step $\delta_{2}H^{k}_{1}=\delta_{2}H_{1}^{k+1}$. Then we get $d_{2}H=d_{2}\ta H$ and
$\ta H\zsu\K^{2}$.
If $H_{1}$ is unendlich with endlich $d_{2}$ dimension and still selfsufficient
we can adopt a similar argument to build its $\ta$-closure which is still selfsufficient and of the same $d_{2}$.

With the use of axiom $\sig{2}{3}$ we see that $\ta$-closed spaces are closed under intersection
and prove the following
\begin{lem}\label{ta-schnitt}
Let $H_{1}$ and $L_{1}$ be $\ta$-closed subspaces of $\K^{2}_{1}$, then $H\cap L=\gena{H_{1}\cap K_{1}}{\K^{2}}$.
\end{lem}
\begin{proof}
Inclusion right-to-left is clear.
Consider instead $w\in\gena{H_{1}}{\K^{2}}\cap\gena{L_{1}}{\K^{2}}$ of weight $2$. Then there exist $x\in H_{1}$ and $y\in L_{1}$ such that
$[a,x]=w=[a,y]$, but then $x-y$ must be a scalar multiple of $a\in H_{1}\cap L_{1}$, thus both $x$ and $y$
lay in $H_{1}\cap L_{1}$. 

\end{proof}

\medskip
For each algebra $M$ in $\nla{3}$ let $\map{\pi}{\fr{3}\tr{2}M}{M}$ be the canonical map as in \pref{communo} above,
$\pi$ is defined by
$\bar w\;(\mathit{mod}\:\jei{\tr{2}M})\longmapsto\bar w \;({mod}\:R)$. Where $M=\fla{3}{X_{1}}\quot R$.
Then define $N^{3}(M)=\ker\pi$, we have $N^{3}(M)\inn(\fr{3}\tr{2}M)_{3}$.

For each subalgebra $H=\gena{H_{1}}{M}$ of $M$ set
$$N^{3}(H_{1})=N^{3}(M)\cap\gena{H_{1}}{\fr{3}\tr{2}M}.$$

Let now $M=\gen{M_{1}}$ be an algebra of $\nla{3}$ such that $\tr{2}M\inn\K^{2}$
%can be $\delta_{2}$-strongly embedded in $\K^{2}$
in particular $M_{1}$ can be identified with a subspace
%selfsufficient space
of $\K^{2}_{1}$.

For each $H_{1}$ subspace of $M_{1}$ such that $d_{2}H_{1}<\omega$ define
(when it make sense)
\begin{labeq}{deltatre}
\delta_{3}H_{1}=d_{2}H_{1}-\dim_{\Fp}(N^{3}(H_{1}))
\end{labeq}
If $H=\gena{H_{1}}{M}$ we will often write simply $\delta_{3}H$ instead of $\delta_{3}H_{1}$.

Assume now $M\inn L\in\nla{3}$ with $\tr{2}L\inn\K^{2}$, $d_{2}L<\omega$ and $M_{1}\zsu L_{1}$
(writing so we will always mean that $\tr{2}M\zsu\tr{2}L$).
If we define $\delta_{3}^{M}$ and $\delta_{3}^{L}$ as above, computing dimensions respectively
in $N^{3}(M)$ and in $N^{3}(L)$,
then for each $H_{1}\inn M_{1}$ we have $\delta_{3}^{M}(H_{1})=\delta_{3}^{M}(H_{1})$.
This because on account of lemma \ref{bellemma} $\fr{3}\tr{2}M\simeq\gena{M_{1}}{\fr{3}\tr{2}L}$ and
$N^{3}(M)\simeq\gena{M_{1}}{\fr{3}\tr{2}L}\cap N^{3}(L)$.

Therefore working with strong $\delta_{2}$-extensions there won't be need to specify
each time which $\delta_{3}$ we are referring to.

\medskip
It's time to define the subclass of $\nla{3}$ that we wish to amalgamate once we give
a notion of $\delta_{3}$-strong embedding.

Assume $M\in\nla{3}$ with $M_{1}\inn\K^{2}_{1}$, we say that $M$ has the
property %$\sig{3}{2}^{\prime}$ if
\begin{flushleft}
%(\sig{3}{2}^{\prime})\textsl{for each finitely generated $\Fp$-subspace }H_{1}\zsu M_{1},\quad\delta_{3}H_{1}\geq2.
$(\sig{3}{2}^{\prime})$\quad\textsl{if for each finitely generated $\Fp$-subspace $H_{1}\zsu M_{1}$ follows
$\delta_{3}H_{1}\geq2$}. (\dots es st\"ort nicht, aber vielleich reicht beliebige $H_{1}$ \dots)
\end{flushleft}

Now consider the following first order sentence,
$$(\sig{3}{3})\quad(\forall z,\,P_{2}(z))(\forall x,y,\,P_{1}(x)\wedge P_{1}(y))([z,x]=[z,y]\,\rightarrow
\textsl{``$x$ lin.{}depends on $y$''})$$
and then define
$$\Kl^{3}=\left\{
M=\gen{M_{1}}\in\nla{3}\mid
\tr{2}M\zsu\K^{2},\,M_{1}\, \ta\text{-closed},\,M\,\text{has}
\,\sig{3}{2}^{\prime}\,\text{and}\,\sig{3}{3}
\right\}.$$
We isolate objects in the class that have finite $d_{2}$-dimension in
$$\Klf=\{M\in\Kl^{3}\mid d_{2}M_{1}<\omega\}.$$
We note that $\delta_{3}$ is always defined on the objects of $\Kl^{3}_\textsf{f}$ and
that the value of $\delta_{3}$ is always bigger than $2$ on each subspace of
every structure in the class.

\begin{dfn}
Let $M\in\Kl^{3}$ and $H_{1}$ a subspace of $M_{1}$ at which $\delta_{3}$ is defined, $H=\gena{H_{1}}{M}\in\Klf$ %mit $d_{2}H<\omega$, %\in\Klf$
we say that $H_{1}$ is a \emph{$\delta_{3}$-strong
subspace} of $M_{1}$ if 
\begin{itemize}
\item[-]$H_{1}\zsu M_{1}$\quad(\"uberflussig, falls $H$ in $\Klf$ liegt \dots)
\item[-]for any $C\in\Klf$ such that $C_{1}\nni H_{1}$ we have $\delta_{3}H_{1}\leq\delta_{3}C_{1}$.
\end{itemize}
We write ambiguously $H_{1}\dsu M_{1}$ or $H\dsu M$, provided $H=\gena{H_{1}}{M}$.
\end{dfn}
If $M\in\Kl^{3}$ and $H_{1}$ is a subspace of  $M_{1}$, the $\delta_{3}$ constructed
with respect to $M$ yields $\delta_{3}H_{1}\geq\delta_{3}\ssc^{M}_{2}H_{1}\geq\delta_{3}\ta\ssc_{2}^{M}H_{1}$. We note therefore that if $H_{1}\dsu M_{1}$ then \emph{for all} $C_{1}\nni H_{1}$ it holds
$\delta_{3}C_{1}\geq\delta_{3}H_{1}$.

Thanks to the $\ta$-closure we can prove subadditivity of $\delta_{3}$ on $\Klf$-structures.
\begin{lem}\label{presubatre}
Let $U$ and $V$ be algebras in $\Klf$ both contained in $M\in\Kl^{3}$. We have
$$\delta_3\left(U_{1}+V_{1}\right)+\delta_{3}(U_{1}\cap V_{1})\leq\delta_{3}U+\delta_{3}V.$$
\end{lem}
\begin{proof}
As $d_{2}$ is a dimensionfunction, on one hand we have
$$d_{2}(U_{1}+V_{1})=d_{2}(\cle(U_{1}\cup V_{1}))=d_{2}(U_{1}\cup V_{1})\leq
d_{2}U+d_{2}V-d_{2}(U\cap V).$$
Since $\delta_{3}(U+V)=d_{2}(U+V)-\dfp(N^{3}(U+V))$ we only have to show
$$\dfp(N^{3}(U_{1}+V_{1})\geq\dfp(N^{3}(U))+\dfp(N^{3}(V))-\dfp(N^{3}(U\cap V)).$$
Let $W=\fr{3}\tr{2}M$ and $N^{3}=N^{3}(M)$, we have
$$
N^{3}\cap\gena{U_{1}+V_{1}}{W}\supseteq
N^{3}\cap\left(\gena{U_{1}}{W}\right)+N^{3}\cap\left(\gena{V_{1}}{W}\right).
$$
The previous inclusion, has to be regarded as
between vector spaces in the weight $3$ part of $W$.

On account of lemma \ref{ta-schnitt}, we have that $\tr{2}U\cap\tr{2}V=\gena{U_{1}\cap V_{1}}{\tr{2}M}$,
moreover the definition of $\Klf$ implies that $\tr{2}U$ and $\tr{2}V$ are $\delta_{2}$-strong
embeddable in $\tr{2}M$, thus we can apply lemma \ref{bellemmino} and obtain
$\gena{U_{1}}{W}\cap\gena{V_{1}}{W}=\gena{U_{1}\cap V_{1}}{W}$.
Therefore we have
\begin{multline*}
\dfp(N^{3}(U_{1}+V_{1}))\geq\dfp(N^{3}\cap\gena{U_{1}}{W}+N^{3}\cap\gena{V_{1}}{W})=\\
=\dfp(N^{3}\cap\gena{U_{1}}{W})+\dfp(N^{3}\cap\gena{V_{1}}{W})-\dfp(\gena{U_{1}}{W}\cap\gena{V_{1}}{W}
\cap N^{3})\geq\\
\geq\dfp(N^{3}(U))+\dfp(N^{3}(V))-\dfp(N^{3}(U\cap V)).
\end{multline*}
This concludes the proof.
\end{proof}

The next lemmas lead to the construction of a $\delta_{3}$-selfsufficient closure.
\begin{lem}
Let $A$ and $B$ be subalgebras of $M\in\Kl^{3}$, $A,B\in\Klf$, and $E_{1}$, be a $\ta$-closed subspace of $M_{1}$,
if $A\dsu B$ then $A_{1}\cap E_{1}\,\dsu\,B_{1}\cap E_{1}$.

The same holds replacing $B$ with $M$.
\end{lem}
\begin {proof}
Take $V_{1}\inn M_{1}$ such that $A_{1}\cap E_{1}\inn V_{1}\inn B_{1}\cap E_{1}$ and $V\in\Klf$.
Observe that $E_{1}\cap A_{1}=V_{1}\cap A_{1}$ and use subadditivity of $\delta_{3}$ to compute
%\begin{multline*}
%\delta_{3}\ta(\ssc_{2}(A_{1}+V_{1}))+\delta_{3}(A_{1}\cap E_{1})\leq\\\leq
$$
\delta_{3}(A_{1}+V_{1})+\delta_{3}(A_{1}\cap E_{1})\leq\delta_{3}A_{1}+\delta_{3}V_{1}\leq\\
\leq\delta_{3}%\ta(\ssc_{2}(A_{1}+V_{1}))
(A_{1}+V_{1})+\delta_{3}V_{1}.
$$
The last equality uses $\delta_{3}A_{1}\leq\delta_{3}(A_{1}+V_{1})$ from
the fact $A\dsu B$. %and $\gena{\ta(\ssc_{2}(A_{1}+V_{1}))}{B}\in\Klf$,

Conclude $\delta_{3}(A_{1}\cap E_{1})\leq\delta_{3}V_{1}$ as desired.
\end{proof}
\begin{cor}
$M\in\Kl^{3}$, $A\dsu B$ an extension of $\Klf$-structures contained in $M$.
If $B\dsu M$ then $A\dsu M$.
\end{cor}
\begin{cor}
Let $A$ and $B$ in $\Klf$ be $\delta_{3}$-strong subalgebras of $M\in\Kl^{3}$,
then  $A_{1}\cap B_{1}\dsu M_{1}$.
\end{cor}

Let $M$ be a $\Kl^{3}$. Consider now a subspace $H_{1}$ of $M_{1}$ with $d_{2}H_{1}<\omega$, then define
$$\ssc_{3}H_{1}\!\!=\bigcap_{\substack{H_{1}\inn F_{1}\dsu M_{1}\\ F\in\Klf}}\!F_{1}$$
and as usual $\ssc_{3}H=\gena{\ssc_{3}H_{1}}{M}$.

We observe that $\ssc_{3}H$ is the smallest $\Klf$-algebra above $H$ which is $\delta_{3}$-strong
in $M$.

\subsection{The theory $T^{3}$}
A crucial definition
\begin{dfn}
A structure $K$ in $\Kl^{3}$ is \emph{rich} if for each $\delta_{3}$-strong extension $N\dsu M$ of
$\Klf$-algebras, if $N\dsu K$ then there is a strong embedding of $M$ in $K$ over $N$.
\end{dfn}

As the trivial subalgebra $\triv$ is $\delta_{3}$-strong in each $\Klf$-algebra,
a rich structure of $\Kl^{3}$ is also $\Klf$-universal. A rich structure $K\in\Kl^{3}$ is also homogeneous
with respect to $\delta_{3}$-strong $\Klf$-subalgebras.

We prove amalgamation property for the class $\Klf$, this yields the existence
of a rich structure $\K^{3}$ in $\Kl^{3}$.
\begin{lem}[Asymmetric Amalgama]
Let $A,B$ and $C$ in $\Klf$. Let $B=\gena{B_{1}}{C}$ and $B\dsu A$ then there exists a $D$ in
$\Klf$ such that $C\dsu D$ and $A=\gena{A_{1}}{D}$.

Moreover if $B$ is $\delta_{3}$-strong in $C$ too, then also $A\dsu D$ holds.
\end{lem}
\begin{proof}
If we denote with $A^{\prime}$, $B^{\prime}$ and $C^{\prime}$, respectively $\tr{2}A$, $\tr{2}B$ and
$\tr{2}C$ then, after the definition of $\Klf$, we have $A^{\prime}\zso B^{\prime}\zsu C^{\prime}$.
We build the free amalgam $D^{\prime}=\fram{A^{\prime}}{B^{\prime}}{C^{\prime}}$ of $A^{\prime}$ and $C^{\prime}$ over $B^{\prime}$ and we embed $\delta_{2}$-strongly $D^{\prime}$ in $\K^{2}$ over $B^{\prime}$,
we keep calling $D^{\prime}$ the isomorphic image in the generic model of $T^{2}$.

Consider $\ta D^{\prime}$ the $\ta$-closure of $D^{\prime}$ in $\K^{2}$,
because both $A^{\prime}$ and $C^{\prime}$ are selfsufficient in $D^{\prime}$
and $D^{\prime}\zsu\ta D^{\prime}$, we have that $A^{\prime}$ and $C^{\prime}$ are both $\delta_{2}$-strong
in $\ta D^{\prime}$. Moreover $\ta$-closure implies
\begin{labeq}{inter}
B^{\prime}=\gena{B_{1}}{\ta D^{\prime}}=\gena{\abu}{\ta D^{\prime}}\cap\gena{\cbu}{\ta D^{\prime}}.
\end{labeq}

Now on account of Lemma \ref{bellemma},
there exist two embeddings $j_{A}$ and $j_{C}$ of $\fr{3}\tr{2}A$ and $\fr{3}\tr{2}C$
into $\fr{3}\ta D^{\prime}$ their images being $\gena{\abu}{\fr{3}\ta D^{\prime}}$ and $\gena{\cbu}
{\fr{3}\ta D^{\prime}}$ respectively. From \pref{inter} and lemma \ref{bellemmino} instead, we get
\begin{labeq}{freeint}
\gena{B_{1}}{\fr{3}\ta D^{\prime}}=\gena{\abu}{\fr{3}\ta D^{\prime}}\cap\gena{\cbu}{\fr{3}\ta D^{\prime}}
\end{labeq}
here $B_{1}$, $\abu$ and $\cbu$ are identified with its images modulo $j_{A}$ and $j_{C}$. 

We set $N^{3}(A)^{\bullet}=j_{A}(N^{3}(A))$ and $N^{3}(C)^{\bullet}=j_{C}(N^{3}(C))$.
And we build
$$D=\fr{3}\ta D^{\prime}\quot(N^{3}(A)^{\bullet}+N^{3}(C)^{\bullet}).$$
As $N^{3}(A)^{\bullet}+N^{3}(C)^{\bullet}$ is an ideal consisting only of weight $3$ elements,
$\fr{3}\tr{2}D=\fr{3}(\tr{2}\fr{3}\ta D^{\prime})%\simeq
=\fr{3}\ta D^{\prime}$, hence we obtain $N^{3}(D)=N^{3}(A)^{\bullet}+N^{3}(C)^{\bullet}$.

From \pref{freeint} we have that
$$N^{3}(B)^{\bullet}:=j_{A}(N^{3}(B))=N^{3}(A)^{\bullet}\cap N^{3}(C)^{\bullet}=j_{C}(N^{3}(B))$$
and the following compatibility equations
\begin{labeq}{compa}
N^{3}(A)^{\bullet}=N^{3}(D)\cap\gena{\abu}{\fr{3}\ta D^{\prime}}
\end{labeq}and
$$
N^{3}(C)^{\bullet}=N^{3}(D)\cap\gena{\cbu}{\fr{3}\ta D^{\prime}}.
$$

So far, are $A$ and $C$ embeddable in $D$ via Lie Algebra monomorphisms,
just regard $A\simeq\fr{3}\tr{2}A\quot N^{3}(A)$ and take the quotient of $j_{A}$:
\begin{eqnarray}
\map{\bar j _{A}}{&\fr{3}\tr{2}A\quot N^{3}(A)}{D}\\
&\bar w\longmapsto \bar w
\end{eqnarray}
on account of \pref{compa} is $\bar j_{A}$ one-to-one, in particular its image is again
$\gena{\abu}{D}=(\gena{\abu}{\fr{3}\ta D^{\prime}}+N^{3}(D))\quot N^{3}(D)$.\quad Do the same for $C$.

\medskip
We show next that if $B_{1}\dsu A_{1}$ then the embedding of $C$ into $D$ is $\delta_{3}$-strong.
We have to prove $\delta_{3}E_{1}\geq\delta_{3}C_{1}$ for each $E_{1}\zsu\ta D_{1}$ such that
$E_{1}\nni\cbu$ and $E_{1}$ $\ta$-closed.

We first observe $\delta_{3}E_{1}\geq\delta_{3} E_{1}\cap D_{1}$. This is because $d_{2}E_{1}\geq
d_{2}E_{1}\cap D_{1}$ and
\begin{multline}\label{ntreonto}
N^{3}(E_{1})=\\
=N^{3}(D)\cap\gena{E_{1}}{\fr{3}\ta D^{\prime}}=
(N^{3}(A)+N^{3}(C))\cap\gena{E_{1}}{\fr{3}\ta D^{\prime}}=\\
=(N^{3}(A)\cap\gena{E_{1}}{\fr{3}\ta D^{\prime}})+N^{3}(C)=\\
=(N^{3}(D)\cap\gena{A_{1}}{\fr{3}\ta D^{\prime}}\cap\gena{E_{1}}{\fr{3}\ta D^{\prime}})+N^{3}(C)=\\
=(N^{3}(D)\cap\gena{A_{1}\cap E_{1}}{\fr{3}\ta D^{\prime}})+N^{3}(C)\inn\\
\inn N^{3}(E_{1}\cap A_{1}+C_{1})\inn N^{3}(E_{1}\cap D_{1})
\end{multline}
Therefore we have to show $\delta_{3}E_{1}\cap D_{1}\geq\delta_{3}C_{1}$.

As $E_{1}\cap D_{1}=C_{1}+(E_{1}\cap \abu)$ we finish once we show
$\delta_{3}(E_{1}\cap D_{1}\quot C)=\delta_{3}(E_{1}\cap\abu\quot B)$ because
%$E_{1}\cap \abu$ is a $\delta_{2}$ strong subspace of $\abu$ and 
$B\dsu A$ and $\delta_{3}(E_{1}\cap \abu\quot B)\geq 0$.

We have
%$$\delta_{3}(E_{1}\cap D_{1}\quot C)
$$\delta_{3}(E_{1}\cap D_{1})-\delta_{3}(C_{1})=
d_{2}(E_{1}\cap D_{1}\quot C_{1})
-\big(\dfp(N^{3}(E_{1}\cap D_{1}))-\dfp(N^{3}(C))\big).$$
Everything behaves good for $d_{2}$, we have $d_{2}(E_{1}\cap D_{1})-d_{2}(C_{1})=
d_{2}(E_{1}\cap\abu)-d_{2}B_{1}$. (\dots eigentlich stimmt das f`\"ur $\delta_{2}$ aber
man kann approximieren \dots)

To conclude we have to show
$$N^{3}(E_{1}\cap\abu)\quot N^{3}(B)^{\bullet}\simeq
N^{3}(C_{1}+(E_{1}\cap \abu))\quot N^{3}(C)^{\bullet}.$$
Consider the map
\begin{eqnarray*}
N^{3}(E_{1}\cap \abu)\quot N^{3}(B)^{\bullet}&\longrightarrow&N^{3}(E_{1}\cap D_{1})\quot N^{3}(C)^{\bullet}\\
\bar\eta&\longmapsto&\bar\eta\quad\quad\forall\eta\in N^{3}(E_{1}\cap\abu).
\end{eqnarray*}
Soundness is immediate. Moreover our map is injective because
\begin{multline*}
N^{3}(C)^{\bullet}\cap\gena{E_{1}\cap \abu}{\fr{3}\ta D^{\prime}}=\\
=N^{3}(D)\cap\gena{\cbu}{\fr{3}\ta D^{\prime}}\cap\gena{E_{1}\cap \abu}{\fr{3}\ta D^{\prime}}=\\
=N^{3}(D)\cap\gena{\abu\cap C_{1}}{\fr{3}\ta D^{\prime}}=N^{3}(B)^{\bullet}.
\end{multline*}
Here we used $\gena{\cbu}{\fr{3}\ta D^{\prime}}\cap\gena{E_{1}\cap\abu}{\fr{3}\ta D^{\prime}}=
\gena{\cbu\cap\abu}{\fr{3}\ta D^{\prime}}$ because both $\cbu$ and $E_{1}\cap\abu$ are $\delta_{2}$ strong in $D_{1}$ and $\ta$ closed and because $\gena{\cbu}{\ta D^{\prime}}\cap\gena{E_{1}\cap\abu}{\ta D^{\prime}}=\gena{\bbu}{\ta D^{\prime}}$.

To see that this map is onto use the arguments in \pref{ntreonto} again to get
$$N^{3}(E_{1}\cap D_{1})=(
N^{3}(D)\cap\gena{E_{1}\cap A_{1}}{\fr{3}\ta D^{\prime}})+N^{3}(C)^{\bullet}.$$
%On the other hand, since again $\gena{\abu}{\ta D^{\prime}}\cap\gena{E_{1}}{\ta D^{\prime}}=\gena{E_{1}\cap\abu}{\ta D^{\prime}}$ we have
%\begin{multline*}
%N^{3}(E_{1})=N^{3}(D)\cap\gena{E_{1}}{\fr{3}\ta D^{\prime}}=\\
%=(N^{3}(A)^{\bullet}+N^{3}(C)^{\bullet})\cap\gena{E_{1}}{\fr{3}\ta D^{\prime}}=\\
%=(N^{3}(A)^{\bullet}\cap\gena{E_{1}}{\fr{3}\ta D^{\prime}})+N^{3}(C)^{\bullet}=\\
%=(N^{3}(D)\cap\gena{E_{1}\cap A_{1}}{\fr{3}\ta D^{\prime}})+N^{3}(C)^{\bullet}
%\end{multline*}
%and this gives that our map is onto.

We have shown that
$$\dfp(N^{3}(E))-\dfp(N^{3}(C))=\dfp(N^{3}(E_{1}\cap\abu))-\dfp(N^{3}(B)^{\bullet})$$
thus $C$ is $\delta_{3}$-strong embeddable in $D$.

A completely analogous argument proves that $A$ is $\delta_{3}$-strong in $D$,
provided $B$ is a $\delta_{3}$-strong substructure in $C$.

\smallskip
To conclude the proof we show that the algebra $D$ we constructed has $\sig{3}{2}^{\prime}$.

Pick a subspace $A_{1}\zsu\ta D_{1}$ of finite linear dimension. We have
%\begin{multline*}
$$
\delta_{3}(\ta A_{1}+C_{1})+\delta_{3}(\ta A_{1}\cap C_{1})
\leq\delta_{3}\ta A+\delta_{3}C_{1}\leq\delta_{3}A_{1}+\delta_{3}(\ta A_{1}+C_{1})
$$
%\end{multline*}
Here we use again $C_{1}\dsu\ta D_{1}$ and $\delta_{3}A_{1}\geq\delta_{3}\ta A_{1}$.

Because $\delta_{3}(\ta A_{1}\cap C_{1})\geq2$, then $\delta_{3}A_{1}\geq2$.
\end{proof}
To deal with the axioms of $T^{3}$ that we are going to define, we have to introduce a new auxiliary
class in which we consider structures of finite linear dimension.

Define
$$\Klz=\{A=\gen{A_{1}}\mid\tr{2}A\in\Kl^{2}_\textsf{0},A\,\text{has}\,\sig{3}{2}^{\prime}\,\text{and}\,\sig{3}{3}\}.$$

If $A\in\Klz$ and $A\inn B$ with $B$ in $\Klz$ or in $\Kl^{3}$, we say that $A_{1}$
is $\delta_{3}$-strong in $B_{1}$ if $\tr{2}A\zsu\tr{2}B$ and if for each  $C_{1}\inn_{\omega}
B_{1}$ with $C_{1}\nni A_{1}$, we have $\delta_{3}A_{1}\leq\delta_{3}C_{1}$.
We write equivalently $A_{1}\dsu B_{1}$ or $A\dsu B$. If $B\in\Klf$ we call
$A\quot B$ a $\Klf$-extension.

We list now the axiom system for $T^{3}$ and enunciate the main theorem. An $\mathcal{L}$-structure
$M$ is amodel for $T^{3}$ if
\begin{itemize}
\punto{$\sig{3}{1}$}$M$ is in $\nla{3}$ and ${\tr{2}M\upharpoonright}_{\mathcal{L}^{2}}\sat T^{2}$ (\emph{vielleicht reichen wenige Axiome davon})
\punto{$\sig{3}{2}$}$(\forall \bar x,\,P_{1}(\bar x))(\delta_{2}\gen{\bar x}-\dfp N^{3}(\gen{\bar x})\geq2$)
\punto{$\sig{3}{3}$}$(\forall z,\,P_{2}(z))(\forall x,y,\,P_{1}(x)\wedge P_{1}(y))([z,x]=[z,y]\,\rightarrow
\textsl{``$x$ lin.{}depends on $y$''})$
\punto{$\sig{3}{4}$}$(\forall y,\,P_{3}(y))(\forall z,\,P_{2}(z))(\exists x)([z,x]=y)$
\punto{$\sig{3}{5}$}For each extension $A\quot B$ in $\Klz$ such that $\delta_{3}A\quot B=0$,
if $B\inn M$ then there exist a $B$-isomorphic copy of $A$ in $M$.
\end{itemize}
\begin{teo}\label{richmodel}
Let $K$ be a structure in $\Kl^{3}$, then $K$ is rich if and only if $K$ is an $\omega$-saturated model
of $T^{3}$.
\end{teo}

In order to prove the theorem we give an uniform way to switch between the classes $\Klf$ and $\Klz$.

Consider a structure $A\in\Klz$, because $\tr{2}A\in\Kl^{2}_\textsf{0}$ and $\K^{2}$ is $\Kl^{2}$-universal, we can identify $A_{1}$ with
a $\delta_{2}$-strong subspace of $\K^{2}_{1}$, so that $\tr{2}A=\gena{A_{1}}{\K^{2}}$.
If we consider the $\delta_{2}$-strong embedding of
$\tr{2}A$ into $\ta(\tr{2}A)=\gena{\ta A_{1}}{\K^{2}}$, we obtain an embedding of $\fr{3}\tr{2}A$ into $\fr{3}\ta(\tr{2}A)$. So we can define
\begin{labeq}{starring}
A^{*}=\fr{3}\ta(\tr{2}A)\quot N^{3}(A).
\end{labeq}
As $\ta\tr{2}A$ is $\delta_{2}$-strong in $\K^{2}$ and $d_{2}A_{1}=d_{2}\ta A_{1}$ it is not difficult to prove that $A^{*}$ belongs to $\Klf$. Moreover $A^{*}$ share the same predimensions of $A$.
\begin{lem}\label{star}
Let $A,B\in\Klz$ and $M\in\Kl^{3}$ then the following holds
\begin{itemize}
\punto{i}$A^{*}$ belongs to $\Klf$, $\delta_{3}A^{*}=\delta_{3}A$ and $d_{2}A^{*}=d_{2}A$
\punto{ii}If $A\dsu B$ then $A^{*}\dsu B^{*}$
%\punto{iii}If $A\dsu M$, then $A^{*}\dsu M$.
\end{itemize}
\end{lem}

Now consider  $M\in\Klf$ we say that $H_{1}\inn M_{1}$ is a \emph{core} of $M$ if $\gena{H_{1}}{M}\in\Klz$,
$H_{1}\zsu M_{1}$, $d_{2}H_{1}=d_{2}M_{1}$ and $N^{3}(M)\inn\gena{H_{1}}{\fr{3}\tr{2}M}$.
It is immediate to see that if $H_{1}$ is a core of $M$, then for each $H_{1}\inn V_{1}\inn M_{1}$, then
$\delta_{3}M_{1}=\delta_{3}V_{1}$.

Moreover we have
\begin{lem}\label{chistocore}
Let $H\in\Klf$, $H\inn M$ for some $M\in\Kl^{3}$ and $H_{1}^{\prime}$ be a core of $H$.
Then $H\dsu M$ if and only if $H^{\prime}$ is strong in $M$.
\end{lem}

There is a natural way to extract cores: consider a structure $H\in\Klf$, because $\delta_{3}H\geq2$ the space $N^{3}(H)$ must
be finitely generated in $\fr{3}\tr{2}H$. So call $S_{1}$ the minimal subspace of $H_{1}$,
such that $N^{3}(H)\inn\gena{S_{1}}{\fr{3}\tr{2}H}$, now as $d_{2}H_{1}$ is finite, let $B_{1}$
be a finite $d_{2}$-basis of $H_{1}$. Now take $H^{c}_{1}$ the selfsufficient closure of $S_{1}+B_{1}$
in $H_{1}$ and call $H^{c}$ the subalgebra $\gena{H^{c}_{1}}{H}$. As $\gena{H^{c}_{1}}{\K^{2}}\zsu\K^{2}$,
we can easily see that $H^{c}\in\Klz$. %We call $H^{c}$ is the \emph{standard core}

A $\delta_{3}$-strong extension $M\quot N$ of algebras in $\Klf$ is \emph{minimal} if there is no algebra
$M^{\prime}\dsu M$ in $\Klf$ such that $N\subsetneq M^{\prime}\subsetneq M$ or equivalently if for
each $N_{1}\subsetneq M^{\prime}_{1}\subsetneq M_{1}$ which is $\ta$-closed and $\delta_{2}$-strong in $M_{1}$,
we have $\delta_{3}M_{1}^{\prime}\geq\delta_{3}M_{1}$.

\begin{dfn}
A structure $K$ in $\Kl^{3}$ is \emph{poorly rich} if for each $A,B\in\Klz$ such that $B\dsu A$ and $B\dsu K$,
there exist a $\delta_{3}$-strong embedding of $A$ in $K$ over $B$.
\end{dfn}
\begin{prop}\label{richnotrich}
A structure $K\in\Kl^{3}$ is rich exactly when it is poorly rich.
\end{prop}
\begin{proof}
Assume first $K$ is rich and $A\quot B$ is an extension of algebras in $\Klz$ such that $B\dsu M$.
Consider $A^{*}\nni A$ as constructed in \pref{starring} being careful of choosing a $\delta_{2}$-strong embedding $B_{1}\inn A_{1}\hookrightarrow \K^{2}_{1}$ such that $\gena{\ta B_{1}}{\tr{2}K}\zsu\gena{\ta A_{1}}{\K^{2}}$. As $d_{2}\ta(B_{1})=d_{2}B_{1}$ and
$B\dsu K$, we have $N^{3}(\ta(B_{1}))=N^{3}(B_{1})$, then apply lemma \ref{star}.(ii) to
$B^{*}\simeq\fr{3}\gena{\ta B_{1}}{\tr{2}K}\quot N^{3}(B)=\gena{\ta B_{1}}{K}$. This yields $\gena{\ta B_{1}}{K}\dsu A^{*}$. Now conclude via richness that there exists a $\delta_{3}$-strong embedding of $A^{*}$ in $K$ over $\gena{\ta B_{1}}{K}$
and in particular, a $B$-isomorphic image of $A$ in $K$, which is strong in $K$ because it is a core for $A^{*}$.

\medskip  
Let $K$ be a poorly rich $\Kl^{3}$-structure and $M\quot N$ be a minimal $\Klf$-extension with $N\dsu K$.

Isolate a core $N_{1}^{c}$ of $N$ then put $M_{1}^{c}\colon=\ssc_{2}^{M}(N_{1}^{c},B_{1},S_{1})$ where
$B_{1}$ is a $d_{2}$-basis of $M_{1}$ over $N_{1}$ and $N^{3}(M)\inn\gena{N^{c}_{1},S_{1}}{\fr{3}\tr{2}M}$.
Then $M_{1}^{c}$ is a core for $M$ and $N_{1}^{c}$ can be chosen so that $N_{1}^{c}=M_{1}^{c}\cap N_{1}$.

We prove next that $N_{1}+M_{1}^{c}\zsu M_{1}$ with respect to the predimension on  $\tr{2}M$ induced by
$\K^{2}$. On this purpose it suffices to show $U_{1}+M_{1}^{c}\zsu M_{1}$ for each $\delta_{2}$-strong
finitely generated subspace $U_{1}$ of $N_{1}$ which contain $N_{1}^{c}$. Let $U_{1}$ be such a space
and $C_{1}\nni U_{1}+M_{1}^{c}$ then $\delta_{2}C_{1}-\delta_{2}(U_{1}+M_{1}^{c})\geq\delta_{2}C_{1}-
\delta_{2}U_{1}-\delta_{2}M_{1}^{c}+\delta_{2}(U_{1}\cap M_{1}^{c})\geq\delta_{2}N_{1}^{c}-\delta_{2}U_{1}=0$
because $M_{1}^{c}\zsu M_{1}$, $U_{1}\cap M_{1}^{c}=N_{1}^{c}$ and $\delta_{2}N_{1}^{c}=\delta_{2}U_{1}$.

It follows $\gena{\ta(N_{1}+M_{1}^{c})}{M}\supsetneq N$ lies in $\Klf$ and share the same $\delta_{3}$ of
$M$. Then by minimality
$$M=\gena{\ta(N_{1}+M_{1}^{c})}{M}=\fr{3}\gena{\ta(N_{1}+M_{1}^{c})}{\tr{2}M}\quot N^{3}(M_{1}^{c}).$$
In other words
%Now as $\gena{\ta\ssc_{2}(N_{1},B_{1},S_{1})}{\tr{2}M}\zsu\tr{2}M$ then
%$\fr{3}\gena{\ta\ssc_{2}(N_{1},B_{1},S_{1})}{\tr{2}M}$ embeds in $\fr{3}\tr{2}M$, so we can consider
%the ideal $N^{3}(M)=N^{3}(M^{c}_{1})$ as in $\fr{3}\gena{\ta\ssc_{2}(N_{1},B_{1},S_{1})}{\tr{2}M}$ and build
%$$M^{\prime}\colon=\fr{3}\gena{\ta\ssc_{2}(N_{1},B_{1},S_{1})}{\tr{2}M}\quot N^{3}(M_{1}^{c}).$$
%As $M\quot N$ is minimal and $\delta_{3}(M\quot M^{\prime})=0$ then $M=M^{\prime}$.
$M$ is generated
over $N$ by $M_{1}^{c}$ up to the operator $\ta\ssc_{2}$.

Put $N^{c}=\gena{N^{c}_{1}}{N}$ and $M^{c}=\gena{M^{c}_{1}}{M}$,
lemma \ref{chistocore} implies $N^{c}\dsu M^{c}$ and $N^{c}\dsu K$.
Therefore, as $K$ is poorly rich, there exists a $V_{1}\dsu K_{1}$ such that
$\gena{V_{1}}{K}\simeq_{N^{c}}M^{c}$. But then $\gena{V_{1}}{K}\simeq_{N}M^{c}$ also holds.

We discuss now two cases: assume first $\delta_{3}(M\quot N)=0$. Because $N\dsu K$ and $\delta_{3}(V_{1}\quot N_{1})=0$ it follows $\gena{N_{1}+V_{1}}{K}\dsu K$, then, as $d_{2}(N_{1}+V_{1})=
d_{2}\ta(N_{1}+V_{1})$, we have  $N^{3}(V_{1}+N_{1})$=$N^{3}(\ta(V_{1}+N_{1}))$.

We conclude noting
$$\gena{\ta(N_{1}+V_{1})}{K}=\fr{3}\gena{\ta(N_{1}+V_{1})}{\tr{2}K}\quot N^{3}(V_{1})\simeq_{N}M.$$

The case of a free minimal extension, that is $\delta_{3}(M\quot N)>0$, can be worked
out in a similar fashon.

%Now if $A_{1}\quot B_{1}$ is free minimal, that is $\delta_{3}(A\quot B)>0$.
%Then we claim $d_{2}A=d_{2}B+1$ and $A_{1}\quot B_{1}$ is also $\delta_{2}$ minimal.

%We must however have $d_{2}A>d_{2}B$, assume $d_{2}A>d_{2}B+1$, then there exist an element $a\in A_{1}$,
%such that $A_{1}\supsetneq\ssc_{2}(B_{1},a)\supsetneq B_{1}$. Now from minimality it follows
%$\delta_{3}\ssc_{2}(B_{1},a)>\delta_{3}A>\delta_{3}B$. But for each $a\in A_{1}$ we have $\delta_{3}\ssc_{2}
%(B_{1},a)\leq\delta_{3}B+1$. And this cannot happen.
%In particular it follows $\delta_{3}A=\delta_{3}B+1$ and $\dfp(N^{3}(A))=\dfp(N^{3}(B))$.

%Now if there exists $A_{1}\supsetneq H_{1}\supsetneq B_{1}$ with $A_{1}\zso H_{1}$, since
%$N^{3}(A)=N^{3}(B)$, then $\delta_{3}H$ is either equal to $\delta_{3}A$ or to $\delta_{3}B$. In both cases
%that is against minimality. This proves minimality of $A_{1}$ over $B_{1}$ with respect to $\delta_{2}$,
%therefore $A_{1}=\gen{B_{1},a}$ for some $a$ in $A_{1}$ free from $B_{1}$.
\end{proof}

\medskip
{\bf Soweit, gelten noch manche schwierigkeiten wobei ich noch nicht zu recht gekommen bin!
Deswegen werde ich diese in einem kleinen (Di)lemma isolieren, das nicht so schwer zu beweisen
scheint.}
\begin{lem}\label{dilemma}
Assume $K\in\K^{3}$ and $A\quot B$ a prealgebraic minimal extension of $\Klz$, with $B \inn K$
as a subalgebra. then there exists a $\Klf$ extension $A^{\prime}\quot B^{\prime}$ such that
$A\inn A^{\prime}$ and $B\inn B^{\prime}$ and $B^{\prime}\inn K$.

Of course we require the inclusion $B^{\prime}\dsu A^{\prime}$ to preserve $B_{1}\dsu A_{1}$.
\end{lem}

We are now able to prove the theorem.
\begin{proofof}{Theorem \ref{richmodel}}

After proposition \ref{richnotrich} it is sufficient to prove\\
{(a)} a rich structure $K\in\Kl^{3}$ is an $\omega$-saturated model of $T^{3}$ and\\
{(b)} an $\omega$-saturated model $M$ of $T^{3}$ is a poorly rich algebra of $\Kl^{3}$.

\medskip
To prove {(a)} we first note that a $\Kl^{3}$-algebra is an $\mathcal{L}^{3}$-structure and,
because $K$ is %$\K^{3}$-universal and $\dsu$-homogeneous, that $\
rich, then $\tr{2}K$ is a rich structure in $\Kl^{2}$ and in particular a model of $T^{2}$.
Therefore $K\sat\sig{3}{1}$.

Axiom $\sig{3}{3}$ we get at once.
To prove $\sig{3}{2}$, instead, just consider an arbitrary finite set
$X$ in $K_{1}$, then, because $K$ has property $\sig{3}{2} ^{\prime}$, we have $\delta_{2}\gen{X}-\dfp N^{3}(\gen{X})\geq
d_{2}X-\dfp N^{3}(\gen{X})\geq\delta_{3}\ssc_{2}\gen{X}\geq2$.

Consider now a $\Klz$-extension $A\quot B$ such that $\delta_{3}(A\quot B)=0$ and $B$ is a subalgebra
of $K$, $B=\gena{B_{1}}{K}$.

%Now construct $B^{*}$ and $A^{*}$ as in \pref{starring}. Because $K$ is $\Klf$-homogeneous and $\Klf$-universal,
%we can assume that $K\nni B^{*}\nni B$ while on the other side we have $B^{*}\dsu A^{*}$ by lemma \ref{star}.
Now find $A^{\prime}$ and $B^{\prime}$ as in lemma \ref{dilemma} and 
put $F=\ssc_{3}B^{\prime}\dsu K$. As $F\nni B^{\prime}\dsu A^{\prime}$ we can build, via asymmetric amalgamation, an algebra $F^{\prime}\in\Klf$ such
that $F\dsu F^{\prime}$ and $A\inn F^{\prime}$. Now by richness, we strongly embed $F^{\prime}$ in $K$ over $F$. We obtain in particular an embedding of $A$ in $K$ over $B$.
This proves $K\sat\sig{3}{5}$.

Axiom $\sig{3}{4}$ follows from (poor) richness and will be proved separately. It is indeed a consequence
of axioms $\sig{3}{i}$ for $i=1,2,3,5$.

We have proven $K\sat T^{3}$, now take an $\omega$-saturated model $K^{\prime}$ of $T^{3}$,
as {(b)} implies that $K^{\prime}$ is poorly rich, then $K$ and $K^{\prime}$ are back-and-forth equivalent.
This yields that $K$ is $\omega$-saturated too.

\medskip
Now to prove {(b)} assume $M\sat T^{3}$ and $M$ is $\omega$-saturated. Suppose $A\quot B$ is
a strong extension in $\Klz$, which is minimal prealgebraic and with $B\dsu M$.
On account of axiom $\sig{3}5{}$, in particular follows that we can realise $A$ in $M$ over $B$.
Now, because $\delta_{3}(A\quot B)=0$, the image of $A$ will be also strong in $M$.

If $A\quot B$ is free minimal, we can show that
also $\tr{2}A\quot\tr{2}B$ is free minimal, then it must be of the form $A_{1}=\gen{B_{1},a}$ for some
$a$ which is $d_{2}$-free from $B_{1}$.

Now because $M$ is saturated and \lqq$x$ $\delta_{2}$-free from $B_{1}$ and $\delta_{3}$-free from
$B_{1}$\rqq  is (a part of) a type over the finite set $B_{1}$, we can realise $a$ in $M$ over $B_{1}$.

So far $M$ is an $\nla{3}$ structure which has $\sig{3}{2}^{\prime}$ and $\sig{3}{3}$ and which
is poorly rich. To see that $M$ belongs to $\Kl^{3}$ we note that in our case, $\tr{2}M$ is an $\omega$-
saturated model of $T^{2}$, then conclude $\tr{2}M\simeq\K^{2}$.
\end{proofof}