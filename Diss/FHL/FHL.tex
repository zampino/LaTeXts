\documentclass[a4paper,10pt,ngerman,english]{report}
\linespread{1.1}
\usepackage{babel}
\usepackage[latin1]{inputenc}
\usepackage{amsmath,amsfonts,amssymb,amsthm}
%\usepackage{eqname}
%\usepackage{boldtensors} 
%\usepackage{frcursive}
%\usepackage{calligra}
\usepackage{ModNet}
\usepackage{FHL}
\usepackage{mathrsfs}
\usepackage[matrix,arrow]{xy}
\usepackage{graphicx}
\usepackage{stackrel}
%\usepackage{pdfsync}
\usepackage{tocbibind}
\usepackage[normalem]{ulem}
%--------------------------------------------------------------------------CB--------------------------------------------
%\usepackage[leftbars,color]{changebar}
%\cbcolor{black}
%\setlength{\changebarwidth}{1pt}
%\setlength{\changebarsep}{1.4cm}
%-------------------------------------------------MARGIN NOTES---------------------------------------------------------------
%\usepackage{marginnote}
%\setlength\marginparsep{1cm}
%\renewcommand*{\marginfont}{\sffamily\small} %\mdseries} %{\sffamily}
%\reversemarginpar
%                                          overwrite \mn to nothing 
%\renewcommand{\mn}[1]{}
%-------------------------------------------------COMMENT::::VERSIONS----------------------------------------------------
%\usepackage{comment}
%\includecomment{lemmaRMU}
%\excludecomment{lemmaRMU}
%\includecomment{provacasanova}
%\excludecomment{provacasanova}
%\excludecomment{cfrbad}
%\includecomment{weicm}
%\excludecomment{weicm}
%-----------------------------------------------
%\renewcommand{\thesection}{\arabic{section}} 
%\setcounter{tocdepth}{3}
%\usepackage{syntonly}
%\syntaxonly
\usepackage[left]{showlabels}
\usepackage[naturalnames,colorlinks=true,linkcolor=black,citecolor=black]{hyperref}
%-------------------------------------------------------------------------------------
\title{Hrushovski Predimensions on nilpotent Lie Algebras}
%title{Fra\"{i}ss�\,-Hrushowski Limits of Lie Algebras}
\author{Andrea Amantini}
%\date{}
%-----------------------------------------------------------------------------------
\begin{document}
%   COMANDI PER definitiva.tex COMMENTATI QUI SOTTO: (RIMANE IL GRUPPO \newpage\thisp?\cleardoubl?)
%\pagenumbering{roman}
%\setcounter{page}{1}
\maketitle
%\selectlanguage{ngerman}
\xsubsection{\abstractname}
\medskip
\begin{minipage}{.9\textwidth}
\setlength{\parindent}{2ex}
In dieser Arbeit wird das Fra\"iss\'e-Hrushowskis Amalgamationsverfahren in Zusammenhang
%mit nilpotenten Gruppen von endlichem Exponent beziehungsweise
mit nilpotenten graduierten Lie Algebren \"uber einem endlichen K\"orper untersucht.

Die Pr\"adimensionen die in der Konstruktion auftauchen sind mit dem gruppentheoretischen Begriff der {\em Defizienz}
zu vergleichen, welche auf homologische Methoden zur\"uckgef\"uhrt werden kann.

Dar�ber hinaus wird die Magnus-Lazardsche Korrespondenz zwischen den oben genannten Lie Algebren
und nilpotenten Gruppen von Primzahl-Exponenten beschrieben.
Dabei werden solche Gruppen durch die Baker-Haussdorfsche Formel
in den entsprechenden Algebren definierbar interpretiert.

\medskip
Es wird eine $\omega$-stabile Lie Algebra von Nilpotenzklasse 2 und Morleyrang $\omega\cdot2$ erhalten, indem
man eine {\em unkollabierte} Version der von Baudisch konstruierten {\em new uncountably categorical group} betrachtet.
Diese wird genau analysiert. Unter anderem wird die Unabh\"angigkeitsrelation des Nicht-Gabelns durch die
Konfiguration des freien Amalgams charakterisiert.
%Ziel dieser Arbeit ist eine Erweiterung auf h\"oheren Nilpotenzklassen der von Baudisch
%konstruierten nil-2 {\sl } von prim Exponent.

\smallskip
Mittels eines induktiven Ansatzes werden die Grundlagen entwickelt, um neue Pr\"adimensionen f\"ur Lie Algebren der Nilpotenzklassen
gr\"o\ss er als zwei zu schaffen.
%F\"ur den nil-3 Fall geben wir eine Notion einer {\em selbstgen\"ugende} Erweiterung;
%damit wird ein erstes Amalgamationslemma bewiesen.

Dies erweist sich als wesentlich schwieriger als im Fall 2.
Wir konzentrieren uns daher auf die Nilpotenzklasse 3, als Induktionsbasis des oben genannten Prozesses.

In diesem Fall wird die Invariante der Defizienz
auf endlich erzeugte Lie Algebren adaptiert. Erstes Hauptergebnis der
Arbeit ist der Nachweis dass diese Definition zu einem vern\"uftigen Begriff selbst-gen\"ugender
Erweiterungen von Lie Algebren f\"uhrt und
sehr nah einer gew\"unschten Pr\"adimension im Hrushovskischen Sinn ist.

Wir zeigen -- als zweites Hauptergebnis -- ein erstes Amalgamationslemma bez\"uglich
selbst-gen\"ugender Einbettungen.
\end{minipage}
\newpage
\thispagestyle{empty}
\cleardoublepage
\selectlanguage{english}
\xsubsection{\abstractname}
\medskip
\begin{minipage}{.9\textwidth}
\setlength{\parindent}{2ex}
In this work, the so called Fra\"iss\'e-Hrushowski amalgamation is applied to nilpotent graded Lie algebras
over the $p$-elements field with $p$ a prime. We are mainly concerned
with the {\em uncollapsed} version of the original process.

The predimension used in the construction is compared with
the group theoretical notion of {\em deficiency}, arising from group Homology.

We also describe in detail the Magnus-Lazard correspondence, to switch between the aforementioned Lie algebras
and nilpotent groups of prime exponent.
In this context, the Baker-Hausdorff formula allows such groups to be definably interpreted in the corresponding
algebras.

Starting from the structures 
which led to Baudisch' {\em new uncountably categorical group}, %Fra\"iss\'e-%
%Hrushovski amalgamation is first applied to a suitable class of
%graded 2-nilpotent Lie algebras over the finite field $\Fp$, for $p$ prime.
%for a class of nilpotent graded Lie algebras in higher nilpotency classes
%to support an analogous construction.
%We limit ourself to the {\em uncollapsed} phase of the procedure and obtain
we obtain an $\omega$-stable Lie algebra of nilpotency class 2,
as the countable rich Fra\"iss\'e limit of a suitable class of finite algebras over $\Fp$.

We study the theory of this structure in detail: we show its Morley rank is $\omega\cdot2$ and
a complete description of non-forking independence is given, in terms of free amalgams.

\smallskip
In a second part, we develop a new framework for the construction of
deficiency-predimensions among graded Lie algebras of nilpotency class
higher than $2$. This turns out to be considerably harder
than the previous case. The nil-3 case in particular has been extensively treated, as
the starting point of an inductive procedure.

In this nilpotency class, our main results concern a suitable deficiency function, which behaves
for many aspects like a Hrushovski predimension.
A related notion of {\em self-sufficient} extension is given.

We also prove a first amalgamation lemma with respect to self-sufficient embeddings.
\smallskip
\end{minipage}
\newpage
\thispagestyle{empty}
\cleardoublepage
%\section*{Lebenslauf}
\medskip
\thispagestyle{empty}
\selectlanguage{ngerman}
%\centering
\begin{tabular}{r@{\extracolsep{2em}}p{8.5cm}}
\multicolumn{2}{l}{\bf Pers\"onliche Daten}\\[+1mm]
\hline\\[+1mm]
Name: 					& Andrea Amantini\\
Geburtsort, Datum: 			& Florenz (Italien), 28.01.1980\\
Staatsangeh\"origkeit: 		& italienisch\\
Anschrift:					& Lausitzerstra{\ss}e 37, 10999 Berlin\\
Tel:						& +49 (0)30 3462 4345\\
%Mob:						& +49 (0)176 2829 7792\\
E-Mail:					& amantini@math.hu-berlin.de\\[+4mm]
\multicolumn{2}{l}{\bf Bildungsgang}\\[+1mm]
\hline\\[+1mm]
Okt 2005 - heute		&Doktorand an der Humboldt Universit�t zu Berlin\\ 	
				&im Fach Mathematik, Schwerpunkt Mathematische Logik -- Modelltheorie\\
				&({\sl Betreuer Prof. Dr. A. Baudisch})\\[+3mm]
%Okt 2005 - Sep 2008	&Stipendium Marie Curie FP6 -- MODNET\\
Sep 2008 - Okt 2008			&Forschungsaufenthalt an der {\em Universit� Lyon 1}, Inst. Camille Jordan -- {\sl Lyon, France}\\[+3mm]
1999-2005	&Studium an der {\em Universit� degli Studi di Firenze} -- {\sl Florenz, Italien}\\
		&Studiengang Mathematik, Abschlusspr�fung:
		Diplom ({\sl Laurea in Matematica}) -- Titel der Abschlu{\ss}arbeit ({\sl Tesi di Laurea)}: {\em Gruppi pseudo-liberi
		localmente nilpotenti}
		Abschlussnote: 110/110 \emph{cum Laude}\\[+3mm]
1994-1999	&Gymnasium ({\em Liceo Scientifico}) -- {\sl Florenz, Italien}\\[+6mm]
\multicolumn{2}{l}{\bf Stellen}\\[+1mm]
\hline\\[+1mm]
Okt 2008 - Dez 2009 	& Wissenschaftlicher Mitarbeiter an der
					HU Berlin\\[+3mm]
Okt 2005 - Sep 2008 	& EU-Gastwissenschaftler an der
					HU Berlin -- im Rahmen des {\sl Marie Curie FP6 research training network} MODNET\\[+6mm]
\multicolumn{2}{l}{\bf Lehrt�tigkeiten an der HUB}\\[+1mm]
\hline\\[+1mm]
WS 08/09	& �bungsleiter im Fach Algebra I\\
SS 09		& �bungsleiter im Fach Gew�hnlichen Differentialgleichungen\\
WS 09/10	& �bungsleiter im Fach Algebra II
\end{tabular}
\newpage
\thispagestyle{empty}
%\centering
\begin{center}
\begin{tabular}{r@{\extracolsep{2em}}p{9cm}}
\multicolumn{2}{l}{\bf Ausgew\"alte Konferenzen, Workshops u. Sommerschulen}\\[+1mm]
\hline\\[+1mm]


August 2004		&Perugia, S.M.I. Sommer Schule. %Prof. G.Alcober - Euskal Herriko Univ. Bilbao-\\
				{\sl Group Actions. Classification of Finite Groups of some ``simple'' order.
				Soluble and Nilpotent Groups}\\
Dezember 2005	&Leeds, MODNET Sommer Schule. {\sl Elements of Stability Theory.
				Intermediate Model Theory}\\
April 2006			&Freiburg, MODNET Sommer Schule. {\sl Advanced Stability. Algebraic Geometry and Model Theory}\\
Juni 2006			&Lyon, MODNET Training Workshop. {\sl Hrushowski Amalgamation and Fusion. Simple Theories}\\
%Juni 2006			&Lyon, Logicum Lugdunensis. \url http://math.univ-lyon1.fr/logicum/logicumlugdunensis\\
September 2006	&Oxford, MODNET Workshop in model theory\\
November 2006	&Antalya, MODNET Mid-Term Conference.\\
Januar 2007		&Oberwolfach, MFO Workshop on Model Theory of Groups.\\
Juni 2007			&Camerino, MODNET Sommer Schule. {\sl Model Theory of Modules. Introduction to o-minimality. Stable Groups}\\
September 2007	&Berlin, MODNET Training Workshop. {\sl Model Theory of Fields and Applications. Construction of o-minimal
				structures}\\
April 2008			&La Roche, MODNET Training Workshop. {\sl Motivic Integration. Model Theory of Valued Fields.
				Interaction between Model Theory and Number Theory}\\
Juni 2008			&Leeds, Around Classification Theory.\\
Juli 2008			&Manchester, MODNET Summer School. {\sl Groups of finite Morley Rank. Finite Model Theory}
\end{tabular}
\vfill
%\vspace{4ex}
%\bigskip
\end{center}
\begin{flushright}
Berlin, den~\today\\[+7mm]
\dots\dots\dots\dots\dots\dots\dots\dots\dots
\end{flushright}
\newpage
\thispagestyle{empty}
\cleardoublepage


%% You can change this text, if needed.
\thispagestyle{empty}
%\raggedright
\section*{Selbst\"andigkeitserkl\"arung}

Ich erkl\"are, dass ich die vorliegende Arbeit selbst\"andig und nur unter Verwendung der angegebenen Literatur und Hilfsmittel angefertigt habe.



\vspace{2\baselineskip}
\noindent Berlin, den \today\\
Andrea Amantini\\[+7mm]
\dots\dots\dots\dots\dots\dots\dots\dots\dots

%\selectlanguage{english}
\tableofcontents
\newpage
\thispagestyle{empty}
\cleardoublepage
\xchapter{Introduction}
%\pagenumbering{arabic}
%\setcounter{page}{1}
%\addtocounter{chapter}{1}
The purpose of this work is twofold: on one side we propose a new treatment of the structures which led to Baudisch' {\em new
$\aleph_{1}$-categorical group} of nilpotency class $2$ constructed in \cite{bad}. On the other hand we settle
a new framework to possibly achieve Groups with similar properties but in higher nilpotency classes. The main efforts
involve the nilpotent-$3$ case.

For what concerns both aspects, the deep contiguity between nilpotent groups of prime exponent and
graded Lie algebras over finite fields, let us work within the second kind of structures, which support in addition
a linear-algebraic approach. %, suitable for a {\em counting} argument.
This correspondence is explained in detail in Section \ref{nilgral}.

\smallskip
The aforemensioned \lqq Baudisch group\rqq arises from a direct translation in combinatorial group-theoretic terms, of the restyled
Fra\"iss\'e amalgamation technique, which led Hrushovski in \cite{hruabi} to confute Zilber's {\em structural conjecture}
(\cite{zil}). % on uncountably categorical theories.
We briefly review these facts below, as they form in part the guidelines of the
present work.

\medskip
A definable set of a complete first-order theory is called strongly minimal if its Morley rank and degree are both equal to one.
%This translates to the fact that no definable set
%can be infinite and co-infinite, in any elementary extension.
%The essential tool to prove Morley's categoricity results (rephrased in a geometric flavour
%by Baldwin and Lachlan in \cite{blsms}) is the fact that, in strongly minimal structures, the
%algebraic closure is a pregeometry and hence allow a dimension theory.
In a strongly minimal structure, the (model-theoretic) algebraic closure yields a {\em pregeometry}. This allowed for instance
Baldwin and Lachlan in \cite{blsms} to reprove Morley's categoricity results by means of a {\em dimensional} approach, derived
by such pregeometries.
Strongly minimal structures are in particular $\aleph_{1}$-categorical
and on the contrary, uncountably categorical structures do always ``contain'' strongly minimal sets as
-- we might say -- building blocks.

For the definition of a (pre)geometry and related notions, the reader is referred to Section \ref{qdim}.

%In \cite{hruabi} and \cite{jbg} it is recalled that t
The pregeometries attached to the strongly minimal sets definable in a
$\aleph_{1}$-categorical structure, have (after localisation) all isomorphic associated geometries. This local isomorphism type
constitutes therefore an invariant of such structures.

Zilber conjectured indeed that each $\aleph_{1}$-categorical theory $T$ is assigned a geometry according to the following trichotomy
(cfr.\,\cite{hruabi,jbg}).
\begin{itemize}
\punto{1}A disintegrated geometry. No infinite %almost strongly minimal
group is definable in $T$.
\punto{2}A nontrivial modular geometry of a vector space. An infinite % rank-$1$
group is definable in $T$, but no infinite field does.
\punto{3}A non locally modular geometry. $T$ is not one-based and an infinite field is interpretable
in $T$.
\end{itemize}

The conjecture was disproved by Hrushovski in \cite{hruabi} by means of {\em new strongly minimal sets},
which have a non-locally modular geometry, but nevertheless do not interpret an infinite field.
%\xsubsection{The {\em ab Initio} Construction}
These counterexamples rely on a Fra\"{\i}ss\'e amalgamation procedure (described in Section \ref{fraisse}), together with
a pregeometric machinery, which modifies ordinary embeddings. This allows in particular
to control the types of the Fra\"iss\'e limit by means of a dimension function: the
structures obtained are stable, which is not in general the case for Fra\"iss\'e constructions.

\smallskip
To summarise the above process, start -- say -- from a ternary, order-invariant relation $(M,R)$
and define an integer valued function of the finite parts of the domain $M$:
\begin{labeq}{urdelta}
\delta(A)=\card{A}-\card{R(A)}
\end{labeq}
where $R(A)$ describes the set of all ternary {\em links} $(a,b,c)$ with $R(a,b,c)$ -- up to permutation -- which insist among
points of $A$.

This $\delta$ turns out to be a {\em predimension function} in the sense of Section
\ref{pregextsec}; there we explain how to derive a pregeometry from any predimension.
This yields a dimension function $d_{M}$ on each $\{R\}$-structure $M$.

The crucial steps -- rather informally -- are given below and summarise the approach of \cite{jbg}. In this paper, Poizat
divides the construction into two distinct subsequent steps:
\begin{itemize}
\punto{Phase One}Define the class $\Kl{}$ of all finite $\{R\}$-structures with non-negative predimension.
Give a notion of {\em strong} %or {\em self-sufficient}
extensions $A\geqslant B$ in terms of $\delta$ and prove $\Kl{}$ has the properties %(HP), (JEP) and (AP)
of {\em Hereditarity, Joint Embedding} and {\em Amalgamation}
described in Section \ref{fraisse}, with respect to $\leqslant$.
%\end{itemize}

The Fra\"iss\'e limit $K$ of $(\Kl{},\leqslant)$ obtained is $\omega$-saturated and $\omega$-stable of Morley rank $\omega$ and is
ultrahomogeneous with respect to $\leqslant$.
Types of elements over a set $B$ are discerned in base of their dimension $d_{K}$: points
which are dependent over $B$ have all finite (unbounded) Morley rank, while transcendent points have all the same type and
rank $\omega$.
The (forking) geometry of the generic type is the $d_{K}$-pregeometry,
this is not locally modular.
No group diagram is allowed by dimension arguments.

If we restrict the class $\Kl{}$ by changing the initial lower bound of $\delta$ to a fixed positive integer $k$, one obtains a Fra\"iss\'e
limit with a $k$-transitive, non $k-1$-transitive automorphism group.

\punto{Phase Two} A proper subclass $\Kl{}^{\mu}$ of $\Kl{}$ is defined, for which an $\N$-valued function $\mu$
bounds the length of realisations of a family of distinguished {\em minimal pre-algebraic} extensions. With a more difficult proof,
the amalgamation property is true of $\Kl{}^{\mu}$ as well.
The theory $T^{\mu}$ of the Fra\"iss\'e limit  $K^{\mu}$ of $(\Kl{}^{\mu},\leqslant)$ is strongly minimal.

This second phase is referred to in the literature as the {\em collapse}, because the
finite-rank pre-algebraic types in Phase one, are collapsed to algebraic ones, while as a consequence,
the infinite rank type is forced to assume Morley rank $1$. The strongly
minimal geometry on $K^{\mu}$ {\em coincides} with the $d$-pregeometry of $K$ above.
\footnote{In his PhD thesis \cite{fer}, Marco Ferreira proves that the geometries of the collapsed structures are isomorphic to the geometry of the regular type in the uncollapsed construction.}
\end{itemize}

\smallskip
In the original paper \cite{hruabi}, this bipartite analysis is not present and the amalgamation
is carried out directly in the collapsed case. Hrushovski proves the non-interpretability of an infinite group in $T^{\mu}$
as a consequence of {\em flatness}, a property attributed to the geometry of $K^{\mu}$. On the other hand Pillay shows in \cite{pilcm},
that $CM$-{\em trivial} structures do not allow the interpretation of an infinite field:
in \cite{hruabi} it is also proved that that the collapsed structure is $CM$-trivial and that flatness implies $CM$-triviality.

\smallskip
F.{\ }Wagner in \cite{wag}, provides
an axiomatic approach to the above constructions which replaces an explicit predimension argument.
\crule
%\xsubsection{Baudisch Group and the Red Collapse Frame}
\bigskip
In \cite{bad} Baudisch starts from a predimension $\delta$ which is very much alike \pref{urdelta}: it computes
the gap between the number of {\em generators} and {\em relators} of a suitably {\em linearised} presentation of groups.

In the perspective of Zilber's trichotomy, he obtains a pure uncountably categorical group of Morley rank $2$ with no infinite field
interpretable: the associated pregeometry is not locally modular -- because the group obtained is connected and non-abelian
(cfr.{\,}\cite{hp}) -- and its theory is shown to be $CM$-trivial.

The following result indicates which classes of groups may allow such feature.
\begin{fact*}[{\cite[Theorem 2.1]{bad}}]
Assume a connected group $G$ of finite Morley rank does not interpret an infinite field. Then {\em either}
a definable section of $G$ contradicts the Cherlin-Zilber algebraicity Conjecture\footnote{
Infinite simple groups of finite Morley rank are conjectured to be algebraic groups over
an algebraically closed field.}, {\em or}
%its connected component $G^{o}$
$G$ is nilpotent.

In the last case $G$ is the central product of a definable divisible abelian subgroup $A$
and a definable nilpotent subgroup $B$ of $G$ of bounded exponent.
\end{fact*}

To eventually place ourselves on the ``bright side'' of Cherlin-Zilber Conjecture,
the objects considered in \cite{bad} are $2$-nilpotent groups of exponent a fixed prime $p$ bigger than $2$.
Such groups can be reconstructed from the pair of $\Fp$-vector spaces $(G_{ab},G^{\prime})$ -- the sections
of the lower central series -- by means
of the %bilinear map $\map{[\,\,,\,]}{G_{ab}\times G_{ab}}{G^{\prime}}$. 
linear map $\map{c_{G}}{\exs G_{ab}%\wedge G_{ab}
}{G^{\prime}}$, induced by the group commutator in $G$ on the exterior square algebra of $G_{ab}$.
This draws our attention to the pair $(G_{ab},\ker(c_{G}))$: step-$2$ nilpotency
yields a 1-1 correspondence of these groups with the
%The Fra\"isse procedure are in fact conducted along embeddings of
structures $(M,N(M))$, where $M$ is a $\Fp$-vector space and $N(M)$ is a subspace of $\exs M$. %(cfr.\,Remark \ref{baucond}).

In case of a finitely generated $M$, one considers
\begin{labeq}{deltabau}
\delta(M)=\dfp(M)-\dfp(N(M))
\end{labeq}
The Hrushovski amalgamation program described above is carried out in \cite{bad} with this $\delta$ directly for the Collapsed case,
once a suitable native function $\mu$ is implicitly given. 

\smallskip
In Chapter \ref{due} we recast all the steps leading to (Phase One) of the Hrushovski-Baudisch construction in terms
of nilpotent Lie algebras over $\Fp$.

\crule
In Section \ref{nilgral} we present a well-known uniform method to associate a group
with a Lie ring, this uses the sections of the lower central series and the group commutator.
As a consequence, this procedure becomes particularly effective when dealing with nilpotent groups.
%of prime exponent $p$.
If we denote by $\ngb{c}{p}$ the variety %In this case in fact the above correspondence supplies 
of $c$-nilpotent and exponent $p$ groups, we isolate a class of $c$-nilpotent graded Lie algebras $\nla{c}$ over the field $\Fp$
in order to obtain a {\em grading} functor $\gr$ of $\ngb{c}{p}$ into $\nla{c}$, which is surjective
at the level of objects.

The literature about this subject is founded on the work of Lazard, Magnus \cite{laz,mag,mag37} and
Witt's \cite{witt}. In a {\em torsion-free} context this phenomenon is
also called {\em Mal'cev Correspondence}: it establishes an equivalence between the categories of
torsion-free divisible nilpotent groups and nilpotent Lie $\Q$-algebras (see \cite[\S6]{bah}). 

We give two different methods to associate a given Lie algebra $L$ of $\nla{c}$, a group $G$ of $\ngb{c}{p}$ with $\gr(G)=L$:
a group theoretical one, which employes a torsion version of the relationship between free groups and free Lie
rings (this is Witt's {\em Treue Darstellung}) and
a more analytical procedure, which uses the Baker-Hausdorff formula.
%(Theorem \ref{faikaha} and Corollary \ref{co:grupphausdorff}).
This last approach, although less transparent for higher classes $c$, has the advantage of establishing a multiplicative group
structure $(L,\circ)$ directly on the Lie algebra domain $L$.
This group law will be in fact first-order definable in terms of the ring signature.

\smallskip
The additional requirement $G^{\prime}=Z(G)$ for the groups $G$ considered in \cite{bad},
is discussed in Remark \ref{baucond}. This property, which is preserved by the algebra-group correspondence,
will be obtained for $\nla{2}$-algebras as a consequence of the positive lower bound chosen for the predimension.

\smallskip
In Section \ref{schur} we are concerned with an existing notion of group theoretical {\em deficiency}, which computes the difference
between the generators and the relators of a finitely presented group $G$. The second integral homology group of $G$ is involved
in such a measurement.
More precisely, the deficiency of $G$ is always bounded from above, by the difference between the $\Z$-rank of $G_{\it ab}$
and the minimal number of generators for $H_{2}(G,\Z)$.
Following Stammbach and Stallings we derive the correspondent notion of deficiency
for groups in the variety $\ngb{c}{p}$ and homology will be taken with coefficients over $\Fp$.

If we consider a presentation $R\to F\to G$, the so called {\em Hopf formula} returns $H_{2}(G)$ as
the quotient $R\cap F^{\prime}/[R,F]$. This term filters in fact the {\em essential} relators in $R$,
those which actually cause the deficiency to drop.

This filter is basically the same adopted in Chapter \ref{tre} for $\nla{c}$-algebras in order to obtain new kinds of
presentations.
Despite the strong similarity between the above notions and the relators space we constructed,
we encountered this group-homological interpretation only in a very late phase of this work.
We decided to include this section as a sort of {\em a posteriori} motivation.

\bigskip
In the first section of Chapter \ref{due}, we start by adapting the deficiency predimension \pref{deltabau} to finite objects of $\nla{2}$.

Any $M=M_{1}\oplus M_{2}$ in $\nla{2}$, is given by a presentation $R\to\fla{2}{M_{1}}\to M$ from the free nil-2 Lie algebra
$\fla{2}{M_{1}}$ over $M_{1}$.
For a subspace $A$ of $M_{1}$, the integer $\delta(A)$ (or $\delta_{2}(A)$ to distinguish from other nilpotency classes) will be defined as
$\dfp(A)-\dfp(R(A))$. The
relators ideal $R(A)\inn\fla{2}{A}$ depends by the ambient relators $R$ and the subspace $A$.

This function is proved to be a predimension {\em over} the $\Fp$-linear closure, as defined in Section \ref{qdim}.
As a consequence, $\delta$ gives rise to a pregeometry on the vector space $M_{1}$ whose closure operator extends the
linear span. We show directly that this pregeometry is actually a non locally-modular geometry {\em over} $\Fp$.

The notion of {\em self-sufficient} extensions $M\zsu[2]{}N$ of $\nla{2}$-algebras will be given in terms of $\delta$: as
usual $\delta(C)$ cannot drop below $\delta(M)$ on spaces $C$ between $M$ and $N$.

\medskip
Section \ref{amalga2} describes the subclass $\Kl{2}$ of $\nla{2}$, for which
an {\em asymmetric} amalgamation lemma is shown: we define a free amalgam in $\nla{2}$, which preserves a positive lower bound
of the deficiency, provided a kind of one-point algebraic extensions are suitably avoided.
Compared to the correspondent statements in \cite{bad}, the proofs here are overall simplified, left aside some
technicalities (Lemmas \ref{basE} and \ref{reldue})%\mn{\bf maybe we can avoid these, cfr. Lemma \ref{amalsigma2}}
, which we have to borrow with minor changes from the original text.

As part of this section we find the treatment of {\em minimal} strong extensions, these will be
fundamental for the rank computations in the uncollapsed theory. To this end we prove that chains of minimal extensions
commute with free amalgamation.

\medskip
Asymmetric amalgamation yields a first-order axiom system $T^{2}$
for the countable Fra\"iss\'e limit of $\Kl{2}$. As it is meant to happen the $\omega$-saturated models
of $T_{2}$ are exactly the rich structures whose age is $\Kl{2}$. This is Theorem \ref{Cazzuola} of Section \ref{t2axioms},
where we also prove $\omega$-stability of $T_{2}$ and give a description of the algebraic closure in $T_{2}$.

In Section \ref{rango}, we explicitly compute the Morley rank of the %theory $T_{2}$, the Morley rank of the
countable rich model $\mathbb{M}$, which is -- as expected\footnote{It is sort of
by chance that this value coincides with the rank of the uncollapsed {\em black field} of Poizat. In that case this factor
is artificially obtained by the shape of the predimension, while in ours it closely reflects the structural nil-$2$ constraint.}
 -- $\omega\cdot2$.
 
The reason for this number comes from the $\nla{2}$-grading $\mathbb{M}=\mathbb{M}_{1}\oplus\mathbb{M}_{2}$ and the
locally-free behaviour imposed by the axioms.
As our predimension takes its entries among the finite parts of $\mathbb{M}_{1}$, we first obtain Morley rank $\omega$ for this
set, by a geometric type analysis \`a la John B. Goode (cfr.\,Phase One above).
On the other hand, to require a positive deficiency, forces the
homogeneous subspaces $\mathbb{M}_{1}$ and $\mathbb{M}_{2}$ to be definably $\Fp$-isomorphic. This doubles the rank.
The same happens in the collapsed case and explains the rank $2$, there in fact the corresponding set
$\mathbb{M}_{1}$ is strongly minimal.

By applying the aforementioned correspondance
we reconstruct a nil-2 group $\mathbb{G}$ which has Morley rank $\omega\cdot2$. Indeed the whole local construction (amalgamation,
self-sufficient embeddings, richness, etc.)
can be traced back at the level of groups; cfr.~Remark \ref{baucond}.

\smallskip
A complete description of forking in $T^{2}$ follows. This is done in Section \ref{forking} by exhibiting a suitable ternary
{\em independence relation} among sets of the monster model $\mathbb{M}$ which satisfies the
axioms of forking in stable theories. This notion of independence reflects both the {\em geometric} information of
the predimension and the {\em structural} condition imposed by free amalgamation.

\smallskip
In the last section of Chapter \ref{due}, we propose a notion of {\em weak canonical base} for
types of self-sufficient tuples over models.
This is compared with the properties of {\em weak elimination of imaginaries} and $CM$-triviality
for the {\em uncollapsed} theory. On this purpose one may also check the notion
of {\em relative} $CM$-triviality proposed in \cite{cmtr}.

\bigskip
In the third Chapter we study a possible construction of deficiency predimensions in the case of nilpotent
Lie algebras from $\nla{c}$ of class $c$ greater than $2$.

The guiding principle here is an inductive approach\label{indunil} over the nilpotency class, suggested by the graded shape of
a (saturated-homogeneous say) object $\mathbb{M}=\mathbb{M}_{1}\oplus\dots\oplus\mathbb{M}_{c}$ of $\nla{c}$.

This corresponds to a presentation $R\to\fla{c}{\mathbb{M}_{1}}\to \mathbb{M}$ from the free Lie nil-$c$ algebra
$\fla{c}{\mathbb{M}_{1}}$, where the homogeneous ideal $R$ equals $R_{2}+\dots+R_{c}$ (cfr.\,Section \ref{nilgral}).
On the other hand, denote by $\mathbb{M}_{*}$ the {\em truncation}
to $\nla{c-1}$, that is $\mathbb{M}_{*}=\mathbb{M}/\mathbb{M}_{c}\simeq\mathbb{M}_{1}\oplus\dots\oplus\mathbb{M}_{c-1}$.

Now assume we have a notion of deficiency $\delta_{c-1}$
which locally measures the gap among linear dimensions in $\mathbb{M}_{1}$
and the numbers of independent relators from $\mathbb{M}_{*}$ in all possible weights $<c$.
Suppose further, such a function behaves like a predimension and yields a dimension function $d_{c-1}$ on $\mathbb{M}_{1}$.
Then we ideally define $\delta_{c}(A)$ for $A\inn \mathbb{M}_{1}$, as the difference between $d_{c-1}(A)$
and the linear dimension of a new {\em relators space} $\rc(A)$.

$\rc(A)$ is able to isolate elements of $R_{c}$,
from Lie products $[\rho,x_{1},\dots,x_{c-k}]\in R_{c}$, involving relations $\rho\in R_{k}$ of a lesser weight $k<c$.
The definition of $\rc(M)$ is found in Section \ref{maxrels}.

\smallskip
For a fixed prime $p$ and $c$ with\footnote{
The constraint $c<p$ lays in the nature of the Hausdorff series development described in Section \ref{nilgral}.
%the rational coefficient of the homogeneous term of degree $i$ is in fact divided by primes not greater than $i$. 
} $c<p$, in its entirety, this recursive program should produce a sequence of pregeometries
$(\mathbb{M}_{1},\cl_{i})_{i\leq c}$, each one extending the previous ($\cl_{i}\inn\cl_{i+1}$) and all insisting upon the 
same domain set $\mathbb{M}_{1}$. Here $\cl_{1}$ is the $\Fp$-linear closure and $\cl_{2}$ is the pregeometry obtained
from the deficiency $\delta_{2}$, associated to $\nla{2}$-algebras.

This aspect motivates the study of extensions among pregeometries and the notion of predimentions {\em over} a given pregeometry
given in Section \ref{pregextsec}.

\medskip
The above operator $\rc$ relies on a {\em free-lift} functor $\map{\frl}{\nla{c-1}}{\nla{c}}$
defined in Section \ref{freelift}. This is such that $\frl(M)_{*}=M$ for all $M$ in $\nla{c-1}$ and obey the following universal property:
for any other $N\in\nla{c}$ with $N_{*}\simeq_{\nla{c-1}}M$, $\frl(M)$ maps uniquely onto $N$. In other words $\frl(M)$ is the freest possible
object in $\nla{c}$ to have a truncation in $\nla{c-1}$ which is $M$. We prove in fact that $\frl$ is left-adjoint to $\map{_{*}}{\nla{c}}
{\nla{c-1}}$ in Proposition \ref{morphifreelift}.

Composed the other way around, the universal property of $\frl$ yields, for any algebra $M$ of $\nla{c}$, the desired {\em shifted presentation} $\rc(M)\to\frl(M_{*})\to M$. The kernel $\rc(M)$ has the properties mentioned above.

\medskip
This formal strategy is applied, in Section \ref{preditre}, in the step from $\nla{2}$ to $\nla{3}$. Already in this {\em induction
basis}, major difficulties are encountered in the reproduction of both the Fra\"iss\'e procedure and the pregeometric approach.

We define a first deficiency for finitely generated $\nla{3}$-algebras $A$, as the difference between $\delta_{2}(A_{*})$
-- the $\nla{2}$-predimension defined in Chapter \ref{due} -- and the $\Fp$-dimension of the space $\rt(A)$ given above.

So defined, this function is unreliable to control deficiencies within a fixed ambient structure $M$ of $\nla{3}$. That is
is because $\rt(A)$ is not in general contained into $\rt(B)$ for extensions $A\inn B$ inside $M$.

\smallskip
This is due to a structural issue intrinsic to the free-lift functor:
for extensions $M\inn N$ of $\nla{2}$-structures, the lifted algebra $\frl(M)$ {\em does not} always embed into $\frl(N)$.
In Section \ref{embiss} we prove however that if $M$ is a {\em self-sufficient} $\nla{2}$-subalgebra of $N$, then
we have a corresponding extension of the lifted $\nla{3}$-algebras, i.e.\,$\frl(M)\inn\frl(N)$.
This crucial result, which influences the whole subsequent construction,
is proved by using the so called {\em Hall's bases} (Definition \ref{basicommutators}) of {\em basic} commutators
for free Lie algebras. In fact a similar approach to Hall's {\em collecting process} in \cite{mhalll} is employed.

\smallskip
Now fixed an $\nla{3}$-algebra $M$, we define a more adaptive deficiency $\ded^{M}(A)$, which reads subspaces $A$ of
$M_{1}$. This is built in terms of the {\em dimension function} $d_{2}^{M}$ -- induced by the pregeometry from $M_{*}$ -- and a
suitable {\em monotone} operator $\rt_{M}(A)$, which returns subspaces of $\rt(M)$ and depends on $\frl(\gen{A})$. %(\gena{A}{M})$.

As a consequence of the above embedding result, the functions $\delta_{3}$ and $\ded^{M}$ do agree
on $\delta_{2}$-strong subalgebras $A$ of $M$.

This behaviour also suggests the following definition of {\em strong} $\nla{3}$-extensions: to write $A\dsu{} M$ and
say $A$ is self-sufficient in $M\in\nla{3}$, we require in fact that the truncated
structures are
self-sufficient with respect to $\delta_{2}$ ($A_{*}\zsu[2]{}M_{*}$) and that the auxiliary deficiency $\ded^{M}$ assumes values
bigger than $\delta_{3}(A)$ on all $C$ between $A$ and $M_{1}$.

\smallskip
Consequently, we exhibits in Section \ref{amalgatre} a strong amalgam of $\nla{3}$-algebras.
This is obtained as follows: start with a strong configuration like $A\dso{}B\dsu{}C$, then
take the truncated preamalgam $A_{*}\zso[2]{}B_{*}\zsu[2]{}C_{*}$ and obtain, with the results in Chapter \ref{due},
a free $\nla{2}$-amalgam $D_{*}$ of $A_{*}$ and $B_{*}$ over $C_{*}$.

This yields strong $\nla{2}$-inclusions $A_{*}\zsu[2]{}D_{*}\zso[2]{}C_{*}$. Now take the free-lift $\frl(D_{*})$ and by virtue of the aforementioned fact, obtain the embeddings $\frl(A_{*})\inn \frl(D_{*}) \nni \frl(C_{*})$.

Since $A$ and $C$ are isomorphic to the quotients $\frl(A)/\rt(A)$ and $\frl(C)/\rt(C)$, the $\nla{3}$-algebra
$D\defeq\frl(D_{*})/(\rt(A)+\rt(C))$, amalgamates $A$ and $C$ over $B$ and we show $A\dsu{} D\dso{}C$ in
Lemma \ref{amalgatrestrong}.

With a modified procedure we were actually able to prove the {\em asymmetric} version of the above result: from
$A\nni B\dsu{} C$, we obtain $A\dsu{} D\nni C$.
As shown in Chapter \ref{due} in fact, asymmetric amalgamation is indispensable
to approximate richness in a possible axiomatisation of the Fra\"iss\'e limit.


\medskip
A further remark, independent of previous issues, settle at this point the following -- and more critical -- problem:
{\sl to decide whether $\rt_{M}(A)\cap\rt_{M}(B)$ equals $\rt_{M}(A\cap B)$, for given subspaces $A$ and $B$ of $M_{1}$.}
%given subspaces $A$ and $B$ of $M_{1}$, $\rt_{M}(A)\cap\rt_{M}(B)$ does not equal $\rt_{M}(A\cap B)$ in general.

The answer is negative in general and two main obstructions follow thereafter:
\begin{itemize}
\item[-]we prove with examples, that $\ded^{M}$ (and $\delta_{3}$) is not in general submodular.
\item[-]We cannot prove the strong $\nla{3}$-embedding $\dsu{}$is transitive, nor find a transitive notion related to
$\dsu{}$\footnote{
there is a standard way to {\em force} transitivity via a local ``cut lemma'' (cfr. Lemma \ref{2cut}) definition of strongness: in our case
one should define \lqq $A$ is strong in $M$\rqq if for any finite part $U$ of $M_{1}$, $\delta_{3}(A_{1}\cap U)\leq\delta_{3}(U)$.
This definition however does not comply with the amalgamation in $\nla{3}$ described in Lemma \ref{amalgatrestrong}.
}.
\end{itemize}
The first makes void the proof-strategies adopted in Chapter \ref{due}. Submodularity %\pref{summo}
is in fact on one hand the key property to turn a deficiency-like function into a predimension,
on the other, it ensures that free amalgamation preserves the same lower bound
for the deficiency, of the amalgamated structures. 

\smallskip
The efforts of Section \ref{classetre}
goes in the direction of finding {\em local} conditions to force a modular behaviour of $\rt_{M}$ and hence be able to
use submodularity of $\ded$ {\sl just where we need it}.

This is strongly connected to the relationship between $\delta_{3}$ and $\ded$. In this
section we prove indeed that they are uniformly comparable, namely in the direction $\ded^{M}(A)\leq\delta_{3}(A)$ for
any finite algebra $A$ of $\nla{3}$.

In accordance to this and the above amalgamation process,
we define a class $\Kl{3}$ of $\nla{3}$-algebras $M$ with $M_{*}$ in $\Kl{2}$ for which
$\delta_{3}$ is non-negative on the finite subalgebras of $M$. By the above, we can use
indifferently $\delta_{3}$ or $\ded^{M}$ to test whether $M$ is in $\Kl{3}$.

We indicate $\Kl{3}$ as a possible candidate to represent the age of the desired rich $\nla{3}$-algebra, although we couldn't
prove the amalgamation property for $\Kl{3}$. 

%\smallskip
%Despite a very long series of attempts, the whole Fra\"iss\'e construction was not accomplished.
%The long time employed, eventually with very slow and rare improvements, forced us to prematurely put an end to this work.
\crule
The exclusive treatment of the uncollapsed case in this work is also motivated by a later project of Baudisch',
{\em The Additive Collapse} (\cite{addcoll}).
Here an $\omega$-stable theory $T$ is considered, which expands the theory of vector spaces over the finite field $\Fp$.
A pregeometry is assigned on the models of $T$ and a notion of {\em strong embedding}
between subspaces is given, which both influence the elementary type of the saturated monster $\K$ of
$T$. Further properties are required of $T$, which capture the essential features of the {\em uncollapsed} infinite rank versions of the known amalgamation examples. 

After {\em prealgebraic codes} and the aforementioned bound-function $\mu$ are chosen,
the collapsed structure $\K^{\mu}$ of finite rank, is constructed directly {\em inside} $\K$.

This new procedure is meant to unify under a common frame, the Red fields \cite{rf}, the new uncountably categorical group and
the fusion over a vector space \cite{fu}.

Should suitable stable rich $\nla{c}$-algebras ($c>2$) be constructed with the methods described in the present work,
then the additive collapse process would give finite rank nilpotent Lie algebras or groups, with underlying Hrushovski
geometries.
%\newpage\thispagestyle{empty}
%\cleardoublepage
%\thispagestyle{empty}
\xsubsection{Acknowledgements}
My first thought goes to my advisor, mostly for the patience he demonstrated in this very long path.
Without his ideas indeed, very few of the present work could exist.
discussions

What this thesis {\em has not become}, is the result of a creative struggle between us, which unfortunately
failed in achieve the full score.

Of course I wish to thank a lot my tutor Martin Hils, which always tried to give me the motivations to go on.
I owe him a lot.

I want to thank Assaf Hasson and Misha Gavrilovich for the discussions about the possible homological implications.
%---------------------------------------CHAPTER ONE---------
\newpage
\thispagestyle{empty}
\cleardoublepage
%
\chapter{Basic Facts and Definitions}\label{uno}
\section{Combinatorial Pregeometries}\label{qdim}
If $M$ is a set, $\ps{M}$ denotes its powerset.
To denote unions of sets, juxtaposition will almost everywhere replace the symbol $\cup$ in the sequel, so that
$AB$ will mean $A\cup B$ and $Ab$ will be $A\cup\{b\}$ for all sets $A,B$ and elements $b$, of $M$.
%\end{presection}

%\cbstart
%\begin{dfn}
%A {\em closure operator}\mn{define a pregeometry directly!!!} $\cl$ on the set $M$, is a map
%$\map{\cl}{\ps{M}}{\ps{M}}$ which satisfies the following properties
%\begin{itemize}
%\punto{cl1}$A\inn\cl(A)$ for all $A\in\ps{M}$
%\punto{cl2}$\cl\circ\cl=\cl$
%%\punto{cl3}$\cl A\inn\cl B$ whenever $A\inn B$
%\punto{cl3}$\cl(B)\inn\cl(A)$ whenever $A\nni B$.
%%\cl(A)$ is the union of all $\cl(A^{\prime})$, where the $A^{\prime}$ range over $\fp{M}$
%\end{itemize}
%
%We say that the closure operator $\cl$ on $M$, has the {\em exchange property} if in addition
%\begin{itemize} 
%\punto{ex}For all $a,b\in M$ and all $A\in\ps{M}$, when $a\in\cl(Ab)\non\cl(A)$, then $b\in\cl(Aa)$
%\end{itemize}
%while $\cl$ is said a {\em finitary} closure operator if for all sets $A$ in $M$
%\begin{itemize}
%\punto{fin}$\cl(A)$ is the union of all $\cl(A^{\prime})$, where the $A^{\prime}$ range over the finite parts of $A$.
%\end{itemize}
%\end{dfn}
%\cbend
\begin{dfn}\label{pregdef}
A \emph{pr{\ae}geometry} %(or a {\em combinatorial matroid})
$(M,\cl)$ is a %couple $(M,\cl)$
set $M$ endowed with a closure operator $\map{\cl}{\ps{M}}{\ps{M}}$ on $M$,
which satisfies the {\em Steiniz exchange property}. This means the following properties are required of $\cl$:
\begin{itemize}
\punto{cl1}$A\inn\cl(A)$ for all $A\in\ps{M}$
\punto{cl2}$\cl\circ\cl=\cl$
\punto{fin}$\cl(A)$ is the union of all $\cl(B)$, where the $B$'s range over the finite parts of $A$.
\punto{ex}For all $a,b\in M$ and all $A\in\ps{M}$, when $a\in\cl(Ab)\non\cl(A)$, then $b\in\cl(Aa)$
\end{itemize}

If in addition $\cl(\vac)=\vac$ and $\cl(a):=\cl(\{a\})=\{a\}$ for all singletons $a\in M$, we say
that $(M,\cl)$ is a {\em geometry}. 
\end{dfn}

Note that (fin) alone implies monotonicity %(cl3)
of the closure operator $\cl$. A {\em closed set} of $M$ is defined, as usual, as a fixed point of $\cl$.
%Property (ex) is referred to as the {\em Steiniz exchange property}.
%The approach of J.B. Goode in \cite{JBG} is rather to introduce a (family of) relations between elements
%and finite sets
%``$cis(x,\bar y)$'' of {\em ciscendency} to express ``$x\in\cl(\bar y)$'', satisfying
%the above conditions. That context is actually the most (\dots) to the statement of lemma \ref{} below
%
%Again Goode refers these notions to be first considered by Van der Waerden and Krasner.

\medskip
From a pregeometry $(M,\cl)$ we obtain a geometry $(M_{*},\cl_{*})$ if we define $M_{*}$ as $(M\non\cl(\vac))/_{\sim}$
for $a\sim b\iff\cl(a)=\cl(b)$, and $\cl_{*}(A/_{\sim})$ to be $\cl(A)/_{\sim}$.
This procedure is exactly the way a projective space is obtained out of a vector space:
each line is identified to a point.

\medskip
If $(M,\cl)$ is a pregeometry and $B$ a subset of $M$ we define its {\em localisation at $B$}
as the pregeometry $(M,\cl_{B})$ given by $\cl_{B}(U):=\cl(BU)$ for all subsets $U\inn M$.

\medskip
We say that the subset $A$ of $M$ is {\em independent over} $B$
(or $B$-{\em independent}) if $a\notin\cl_{B}(A\non\{a\})$ for all $a\in A$. 

We say that a subset $C$ of $A\inn M$ is a {\em base for $A$ over $B$}, if it is independent over $B$ and $A\inn\cl_{B}(C)$.
The definition of an independent set or of a base of a set are obtained if we put $B$ above to be the empty set.

By the exchange property, given any set $A$, a maximal $B$-independent subset of $A$ is a base over $B$. Moreover
all bases have the same cardinality\footnote{Exchange property is essentially needed to prove that {\em finite} bases have all the
same size.}, which is defined as the {\em dimension of $A$ over $B$} and denoted
with $\dim(A/B)$. This (ordinal) number satisfies the following additivity property
\begin{labeq}{dimadd}
\dim(AB)=\dim(B)+\dim(A/B)
\end{labeq}
for any sets $A$ and $B$.

We may also say that a set $D$ is independent of $C$ over $B$ if $\dim(D/B)=\dim(D/CB)$.
\begin{dfn}
A pregeometry $(M,\cl)$ is {\em trivial} or {\em disintegrated} if for any sets $A,B\inn M$ we have $\cl(AB)=\cl(A)\cup\cl(B)$.

We say that a pregeometry $(M,\cl)$ is {\em modular} if for all closed sets $A$ and $B$, we have
$\dim(A/B)=\dim(A/A\cap B)$.

A pregeometry $(M,\cl)$ is {\em locally modular} if the above equality is true whenever $\dim(A\cap B)>0$
or equivalently if $(M,\cl_{\{a\}})$ is modular for all $a\in M$.
%\uwave{for some finite set $A$ of $M$}.\mn{\bf check def!!}
\end{dfn}
Remark that a trivial pregeometry is always modular, and that a modular geometry is also
locally modular. Moreover a pregeometry is modular exactly if any closed set $A$ is independent of any closed $B$
over their intersection and also iff the following equality holds on finite-dimensional closed sets $A,B$
\begin{labeq}{mod}
\dim(AB)+\dim(A\cap B)=\dim(A)+dim(B).
\end{labeq}
%It follows that a pregeometry is locally modular exactly when condition \pref{mod}
%is satisfied for all closed $A$ and $B$ with $\dim(A\cap B)>0$.

It is routine to mention the following examples:
\begin{itemize}
\item[-]A vector space $V$ over a field $\cur$ is a pregeometry if we set $\cl(A)=\gen{A}_{\cur}$,
the $\cur$-linear span of a subset $A$ in $V$. This is a non-trivial modular pregeometry.

\item[-]If $\mathbb{A}$ is an affine space with underlying vector space $V$, the affine closure
turns $\mathbb{A}$ into a non-modular, locally modular pregeometry.

\item[-]Algebraic closure in an algebraically closed field (of large enough transcendence degree)
gives rise to a non-locally modular pregeometry.
\end{itemize}

%\subsection{Zil'ber's Structural Conjecture}
%--- MUTE-----Zilber Trichotomy
%We recall that a complete first-order theory is called strongly minimal if Morley rank and degree of its
%monstermodel $\mon$ are both equal to one. This translates to the fact that no definable set
%can be infinite and co-infinite, in any elementary extension.
%
%The esential tool which reproves Morley categoricity results (rephrased in a pregeometric flavour
%by Baldwin and Lachlan in \cite{blsms}) is the fact that, in strongly minimal structures, the
%algebraic closure is a pregeometry. Strongly minimal structures are in particular $\aleph_{1}$-categorical,
%and on the contrary uncountably categorical structures do always ``contain'' strongly minimal sets as
%-- we maight say -- building blocks.
%%Therefore it makes sense to classify a strongly minimal theory
%%according to the pregeometry $(\mon,\acl)$.
%
%In \cite{hruabi} and \cite{jbg} it is recalled that the pregeometries attached to the strongly minimal sets definable in a $\aleph_{1}$-categorical structure, have (after localization) all isomorphic associated geometries. This {\em local} isomorphism type constitutes therefore an
%invariant of such structures.
%
%Zilber's {\em trichotomy conjecture} -- formulated in \cite{} -- essentially assigned
%to an $\aleph_{1}$-categorical theory $T$ one of the following geometries:\mn{Compare/Consult John B. Goode}
%\begin{itemize}
%\punto{1}A disintegrated geometry. No almost strongly minimal group is definable in $T$.
%\punto{2}A non-trival modular geometry of a vector space. A rank-$1$ group is definable in $T$, but no infinite field does.
%\punto{3}A non locally modular geometry. $T$ is not one-based and an infinite field is interpretable
%in $T$.
%\end{itemize}
%
%The conjecture was disproved by Hrushovski in \cite{hruabi} by means of a {\em new strongly minimal set}:
%which has a non-locally modular geometry, but nevertheless does not interpret a field. Some detail
%of this are given in section \ref{abi}.

\subsection{Predimensions and associated Pregeometry Extensions}\label{pregextsec}
We denote by $\fp{M}$ the set of the finite parts of $M$.
\begin{dfn}\label{pregext}
Assume $\cl$ and $\cl_{0}$ are closure operators which both turn $M$ into a pregeometry.
We say that $\cl$ {\em extends} $\cl_{0}$ if for all $A\inn M$ we have $\cl_{0}(A)\inn\cl(A)$.

We say that $(M,\cl)$ is a {\em geometry over} $\cl_{0}$, if $\cl$ extends $\cl_{0}$, if $\cl(\vac)=\cl_{0}(\vac)$
and if $\cl(a)=\cl_{0}(a)$ for all $a\in M$. In the case $\cl_{0}$ is the identical closure ($\cl_{0}(A)=A$, for all $A\inn M$)
$(M,\cl)$ is called a {\em geometry}.
\end{dfn}

If $\cl$ extends $\cl_{0}$ and $\dim$, $\dim_{0}$ denote the associated dimensions, then for each
$A\in\fp{M}$ we have clearly $\dim_{0}(A)\geq\dim(A)$, moreover
$$
\dim(\cl_{0}(A))\leq\dim(\cl(A))=\dim(A)\quad\text{and}\quad\dim(A)\leq\dim(\cl_{0}(A)).
$$
In particular $\dim(A)=\dim(\cl_{0}(A))$, that is, $\dim$ is determined by its value on $\cl_{0}$-closed sets.

\smallskip
Let now $(M,\cl)$ be a pregeometry, %with dimension $d$, we
we denote the set of \emph{finitely generated $\cl$-closed parts} of $M$ by
$${\fp{M}}_{\cl}=\{B\inn M\mid B\,\text{is $\cl$-closed with $\dim(B)$ finite}\}.$$
\begin{dfn}\label{clpred}
We call a map $\map{\delta}{\fp{M}_{\cl}}{\N}$ a {\em predimension over $\cl$} or a \emph{$\cl$-predimension}
on $M$ if the following holds:
\begin{align}
\tag{normalization}\qquad&\delta(\cl(\vac))=0\quad\text{and}\quad\delta(\cl(a))\leq1&\\[+2mm]
\qquad&\delta(\cl(UV))\leq\delta(U)+\delta(V)-\delta(U\cap V)&\label{summo}
%\intertext{for all $U,V\in\fpe{M}$.}
\end{align}
for all $a\in M$ and $U,V\in\fp{M}_{\cl}$. Compared to \pref{mod}, property \pref{summo} above is referred to as
{\em submodularity}.

A {\em predimension} on $M$ is, by definition, a $\cl$-predimension where $\cl$
is the identical closure on $M$.%: $\cl(A)=A$, for all sets $A$. %$\cl=id_{\ps{M}}$.

A predimension $d$ on $M$ which is {\em monotone}, that is $d(B)\leq d(A)$ for all finite $B\inn A$ in $\fp{M}$
%\begin{itemize}
%\punto{monotonicity}$d(B)\leq d(A)$ for all finite subsets $B\inn A$ of $M$
%\end{itemize}
is called a {\em dimension function} on $M$.
\end{dfn}

\medskip
Assume $\delta$ is a $\cl$-predimension on $M$ and set, for all $A$ in $\fp{M}$
%$$\delta^{*}(B)=\min\{\delta(C)\mid C\in\fp{M}_{1},\,C\nni B\}$$
%and successively
%\begin{labeq}{didef}
%d_{2}(A)=\delta^{*}(\cle(A))\quad\text{for all}\,A\in\fp{M}.
%\end{labeq}
%(It can also be defined:
\begin{labeq}{ddidelta}
d(A):=\min(\delta(C)\mid C\in\fp{M}_{\cl},\:C\nni A).
\end{labeq}
With the above definition we still have $d(\vac)=0$ and $d(a)\leq1$ for all singleton $a$.
Moreover for finite  $A,B$ in $M$ let us choose $\cl$-closed oversets $A^{\prime}\nni A$ and $B^{\prime}\nni B$
%with minimal $\delta$, that is
with $d(A)=\delta(A^{\prime})$ and $d(B)=\delta(B^{\prime})$.
Since closed sets are closed under intersection, we have
$$d(AB)+d(A\cap B)\leq\delta(\cl(A^{\prime}B^{\prime}))+\delta(A^{\prime}\cap B^{\prime})\leq d(A)+d(B),$$
that is $d$ is a dimension function on $M$ after Definition \ref{clpred} above, and
is called {\em the} dimension function {\em associated to} $\delta$.
Also note that, in the definition of $d$, is crucial to require $\delta$ to be non-negative.

The next lemma shows that $d$ is actually
the dimension associated to a prescribed pregeometry.
\begin{lem}\label{preg}
Assume $d$ is the dimension function associated to a $\cl$-predimension $\delta$ on
the set $M$ via \pref{ddidelta}.

For any $A\in\fp{M}$ define $\cl_{d}(A)$ to be the set of all $b$ of $M$ such that $$d(Ab)=d(A).$$

If we define $\cl_{d}$ on arbitrary sets in the natural way, that is by putting $\cl_{d}(A)=\bigcup\{\cl_{d}(F)\mid F\in\fp{A}\}$,
then $(M,\cl_{d})$ is a pregeometry which extends $(M,\cl)$ and has dimension $d$.
\end{lem}
\begin{proof}
It is enough to show properties (cl2) and (ex) holds for $\cl_{d}$ over finite sets, while (cl1) is clear.

For (cl2) assume %then that each element $b$ of
that a finite set $B$ is contained in %in the closure of a finite set
$\cl_{d}(A)$ for some finite set $A\inn M$. That is $d(Ab)=d(A)$ for all $b$ in $B$.
By induction on the cardinality of $B$, using submodularity \pref{summo}, it follows $d(BA)=d(A)$.

We need to show that if an element $a$ is in $\cl_{d}(B)$ then it is in $\cl_{d}(A)$.
Applying submodularity we have $d(aBA)\leq d(aB)+d(BA)-d(B)=d(BA)$. 

We can conclude $d(aA)\leq d(aBA)\leq d(A)$ as desired. This gives $\cl_{d}(A)\nni \cl_{d}(\cl_{d}(A))$ and (cl2) follows.

\smallskip
To obtain the exchange property (ex), observe first that, as as a dimension-function, $d$ satisfies $d(Ab)\leq d(A)+1$, for any finite $A\inn M$.

Assume $a\in\cl_{d}(A\,b)\non\cl_{d}(A)$, this means $d(A)<d(aA)\leq
d(aAb)=d(Ab)\leq d(A)+1$. This forces $b$ to be in the closure of $Aa$.

\smallskip
That $\cl_{d}$ extends $\cl$ is readily seen, as $d(Ab)=d(A)$ for all $b\in\cl(A)$ by definition \pref{ddidelta}, for all $A\in\fp{M}$.

And it is trivial to verify that $d(S)$ %=\dim(S)$
is the $\cl_{d}$-dimension of $S$, for every set $S$ in $M$.
%It remains to check that $d$ is actually the dimension of $\cl_{d}$. We have namely to
%check that, for any finite $S$ in $M$, $d(S)$ coincide with the number of elements in a basis of $S$. Let $\{s_{1},\,\dots,\,s_{n}\}$ be such a basis. If $n=1$ then since $\{s_{1}\}$ is independent $d(s_{1})=1$, assume it is true of $n-1$.
\end{proof}

%We also remark that since the axioms for a predimension are universal (modulo $\cl$), the following 
%
%Assume $\delta$ is a $\cl$-predimension on the set $M$ and $d$
%is the dimension function associated to $\delta$.
%Assume $\cl^{\prime}$ is the pregeometry  on $M$ extending $\cl$ derived from $d$.
%
%If now $N$ is a $\cl$-closed subset of $M$, then the restriction of $\cl^{\prime}$ to $N$, $\res{\cl^{\prime}}{\ps{N}}$
%coincide with the pregeometry obtained by the dimension function on $N$ associated to $\res{\delta}{\fp{N}_{1}}$.
\section{Fra\"iss\'e Limits}\label{fraisse}
We refer to a {\em Fra\"iss\'e amalgamation construction} in general as the technique introduced by Roland Fra\"iss\'e in \cite{fra}
to recover universal-homogeneous structures from a prescribed class of finite ones. We follow the treatment
of Ziegler-Tent \cite{zietent} for the countable setting. One may also check \cite{bs} for similar constructions in arbitrary cardinality.

The original results of \cite{fra} are stated in a relational language, but they remain true in the following wider context.

\medskip
Let $\La$ be a countable language; if we say embedding below, we mean $\La$-embedding.
Given a class $\mathcal{K}$ of finitely generated $\La$-structures which is closed under isomorphisms,
we define the following properties for $\Kl{}$:
\begin{itemize}
\punto{HP} For each object $A$ in $\Kl{}$ and substructure $B\inn A$, we have $B\in\Kl{}$ (Hereditary Property).
\punto{JEP} Given any two objects $B$ and $C$, there exists
$A$ in $\Kl{}$ which embeds $B$ and $C$ (Joint Embedding Property).
\punto{AP} For all $D\in\Kl{}$ and embeddings $\emb{\beta}{D}{B}$ and
$\emb{\gamma}{D}{C}$ there exists an object $A$ of $\Kl{}$ and embeddings $\emb{b}{B}{A}$ and $\emb{c}{C}{A}$
such that $\beta b=\gamma c$ (Amalgamation Property).
\end{itemize}
Remark that (JEP) does not follow in general by Amalgamation, as provided by the class of (finite) fields without a
specified characteristic.

Given an $\La$-structure $M$ we define {\em the age of} %l'\^{a}ge} of
$M$, denoted $\age(M)$ to be the class of all finitely generated $\La$-structures which are isomorphic to a substructure of $M$.
Define $\widetilde{\Kl{}}$ to be the class of all $\La$-structures whose age is contained in $\Kl{}$.

For a given $M$, $\age(M)$ satisfies of course (HP) and (JEP), while for $\age(M)$ to have (AP)
it is necessary to require $M$ is {\em strongly homogeneous}. How much, is explained by the next definition and facts.
\begin{dfn}\label{ricca}
An $\La$-structure $M$ is said {\em $\Kl{}$-rich}
if $\age(M)=\Kl{}$ and 
for any embedding $\emb{\beta}{B}{A}$ of $\Kl{}$-objects $A$ and $B$, if $b$ is an embedding of
$B$ into $M$ then there exists an embedding $\emb{a}{A}{M}$ such that $\beta a=b$.
\end{dfn}

For the proof of the following result we refer to \cite[Theorem 13.4]{zietent}.
\begin{fact}[Fra\"iss\'e Limits]\label{fraissteo}
Let $\Kl{}$ be a denumerable class of $\La$-structures for a countable language $\La$ which is closed under isomorphism, then
\begin{itemize}
\punto{i}there exists a countable $\Kl{}$-rich $\La$-structure $M$ in $\widetilde{\Kl{}}$ iff
$\Kl{}$ satisfies \rm{(HP), (JEP)} and {\rm (AP)}.

\punto{ii}Any two countable $\Kl{}$-rich structures are isomorphic. More generally, any two $\Kl{}$-rich structures
are $\La_{\infty,\omega}$-equivalent, that is, they can be matched up by an infinite back and forth correspondence.
\end{itemize}
\end{fact}
The isomorphism type of the countable rich structure is called the {\em Fra\"iss\'e limit} of the class $\Kl{}$.

The class $\widetilde{\Kl{}}$ is not in general elementary.
The classical first examples of this construction are
the class of finite linear orders, which has $(\Q,<)$ as Fra\"iss\'e limit and
the class of finite undirected graphs, of which the Random Graph is the limit.

\medskip
Let now a denumerable class $\Kl{}$ be given, of finitely generated $\La$-structures, for a countable language $\La$.
Assume $\zsu{}$ is a binary relation among objects of $\Kl{}$, which is contained in the $\La$-embedding relation
and which is invariant under $\La$-isomorphisms.

\begin{rem}\label{ModiFra}
Suppose $(\Kl{},\zsu{})$ is a partial order and the properties (JEP) and (AP) are true of $\Kl{}$ with $\zsu{}$
replacing $\La$-embeddings, while (HP) holds in the original fashion.
Then Fact \ref{fraissteo} applies to this situation: there exists a countable structure $\K$ in $\Klt{}$, which is rich with respect to $\zsu{}$.
With this we mean just $\beta$, $b$ and $a$ are to be replaced with $\zsu{}$-embeddings in
Definition \ref{ricca}.
%Moreover we have to redefine $\age(\K)$ as the class of all finitely generated $\Kl{}$-structures, which $\zsu{}$-embeds into $\K$.

We may write in this case that $\K$ is the Fra\"iss\'e limit {\em of} $(\Kl{},\zsu{})$.
\end{rem}

Hrushovski's construction relies on the above modification.
As described in the Introduction, the ``ab Initio'' example substitutes embeddings among relations
with {\em pre-dimensionally strong} embeddings.

In Secton \ref{amalga2}, we describe a similar approach: Lie algebra embeddings are replaced by a suitable
stronger notion.
\section{%Facts from General and Geometric Stability Theory
A few Notions from Stability Theory}\label{stab}
The facts from stability theory we use are quite basic.
In the case of the uncollapsed nil-$2$ Lie algebra we construct in Section \ref{t2axioms}, the theory obtained is $\omega$-stable.
Only properties of such theories will therefore be needed. For the concepts and the definitions of this section,
we essentially follow \cite{ziebous} or \cite{zietent}.

\medskip
By a {\em totally transcendental theory} we mean a theory in which each formula $\varphi(\bar x)$
in $n$ variables has ordinal Morley rank, for all $n<\omega$.
By Fact \ref{ziemr} below this is equivalent to require every $1$-formula to have ordinal Morley rank or
to require the formula $x=x$ to have such a rank.
\begin{fact*}
An $\omega$-stable theory $T$ is totally transcendental (short t.t.). Moreover the two notion coincide
if the language of $T$ is countable.
\end{fact*}
The facts recalled in the rest of the section, if not otherwise specified, concern a fixed large saturated monster $\mon$ of
a totally transcendental theory $T$.
{\em Small} sets are subsets of $\mon$ whose cardinality is less than $\card{\mon}$ and {\em models} are
small elementary substructures of $\mon$.

We assume Morley rank and degree are defined on partial types $p$ in $T$ over parameters from $\mon$
and write respectively $\mr(p)$ and $\md(p)$. We also denote by $\mrd(p)$ the ordered pair $(\mr(p),\md(p))$.

For a formula $\varphi(\bar x)$, $\mrd(\varphi(\bar x))$ stands for $\mrd(\{\varphi(\bar x)\})$ while $\mr(\bar a/B)$ will be the Morley rank of $\tp{\bar a}{B}$, for any tuple $\bar a$ and small subset set $B$ of $\mon$.

\begin{rem}\label{belrango}
Morley rank is {\em continuous}, that is for any complete type $p$, $\mr(p)$
is the rank of a formula $\phi$ in $p$, and for any complete type $q\ni\phi$, $\mr(q)\leq\mr(\phi)$.
Moreover for any formula $\psi(\bar x)$ we have dually
$$\mr(\psi(\bar x))=\max(\mr(p)\mid p\in\ssp{\bar x}{A},\psi\text{ is over $A$ and }\psi\in p).$$
\end{rem}

\smallskip
Both statements of the following fact will be used in Section \ref{rango} further below.
The first is an easy exercise on rank computation under
algebraicity, the second is due to a result of Erimbetov (\cite{eri}), in the formulation of which we follow \cite{ziemr}.
The {\em product} of two ordinals $\alpha$ and $\beta$, denoted by $\alpha\cdot\beta$, is defined as
the order type of the lexicographic order on $\alpha\times\beta$.

\begin{fact}\label{ziemr}
Let $\map{f}{\mathscr{D}}{\mathscr{E}}$ be a definable map between definable (possibly with parameters) classes of the monster
model of an arbitrary theory.

\begin{itemize}
\item[1.]If $f$ is finite-to-one and onto $\mathscr{E}$, then $\mr(\mathscr{D})=\mr(\mathscr{E})$.
\item[2.]If $\mathscr{E}$ has Morley rank $\beta$ and the Morley rank of all fibres $f^{-1}(e)$ is bounded by an ordinal $\alpha>0$, the Morley rank of $\mathscr{D}$ is bounded by $\alpha\cdot(\beta+1)$.
\end{itemize}
\end{fact}

\medskip
We assume a notion of {\em non-forking} extension of types is given (through {\em dividing}).
In a totally transcendental setting, non-forking is expressed in terms of Morley rank.
For tuples $\bar a$ and small subsets $A,B$ in $\mon$ we write
\begin{labeq}{forkrank}
\ffin{\bar a}{B}{A}%\stackrel{def}{
\iff\mr(\bar a/B)=\mr(\bar a/AB).
\end{labeq}
to say that $\tp{\bar a}{AB}$ {\em does not fork over} $B$.
 
The {\em Lascar rank} (\cite{las}) on complete types $p$ of a stable theory, denoted by $U(p)$ is the smallest {\em connected} notion
of rank on complete types, whose gap on type extensions witness forking (see also \cite{haha} or and \cite[\S6]{bue}).
This means, if $q$ extends $p$, $q$ is a forking extension of $p$ iff $U(q)<U(p)$.

Moreover connectedness means, that $U(p)=\alpha$ and $\beta\leq\alpha$ implies the existence of a
complete type $q\nni p$ with $U(q)=\beta$.\footnote{Morley rank is connected with respect to {\em formulas} but
not on complete types.}

%is  or $U$-rank as emerges, for instance
%in \cite{las} and \cite{haha} or %. This is the $\mathit{L}$-rank defined by Shelah in
%\cite[V.7.5]{sh90}.
%which is a priori defined in an unstable context\mn{clarify the setting!}: $U$ is the unique {\em connected} (and hence
%{\em minimal}) notion
%of rank on complete types (cfr.\,\cite{la}); for an extension $p\inn q$ of
%and hence, in a stable theory, witness the {\em non-forking} relation on complete type extensions:
%for $q$ to be a {\em forking extension} of $p$ is equivalent to $U(p)<\infty$ and $U(q)<U(p)$.
%It is in fact the smallest connected ordinal rank to isolate the properties of forking, 
In particular we have $U(p)\leq\mr(p)$ on all complete types $p$ of $T$.

\smallskip
The strength of Lascar rank lays in its additive property. We refer to the book
of Buechler cited, for the definition of the {\em connected sum} $\alpha\oplus\beta$ of two ordinal numbers
$\alpha,\beta$.
\begin{fact}[{\cite[Theorem 8]{las}}]\label{uddi}
In a superstable theory $T$, for all tuples $\bar a,\bar b$ and sets $B$, we have
$$
U(\bar a/B\bar b)+U(\bar b/B)\leq U(\bar a\bar b/B)\leq U(\bar a/B\bar b)\oplus U(\bar b/B).$$
Moreover since the ordinal sum $+$ and $\oplus$ coincide on finite ordinals, when
$U(\bar a/B\bar b)$ and $U(\bar b/B)$ are both finite, we have
$$
U(\bar a\bar b/B)=U(\bar a/B\bar b)+U(\bar b/B).
$$
\end{fact}
This additive behaviour resembles additivity of Morley rank in strongly minimal sets and will turn out
very useful when computing the rank of the theory $T_{2}$ in Section \ref{t2axioms}.

Unfortunately the two notions of rank introduced so far do not in general coincide on complete types even in
an $\omega$-stable context.  In \cite[\S6 and \S7]{bue} one finds an extensive account of 
examples and conditions under which these ranks do or do not coincide. Among the affirmative cases we find, for instance, the
uncountably categorical theories.

Rank computations in Section \ref{rango} involve the following very special instance, which we prove below
\begin{lem}\label{RMU}
Let $\mathfrak{X}$ be a family of complete isolated types in $T$, %each type in $\mathfrak{X}$ is
over finite sets of parameters.

Suppose further that for any type $p\in\mathfrak{X}$ and each finite set $C$ containing the parameters of $p$,
any complete extension of $p$ over $C$ lays again in ${\mathfrak X}$.

Then Morley rank and $U$-rank agree on $\mathfrak{X}$.
\end{lem}
\begin{proof}
Let $p\in\ssp{\bar x}{A}$ be a type in $\mathfrak{X}$ for a finite set $A$, and assume that $M\!R(p)\geq\alpha$ for some ordinal number $\alpha$, we show
by induction on $\alpha$, that $U(p)\geq\alpha$. Let the statement by true of types from ${\mathfrak X}$
for all ordinals $\alpha<\kappa$. If $\kappa$ is a limit
ordinal, then by the definition of ranks it follows $U(p)\geq\kappa$.

Let $\kappa$ be $\alpha+1$ for some ordinal $\alpha$, and $\mr(p)\geq\alpha+1$. Since $p$ is isolated, there is a formula
$\varphi(\bar x)$ over $A$ which implies $p$, hence $\mr(\varphi)\geq\alpha+1$. Since $T$ is t.t., let $\psi(\bar x)$ be a formula
over some finite $C\nni A$ implying $\varphi$ with $\mr(\psi)=\alpha$. Choose a type $q$ in $\ssp{\bar x}{C}$ generic
in $\psi$, then we have $\mr(q)=\alpha$, $q$ implies $\varphi$ and hence $q$ is a forking extension of $p$.
This yelds $q\in\mathfrak{X}$ and
since $\mr(q)\geq\alpha$, by induction, $U(q)\geq\alpha$, this means exactly $U(p)\geq\alpha+1$.

We actually showed that the assumptions force Morley rank to be connected on ${\mathfrak X}$.
\end{proof}

\medskip
We will also need the characterisation of forking in terms of a {\em notion of independence}:
a ``stable version'' of Kim-Pillay results for simple theories.

With this respect, we follow the approach of \cite[Theorem 5.8]{haha} and Ziegler and Tent in (\cite[Theorem 36.10]{zietent}).

The last authors seem to exhibit an overall {\em shortest} list of properties for a distinguished class of type extensions to
coincide with the non-forking relation. We stick however to the equivalent formulation in terms of an independence relation
among {\em sets} rather than types.
\begin{fact}\label{stableforking}
Assume a complete theory $T$ is endowed with a ternary relation $\fin{\bar x}{X}{Y}$ between tuples $\bar x$ and pairs of
(small) sets $X,Y$ of $T$,
which is invariant under $\aut(\mon)$.
Then $T$ is stable if an only if $\ind$ satisfies:
%\begin{itemize}
%\punto{\small{\sc Local Character}}there is a cardinal $\kappa$ such that for all tuple $\bar a$ and set $C$, there is $C_{0}\inn C$ of
%cardinality at most $\kappa$ such that $\fin{\bar x}{C_{0}}{C}$.
%\punto{\small{\sc Boundedness}}There is a cardinal $\mu$ such that for all $A\nni B$ and any tuple $\bar a$, there are at most
%$\mu$ $\aut_{A}(\mon)$-orbits among tuples $\bar a^{\prime}$ with $\fin{\bar a^{\prime}}{B}{A}$ and $\bar a\equiv_{B}\bar a^{\prime}$.
%\end{itemize}
\begin{description}
\item{\small{\sc (Local Character)}} there is a cardinal $\kappa$ such that for all tuple $\bar a$ and set $C$, there is $C_{0}\inn C$ of
cardinality at most $\kappa$ such that $\fin{\bar x}{C_{0}}{C}$.
\item{\small{\sc (Boundedness)}} There is a cardinal $\mu$ such that for all $A\nni B$ and any tuple $\bar a$, there are at most
$\mu$ $\aut_{A}(\mon)$-orbits among tuples $\bar a^{\prime}$ with $\fin{\bar a^{\prime}}{B}{A}$ and $\bar a\equiv_{B}\bar a^{\prime}$.
\end{description}
\smallskip
If in addition $\ind$ satisfies, for all sets $A\nni B\nni C$:
\begin{description}
\item{\small\sc(Transitivity)} for any tuple $\bar a$, %and $A\nni B\nni C$,
from $\fin{\bar a}{C}{B}$ and $\fin{\bar a}{B}{A}$, follows $\fin{\bar a}{C}{A}$.
\item{\small\sc(%Weak
Monotony)} For all %$A\nni B\nni C$ and
$\bar a$, $\fin{\bar a}{C}{A}$ implies $\fin{\bar a}{C}{B}$.
\item{\small\sc(Existence)} %For all sets $A\nni B$ 
There always exists a tuple $\bar a$ with $\fin{\bar a}{B}{A}$.
\end{description}
then $\ind$ coincides with non-forking independence,
that is $\fin{\bar a}{B}{A}$ holds, exactly when $\tp{\bar a}{AB}$ {\em does not fork over} $B$.
\end{fact}
Of course properties above specialise to the case t.t.\,theories, i.{}e. Local Character becomes
{\em finite} Local Character and a {\em finite} instance of Boundedness property is satisfied. 

On the contrary, finite local character and finite boundedness of a notion of independence in a
small theory imply $\omega$-stability.
\begin{rem}\label{re:extrafking}
Stable forking independence satisfies in addition:
\begin{align}
\tag{Symmetry}\forall\bar a,\bar b,B,\quad\ffin{\bar a}{B}{\bar b}&\iff\ffin{\bar b}{B}{\bar a}\\
\tag{Irreflexivity}\forall\bar a,B,\quad\ffin{\bar a}{B}{\bar a}&\Rightarrow \bar a\in\acl(B)\\
\tag{Algebraicity}\forall\bar a,\; C\inn\acl(A,B)\text{ and }\ffin{\bar a}{B}{A}\quad&\Rightarrow\ffin{\bar a}{B}{C}\\
\tag{Base Monotonicity}\forall \bar a,A\nni B\nni C,\quad\ffin{\bar a}{C}{A}&\Rightarrow\ffin{\bar a}{B}{A}
\end{align}
\end{rem}
%\begin{itemize}
%\punto{Irreflexivity}$\ffin{\bar a}{B}{\bar a}\Rightarrow \bar a\in\acl(B)$
%\punto{Algebraicity}If $C\in\acl(A,B)$ then $\ffin{\bar a}{B}{A}\Rightarrow\ffin{\bar a}{B}{C}$
%\punto{Base Monotonicity}For $A\nni B\nni C$, $\ffin{\bar a}{C}{A}\Rightarrow\ffin{\bar a}{B}{A}$.
%\punto{Symmetry}For all $\bar a,\bar b$ and $B$, $\ffin{\bar a}{B}{\bar b}\iff\ffin{\bar b}{B}{\bar a}$.
%\end{itemize}
For a comprehensive account on the possible axiomatic choices for a notion of independence
we refer to \cite{ad}.

\bigskip
In the last section of Chapter \ref{due}, we prove some results around
weak elimination of imaginaries and also draw a strategy toward a proof of $CM$-triviality for our uncollapsed structure.

We recall below some essential facts about these notions, following
\cite{ziebous} and \cite{cafa}.
\begin{dfn}\label{wei}
A theory $T$ has {\em weak elimination of imaginaries} (WEI) if for every imaginary element $e$,
in $M^{eq}$ for any model $M$ of $T$, there is
a real tuple $\bar c$ such that $e$ is definable over $\bar c$ and $\bar c$ is algebraic over $e$, that is
$$e\in\dcl^{eq}(\bar c)\quad\text{and}\quad\bar c\in\acl^{eq}(e).$$
\end{dfn}
Imaginary elements are used essentially to deal with canonical bases of types and definable sets.

In our t.t.{\,}theory $T$ for a complete stationary type $p=p(\bar x)$ ($\md(p)=1$) over a set $A$,
the {\em canonical base} of $p$ is the definable closure $Cb(p)$ of the -- at most $\card{T}$-many -- canonical
parameters of the $p$-definition formulas $d_{p}x\varphi(x,y)$ (\cite[p.29]{ziebous}) as $\varphi(x,y)$ ranges over the language of $T$.
In our context $Cb(p)$ is the definable closure of a finite sequence of imaginaries.

$Cb(p)$ lays a priori in $\mon^{eq}$ and is point-wise fixed by exactly those automorphism $\sigma$ of $\mon$
for which $p$ and $p^{\sigma}$ have the same global non-forking extension.
Therefore if ${\bf p}$ is a global type, ${\bf p}$ is fixed by exactly the automorphisms which fixes $Cb({\bf p})$ point-wise.
For a global type ${\bf p}$ and a set $A$
of parameters we will also need the following renown property of canonical bases:
\begin{fact}[{\cite[Theorem 4.2]{ziebous}}]\label{ziecb}{\ }
\begin{itemize}
\punto{1.}{\bf p} does not fork over $A$ iff $Cb({\bf p})\inn\acl^{eq}(A)$
\punto{2.}{\bf p} is the unique non-forking extension of the (stationary) type $\res{{\bf p}}{A}$ iff $Cb({\bf p})\inn\dcl^{eq}(A)$.
\end{itemize}
\end{fact}
%I refer in general to \cite{ziebous,cafa} for the aforementioned definitions and facts about canonical parameters and imaginaries.
We will write $Cb(\bar a/B)$ to denote $Cb(\tp{\bar a}{B})$, provided $\tp{\bar a}{B}$ is stationary.

The following result which may be derived from \cite[Proposition 2.5]{cafa} will be also mentioned in Section \ref{cmt}. It is a statement
about the existence of {\em weak} canonical bases for types over models.
For ease of reference, we adapt the proof to the total transcendental setting.
\begin{lem}\label{weimodels}
%Let $\mon$ the monster model of a totally transcendental theory $T$, then
$T$ has {\em (WEI)} if and only if for any (small) model $M\ess\mon$, and any type $p\in\ssp{}{M}$, there exists a {\em real} tuple $\bar c$
in $M$ such that:
\begin{itemize}
\punto{{\it i\,}}the pointwise stabiliser of $\bar c$ in $\aut(M)$ fixes the type $p$,
\punto{{\it ii\,}}$\bar c$ has finitely many conjugates under the automorphisms of $M$ which leave $p$ fixed.
\end{itemize}
\end{lem}
\begin{proof}
Let then $e$ be an imaginary of $T$ such that $e=\bar a/\epsilon$ where $\epsilon(\bar x,\bar y)$ is a $0$-definable
equivalence relation, and $\bar a$ is in $\mon$. Let $\mathbf{p}$ a global generic type in $\epsilon(\bar x,\bar a)$.

By taking a small but sufficiently saturated model $M$ ($\omega$-saturation will do), containing $\bar a$ and such that ${\bf p}$ does not
fork over $M$, we obtain a real tuple $\bar c$ with properties (i) and (ii) related to $\aut(\mon)$ and ${\bf p}$.
%For a totally transcendental theory, $\omega$-saturation of $M$ is enough.

But then we have $e\in\dcl^{eq}(\bar c)$, for if $\sigma\in\aut_{\bar c}(\mon)$, then ${\bf p}^{\sigma}={\bf p}$ and
this implies that $\epsilon(\bar x,\bar a)\wedge\epsilon(\bar x,\bar a^{\sigma})$ must be consistent, thus $\sigma$ fixes $e$.

On the other side, the group $\aut_{e}(\mon)$ {\em transitively} permutes the generic global types of the formula
$\epsilon(\bar x,\bar a)$. Since these are but in a finite number,
if $\aut(\mon)_{\bf p}$ denotes the stabiliser of the type ${\bf p}$ under the action of $\aut_{e}(\mon)$,
then the index of $\aut(\mon)_{\bf p}$ in $\aut_{e}(\mon)$ is finite. By the hypothesis, $\bar c$ has a finite orbit under $\aut(\mon)_{\bf p}$, then it has
necessarily a finite orbit under $\aut_{e}(\mon)$. This gives $\bar c\in\acl^{eq}(e)$.

\smallskip
For the converse statement, (WEI) implies that for any type $p$ over a model $M$, we can find a real tuple $\bar c$ such that
\begin{gather}
\begin{cases}\label{ciBi}
\bar c\in\acl^{eq}(Cb(p))\\Cb(p)\in\dcl^{eq}(\bar c).
\end{cases}
\end{gather}
These properties imply (i) and (ii) above.
\end{proof}
A real finite set with property \pref{ciBi} above will be found -- for types of self-sufficient tuples -- in Lemma \ref{teowei} of
Chapter \ref{due}.

The following result from \cite{pilcm} will also be useful
\begin{fact}\label{pilcb}
Assume $M\ess\mon$ is a model of a stable theory, $\mon$ its monster model and let $c,d$ be tuples in $\mon^{eq}$.

If any of the following two conditions
%and denote by $\mathscr{C}$, $Cb(c/M)$ and by $\mathscr{D}$, $Cb(d/M)$.T
%then the following holds:
\begin{itemize}
\punto{i}$c\in\acl(d)$ %\quad\Rightarrow\quad Cb(c/M)\inn\acl^{eq}(Cb(d/M))$
\punto{ii}$\ffin{c}{d}{M}$ %\quad\Rightarrow\quad Cb(d/M)\inn\acl^{eq}(Cb(c/M))$
\end{itemize}
holds, then $Cb(c/M)\inn\acl^{eq}(Cb(d/M))$.
\end{fact}

\smallskip
We recall next the definition of $CM$-triviality.
%which actually 
%This first appears in \cite{hruabi} together with two other equivalent conditions.
%The formulation below is particularly suitable to work with the notion of weak canonical base
%for self-sufficient tuples we will give after Theorem \ref{teowei}.
\begin{dfn}\label{cmtdef}
A theory $T$ is said to be {\sl CM}-trivial, if for any algebraically closed sets $B\inn A$ of the monster $\mathbb{C}^{eq}$ of $T$,
and all tuple $c$ in $\mathbb{C}^{eq}$ with $\acl^{eq}(B,c)\cap A=B$ we always have $Cb(c/B)\inn\acl^{eq}\left(Cb(c/A)\right)$.
\end{dfn}

With Pillay's \cite[Corollary 2.5]{pilcm}, we can rephrase the definition above in terms of models of $T$ and {\em real} tuples:
\begin{fact}\label{pilcmt}
A theory $T$ is {\sl CM}-trivial iff for all small models $M\ess N$ and (real) tuples $\bar c$ from $\mathbb{C}$ with
$acl(M,\bar c)\cap N=M$ we have $Cb(\bar c/M)\inn\acl^{eq}\left(Cb(\bar c/N)\right)$.

Moreover $T$ is $CM$-trivial iff it is such after adding some set of parameters to $T$.
\end{fact}
Such a property for $T$ prevent the theory from interpreting fields:
\begin{fact}[{\cite[Proposition 3.2]{pilcm}}]
No infinite field is interpretable in a $CM$-trivial theory $T$.
\end{fact}
\section{Nilpotent Groups and graded Lie Algebras}\label{nilgral}
We collect in this section some facts and notations from group theory and Lie algebras. We give a picture
of the {\em Magnus-Lazard} correspondence between groups and Lie rings.

\medskip
We refer to the (group) word $\gamma_{k}(x_{1},\dots,x_{k})$ as the {\em left-normed} or {\em simple} group commutator of length $k$
\begin{labeq}{simplecomm}
[x_{1},\dots,x_{k}]=[[[\dots[[x_{1},x_{2}],x_{3}],\dots],x_{k-1}],x_{k}]
\end{labeq}
where $\gamma_{1}(x)=x$ and $\gamma_{2}(x_{1},x_{2})$ is $[x_{1},x_{2}]=x_{1}^{-1}x_{1}^{x_{2}}=x_{1}^{-1}x_{2}^{-1}x_{1}x_{2}$.
We will not formally distinguish between group commutators and Lie brackets in the sequel.

For any group $G$, with $\gamma_{k}(G)$ we denote the verbal subgroup of $G$ determined by $\gamma_{k}$,
that is $\gen{[g_{1},\dots,g_{k}]\mid g_{i}\in G}$. The subgroups $(\gamma_{k}(G))_{k<\omega}$ forms the so called
{\em lower central series} of $G$, the most rapidly descending central series of $G$

Recall that in general a (descending) central series in the group $G$, is a chain $(H^{i})_{1\leq i<\omega}$
of subgroups of $G$ such that $H^{i}\nni H^{i+1}$, $H^{1}=G$ and $[H^{i},G]\inn H^{i+1}$.

We dually define the {\em upper central series} $(\zeta_{k}(G))_{k<\omega}$, where each
subgroup $\zeta_{k}(G)$ can be defined as the set $\{g\mid [g,h_{1},\dots,h_{k}]=1,\forall h_{i}\in G\}$ for all $k<\omega$.
For these, and related notions we refer to $\cite{khk,rob}$.

%Moreover
%if $S$ is a system of generators for $G$, then the $k^{th}$-term of the lower central series is
%the normal closure of all $\gamma_{k}$
%\begin{labeq}{commugen}
%\gamma_{k}(G)=\gen{\gamma_{k}(s_{1},\dots,s_{k})\mid s_{i}\in S}^{G}
%\end{labeq}

We denote by $\ngb{c}{}$ the variety defined by the word $\gamma_{c+1}$: the class of groups $G$
with $\gamma_{c+1}(G)=1$. These are by definition all groups of {\em nilpotency class} (at most) $c$.
We have $\ngb{c}{}\inn\ngb{c+1}{}$ for all $c<\omega$, and we may call for short {\em nil-$c$} groups, the objects of $\ngb{c}{}$.

If $p$ is a prime, by $\ngb{c}{p}$ we denote the variety defined by the words $\gamma_{c+1}(\bar x)$ and $x^{p}$, this is the
class of all nilpotent groups of class $c$ and of bounded exponent $p$. 

\smallskip
The lower central series, is in particular a {\em Lazard series}, that is a decreasing chain $(H^{n})_{n<\omega}$ of
subgroups in $G$, with
$H^{1}=G$ and $[H^{i},H^{j}]\sg H^{i+j}$ for all $i,j<\omega$. The properties we are going to state for
the lower central series, also hold for Lazard series in general. Any Lazard series is a central series.

%For simplicity $G^{n}$ will denote $\gamma_{n}(G)$ for any group $G$ in what follows. %,$(G^{n})$ is a Lazard series.
%If we define $\wg(g)\in\N\cup\{\infty\}$ to be the
%supremum of the set $\{n<\omega\mid g\in G^{n}\}$\footnote{we may assume that the series $(G^{n}$) is {\sl s�parante (fr.)}
%that is $\cap_{n<\infty} G^{n}=1$. In the definition of $L^{G}$ below,
%we have $L^{G}=L^{\overline{G}}$ if $\overline{G}$ is $G/{\cap_{n<\omega}G^{n}}$}. for all group elements $g$ of $G$, then we obtain an {\em integral filtration} on $G$ in the sense of \cite{ser}. We set by definition $G^{\infty}$ to be
%$\bigcap_{n<\infty}G^{n}$.
%
\medskip
For any group $G$ and all $k<\omega$, set for short $G^{k}$ to be $\gamma_{k}(G)$ in the sequel.
The series $G=G^{1}\og G^{2}\og\dots\og G^{k}\og\dots$ gives rise to a Lie ring associated to the group. This
will be discussed below, following \cite[\S3.2]{khk} or the first chapter of \cite{laz}.
\begin{rem}
Let $x,y,z$ be elements of the group $G$. % and take a maximal $n<\omega$ %=\wg(x)+\wg(y)+\wg(z)+1$,
%to be highest ordinal in $\omega\cup\{\omega\}$,
%such that any element among $x,y$ or $z$ belongs to $G^{\alpha}$. Here
%and below, by $G^{\omega}$ and $G^{\omega+1}$ is meant $\bigcap\{G^{k}\mid k<\omega\}$.
%ThenThere exists an $n<\omega$, such that
The following well known identities hold:
\begin{align}
[x,y]=[y,x]&^{-1}\label{commin}\\
[xy,z]=[x,z][x,z,y][y,z]&\label{commbil}\\
[x,y,z^{x}][y,z,x^{y}][z,x,y^{z}]&=1\tag{Witt's identity}
\end{align}
\end{rem}

Now for all $i<\omega$ consider the sections
\begin{equation}
%{L^{G}}_{\!i}=
\gr_{i}G:=G^{i}/G^{i+1}\quad\text{and define}\quad\gr(G)
:=\bigoplus_{0\leq i<\omega}\!%{L^{G}}_{\!i}
\gr_{i}G\tag{$\gr$}
\end{equation}
where $\gr_{0}G\defeq\triv$.

The above remarks and \pref{commbil} provide $\gr(G)$ with a natural ring structure $(\gr(G),+,[\,\,,\,],\triv)$ where
the sum is the componentwise quotient group operation (written additively) and the product is induced by the group commutator:
$$[\boldsymbol{u},\boldsymbol{v}]=\sum_{k}\left(\sum_{i+j=k}\overline{[u_{i},v_{j}]}\right)\quad\text{for $\boldsymbol{u}=(\bar u_{i})$ and
$\boldsymbol{v}=(\bar v_{i})$}.$$

Now Witt's identity and \pref{commin} above turn $\gr(G)$ into a non-associative,
anti-commutative ($[a,b]=-[b,a]$ for all $a,b$) ring in which the {\em Jacobi identity}
\begin{labeq}{jacob}
J(a,b,c)\defeq[[a,b],c]+[[b,c],a]+[[c,a],b]=0
\end{labeq}
holds for all $a,b,c$ in $L$. That is $\gr(G)$ is a {\em Lie ring} (a Lie $\Z$-algebra) according to the
next definition. Notice that $\gr(G)\simeq\gr(G/\cap_{n<\omega}G^{n})$.

\begin{dfn*}
If $\cur$ is a commutative unitary ring, a {\em Lie algebra $L$ over $\cur$} or
a Lie $\cur$-algebra is a $\cur$-module
endowed with a $~k$-bilinear map $[\,\,,\,]$ which factorises through $\exs L$, that is $[a,a]=0$
for all $a$ in $L$, and such that the Jacobi identity $J(a,b,c)=\triv$ is satisfied for all $a,b,c$ in $L$.
\end{dfn*}

For a subset $S$ of a Lie $\cur$-algebra $L$ we denote by $\gen{S}$ or $\gena{S}{L}$ the subalgebra
generated by $S$ in $L$ while $\gen{S}_{\cur}$ denotes the $\cur$-submodule of $L$ generated by $S$.
The product $[S,T]$ of subsets $S$ and $T$ of $L$ is $\genp{[s,t]\mid s\in S,t\in T}$,
while the ideal generated in $L$ by $S$ is denoted by $\genid{S}{}$. This is -- by means of anti-commutativity and repeated applications
of the Jacobi identity -- also $\gen{S,[S,L],[S,L,L],\dots}_{\cur}$.

Exactly like for groups, we define the terms of the lower central series $\gamma_{n}(L)$ of $L$ recursively as $[\gamma_{n-1}(L),L]$ for
all $n<\omega$ where $\gamma_{1}(L)=L$. These builds a decreasing chain of ideals of $L$ and
$\gamma_{n}(L)=\gen{\gamma_{n}(s_{1},\dots,s_{n})\mid s_{i}\in L}_{\cur}$. The definition
for the upper central terms $\zeta_{n}(L)$ is exactly the same defined above.

\smallskip
We say that $L$ is {\em nilpotent of class} (at most) $c$, if $\gamma_{c+1}(L)=\triv$.

\smallskip
If a Lie $\cur$-algebra $L$ is generated by a set $S$, then an inductive argument on Jacobi identities shows, that $L$ is generated
as a $\cur$-module, by all the {\em simple monomials} with entries in $S$, that is by left-normed products $[s_{1},\dots,s_{k}]$
like \pref{simplecomm} of {\em weight} $k$, for all $k<\omega$. In particular $\gamma_{n}(L)$ is the ideal generated by all simple monomials of length $n$ in elements of $S$ or the $\cur$-module generated by monomials of weight $\geq n$.
%%\footnote{for the $S$-weight to be well defined one should also
%%require for all $s\in S$, $s\notin\gen{S\non\{s\}}$.}
%%
%%We recursively define {\em monomials of $S$-weight $n$} as follows. Put $\wg_{S}(\triv)=0$ and
%%$\wg_{S}(s)=1$ for all $s\in S$. Assume all monomials $m$ of $S$-weight less than $n$ are defined and identified
%%by writing $\wg_{S}(m)=k$ for some $k<n$. Then an element $m$ of $L$ is of $S$-weight $n$ -- short $\wg_{S}(m)=n$ -- if
%%$w=[a,b]$ for $a,b$ monomials of $S$-weight less than $n$ such that $\wg_{S}(a)+\wg_{S}(b)=n$.
%%
%%We denote by $L_{S,i}$, the submodule
%$\gen{a\in L\mid\wg_{S}(a)=i}_{\cur}$ and call it the {\em homogeneous submodule of $S$-weight $i$}.
%%\end{dfn}
%Remark that in general $L_{S,i}\cap\sum_{j\neq i}L_{S,j}$ may not be trivial and $\wg_{S}$ may not be a function,
%this is however the case for free Lie algebras and for the object of the class $\nla{c}$  below.
%We call an ideal $R$ of $L$ {\em homogeneous} (in $S$) if $R=\sum_{i}(R\cap L_{S,i})$.
%\begin{rem}
%For a generating system $S$ of $L$ we have $L=\sum_{i<\omega}L_{S,i}$ and $\gamma_{n}(L)=\sum_{i\geq n}L_{S,i}$.
%\end{rem}


\begin{rem}\label{gradigroup}
For any group $G$, $\gr(G)$ is generated as a ring, by the {\em abelianised} group $G_\textit{ab}$: the quotient
of $G$ modulo $G^{\prime}=\gamma_{2}(G)$. This follows by the natural
surjective map of $\otimes_{\Z}^{n}G_{ab}$ onto $\gr_{n}G$ (see \cite[\S2]{khk}) induced by the group commutator.

In particular $\gr(G)$ is a {\em graded algebra}: it is the direct sum of its homogeneous submodules $\gr_{i}G$
{\em of weight $i$ in $G_{ab}$} and $[\gr_{i}G,\gr_{k}G]\inn\gr_{i+k}G$ for all $i,k$.

If $G$ is a member of $\ngb{c}{p}$, the Lie ring $\gr(G)$ carries a natural
Lie $\Fp$-algebra structure which is nilpotent of class $c$.
\end{rem}




\subsubsection*{Free Algebras and basic commutators}
For the following definitions we follow \cite{ser} and \cite{bah}.

\smallskip
For any set $X$, the {\em free magma} $(\mathcal{M}(X),\cdot\,)$ is -- roughly speaking --
the image of $X$ under the free functor $\mathcal{M}$ from {\em sets} to the category
%the free object in the {\em category}
of all structures which interpret
%a signature consisting of constants $X$ and
a binary operation.

%inductively defined as the disjoint union over $n<\omega$, of all the
%$$\mathcal{M}_{n}(X):=\amalg_{p+q=n}\mathcal{M}_{p}(X)\times\mathcal{M}_{q}(X)$$
%where $\mathcal{M}_{1}(X)=X$. In this way we obtain the natural binary operation
%\begin{align*}
%\mathcal{M}(X)\times\mathcal{M}(X)&\longmapsto\mathcal{M}(X)\\
%u,v&\longmapsto(u,v)=:u\cdot v
%\end{align*}

The elements of $\mathcal{M}(X)$ which are referred to as {\em non-associative words over $X$}
are the disjoint union of $\omega$ subsets $\mathcal{M}_{n}(X)$, each one collecting the {\em words of weight} or {\em length} $n$, for $n\geq1$ (cfr.\,\cite[\S2]{bbk}). We have $\mathcal{M}_{1}(X)=X$ and the product $\cdot$ maps $\mathcal{M}_{i}(X)\times\mathcal{M}_{k}(X)$
into $\mathcal{M}_{i+k}(X)$.

\smallskip
Define the {\em free $\cur$-algebra on $X$} as the free $\cur$-module $\mathcal{F}=\mathcal{F}(X,\cur)$ with
basis $\mathcal{M}(X)$ and with a $\cur$-bilinear multiplication $\cdot$ extended from the product on $\mathcal{M}(X)$,
which in particular makes $(\mathcal{F},\cdot,\triv)$ a non-associative ring without unit, such that
$(ta)\cdot b=t(a\cdot b)=a\cdot tb$, for all $t\in\cur$ and all $a,b$ from $\mathcal{F}$.

$\mathcal{M}(X)$ induces a natural grading on $\mathcal{F}$ given by $\mathcal{F}=\bigoplus_{n<\omega}\mathcal{F}_{n}$ for
$\mathcal{F}_{n}:=\gen{\mathcal{M}_{n}(X)}_{\cur}$.
Each element $a$ of $\mathcal{F}$ is henceforth expressible in a unique manner as a finite sum $\sum a_{n}$, where
$a_{n}\in\mathcal{F}_{n}$ are called the {\em homogeneous components} of $a$.

\smallskip
Let now $\mathcal{A}$ and $\mathcal{B}$ be the ideals of $\mathcal{F}$ respectively generated by the
sets $\{(u\cdot v)\cdot w-u\cdot(v\cdot w)\mid u,v,w\in\mathcal{F}\}$ and $\{u\cdot u, J(u,v,w)\mid u,v,w\in\mathcal{F}\}$
where $J$ is the homogeneous term associated to the Jacobi identity \pref{jacob}.
We define
$$A^{+}(X,\cur)=\mathcal{F}(X,\cur)/\mathcal{A}\quad\text{and}\quad L(X,\cur)=\mathcal{F}(X,\cur)/\mathcal{B}$$
as respectively the {\em free associative} and the {\em free Lie} algebra {\em on $X$ over $\cur$}.

In \cite[2.1]{bah} is proved, that both $\mathcal{A}$ and $\mathcal{B}$ are {\em homogeneous ideals},
that means $\mathcal{A}=\sum_{i}\mathcal{A}\cap\mathcal{F}_{i}$ and $\mathcal{B}=\sum_{i}\mathcal{B}\cap\mathcal{F}_{i}$.

It follows $A^{+}(X,\cur)$ and $L(X,\cur)$ inherit from $\mathcal{F}$ the grading:
\begin{labeq}{gradal}
A^{+}(X,\cur)=\bigoplus_{i\geq1}\mathcal{F}_{i}/\mathcal{A}\cap\mathcal{F}_{i}\quad\text{and}\quad
L(X,\cur)=\bigoplus_{i\geq1}\mathcal{F}_{i}/\mathcal{A}\cap\mathcal{F}_{i}.
\end{labeq}

We denote  by $A_{i}=A_{i}(X,\cur)$ and $L_{i}=L_{i}(X,\cur)$ %, for $1\leq i<\omega$
the $\cur$-submodules in the grading above, and call them {\em homogeneous submodules} of weight $i$.

$A_{i}$ and $L_{i}$ are generated by monomials of weight $i$:
the images in $A^{+}$ and $L$ of words in $\mathcal{M}_{i}(X)$.
It is customary to speak of {\em degree} $i$ for the elements of $A_{i}(X,\cur)$ instead of weight.
\begin{rem}\label{asslie}
Any Lie $\cur$-algebra $M$ is the image of a free Lie algebra $L(X,\cur)$ for some $X$.
Moreover any associative
algebra $A$ is endowed with a product $[a,b]=ab-ba$ which turns $(A,+,[\,\,,\,])$ into a Lie algebra.
As a consequence, there exists a natural Lie $\cur$-algebra homomorphism $\epsilon$ of $L(X,\cur)$ onto the Lie subalgebra
of $A(X,\cur)$ generated by $X$ such that $\epsilon\colon x\mto x$ for all $x\in X$.
\end{rem}

\medskip
Whether for the $\cur$-module $A(X,\cur)$ a $\cur$-basis is provided by all ordered products over $X$, for $L(X,\cur)$ we recur
to the so called basic monomials, also called {\em Hall's Families}. In the groups context the very same definition applies to {\em basic
commutators}.
\begin{dfn}[Basic Monomials]\label{basicommutators}
We inductively construct a linearly ordered set of Lie monomials $\mathscr{B}=\bigcup_{n\geq1}\mathscr{B}_{n}$,
where each $\mathscr{B}_{n}\inn L_{n}(X,\cur)$ will be called the set of {\em basic} monomials of weight $n$.

Let $\mathscr{B}_{1}$ coincide with some linear order on $X$.

Assume a set of {\em basic monomials} $\mathscr{B}_{<n}=\bigcup_{i<n}\mathscr{B}_{k}$ {\em of weight less than $n$} has been defined and totally
ordered, by choosing a linear ordering for each $\mathscr{B}_{k}$ and following the rule: $a>b$ holds whenever
the weight of $a$ is greater than the weight of $b$. %, for all $a,b\in\mathscr{B}_{<n}$.

Now consider any pair of monomials $u,v$ in $\mathscr{B}_{<n}$, the sum of whose weights is $n$.
% equals the sum of their %$\wgt(u)+\wgt(v)=n$
Then the product $[u,v]$ is a {\em basic monomials of weight} $n$ and lays in $\mathscr{B}_{n}$ if both of the
following conditions are satisfied:
\begin{itemize}
\item[-]$u>v$,
\item[-]if $u=[z,w]$ for $z,w\in\mathscr{B}_{<n}$, then $w\leq v$.
\end{itemize}
\end{dfn}

The following result is referred to as {\em Hall's Basis Theorem}.\footnote{Although ambiguous, the name doesn't harm
the fatherhood of both Philip and Marshall.

To the former, one attributes the so called {\em collecting process} (see \cite{mhall}), from which Definition \ref{basicommutators} arise.
It is an algorithm to stepwise transform a group word into an ordered expression of basic commutators.
In \cite{mhalll}, Marshall Hall describes a collecting process in the context of Lie rings but claims the same results to hold
for Lie algebras over any field.

The proof in \cite{bah} -- attributed to A.{}I.\,Shirshov -- holds for arbitrary commutative rings.}
\begin{teo}[{\cite{mhalll},\cite[Theorem 2.2.1]{bah}}]\label{hallstheor} 
Let $\mathscr{B}$ a set of basis commutators on $X$ in $L=L(X,\cur)$. Then $L$ is a free $\cur$-module with
basis $\mathscr{B}$.

In particular each homogeneous submodule $L_{n}(X,\cur)$ is free, with basis $\mathscr{B}_{n}$ for all $n\geq1$. 
\end{teo}
For a clear account of the group theoretical analogous around Hall's Theorem and the {\em collecting process}
we refer to \cite{khk}.

As a corollary to the above theorem in \cite{bah} we find
\begin{fact}\label{ellea}
The canonical Lie morphism $\epsilon$ of Remark \ref{asslie}
mapping $L(X,\cur)$ into $A(X,\cur)$ is a Lie algebra embedding.
\end{fact}
The above $\epsilon$, coincide with the canonical map of a Lie algebra $L$ in its
{\em universal envelope} $U(L)$ (see also \cite{ser}), this is always an embedding provided
$L$ is a free $\cur$-module.

The following fact will also be needed.
\begin{fact}[{\cite[Lemma 2.3.3]{bah}}]\label{basisgen}
Let $L$ be the free Lie algebra $L(X,\cur)$ over the set $X$. If $\mathcal{B}$ is a basis
for the free $\cur$-submodule $\gen{X}_{\cur}=L_{1}(X,\cur)=\mathcal{F}_{1}(X,\cur)$ of $L$, then $L(\mathcal{B},\cur)=L$.
\end{fact}

\bigskip
We introduce below the class of nilpotent Lie algebras, which are to be associated a notion of Hrushowski predimension
in Chapters \ref{due}  and \ref{tre}. 
We show that these structures isolate exactly those algebras which arise from $\ngb{c}{p}$-groups as images under the
functor $\gr$.

\begin{dfn}\label{lcp}
For a prime number $p$ and a positive integer $c$, we define by $\nla{c}$ the class of all $c$-nilpotent (graded)
Lie algebras $M$ over the $p$-element field $\Fp$, which satisfy the following properties:
\begin{enumerate}
\item there are $\Fp$-subspaces $M_{i}$ for $1\leq i\leq c$  with $M=M_{1}\oplus\dots\oplus M_{c}$,
\item$[M_{i},M_{j}]\inn M_{i+j}$ for all $i,j$ ($M_{k}$ is defined to be $\triv$ for $k>c$),
\item\label{genero}$M=\gen{M_{1}}$
\end{enumerate}
\end{dfn}
\begin{rem}
The whole {\em grading} $(M_{i})_{i<\omega}$ depends indeed only on the {\em choice} for the space $M_{1}$: by property $3.$
above each subspace $M_{i}$ is the $\Fp$-subspace of $M$ generated by simple monomials of weight $i$ in the elements from
(a $\Fp$-basis of) $M_{1}$.
\end{rem}

\smallskip
Lie subalgebras of an $\nla{c}$-algebra $M$ are not always again $\nla{c}$-objects,
we hence {\em define} an $\nla{c}$-subalgebra $H$ of $M$, if $H=\gena{H_{1}}{M}$ for some $\Fp$-subspace
$H_{1}$ of $M_{1}$.
By an $\nla{c}$-morphisms we mean a graded homomorphism of Lie algebras. That is, if $\map{\phi}{M}{N}$ for $M,N\in\nla{c}$, then
$\phi(M_{i})\inn N_{i}$ for all $i$.

\begin{rem*}
As observed above, we get a correspondence
\begin{labeq}{grafu}
\map{\gr}{\ngb{c}{p}}{\nla{c}}
\end{labeq}
where, if $G\in\ngb{c}{p}$ and $M$ denotes $\gr(G)$, $G_{\it ab}$ corresponds to $M_{1}$ of Definition \ref{lcp}.
Since the terms of the lower central series are fully invariant, the map above is a functor provided we allow
Lie morphisms in general. Group homomorphisms may have non-graded images under $\gr$.

In the next section, it will be shown however that $\gr$ is onto of the respective objects.
\end{rem*}

\smallskip
For a first-order treatment of $\nla{c}$ we choose the signature $\Lan{c}$ consisting of
ring symbols $\triv$, $+$ and $[\,\,,\,]$, of the scalar functions from $\Fp$ %\mn{\bf Redundant nach Ziegler (?)}
and of predicates $P_{i}$ which are interpreted by the grading ($P_{i}(M)=M_{i}$).
Notice that $\nla{c}$ is not an elementary $\Lan{c}$-class. Property \ref{genero}.\,(as opposed to 1.\,and 2.) of Definition \ref{lcp}
cannot be expressed at the first order in $\Lan{c}$ unless a bound to the length of sums in $M$ is given.

\medskip
By the previous discussion, for any set $X$, the {\em free $c$-nilpotent Lie $\Fp$-algebra over $X$}, which we define as
$$\fla{c}{X}:=L(X,\Fp)/\gamma_{c+1}(L(X,\Fp))$$ is an object of $\nla{c}$.
This is for $L(X,\Fp)$ is a graded Lie algebra and $\gamma_{c+1}(L(X,\Fp))$ is an homogeneous ideal which is equal to $\sum_{i>c}L_{i}(X,\Fp)$.

Similarly, for any object $M$ of $\nla{c}$, $\gamma_{i}(M)=\sum_{j\geq i}M_{j}$.

\medskip
As a corollary to Hall's Theorem above we get (cfr.\,\cite[Corollary 2.7.3]{khk})
\begin{fact}\label{ubc}
For any given set $\mathscr{B}$ of basic monomials over $X$, % in $L(X,\Fp)$,
denote by $\mathscr{B}_{\!{\sss\leq} c}$
the set of elements in $\mathscr{B}$ of weight not greater than $c$. Then $\mathscr{B}_{\!{\sss \leq} c}$ is
a $\Fp$-basis of $\fla{c}{X}$ and in particular
every element $w$ of $\fla{c}{X}$ admits a unique linear combination
\begin{gather*}\label{BC}\tag{\sf BC}
\left.\begin{split}
w=\sum_{b \in\mathscr{B}_{\leq c}} s_{b}{b}\quad\text{with ${b}$ in $\mathscr{B}_{\leq c}$ and non-trivial $s_{b}\in\Fp$}
\end{split}\right.
\end{gather*}
\end{fact}
\begin{dfn}\label{supp}
If $\mathscr{B}_{\leq c}$ and $\fla{c}{X}$ are as above, we define the {\em support} of an element $w\in\fla{c}{X}$,
as the minimal subset $\supp(w)$ of $X$, for which each basic monomial $b$ in the sum \pref{BC} above, carries entries from
$\supp(w)$ according to Definition \ref{basicommutators}. We may specify the set over which $\mathscr{B}$ is constructed,
by writing $\supp_{X}(w)$. 

If a basic monomial $b\in\mathscr{B}$, has support in a subset $Y$ of $X$, we refer to
$b$ also as a monomial {\em over $Y$}, or shortly, as a basic $Y$-monomial.
\end{dfn}

\begin{rem}
With Fact \ref{basisgen} and Remark \ref{asslie}, $L^{c}(\,\cdot\,)$ may be seen as a free functor of
$\Fp$-vector spaces into $\nla{c}$-algebras, adjoint to the predicate $P_{1}$ of $\Lan{c}$:
for any $\Fp$-vector space $V$ and $\nla{c}$-algebras $M$, we have -- with the obvious maps -- a bijection
\begin{labeq}{freeadjoint}
\Hom_{\Fp}(V,P_{1}(M))\to\Hom_{\nla{c}}(\fla{c}{V},M).
\end{labeq}
\end{rem}

\smallskip
In particular any object $M$ of $\nla{c}$ is the quotient of $\fla{c}{M_{1}}$ modulo an homogeneous ideal. In the above notations
$R=\sum_{i\leq c}L_{i}(M_{1},\Fp)\cap R$ with $M_{1}\cap R=\triv$. We will write $R=R_{2}+\dots+R_{c}$.

Since the subspace $M_{1}$ is intrinsic to the structure $M$, the choice of the relators ideal $R$ may be regarded
as canonically associated to $M$.

By a {\em $\nla{c}$-presentation} of $M$ we denote both the expression
$M=\gen{M_{1}\mid R}$ and the associated exact sequence
\begin{labeq}{pres}
R\lto\fla{c}{M_{1}}\stackrel{\epsilon_{M}}{\lto}M
\end{labeq}

On the other hand, to any homogeneous ideal $R$ of $L=\fla{c}{X}$, the quotient $L/R$ is an object of $\nla{c}$.

We say that $M$ is finitely generated if $M_{1}$ has finite $\Fp$-dimension, hence exactly if $M_{1}$ (and $M$) is finite.
Note that in the category $\nla{c}$ the notion of {\em finitely presented} (that is $M_{1}$ and $R$ are
finite dimensional) coincide with being finitely generated. The same holds in general for nilpotent groups.

\medskip
As a result of Definition \ref{lcp}, morphisms among objects of $\nla{c}$ aren't richer than those among their generating $\Fp$-vector spaces.
%Let $M$ and $N$ be presented from $\fla{c}{M_{1}}$ and $\fla{c}{N_{1}}$
%by means of epimorphisms $\epsilon_{M}$ and $\epsilon_{N}$ respectively as in \pref{pres}.
%
%Assume an $\nla{c}$-morphism $\phi$ of $M$ into $N$ is given.
%
%If $\phi_{1}$ denotes the restriction of $\phi$ to the subspace $M_{1}$ with image in $N_{1}$, then
%we may assume that $\map{\phi_{1}}{M_{1}}{\fla{c}{N_{1}}}$. Now apply \pref{freeadjoint} to $\phi_{1}$
%to get a morphism $\map{\widehat\phi}{\fla{c}{M_{1}}}{\fla{c}{N_{1}}}$. It is clear that the following holds.

\begin{lem}\label{commufreeno}
For any $M$ and $N$ in $\nla{c}$,
%once two presentations $\gen{M_{1}\mid R_{M}}$ and $\gen{N_{1}\mid R_{N}}$ as above are chosen,
to any $\nla{c}$-morphism $\phi$ of $M$ to $N$, there is a unique $\widehat\phi\in\Hom_{\nla{c}}(\fla{c}{M_{1}},\fla{c}{N_{1}})$
%As $\phi$ and $\phi_{1}\epsilon_{N}$ coincide on $M_{1}$ and $\phi_{1}=\iota_{M}\widehat\phi$,
which makes the square below commute
\begin{labeq}{commufresco}
\begin{split}
\xymatrix{
\fla{c}{M_1}\ar[r]^{\widehat\phi}\ar[d]^{\epsilon_{M}}&\fla{c}{N_{1}}\ar[d]^{\epsilon_{N}}\\
M\ar[r]^{\phi}&N
}
\end{split}
\end{labeq}
\end{lem}

\subsection*{Relations between the $\ngb{c}{p}$-free  group and the free $\nla{c}$-algebra}
%We will describe two ways by which is possible to recover a group from a given Lie algebra,
%the first method, which is more group theoretical furnishes a functor $\mathscr{G}$ of $\nla{c}$
%to $\ngb{c}{p}$ %``{inverse}'' to $\gr$
%such that for all groups $G$ in $\ngb{c}{}$ gives $\mathscr{G}(\gr(G))=G$ and $\gr(\mathscr{G}(L))=L$ for all
%nilpotent Lie rings $L$. 
%
%The second process uses more {\em topological} Campbell Hausdorff formula to
%equip a Lie ring with a a group structure, which in the nilpotent case allows a model theoretical interpretation
%of groups in Lie rings.

Before we prove that \pref{grafu} is onto, we first establish a correspondence between the free objects in the
classes $\nla{c}$ and $\ngb{c}{p}$.

\smallskip
Let $A^{+}(X)$ be the free associative algebra $A^{+}(X,\Fp)$ over $\Fp$ defined above. We add a
multiplicative unit -- and hence {\em elements of zero degree} -- by defining
$A(X)$ to be $\Fp\oplus A^{+}(X)$ and extending addition and multiplication in the natural way.
$A(X)$ inherits the grading \pref{gradal} and we set $A_{0}(X)=\Fp$.

Let $A^{c}(X)$ be the quotient algebra of $A(X)$ modulo the ideal $\sum_{i>c}A_{i}(X)$,
that is the {\em free} unitary associative nilpotent algebra of class $c$. In particular
$$A^{c}(X)\simeq A_{0}(X)\oplus A_{1}(X)\oplus\cdots\oplus A_{c}(X).$$

Let now $F_{p}(X)$ denote the free group of exponent $p$ on the set $X$, then
$F_{p}^{c}(X):=F_{p}(X)/\gamma_{c+1}
(F_{p}(X))$ is the free group in $\ngb{c}{p}$.

\smallskip
Now assume $c<p$, since $(1+x)^{p}=1+x^{p}=1$ in $A^{c}(X)$, one can extend the map $X\ni x\mapsto1+x$ to a group homomorphism
$\phi$ of $F_{p}(X)$ onto the subgroup $\gen{1+x\mid x\in X}$
of the units of $A^{c}(X)$ (the multiplicative inverse of $1+a$ being $1-a+a^{2}-a^{3}+\cdots+(-1)^{c-1}a^{c-1}$).

\smallskip
If we put together \cite{witt},\cite{mag37},\cite{mag},\cite[Lemma 11.2.2]{mhall} and \cite[Theorem 6.3]{ser}, we find that
the nucleus of $\phi$ coincides with $\gamma_{c+1}(F_{p}(X))$ and the following facts hold.
\begin{fact}%[{\cite{witt},\cite{mag},\cite[Lemma 11.2.2]{mhall}}]
%\label{lambda}
If we assume $c<p$ we have an injective group homomorphism $\phi$ of $F_{p}^{c}(X)$ into the units of $A^{c}(X)$
extending $x\mto1+x$ for $x\in X$.

For all words $w$ of $F_{p}^{c}(X)$ we set
$$\phi\colon w\longmapsto 1+\lambda(w)+W$$
where $\lambda(w)$ is the %$n^{\text{th}}$
homogeneous component $\phi(w)_{n}\in A_{n}(X)$ of $\phi(w)$ of minimal positive degree $n\leq c$ such that
%to be preceded by zero components only. That is if $$\phi(w)=1+\phi(w)_{1}+\cdots+\phi(w)_{c}
%\quad\text{with $\phi(w)_{i}$ in $A_{i}(X)$}$$ then
$\phi(w)_{1}=\dots=\phi(w)_{n-1}=\triv$ and $W$ is a sum of components of higher degree.
We also say that $n$ is the {\em weight} of $w$ and write $\wg(w)=n$. $\lambda(w)$ is called the {\em leading term} of $\phi(w)$ and
has the properties:
\begin{itemize}
\item[-]$\lambda(gh)$ is $\lambda(g)$ or $\lambda(h)$ according to which among $g$ and $h$ has lower weight.
If $g$ and $h$ have the same weight and $\lambda(g)+\lambda(h)\neq\triv$, then $\lambda(gh)=\lambda(g)+\lambda(h)$.
\item[-]$\lambda(g^{-1})=-\lambda(g)$.
\item[-]If $[\lambda(g),\lambda(h)]\neq\triv$, then $\lambda([g,h])=[\lambda(g),\lambda(h)]$.
%The weight of  $[g,h]$ is not smaller than $\wg(g)+\wg(h)$.
$\wg([g,h])\geq\wg(g)+\wg(h)$ and if $[\lambda(g),\lambda(h)]=\triv$, then $\wg(g)=\wg(h)$ and $\lambda(g^{h})=\lambda(g)$.
\item[-]$\gamma_{i}(F^{c}_{p}(X))$ coincides with the set of all $g$ in $F^{c}_{p}(X)$ with $\wg(g)\geq i$.
\end{itemize}
Here above, $[\lambda(g),\lambda(h)]$ denotes the Lie product $\lambda(g)\lambda(h)-\lambda(h)\lambda(g)$
in the associative algebra $A^{c}(X)$.
\end{fact}
If $\epsilon$ is the canonical embedding of Fact \ref{ellea}, since we have $\epsilon(\gamma_{k}(L(X)))=
\epsilon(L(X))\cap\sum_{i\geq k}A_{i}(X)$ for all $1\leq k$, $\epsilon$ factorises to a Lie monomorphism
of the free nilpotent Lie algebra $L^{c}(X)$ into $A^{c}(X)$, hence we identify
$\fla{c}{X}$ with the Lie subalgebra generated by $X$ in $A^{c}(X)$.

By the above facts we obtain a map
\begin{labeq}{lambda}
\map{\lambda}{F^{c}_{p}(X)}{L^{c}(X)}
\end{labeq}
which gives rise to a well defined {\em injective} $\nla{c}$-morphism
\begin{align}\label{barlambda}
\notag\lmap{\bar\lambda}{\gr(F^{c}_{p}(X))}{&\fla{c}{X}}\\
\notag\sum\bar u_{i}\lmto &\sum\lambda(u_{i})
\end{align}
which maps $X$ identically onto $X$.

Since on the contrary, $\gr(F^{c}_{p}(X))$ is the image of an epimorphism of $\fla{c}{X}$ which extends the identity on $X$,
it follows
\begin{rem}\label{barlambdarem}
$\gr(F^{c}_{p}(X))$ and $\fla{c}{X}$ are $\nla{c}$-isomorphic via $\bar\lambda$.
\end{rem}

Notice that the condition $p>c$ is necessary. We have for instance, in $F_{p}(X)$, that Engel elements $[x,y,\dots,y]$ of length $p$
are congruent to $1$ modulo $\gamma_{p+1}(F_{p}(X))$ for all $x,y$ (cfr.{\,}\cite[Theorem 2.8.11]{khk}).
%On this theme, one should also consult \cite[IV.6]{ser}, \cite{mag37} and \cite{mag}.
%By the fact above follows that $\phi(\gamma_{c+1}(F_{p}(X)))=1$ and hence we may factorise $\phi$ to a group
%homomorphism
%$$\map{\phi}{F_{p}^{c}(X)}{A^{c}(X)^{*}}.$$
%Moreover, we still have a well defined map $\lambda\colon\bar w\mapsto\lambda(w)$ for all $\bar w=\gamma_{c+1}(F_{p}(X))w$ in
%$F_{p}^{c}(X)$.
%
%\smallskip
%Now for $1\leq n\leq c+1$ %for $1\leq n\leq c+1$
%%let ${\mathfrak m}_{n}$ be the ideal $\sum_{i\geq n}A^{c}_{i}(X)$. And
%%$\lambda(g)\in\sum_{i\geq n} A_{i}(X)$.
%Fact \ref{lambda} above implies $H^{n}$ are subgroups of $F^{c}_{p}(X)$ with $[H^{i},H^{j}]\inn H^{i+j}$ for all $i+j\leq c+1$,
%%and $H^{c}=1$,
%that is, if we set $H^{i}=\{1\}$ for all $i>c+1$, then $(H^{i})_{i<\omega}$ is a Lazar series, as defined above. %of length $c$.
%This is in particular a central series and hence $\gamma_{i}(F^{c}_{p}(X))\inn H^{i}$.
%
%If we assume $p>c$ then the same arguments used in a theorem of \cite{ser} which is proved in characteristic $0$,
%applies to our means:
%\begin{fact}[{\cite[Theorem 6.3]{ser}}]\label{serreth} If $p>c$, then in the above notations we have
%$$H^{i}=\gamma_{i}(F^{c}_{p}(X))$$
%for all $i$.
%\end{fact}
%\uwave{The assumption on $p$ is needed, to prevent that commutators of length $p$ to be trivial}. See on this purpose \cite[\S3.3]{khk}. 
%
%
%It follows in particular $\ker(\phi)=H^{c+1}=\gamma_{c+1}(F^{c}_{p})=1$.
%%Let $\mathcal{H}$ be the Lie ring obtained from the Lazard series $(H^{i})$ as in $(\gr)$ above.
%%We have in particular a canonical map of $\gr(F^{c}_{p}(X))$ into $\mathcal{H}$ and the map $\phi$ above
%%is a monomorphism.
%
%\smallskip
%Now by means of $\lambda$ with facts \ref{lambda} and \ref{serreth} we obtain for all $i$, a $\Fp$-linear monomorphism
%$\lambda_{i}$ of the $\Fp$-vector space $H^{i}/H^{i+1}$ into
%$A_{i}(X)$, given by $\lambda_{i}\colon\bar w\mapsto\lambda(w)$. This
%gives rise to a Lie homomorphism $\tilde\lambda$ of $\gr F_{p}^{c}(X)$ into the Lie subring of $A^{c}(X)$ generated by $X$.
%
%Now since $\gr F_{p}^{c}(X)$ is generated by the image of $X$ modulo $H^{2}$ and is a nilpotent
%Lie algebra of class $c$, there is a Lie epimorphism $\eta$
%of $L^{c}(X)$ onto $\gr F_{p}^{c}(X)$. Since the image of $X$ under the composition
%$$L^{c}(X)\stackrel{\eta}{\longrightarrow}\gr F_{p}^{c}(X)\stackrel{\tilde\lambda}{\longrightarrow}A^{c}(X)$$
%coincide with that of $\map{\epsilon}{L^{c}(X)}{A^{c}(X)}$ above, one obtains $\eta\tilde\lambda=\epsilon$ and hence
%the following
%\begin{fact}\label{lambdiso}
%$\gr F_{p}^{c}(X)$ is $\nla{c}$-isomorphic as a Lie $\Fp$-algebra with $L^{c}(X)$, and maps via $\tilde\lambda$ isomorphically onto the Lie subring of $A^{c}(X)$ generated by $X$.
%
%Moreover $\tilde\lambda$ maps $H^{i}/H^{i+1}$ $\Fp$-isomorphically over $L_{i}(X)$.
%\end{fact}
\subsubsection{Retrieving groups from $\nla{c}$-algebras}\label{algegruppi}
Let $M$ be a Lie algebra of $\nla{c}$ with $p>c$. As observed above $M$ is isomorphic to the quotient $\fla{c}{X}/R$,
where $X$ is a $\Fp$-basis of $M_{1}$ and $R$ is a homogeneous ideal of $\fla{c}{X}$.

The idea is to associate $M$ to a quotient of $F^{c}_{p}(X)$: we need to find a suitable normal subgroup.
Consider the map \pref{lambda} above and define
$$N=\left\{w\in F_{ p}^{c}(X)\mid\lambda(w)\in R\right\}$$
then by Fact \ref{lambda}, as $\lambda(hg^{-1})$ equals $\lambda(h)$ or $-\lambda(g)$ or
again $\lambda(h)-\lambda(g)$, $N$ is a subgroup of $F_{p}^{c}(X)$.

Moreover, the same fact implies that for all $g$ in $N$ and all $x$ in $X$, %$\lambda(g^{x})=
either $\lambda(g^{x})=\lambda(g)$ or $\lambda([g,x])=[\lambda(g),x]$ is in the ideal $R$. This yields that $N$ is a normal subgroup
of $F_{p}^{c}(X)$. Hence the quotient $F_{p}^{c}(X)/N$ which we denote by $\mathscr{G}(M)$,
is a group in the variety $\ngb{c}{p}$.

\medskip
We can now prove the following somewhat dual result to \cite[I.]{mag}.
\begin{prop}\label{pr:gruppozzo}
Let $p$ be a prime number greater than $c$.
With the above definition and Lemma \ref{commufreeno}, the map $M\mto\mathscr{G}(M)$
is a functor of $\nla{c}$ into $\ngb{c}{p}$-groups
such that $\gr(\mathscr{G}(M))\simeq_{\nla{c}}M$ for all $M$ in $\nla{c}$.

For a fixed $M$ in $\nla{c}$, then $\nla{c}$-subalgebras (ideals) of $M$ correspond -- via $\lambda$ --
to subgroups (normal subgroups) of $\mathscr{G}(M)$.
%and all groups $H$ of $\ngb{c}{p}$ with $\gr(H)=M$ are epimorphic image of $\mathscr{G}(M)$. 
\end{prop}
\begin{proof}
Assume $M=\gen{M_{1}\mid R}$ and $X$ is a $\Fp$-basis of $M_{1}$. Put $F=F_{p}^{c}(X)$ and
let $G=\mathscr{G}(M)$ be the quotient of $F$ modulo the subgroup $N$ defined above.

Since $R$ is a homogeneous ideal and $R=R_{2}+\dots+R_{c}$, then $M_{n}\simeq_{\Fp}L_{n}(X)/R_{n}$, for all $n\leq c$ and if $F^{n}$ denotes $\gamma_{n}(F)$
then, as abelian groups
$$%\begin{labeq}{gammapres}
\gr_{n}G=\gamma_{n}(G)/\gamma_{n+1}(G)\simeq F^{n}/F^{n+1}(F^{n}\cap N)\simeq\frac{\gr_{n}F}{F^{n+1}(F^{n}\cap N)/F^{n+1}}.
$$%\end{labeq}
On the other hand, Remark \ref{barlambdarem} and the definition of $N$ imply that the $\Fp$-isomorphism $\map{\bar\lambda}
{F^{n}/F^{n+1}}{L_{n}(X)}$ maps $F^{n+1}(F^{n}\cap N)/F^{n+1}$ exactly onto $R_{n}$.
It follows $M_{n}$ is isomorphic to $\gamma_{n}(G)/\gamma_{n+1}(G)$ as a $\Fp$-vector space, for all $n$.
Moreover, since %the Lie algebra structure on $\gr(G)$ is given by the quotient of $\gr(F)\simeq L^{c}(X)$ modulo
$\sum_{n} F^{n+1}(F^{n}\cap N)/F^{n+1}$ is an ideal of $\gr(F)\simeq L^{c}(X)$, then $\gr(G)$ is $\nla{c}$-isomorphic to $M$.

The remaining statements directly descend from the construction of $\mathscr{G}(M)$.
\end{proof}
\subsubsection{The Baker-Hausdorff Formula}
There is a second and more classical way to reconstruct groups from Lie algebras, which has a topological-analytical approach.
In our nilpotent context, this yields a more effective model-theoretical interpretation of the aforementioned correspondence.
To describe this method, we have to restart from the original Witt's {\em Treue Darstellung} \cite{witt} in characteristic zero.

\medskip
We mention here that a filtration $({\mathfrak g}_{i})_{i<\omega}$ of a $\cur$-algebra ${\mathfrak g}$ is a decreasing series of ideals
${\mathfrak g}_{i}$ with ${\mathfrak g}_{i}\cdot{\mathfrak g}_{j}\inn{\mathfrak g}_{i+j}$. 

In fact the lower central series $(\gamma_{k}(L))_{k}$ of a Lie algebra $L$ constitutes an example of (central) filtration.
We say that ${\mathfrak g}$ is {\em separated} with respect to the filtration $({\mathfrak g}_{i})$ if $\bigcap{\mathfrak g}_{i}=\triv$.
A separating filtration induces an Hausdorff (T$_{2}$) topology on ${\mathfrak g}$. We refer to \cite{bbk,laz} for
these notions.

\medskip
Consider the Magnuss algebra (\cite[\S5.1]{bbk}) $\widehat{A}=\widehat{A}(X,\Q)$
over the rationals.
This is the topological completion of the free associative unitary $\Q$-algebra
$$A=A(X,\Q)=\Q\cdot\!1\oplus A^{+}(X,\Q)$$ with respect to the
topology induced by the natural degree-filtration.

Elements of $\widehat{A}$ are noncommutative formal power series in the {\em indeterminates} $X$ and coefficients in $\Q$:
$$a=\sum_{i<\omega}a_{i}\qquad\text{for $a_{i}\in A_{i}(X,\Q)$, $a_{0}\in\Q$}.$$

As $L(X,\Q)$ is identified with the Lie subalgebra of $A(X,\Q)$ generated by $X$,
we define the elements of $\widehat{L}$ as the formal series $\sum_{i\geq1}b_{i}$ of $\widehat{A}$
with each homogeneous component $b_{i}$ belonging to $L_{i}(X,\Q)$.

If ${\mathfrak m}$ denotes the ideal $\sum_{i\geq1}A_{i}(X)$ of $\widehat{A}$, then $1+{\mathfrak m}$ is a  multiplicative group.
We obtain the continuos bijections (cfr.\cite[IV.7]{ser})
\begin{align*}
\lmap{\exp}{{\mathfrak m} &}{ 1+{\mathfrak m} }&\lmap{\log}{1+{\mathfrak m}&}{{\mathfrak m}}\\
a&\longmapsto\sum_{i\geq0}\frac{a^{i}}{i!}&1+b&\longmapsto\sum_{i\geq1}(-1)^{i+1}\frac{b^{i}}{i}
\end{align*}
with the usual properties $\log(\exp a)=a$ and $\exp(\log(1+b))=1+b$.
%and also if $[a,b]=\triv$ then $\exp(a+b)=exp(a)\exp(b)$. Also $\widehat{L}\inn{\mathfrak m}$ of course.

\begin{fact}[{\cite[1.IV.7]{ser}\cite[Theorem 6.1,1]{bah}}]\label{expelle}
In the above notations,
$\exp(\widehat{L})$ is a multiplicative subgroup of $1+{\mathfrak m}$.

Moreover if $\epsilon$ denotes the homomorphism of the free group $F(X)$
on $X$ into $1+{\mathfrak m}$ which extends $x\mapsto\exp(x)$ for all $x$ in $X$,
then $\epsilon\log$ is a goup monomorphism of $F(X)$ into $(\widehat{L},\circ)$
where $\circ$ is the group law on $\widehat{L}$ given by
$$\xi\circ\eta=\log(\exp(\xi)\exp(\eta))$$
for all $\xi,\eta\in\widehat{L}$.
\end{fact}

\begin{teo}[{\cite[Proposition 6.2.1]{bah}},{\cite[Proposition \S5.4]{bbk},{\cite[{\sc Th\'eor\`eme 4.2}]{laz}}}]\label{faikaha}
Let now $X$ be the set $\{x,y\}$ the element of $\widehat{A}(x,y,\Q)$, then in the previous notation, we have
\begin{labeq}{hausdseries}
x\circ y=:\mathsf{H}(x,y)=\sum_{i=1}^{\infty}t_{i}\mathsf{h}_{i}(x,y)
\end{labeq}
where $\mathsf{h}_{i}(x,y)$ is a homogeneous term in $L_{i}(\{x,y\},\Z)$ of total weight $i$ in $x$ and $y$ and $t_{i}\in\Q$.

For any complete, separated, filtered Lie Algebra ${\mathfrak g}$ with filtration $({\mathfrak g}_{\alpha})$, over a characteristic zero
field $\cur$, the map
\begin{align}\label{hausdgroup}
\lmap{\circ}{&{\mathfrak g}\times{\mathfrak g}}{{\mathfrak g}}\\
%\intertext{defined by}
&(a,b)\longmapsto\mathsf{H}(a,b)\tag*{}
\end{align}
induces a {\em group structure} on ${\mathfrak g}$ compatible with the topology and such that
\begin{itemize}
\item[-]the neutral element of $({\mathfrak g},\circ)$ is the additive zero $\triv$ and for any element $m$, the $\circ$-inverse $m^{-1}$ of
$m$ coincides with the additive inverse $-m$. Moreover %$m\circ\triv=m=\triv\circ m$
%\item[-]$m\circ-m=\triv=-m\circ m$
%\item[-]
the $n${\sl -th} power $a^{n}$ in $\circ$ of any element $a$ of ${\mathfrak g}$ is $n\cdot a$ for all $n\in\Z$

\item[-]the {\em group commutator} $[l,m]$ built from the group operation $\circ$
equals the Lie product $[l,m]$ in ${\mathfrak g}$ modulo the ideal ${\mathfrak g}_{\alpha+1}$ provided
$l$ or $m$ is in ${\mathfrak g}_{\alpha}$.

\item[-]the chain $({\mathfrak g}_{\alpha})$ becomes a central series of $({\mathfrak g},\circ)$. The quotient group operation induced by $\circ$ on ${\mathfrak g}_{\alpha}/{\mathfrak g}_{\alpha+1}$ coincides with the abelian structure of the quotient algebras.
\end{itemize}
\end{teo}
For an explicit calculation of the terms $s_{i}\mathsf{h}_{i}(x,y)$ in \pref{hausdseries} one may see
\cite[IV.8]{ser}. A first segment of $\xi\circ\eta$ is given by
\begin{labeq}{hausdexpl}
\xi\circ\eta=\xi+\eta+\frac{1}{2}[\xi,\eta]-\frac{1}{12}([\xi,\eta,\eta]+[\eta,\xi,\xi])+\cdots
\end{labeq}

\medskip
Now the crucial fact which allows us to apply the above machinery to $\Fp$-algebras in $\nla{c}$ for $p>c$,
is the following observation.
\begin{fact}[\cite{mag,laz}]\label{modp}
Let $\Q_{c}$ denote the subring of $\Q$ which consists of all quotients $r/s$ for coprime $r,s$ such that
if a prime $q$ divides $s$, then $q\leq c$.
In \pref{hausdgroup} above we have $t_{i}\in \Q_{i}$ for all $i<\omega$.
\end{fact}
Notice that, since $p>c$ as a $\Fp$-vector space any object $M$ in $\nla{c}$ carries
a $\Q_{c}$-algebra structure, simply letting $r/s\cdot m$=$\bar r\bar{s}^{-1}m$ where
$\bar r$ and $\bar s$ denote $r$ and $s$ modulo $p$.

As observed in \cite{bbk}, to a {\em finite} central filtration
%in an algebra ${\mathfrak g}$ over $\cur$ as above, then ${\mathfrak g}$ 
automatically corresponds a complete and separated (discrete) topology. This is the case for nilpotent algebras.
In particular Theorem \ref{faikaha} and Fact \ref{modp} yield (cfr.{\,}\cite[Theorem II,4.2]{laz}):
\begin{cor}\label{co:grupphausdorff}
For $p>c$, considering the lower central filtration on $\nla{c}$-algebras, we obtain
\begin{align}
\lmap{G}{\nla{c}&}{\ngb{c}{p}}\label{haugrp}\\
M&\longmapsto G(M)=(M,\circ,\triv)\notag
\end{align}
By Theorem \ref{faikaha} and by the definition of $\nla{c}$, for each such algebra $M$,
since $M=\gen{M_{1}}$ we have $\gr(G(M))=M$. Moreover for any $\nla{c}$-extension $M\nni N$,
the corresponding groups $G=G(M)$ and $H=G(N)$ satisfy $\gamma_{k}(H)=\gamma_{k}(G)\cap H$.
In particular $\gr(H)$ is an $\nla{c}$-subalgebra of $\gr(G)$.
\end{cor}

\medskip
\begin{cor}\label{co:interp}
For any $\nla{c}$-algebra $M$, the group $G(M)$ is {\em definably interpretable} {\rm(\cite[p.{}24]{mar})} in the $\Lan{c}$-structure $M$.
\end{cor}
\begin{rem}\label{baucond}
In Lemma 3.1 of \cite{bad}, a different approach is described, to reconstruct a group law from a class
of $\Fp$-vector spaces, identifiable with our $\nla{2}$.
That is motivated by the following instance of the collecting process, peculiar of nilpotency class $2$.

Let $G$ be a $\ngb{2}{p}$-group and assume a subset $\{a_{\alpha}\mid \alpha<\kappa\}\inn G$ has been chosen, which -- modulo $G^{\prime}$ --
is a base of the $\Fp$-vector space $G_{ab}$. Then, any element $g$
of $G$ writes in a unique way as a product $g=\prod_{\alpha}a_{\alpha}^{r_{\alpha}}x$ for some $x\in G^{\prime}$ and (with the
due precautions) $r_{\alpha}\in\Fp$. %$r_{\alpha}<p$.

 As $G^{\prime}\inn Z(G)$, if $h=\prod_{\alpha}a_{\alpha}^{s_{\alpha}}y$, then
$gh=\prod_{\alpha}a_{\alpha}^{r_{\alpha}+s_{\alpha}}\prod_{\alpha>\beta}[a_{\alpha},a_{\beta}]^{r_{\alpha}s_{\beta}}xy$.

This yields %another way to define
a group operation $\bullet$ on $\gr(G)$ defined as follows:\footnote{This is not the Hausdorff formula \pref{hausdexpl},
\pref{hausdseries}. To obtain it
one has to replace $r_{\alpha}s_{\beta}$ with $\dfrac{1}{2}(r_{\alpha}s_{\beta}-r_{\beta}s_{\alpha})$ in the last summand.}
$$ %\begin{labeq}{bullo}
\left(\sideset{}{_{\alpha}}\sum r_{\alpha}\bar a_{\alpha}+x\right)\bullet\left(\sideset{}{_{\alpha}}\sum r_{\alpha}\bar a_{\alpha}+y\right)=
\sideset{}{_{\alpha}}\sum(r_{\alpha}+s_{\alpha})\bar a_{\alpha}+x+y+\sideset{}{_{\alpha>\beta}}\sum r_{\alpha}s_{\beta}[a_{\alpha},a_{\beta}]
$$ %\end{labeq}
The peculiarity of this setting, is now the fact that $(\gr(G),\bullet)$ is now isomorphic-- as a group -- to $G$. If we
define $\bullet$ on arbitrary $\nla{2}$-algebras, we obtain a 1-1 correspondence of $\ngb{2}{p}$ with $\nla{2}$ at level of objects.

\medskip
In addition, Baudisch works in the subclass ${\mathfrak G}=\{G\in\ngb{2}{p}\mid G^{\prime}=Z(G)\}$. For if $H\sg G$ and both
$H,G\in{\mathfrak G}$, then we have an $\nla{2}$-inclusion of $\gr(H)$ into $\gr(G)$.
This is because, the condition in ${\mathfrak G}$ implies $H^{\prime}=G^{\prime}\cap H$ and hence $H_{ab}$ embeds as a vector space
into $G_{ab}$.

The class ${\mathfrak G}$ allows therefore to switch between groups and algebras in a clean way when we manipulate
group embeddings in the Fra\"iss\'e construction.

On the other hand since $Z(G)$ -- like every term of the upper central series -- is a definable set in the pure group language,
properties of $\gr(G)$ may be described at the first order with the signature of groups only.

\smallskip
In Chapter \ref{due} we re-obtain this property for $\nla{2}$-algebras $M$, by imposing $2$-generated $\nla{2}$-subalgebras to be free.
\end{rem}

\smallskip
If we consider the analogous property in $\ngb{c}{p}$, namely
\begin{labeq}{gammazeta}
\zeta_{k}(G)=\gamma_{c+1-k}(G)\quad\text{for $G\in\ngb{c}{p}$},
\end{labeq}
then the group-algebra correspondences introduced so far, like $\gr$ and -- reversely -- Proposition \ref{pr:gruppozzo} and \pref{haugrp},
all preserve this feature from $G$ to $M$ and vice-versa: \pref{gammazeta} holds iff
$\zeta_{k}(M)=\gamma_{c+1-k}(M)$ for $M\in\ngb{c}{p}$. Note that in both cases the class $c$ condition, always imply
the inclusion $\zeta_{k}(\,\cdot\,)\nni\gamma_{c+1-k}(\,\cdot\,)$.
%%\section{Nilpotent Lie algebras}
We are working in  $\nla{n}{}=\nla{n}(\Fp)$ the category of Lie Algebras over the field $\Fp$ which are nilpotent of class $n$.
 objects $M\in\nla{n}$ such that $M=\gen{M_{1}}$, for $M_{1}$ a free $\Fp$-module in $M$. We obtain a natural graduation of $M$ according to the \emph{commutator weight}, that is
$M=\oplus_{i\leq n}M_{i}$, where $M_{i}$ denote the $i^{\mathrm{th}}$-\emph{homogeneous component} of $M$, the $\Fp$-vector subspace of $M$ generated by all Lie products of $M_{1}$-weight $i$.

With $M^k$ we name the $k$-th term of the \emph{lower central chain} of $M$, namely the ideal
$M^k=\sum_{k\leq j}M_j$.
%We will always assume $A_{1}$ to be a free $\Fp$ module and $A=\gen{A_1}^{A}$ forevery $A\in\nla{n}$.
 
Plus sportgesundheitspark concerning basic monomials.

\smallskip
With $\fla{n}{X}$ we denote the \emph{free $n$-nilpotent Lie Algebra} over $\Fp$ with set of generators $X$.


%Once a prime number $p$ is fixed, we define $\nla{n,p}$, or
short $\nla{n}$, to be the class of nilpotent Lie Algebras of nilpotency class $n$,
over the finite field $\Fp$,
with the additional property that each algebra $M$ of $\nla{n}$,
is generated by some $\Fp$-vector space $M_{1}$, that is $M=\gen{M_{1}}$.
In this way we obtain a natural graduation of $M$ according to the \emph{commutator weight} with respect to
$M_{1}$, that is
$M=\oplus_{i\leq n}M_{i}$, where $M_{i}$ denote the $i^{\mathrm{th}}$-\emph{homogeneous component}
of $M$, the $\Fp$-vector subspace of $M$ generated by all Lie products of $M_{1}$-weight $i$. By convention $M_{0}=\triv$.

If $M$ belongs to $\nla{n}$, we say that $H$ is an $\nla{n}$-subalgebra
of $M$ if $H=\gena{H_{1}}{M}$ where $H_{1}$ is a vector
subspace of $M_{1}$. Of course $H\in\nla{n}$ as well.
In the future for a subalgebra of $M\in\nla{n}$ we will
always mean a subalgebra of this special kind.

For a morphism $\map{\phi}{L}{M}$ of $\nla{n}$-algebras we mean a \emph{graded} Lie morphism of $L$ to $M$, that
is $\phi(L_{i})\inn M_{i}$ for all $i$.

With $M^k$ we name the $k$-th term of the \emph{lower central chain} of $M$, namely the ideal
$M^k=\sum_{k\leq i}M_i$.
We define as $\map{\tr{n}}{\nla{n+1}}{\nla{n}}$ the map defined by quotienting out the last component:
$\tr{n}A=A\quot A_{n+1}$, note that here $A_{n+1}=A^{n+1}$, in particular $A_{n+1}$ is an ideal.
Moreover if $A=\oplus_{i\leq n+1}A_{i}$ then $\tr{n}A\simeq\oplus_{i\leq n}A_{i}$, therefore, since $\nla{n}\inn\nla{n+1}$, we can regard $\tr{n}$ as an $\nla{n+1}$ morphism which maps identically the homogeneous components up to the $n+1^{th}$,
which is mapped to $\triv$. 

\medskip
$\fla{n}{X}$ will denote the \emph{free $n$-nilpotent Lie Algebra} over $\Fp$ with set of generators $X$.
We know, see for example [Bh], that if $V$ is an $\Fp$-vector space with basis $X$, then $\fla{n}{V}=\fla{n}{X}$. $\fla{n}{X}$ is the $\mathfrak{N}_{n}$-free algebra with free generator set $X$, where $\mathfrak{N}_{n}^{p}$ denotes the variety\footnote{
whose defining word is $[x_{1},\,\dots,\,x_{n+1}]$}
of nilpotent Lie algebras of class $\leq n$ over the field $\Fp$. One sees easily that $L^{n}(X)$ lies in $\nla{n}$ as well. Note that $\nla{n}$ is not a subvariety of $\mathfrak{N}_{n}$.

\bigskip
If $A\in\nla{n}$ we can assume $A=\fla{n}{X}\quot R$, where $X$ is an $\Fp$ basis of the vector space $A_1$,
and $R$ is an ideal of $\fla{n}{X}$ which contains what we call the \emph{relators} of $A$.
According to the previous observation we have, for $\nla{n}$-algebras, a canonical \emph{presentation} $A=\fla{n}{A_{1}}\quot
R$. The notation $A=\gen{x\in A_{1}\mid \rho\in R}$ or $A=\gen{A_{1}\mid\mathcal{R}}$ will also be used, where $\mathcal{R}$ is a set of words generating the verbal ideal $R$.
By our assumpion on $\nla{n}$, we can assume that $R$ is
a \emph{homogeneous} ideal\footnote{\dots be sure.},
that is $R=\sum_{i\leq n}R\cap{\fla{n}{A_{1}}}_{i}=:\sum_{i\leq n}R_{i}$. Moreover, in this case we have $R_{1}=\triv$.

If now $M$ is in $\nla{n}$ presented as $M=\gen{M_{1}\mid\mathcal{R}}$, where
the set of words $\mathcal{R}$ are, by definition, objects of the (absolutely) free Lie algebra $L_{p}(M_{1})$ over the field $\Fp$. We define the subvariety $\mathcal{V}^{n+1}(M)$ of $\mathfrak{N}
_{n+1}^{p}$ as follows
$$\mathcal{V}^{n+1}(M)=\left\{L\in\mathfrak{N}_{n+1}^{p}\mid\mathcal{R}(L)=\triv\right\}=
\mathfrak{N}_{n+1}^{p}\cap\var^{p}(M).$$

%Define now $\fr{n+1}M$ to be the free object in the variety $\mathcal{V}^{n+1}(M)$
%with set of free generators $M_{1}$. We have then
%$\fr{n+1}M=\fla{n+1}{M_{1}}\quot\J(M)$ where $\J(M)$ is the verbal ideal of $\fla{n+1}{M_{1}}$
%generated by the words $\mathcal{R}$. Also $\fr{n+1}M\in\nla{n+1}$.
%
%\medskip
%From the universal property of free objects in varieties we deduce the following.
%Let an algebra $L$ of $\nla{n+1}$ belong to $\mathcal{V}^{n+1}(M)$ with $L_{1}$ isomorphic to $M_{1}$,
%in other words $\tr{n}L\simeq M$, then there exists a unique Lie morphism $\pi$ of
%$\fr{n+1}M$ onto $L$, such that (up to $M$-isomorphic translates of $\tr{n}$) the following diagram commutes.
%\begin{labeq}{communo}
%\xymatrix{
%{\fr{n+1}M}\ar[dr]_{\tr{n}}\ar[rr]^{\pi}&&N\ar[dl]^{\tr{n}}\\
%&M&}
%\end{labeq}
%
%The epimorphism $\pi$ may be constructed as follows. Assume for brevity $\tr{n}N=M$ and
%assume $\map{\chi}{\fla{n+1}{N_1}}{N}$ is a presentation of $N$.
%We may further assume that $N_{1}=M_{1}$, thus $\chi$ induces a presentation $\chi^M$
%for $M$ to the quotient $\fla{n}{M_1}$ such that
%$$\xymatrix{
%&\fla{n+1}{M_1}\ar[dl]_{\tr{n}}
%\ar[dr]^{\chi}&\\\fla{n}{M_1}%\ar[ur]^{i}
%\ar[dr]^{\chi^M}&&N\ar[dl]_{\tr{n}}\\&M&
%}$$
%commutes.
%
%Since in the actual setting $\jei{M}$
%is the verbal ideal $\mathcal{R}(\fla{n+1}{M_{1}})$, where the set of words $\mathcal{R}$
%generates $\ker\chi^{M}$ in $\fla{n}{M_{1}}$, it follows $\J(M)\inn\ker\chi$.
%%=\genid{i(\ker(\chi^{M}))}{\fla{n+1}{M_{1}}}\inn\ker\chi$, 
%Therefore the identical map of $\fla{n+1}{M_{1}}$ induces a morphism $\pi$ of $\fla{n+1}{M_{1}}\quot\J(M)$ onto $\fla{n+1}{M_{1}}\quot\ker\chi$, which
%satisfies \pref{communo}.
%Note also that $\ker\pi\inn(\fr{n+1}M)_{n+1}$.
\subsection{Deficiency and Group Homology}\label{schur}
In this section we see how the second homology of a finitely presented group or Lie algebra, together with the first
lower central section, entirely captures the relevant informations expressible in terms of generators and relations.

The objects we illustrate below present strong similarities with those defined in the second and especially the third
chapter. The last were developed independently from appealing to homology.

We refer to the book \cite{hilsta} for the basic facts concerning group homology.

\medskip
A {\em finitely presented} group $G$ is $(n,r)$-presented, if it admits a presentation
with $n$ generators and $r$ relators. The {\em deficiency} of $G$ is defined as
$$\mathrm{def}(G)=\max(n-r\mid G\,\,\text{is $(n, r)$-presented}).$$

%\cbstart
%In a nilpotent setting\mn{maybe just as footnote}, the \sout{{\em finitely presentation} property} -- which is closed under extensions -- is
%equivalent to being finitely generated. On the other hand in a Fra\"iss\'e construction, the starting structures
%are always finitely generated.
%\cbend

It is possible to estimate the deficiency of a finitely presented group $G$ in terms of the {\em Schur multiplicator}
$H_{2}(G)=H_{2}(G,\Z)$, the second homology group of $G$ with integer coefficients.
The following result, in \cite[14.1.5]{rob}, is attributed to Philip Hall.

If $A$ is a finitely generated abelian group, we denote by $\mathit{rk}(A)$ the rank of $A$ and $d(A)$ the minimal number of
elements required to generate $A$. %For any group $G_{ab}$ is $G/G^{\prime}$.
\begin{fact}
If $G$ is a finitely presented group, then $H_{2}(G)$ is finitely generated. Moreover
\begin{labeq}{defi}
\mathrm{def}(G)\leq\mathit{rk}(G_{ab})-d(H_{2}(G)).
\end{labeq}
\end{fact}

{\em Hopf's formula} (\cite{hopf}) expresses $H_{2}(G)$ in terms of any presentation of the group $G$:
%very close to our definitions in Chapter \ref{tre}:
\begin{labeq}{hopf}
H_{2}(G)=\frac{F^{\prime}\!\cap R}{[F,R\,]}
\end{labeq}
provided $R\to F\to G$ presents $G$.

Hopf's formula was later recognised independently by Stallings (\cite{stall}) and Stammbach (\cite{stam},\cite[\S8]{hilsta}) to stem from the {\em 5-term} Homology sequence 
\begin{labeq}{5term}
H_{2}(E)\to H_{2}(Q)\to N/[E,N]\to E_{ab}\to Q_{ab}\to\triv
\end{labeq}
associated to any short exact sequence $N\to E\to Q$, by applying \pref{5term}
to a presentation $R\to F\to G$ of the group $G$. Now \pref{hopf} follows by the fact $H_{2}(F)=\triv$.

\medskip
%The 5-term sequence above finds a corresponding sequence when we specialise to
%a group variety $\mathcal{V}$ (cfr.{}\cite[]{stahom}). In particular for a subvariety  of ${\mathfrak B}_{p}$, the groups
%with exponent $p$ we can use  homology with coefficients in $\Z/p\Z$ and obtain.
We can specialise -- with Stammbach's \cite[\S III]{stahom} -- the 5-term sequence above to a group variety $\mathcal{V}$, obtaining
a notion of schur multiplier $H_{2}(G;\mathcal{V},B)$ relative to $\mathcal{V}$ for any $G$-module $B$ and group $G\in\mathcal{V}$.
With this technique, an Hopf formula in terms of $\mathcal{V}$-presentations is achieved. %(cfr.{}\cite[\S III,(2.10)]{stahom}).
On the other hand Stallings (\cite[Theorem 2.1]{stall}) points out how group homology with coefficients in $\Z/p\Z$,
is connected to a {\em $p$-exponent modification\footnote{
the group words $\gamma_{k}(\bar x)$, which define the lower central series are replaced by $\gamma_{k}(\bar x)y^{p}$. In a group
of exponent $p$ we reobtain the old series.}} of the lower central series. 

Inspired by the results cited above, similar features concerning finitely presented
$\ngb{c}{p}$-groups may be derived. Note that in general a finitely generated nilpotent group is also
finitely presented, being this property closed under extensions of groups.
\begin{lem}\label{pdeficienza}
%Let $\mathcal{V}$ be an exponent $p$ variety of groups.
We say that $G\in\ngb{c}{p}$ is $(n,r)$-{\em presented in} $\ngb{c}{p}$
if it admits a presentation by the $n$-generated $\ngb{c}{p}$-free group $F$ modulo a normal subgroup $R$, which
is the normal closure of $r$ elements of $F$.

If ${\rm def}_{\ngb{c}{p}}(G)=max(n-r\mid G\text{ is $(n,r)$-presented in $\ngb{c}{p}$})$, then this number exists finite
and we have
\begin{labeq}{vardefi}
{\rm def}_{\ngb{c}{p}}(G)\leq\dfp(G_{ab})-\dfp(H_{2}(G;\ngb{c}{p}))
\end{labeq}
where $H_{2}(G;\ngb{c}{p})$ is {\em defined} as the %leftmost
kernel of the natural map $\phi$ in the exact sequence of $\Fp$-vector spaces
\begin{labeq}{varschur}
%\triv\to H_{2}^{\ngb{c}{p}}(G,\Fp)\to
R/[F,R]\stackrel{\phi\:}{\lto}F_{\sl a{}b}\to F/RF^{\prime}\to\triv
%\defeq\ker\left(    \right)
\end{labeq}
where $R\to F\to G$ is {\em any finite} $\ngb{c}{p}$-presentation of $G$.
\end{lem}
\begin{rem*}
By \pref{varschur} we have
$$H_{2}(G;\ngb{c}{p})=\frac{F^{\prime}\!\cap R}{[F,R\,]}$$
and by \cite[\S III.1,2]{stahom} this group does not depend of the chosen $\ngb{c}{p}$-presentation.
\end{rem*}
\begin{proofof}{Lemma \ref{pdeficienza}}
Assume the group $G$ is $(n,r)$-presented in $\ngb{c}{p}$ by $F$ modulo $R$.

Since $F$ is the $n$-generated $\ngb{c}{p}$-free group, we have $\dfp(F_{ab})=n$.
%Moreover $\dfp(R/[F,R])\leq r$, $G_{ab}$ is $F/RF^{\prime}$ and all terms of \pref{varschur} are $\Fp$-vectorspaces.
%\pref{vardefi} easily follows with some dimension calculus
Exactness in \pref{varschur} now yields $n-r\leq\dfp(F_{\sl ab})-\dfp(R/[F,R])=
\dfp(G_{ab})-\dfp(H_{2}(G;\ngb{c}{p}))$.
\end{proofof}
We list some facts to underline the strength of these concepts.
\begin{fact*}[{\cite[Theorem 6.5]{stall}}]
Let $G$ be a $\ngb{c}{p}$-group with $H_{2}(G;\ngb{c}{p})=\triv$ and $(x_{i})_{i\in I}$ a set of elements in $G$ whose images in
$G_{ab}$ are $\Fp$-linearly independent. Then the $x_{i}'s$ generate a $\ngb{c}{p}$-free subgroup of $G$.
\end{fact*}

\begin{fact*}[\cite{stall,stam}]
Let $\phi$ be a group homomorphism of $G$ in $K$, if $\phi$
induces an isomorphism of $G_{ab}$ to $K_{ab}$ and an epimorphism
$\phi_{*}$ of $H_{2}(G)$ onto $H_{2}(K)$, then $\phi$ induces
isomorphisms of $G\quot\gamma_{i}(G)$ to $K\quot\gamma_{i}(K)$ for all
$i<\omega$.

In particular if $G$ and $K$ are nilpotent, they are isomorphic.
\end{fact*}

A finitely presented group is called {\em efficient} if equality holds in \pref{defi} and $\mathcal{V}$-efficient
if the same equality holds for the corresponding $\mathcal{V}$-deficiency.
\begin{fact*}[{\cite[Theorem 6.5]{stahom}}]
Let $G$ be a group in $\mathcal{V}$, given by a finite $\mathcal{V}$-presentation.
Then there exists an efficient group $K\in\mathcal{V}$ and a surjective homomorphism
$\map{f}{K}{G}$ which induces an isomorphism $\map{f_{i}}{K/\gamma_{i}(K)}{G/\gamma_{i}(G)}$ for every $i\geq1$.

In particular $\ngb{c}{p}$-groups are $\ngb{c}{p}$-efficient: equality in \pref{vardefi} holds!
\end{fact*}

\medskip
The objects and facts reported above apply, in the very same fashion, to Lie algebras. One may check \cite{staknu} or
\cite[\S VII]{hilsta}.

In particular for a presentation $\mathfrak{r}\to\mathfrak{f}\to\mathfrak{g}$, the second integral homology group of ${\mathfrak g}$
is given by $$H_{2}(\mathfrak{g})=\mathfrak{f}^{\prime}\cap\mathfrak{r}\quot[\mathfrak{f},\mathfrak{r}]$$

As before, we find the analogous notion related to our special class of Lie algebras $\nla{c}$ over $\Fp$.
In particular for $M=\gen{M_{1}\mid R}$ in 
%Note that if ${\mathfrak g}$ belongs to our graded category
$\nla{c}$, as $R$ is contained in the {\em commutator algebra} $(\fla{c}{M_{1}})^{\prime}=\gamma_{2}(\fla{c}{M_{1}})$,
we find
\begin{labeq}{LieSchur}
H_{2}(M,\nla{c})=\frac{R}{[R,L]}
\end{labeq}
where $L$ denotes $\fla{c}{M_{1}}$.

Of course we can define -- as in Lemma \ref{pdeficienza} -- the corresponding
notion ${\rm def}_{\nla{c}}$ of $\nla{c}$-deficiency for finitely generated algebras $M$.
We may as well speak of {\em efficient $\nla{c}$-algebras} and in particular, we have 
\begin{labeq}{LieDef}
{\rm def}_{\nla{c}}(M)=\dfp(M_{1})-\dfp(H_{2}(M,\nla{c})).
\end{labeq}
In our case recall that the ideal $R$ is homogeneous and $R=R_{2}\oplus\dots\oplus R_{c}$.
Roughly speaking the group $R/[L,R]$ mods out for all $i\leq c$, the relators $R_{i}$ of
weight $i$ of the redundant terms: the elements of $R_{i}$ which arise as brackets $[r^{\prime},x_{1},\dots,x_{i-k}]$,
for relators of {\em lower weight} $r^{\prime}\in R_{k}$.

In section \ref{freelift} of Chapter \ref{tre} we will encounter exactly this {\em shifting} phenomenon.

\medskip
It is worth to note that the above notions interact with free products with amalgamated subgroup. This is connected with
the Mayer-Vietoris sequence (cfr.{\,}\cite[\S II.6]{stahom}).
\newpage
\thispagestyle{empty}
\cleardoublepage
%----------------------------------------CHAPTER TWO---------------------
\chapter{Nilpotency Class $2$}\label{due}
In this chapter we develop a Fra\"iss\'e-Hrushowski construction within the class $\nla{2}$, described
in the previous section. This will lead to
the uncollapsed theory $T^{2}$ of a rich $2$-nilpotent Lie algebra. As pointed out before the prime $p$ has to
be chosen greater than $2$.

\smallskip
The language $\Lan{2}$ adopted in Section \ref{nilgral} contains,
along with the Lie ring signature, two predicates $P_{1}$ and $P_{2}$ to interpret the grading $M=P_{1}(M)\oplus P_{2}(M)$
of any $\nla{2}$-algebra $M$.
As defined in Section \ref{nilgral} we have $M=\gen{P_{1}(M)}=\gen{M_{1}}$ and hence $M_{2}=P_{2}(M)$ is the subspace
generated by commutator-lenght $2$ elements.
\section{$\delta$-Calculus}\label{sec:deltadue}
Recall that, any object $M$ of $\nla{2}$ is associated a {\em presentation}
$$R\linto \fla{2}{M_{1}}\lonto M$$
and we write $M=\gen{M_{1}\mid R}$,
where $R$ is an homogeneous ideal of the free nil-$2$ Lie algebra $\fla{2}{M_{1}}$, of total weight weight $2$. %$(\fla{2}{M_{1}})_{2}$,
That is, $R$ is a $\Fp$-vector subspace of $(\fla{2}{M_{1}})_{2}$.

We let the homogeneous subspace $(\fla{2}{M_{1}})_{2}$ coincide with the exterior square of the $\Fp$-vector space $M_{1}$ and hence
$\fla{2}{M_{1}}\simeq M_{1}\oplus\exs M_{1}$ (see \cite[\S I.1]{ser}).

Hence if $M$ is given by the presentation\footnote{The couple $\left(M_{1},\rd(M_{1})\right)$ informally represents the kind of structures
utilised in \cite{bad}.}
above, we denote $R$ by $\rd(M)$ %or ambiguously $\rd(M_{1})$,
and we have $$M\simeq M_{1}\oplus\frac{\exs M_{1}}{\rd(M)}.$$

\smallskip
If $M$ is an object of $\nla{2}$, an $\nla{2}$-{\em subalgebra} is by definition, a subalgebra $H$ of $M$,
which is generated by a $\Fp$-subspace $H_{1}$ of $M_{1}$. We write in this case $H=\gena{H_{1}}{M}$.
Conversely for any subspace $H_{1}$ of $M_{1}$, we adopt the convention to denote
by $H$ the $\nla{2}$-subalgebra $\gena{H_{1}}{M}$. % generated by $H_{1}$ in $M$.
By {\em subalgebras} we will exclusively mean $\nla{2}$-subalgebras in the future.

For $\nla{2}$-subalgebras $A$ and $B$ of $M$,
with abuse of the common meaning we {\em denote} by $A+B$ the subalgebra $\gena{A_{1}+B_{1}}{M}$.
%and let $A\sqcap B$ denotes $\gena{A_{1}\cap B_{1}}{M}$.
As $\Fp$-vector spaces, {\em finitely generated} $\nla{2}$-algebras are {\em finite}.

\smallskip
For $M\in\nla{2}$ and a subspace $H_{1}$ of $M_{1}$ we consider $\exs H_{1}$ as a natural subspace of $\exs M_{1}$.
To any such $H_{1}$ or equivalently, to any $\nla{2}$-subalgebra $H=\gena{H_{1}}{M}$ of $M$ we set
\begin{labeq}{erredi}
\rd_{M}(H_{1})=\rd_{M}(H):=\rd(M)\cap\exs{H_{1}}.
\end{labeq}
If the ambient structure $M$ is clear from the context, we simply  write $\rd(H_{1})$ or $\rd(H)$.
%Observe that if $H$ is $\gena{H_{1}}{M}$, then
In any case we have\footnote{Here below instead, $+$ indicates the ordinary sum between a subalgebra and an {\em ideal}. In the sequel
this will be almost never the case.}
$$H\simeq(\fla{2}{H_{1}}+\rd(M))\quot\rd(M)\simeq\fla{2}{H_{1}}\quot \rd_{M}(H).$$

\bigskip
We now introduce an integer valued %\emph{predimension}
function $\delta$ with entries on the finite $\Fp$-subspaces of $M_{1}$, which measures
in terms of $\Fp$-dimension, {\sl how much a finitely generated structure differs from a free one.}
The term {\em deficiency} is also motivated by section \ref{schur}, observe in this case $H_{2}(M)=\rd(M)$.

\begin{dfn}\label{deficienzwei}
Assume an algebra $M$ of $\nla{2}$ has been fixed.
For a finite $\nla{2}$-subalgebra $A$ of $M$ set
\begin{labeq}{deltadue}
\delta(A)=\dfp(A_{1})-\dfp\left(\rd_{M}(A)\right).
\end{labeq}
We call $\delta(A)$ the {\em deficiency} of the subalgebra $A=\gena{A_{1}}{M}$ of $M$.
\end{dfn}
Observe that if an algebra $M$ is fixed, then $\delta(A)$ depends -- by \pref{erredi} --
only on the subspace $A_{1}$ of $M_{1}$. In fact we will write indifferently $\delta(A_{1})$ or $\delta(A)$
for the deficiency of $A$.

Also, $\delta(A)$ is an invariant of the isomorphism type of the structure $A$ and $\delta(A)=
\dfp(A_{1})$ implies $A\simeq\fla{2}{A_{1}}$.

For arbitrary $\nla{2}$-subalgebras $H$ of $M$, and  finite $C_{1}$ (over $H_{1}$), we introduce a {\em relative} deficiency\footnote{with values in $\Z\cup\{-\infty\}$.}
by means of
\[
\delta(C\quot H)=\dim_{\Fp}(C_{1}/H_{1})-\dfp(\rd_{M}(C/H))
\]
provided we define $\rd_{M}(C/H)$ to be the quotient space $\rd(H_{1}+C_{1})\quot \rd(H_{1})$.
We also allow expressions $\delta(C_{1}/H)$ and $\delta(C_{1}/H_{1})$ to denote the above.

For finite $A$ and $B$, we have of course $\delta(A\quot B)=\delta(A+B)-\delta(B)$, while
for finite {\em sets} $\mathcal{U}$ or {\em tuples} in $M_{1}$ and arbitrary $H$, % _{1}$ of $M_{1}$,
we set $\delta(\mathcal{U}/H)=\delta(\genp{H_{1},\mathcal{U}}/H)$ and $\delta(\bar a/H)=\delta(\genp{H,\bar a}/H)$.

\smallskip
Now let $M$ be an $\nla{2}$-algebra, by virtue of Fact \ref{ubc} we have
\begin{rem}\label{rem:exsmod}
For all $H_{1}$ and $K_{1}$ in $M_{1}$
\begin{labeq}{exsmod}
\exs(H_{1}\cap K_{1})=\exs H_{1}\cap\exs K_{1}
\end{labeq}
\end{rem}
\begin{cor}\label{cor:2modker}
Fixed an algebra $M$ of $\nla{2}$, we observe a modular behaviour of the operator
$\rd$ on the subspaces of $M_{1}$, that is for all $H_{1}$ and $K_{1}$,
\begin{labeq}{2modker}
\rd(H_{1}\cap K_{1})=\rd(H_{1})\cap \rd(K_{1})
\end{labeq}
\end{cor}

\smallskip
As a first consequence of the above results, % of modularity in $\rd$
we obtain that $\delta$ is actually a predimension {\em on} $M_{1}$. In fact the
relative deficiency satisfies a stronger version of submodularity.

\begin{lem}\label{2transmogrifer}
Let $H_{1}\nni V_{1}$ and $C_{1}$ be subspaces of $M_{1}$, for $M$ in $\nla{2}$. If $C_{1}$
is finite and $H_{1}\cap C_{1} \inn V_{1}$, %\inn H_{1}$,
then $\delta(C/H)\leq\delta(C/V)$.
\end{lem}
\begin{proof}
On one side, the assumption yields $\dfp(C_{1}\quot H_{1})=\dfp(C_{1}/V_{1})$.

For the negative part of $\delta$, observe that $\rd(C/V)$ embeds into
$\rd(C/H)$. This follows from
$$\rd(V+C)\cap\rd(H)=\rd((V_{1}+C_{1})\cap H_{1})=\rd(V_{1}+(H_{1}\cap C_{1}))=\rd(V).$$
\end{proof}

As an extremal case, we get \emph{submodularity} for $\delta$ on $M_{1}$, that is 
\begin{labeq}{submod}
\delta(C\quot H)\leq\delta(C_{1}\quot C_{1}\cap H_{1})
\end{labeq}
for any $H_{1}$ and finite $C_{1}$.
On finite spaces, if $\cl$ denotes the $\Fp$-linear span in $M_{1}$, this is exactly \pref{summo}.

\smallskip
The next obliged step is to force $\delta$ to be non-negative. With this purpose define the property
\begin{itemize}
\punto{$\sig{2}{2}$}\label{sig22}
\quad for any finite $A_{1}\inn M_{1}$,\;\;$\delta(A)\geq\min(2,\,\dfp(A_{1}))$.
\end{itemize}
As $\delta$ is an invariant of
the isomorphism type of finite $\nla{2}$-algebras, property $\sig{2}{2}$ is first order expressible in
the language $\Lan{2}$ by a denumerable axiom system: just negate the diagrams of those which \emph{do not}
have the desired property.

Some remark about the choice of the number $2$ as a lower bound are to be given.
Of course $1$-generated subalgebras are isomorphic to $\Fp$ in any $\nla{c}$-algebra.

Condition $\sig{2}{2}$ imposes that $2$-generated subalgebras are free. Equivalently,
for any fixed element $a\in M_{1}$ for $M$ with $\sig{2}{2}$, the the kernel of the natural
derivation $ad_{a}\colon x\mapsto[a,x]$ coincides with $\genp{a}$ and,
as a consequence the centre $Z(M)$ is forced to coincide with $M_{2}$. This last condition -- which is equivalent
to require the form $[\,\cdot,\cdot]$ to be non-degenerate -- is therefore weaker than $\sig{2}{2}$.

%By Theorem \ref{faikaha}, this property
This feature reflects to the associated group $G(M)$ (or $\mathscr{G}(M)$) reconstructed from $M$
in Section \ref{algegruppi} (cfr.{\,}Remark \ref{baucond}).
In particular, $\ngb{2}{p}$-groups obtained via $G(\cdot)$ from $\nla{2}$-algebras with $\sig{2}{2}$ all share the property
$G^{\prime}=Z(G)$.

On the other hand, $\sig{2}{2}$ influences the pregeometry on $M_{1}$ associated to $\delta$ (cfr.{ }Corollary \pref{2geom} below).

\smallskip
\begin{rem}\label{preg2}
Assume $M$ has $\sig{2}{2}$, if $\cl$ denotes the $\Fp$-linear closure in $M_{1}$, then
$\delta$ defines a $\cl$-predimension on $M_{1}$ according to definition \ref{clpred}.

Denote by $d^{M}$ or simply $d$,
the dimension function on $M_{1}$ associated to $\delta$
with Lemma \ref{preg} and and by $\cl_{d}^{M}$ or $\cl_{d}$ the resulting closure. For finite $\nla{2}$-subalgebras $A$, we have
\begin{itemize}
\item[-]$d(A)\defeq d(A_{1})=\min(\delta(C)\mid C_{1}\nni A_{1})$ and $d(A)\leq\dfp(A_{1})$
\item[-]$\cl_{d}$ extends $\cl$ and $b\in\cl_{d}(A_{1})$ exactly if $d(A_{1},b)=d(A)$.
\end{itemize}
\end{rem}

\smallskip
In presence of $\sig{2}{2}$ the notion of {\em self-sufficiency} which follows, let us {\em choose} for any given $A_{1}$, a distinguished
minimal space of deficiency $d(A)$ above $A_{1}$.

\begin{dfn}\label{2strong}
Let $H_{1}$ be a subspace of $M_{1}$, for $M\in\nla{2}$.
We call both $H_{1}$ and the $\nla{2}$-sublagebra $H$,
\emph{strong} or \emph{self-sufficient} in $M_{1}$ or $M$ respectively if
for any finite subspace $C_{1}\inn M_{1}$, we have $\delta(C\quot H)\geq0$.
This is written $H_{1}\zsu{}M_{1}$ or $H\zsu{}M$.

For any integer $n<\omega$, we say that $H$ is $n$-strong in $M$, if
$\delta(C_{1}\quot H_{1})\geq0$ holds for all subspaces $C_{1}$ of $M_{1}$
with $\dfp(C_{1}\quot H_{1})\leq n$. We write in this case $H\zsu{n}M$.
We say that an $\nla{2}$-embedding $\phi$ of $H$ into $M$ is ($n$-){\em strong} %or ($n$-){\em self-sufficient}
if $\phi(H)$ is ($n$-) strong in $M$. %(for $n<\omega$).
\end{dfn}
\begin{rem}\label{deltadi}
A finite subspace $A_{1}$ of $M_{1}$ is self-sufficient in $M_{1}$ if and only if $d^{M}(A)=\delta(A)$ and
in general $d(A)\leq\delta(A)$.
\end{rem}

\begin{dfn}
Let $B_{1}$ be a finite subspace of $M_{1}$, define a \emph{self-sufficient closure} of $B_{1}$ in $M_{1}$ to be an
$\inn$-minimal subspace $A_{1}$ of $M_{1}$ containing $B_{1}$ with $\delta(A)=d(B)$.
By Lemma \ref{interstrong} below, the family of strong subspaces of $M_{1}$ is closed under
intersection, as a consequence the notion of self-sufficient closure of a finite space $A_{1}$ depends on
$A$ and $M$ only and is univocally determined as:
$$\ssc^{M}(A_{1})=\ssc(A_{1})\defeq\bigcap\{C_{1}\zsu{}M_{1}\mid C_{1}\, \text{finite and}\, C_{1}\nni A_{1} \}$$
We define the $\nla{2}$-subalgebra $\ssc^{M}\!(A)=\ssc(A)$ of $M$ as $\gena{\ssc(H_{1})}{M}$and we call it
the self-sufficient closure {\em of} $H$ in $M$. For a finite {\em subset} $\mathcal{U}$ of $M_{1}$, we set $\ssc(\mathcal{U})$
to be $\ssc(\genp{\mathcal{U}})$.
\end{dfn}

\smallskip
Note that this definition implies the operator $\ssc$ is actually a closure operator:
it is monotone, and has properties (cl1) and (cl2) of definition $\ref{pregdef}$, moreover
$\ssc(A_{1})\inn\cl_{d}(A_{1})$.

%the intersection above may be considered  to involve finitely many subspaces only.
%In the next section it will be clear how the {\sl ambient structure} $M$ influences the nature of $\ssc$.
%By the fact $d(A_{1})=\delta(\ssc(A_{1}))=d(\ssc(A_{1}))$, it follows now 
%for all $A_{1}$ and by lemma \ref{preg2}
%$d(A_{1})\leq\dfp(A_{1})$. 

\begin{cor}\label{2geom}
For any algebra $M$ with $\sig{2}{2}$, the pregeometry $(M_{1},\cl_{d})$ associated to $\delta$ is actually a
geometry over the $\Fp$-linear closure according to Definition \ref{pregext}.

For a given $M$, the geometry  $\cl_{d}$ is not in general locally-modular.
\end{cor}
\begin{proof}
Axiom $\sig{2}{2}$ implies any two linearly independent couple $a,b$ generates a self-sufficient
subalgebra $\gena{a,b}{M}\simeq\fla{2}{a,b}$ and $d(a,b)=2$. Analogously
$d(\genp{\vac})=d(\triv)=\delta(\triv)=0$ and for any $a\in M_{1}$, $d(\genp{a})=
\delta(a)=1$. It follows $\cl_{d}(\vac)=\triv$ and $\cl_{d}(a)=\genp{a}$.

\smallskip
Consider now the finite $\nla{2}$-algebra $M=\gen{M_{1}\mid\rd(M)}$ whose $M_{1}$ has $\Fp$-base $\{a,b,c,x,y\}$ and such that the relator ideal $\rd(M)$
is spanned in $\exs M_{1}$ by the independent homogeneous elements
$$[a,b]+[x,y],\quad[c,x]+[y,b],\quad[a,y]+[b,c].$$
One checks $M$ has $\sig{2}{2}$ and the subspace $\genp{a,b,c}$ is not self-sufficient in $M$: $\delta(M/a,b,c)=-1$. It follows
$2=d(a,b,c)<\delta(a,b,c)=3$ and this yields
$$d(a,b)+d(b,c)=4>3=d(a,b,c)+d(b).$$
\pref{mod} is not (even locally) satisfied.
\end{proof}

\medskip
Some properties of sef-sufficient spaces will now follow. We assume an algebra $M$ of $\nla{2}$ has been fixed
with $\sig{2}{2}$. All the subspaces and subalgebras considered, lay in $M_{1}$ and $M$ respectively.

\begin{rem}\label{finitedeltabase}
For a self-sufficient $H$ and a finite $A$, we can always find a {\em finite} strong subalgebra $H^{\rm o}$
such that $\delta(A/H)=\delta(A/H^{\rm o})$.

For an arbitrary $H$, one has
$$\delta(A/H)=\inf\left(\delta(A/C)\mid C_{1}\text{ finite and }A_{1}\cap H_{1}\inn C_{1}\zsu{}H_{1}\right).$$
%This also follows by the definition itself of $\delta(A_{1}\quot H_{1})$ above: assume $H$ is self sufficient,
%since $\rd(A_{1}\quot H_{1})$ has finite dimension,
%choose $C_{1}$ in $H_{1}$ such that $\exs (C_{1}+A_{1})$ supports each element of a basis of
%$\rd(H_{1}+A_{1})$ over $\rd(H)$.
\end{rem}
\begin{proof}
For the first part, since $\rd(A/H)$ has to be finite dimensional, pick a finite $\nla{2}$-subalgebra
$H^{\rm o}$ in $H$ with $H^{\rm o}_{1}\nni H_{1}\cap A_{1}$ and such that $\rd(H+A)$ has a basis in $\exs\genp{H^{\rm o}_{1},A_{1}}$ over
$\exs H_{1}$. By Corollary \ref{interstrong} below we can choose $H^{\rm o}$ to be self-sufficient. 

The second part follows by Lemma \ref{2transmogrifer} and the above arguments.
\end{proof}
%As a  consequence of this, self-sufficiency is closed under increasing chains of subspaces: assume $B_{1}^{i}\zsu{} M_{1}$ for all $i<\omega$ is an increasing family of finite subspaces
%whoose union subspace is $B_{1}\inn M_{1}$. Then by the previous remarks, for any finite $A_{1}$, we can prove $\delta(A_{1}\quot B_{1}^{i})\searrow\delta(A_{1}\quot B_{1})$ for $i\rightarrow\infty$.
%%[EXT] -->  \dfp K_{1}\quot V_{1}^{i} is def.ly constant = \dfp K_{1}\quot V_{1} while \rdK_{1}\quot V_{1}^{i} converges increasing to \rdK_{1}\quot V_{1}.
%This gives
%$V_{1}\zsu{} M_{1}$.

The next lemma shows {\em transitivity} of strong embeddings.
\begin{lem}\label{2trans}
If $H$ is $n$-strong in $K$ and $K$
is self-sufficient in $M$, then $H$ is $n$-strong in $M$.

In particular from $H\zsu{} K$ and $K\zsu{} M$, follows $H\zsu{}M$.
\end{lem}
\begin{proof}
Let $C_{1}$ be a finite subspace, both statements of the lemma follow from the
inequality $\delta(C_{1}/H)\geq\delta(C_{1}\cap K_{1}/H)+\delta(C_{1}/K)$.

We have equality for the $\Fp$-linear dimensions and for the negative parts, we observe that
$\rd(C/H)$ maps to $\rd(C/K)$ with kernel
$$\frac{\rd(H+C)\cap\rd(K)}{\rd(H)}=\frac{\rd((H_{1}+C_{1})\cap K_{1})}{\rd(H)}=\frac{\rd(H_{1}+(C_{1}\cap K_{1}))}{\rd(H)}.$$
\end{proof}


Another straightforward application of lemma \ref{2transmogrifer} is the following:
\begin{lem}[Cut Lemma]\label{2cut}
If $H$ is self-sufficient in $K$, then for any subspace $V_{1}$ of $M_{1}$, we have
$H_{1}\cap V_{1}\zsu{} K_{1}\cap V_{1}$.
\end{lem}
\begin{proof}
Observe $\delta(E_{1}/H_{1}\cap V_{1})\geq\delta(E/H)\geq0$ whenever $E_{1}\inn K_{1}\cap V_{1}$, since
$H_{1}\cap V_{1}$ contains $E_{1}\cap H_{1}$.
\end{proof}
\begin{cor}\label{interstrong}
If $H$ and $K$ are self-sufficient,
then the intersection $H_{1}\cap K_{1}$ is also strong in $M_{1}$.
\end{cor}
\begin{proof}
By Lemma \ref{2cut} we have $H_{1}\cap K_{1}\zsu{} K_{1}$.
Then conclude by transitivity of $\zsu{}$ (Lemma \ref{2trans}).
\end{proof}

%We conclude this section with further results involving the predimension $\delta$ and
%its interactions with $d^{M}$.

\begin{lem}\label{finchar}
Let $H_{1}$  be a subspace of $M_{1}$ then
$H$ is strong if and only if for any finite subspace $C_{1}$ of $H_{1}$ there exists a finite subspace
$C_{1}^{\rm o}\inn H_{1}$ containing $C_{1}$, such that $C^{\rm o}\zsu{}M$.
\end{lem}
\begin{proof}
If $H$ is strong, given any finite $C_{1}$ in $H_{1}$, then take $C^{\rm o}$ to be $\ssc(C)$.
Because of Lemma \ref{interstrong} $C^{\rm o}$ is contained in $H$.

\smallskip
For the converse, if $A_{1}$ is finite in $M_{1}$, we want $\delta(A/H)$ to be non negative.
But this follows by the hypothesis applying Remark \ref{finitedeltabase}.
\end{proof}

Given two algebras $N\inn M$ of $\nla{2}$ the self-sufficient
closure of a finite subspace $A_{1}$ of $N_{1}$ computed in $N$ may differ from $\ssc^{M}(A_{1})$.
But as expected we have
\begin{rem}\label{samed2}
Assume $N$ is an $\nla{2}$-subalgebra of $M$. Then $N$ is strong in $M$ if and only if for all subspaces
$V_{1}$ of $N_{1}$ the closures $\ssc^{N}(V)$ and $\ssc^{M}(V)$ %$d^{N}(V_{1})$ and $d^{M}(V_{1})$
coincide.
\end{rem}
\begin{proof}
We may suppose $V$ are {\em finite} subalgebras in the statement and in general $\ssc^{M}(V)\inn\ssc^{N}(V)$ as strongness
is expressible via universal sentences.

The first condition is clearly sufficient. It is necessary by virtue of Lemma \ref{finchar},
since for any finite $V_{1}\inn N_{1}$, we now know $\ssc^{N}(V)=\ssc^{M}(V)$ is {\em inside} $N$, but strong {\em in} $M$.
\end{proof}
We might have stated Remark \ref{samed2} in terms of $\cl_{d}$-dimensions: $N\zsu{}M\iff d^{N}(V_{1})=d^{M}(V_{1})$
for any subspace $V_{1}\inn N_{1}$.

\begin{lem}\label{samedelta2}
Assume $H\zsu{} M$ and $\delta(A/H)=0$ for some finite subspace $A_{1}$ of $M_{1}$, then $H+A$ is %$\delta$-
self-sufficient in $M$ as well.

Moreover if an element $a$ of $M_{1}$ is $\cl_{d}$-independent of $H_{1}$, i.{}e.\,$d^{M}(a/H)=1$, then
$\gena{H_{1},a}{M}$ is strong in $M$.
\end{lem}

\begin{proof}
Consider a finite subspace $E_{1}$ of $M_{1}$, then the first statement follows by computing
$$\delta(E/H+A)=\delta(E+A/H)-\delta(A/H).$$

For the second one, note that any finite subspace of $\genp{H_{1},a}$ is
contained in some $\genp{A_{1},a}$ where $A_{1}$ is a finite strong subspace of $H_{1}$.
Since $d=d^{M}$ is a dimension, $1=d(a/H_{1})\leq d(a/A_{1})\leq1$ implies
$d(A_{1})+1=d(A_{1},a)$.

We conclude by Lemma \ref{finchar} showing that $\genp{A_{1},a}\zsu{}M_{1}$.
We have indeed, since $A\zsu{}M$, $\delta(A_{1},a)\leq\delta(A)+1=d(A_{1})+1=d(A_{1},a)$.
This yields $\delta(A_{1},a)=d(A_{1},a)$.
\end{proof}

\bigskip
For an arbitrary space $H_{1}$ we
define the {\em self-sufficient closure} $\ssc^{M}(H_{1})$ of $H_{1}$ as the subspace of $M_{1}$ generated by the
self-sufficient closures of all the finite parts of $H_{1}$. This space is strong on account of Lemma \ref{finchar}.
As before, by $\ssc(H)$ we mean $\gena{\ssc(H_{1})}{M}$. This is the minimal strong $\nla{2}$-subalgebra of $M$
containig $H$.

\smallskip
We adopt for the sequel the following notation: for any subspace$H_{1}$ and tuple $\bar a$ of $M_{1}$, we write
$\ssc(H_{1},\bar a)$ for $\ssc(\genp{H_{1},\bar a})$ and $\ssc(H,\bar a)$ for $\gena{\ssc(H_{1},\bar a)}{M}$.
On the other hand, by default $d^{M}$ reads indifferently sets, subspaces or tuples of $M_{1}$.

\begin{prop}\label{fincharssc}
Assume $H_{1}$ is a strong subspace of $M_{1}$ and $\bar a$ is a finite tuple in $M_{1}$, then
\begin{itemize}
\punto{i}$d(\bar a\quot H_{1})\leq\delta(\bar a\quot H_{1})$ %for any finite tuple $\bar c\inn M_{1}$
\punto{ii}$\ssc(H_{1},\bar a)$ is a {\em finite} extension of $H_{1}$,
\punto{iii}$d(\bar a\quot H_{1})=\delta(\ssc(H_{1},\bar a)\quot H_{1})=\min(\delta(A_{1}\quot H_{1})\mid A_{1}\nni\bar a)$
\punto{iv}$d(\bar a\quot H_{1})=\delta(\bar a\quot H_{1})$ iff $\genp{H_{1},\bar a}\zsu{}M_{1}$.
\punto{v}There exists a finite $H_{1}^{\rm o}\zsu{}H_{1}$ such that $\ssc(H,\bar a)=H+\ssc(H^{\rm o},\bar a)$, 
$H_{1}\cap\ssc(H_{1}^{\rm o},\bar a)=H_{1}^{\rm o}$ and that %$\delta(\ssc(H_{1}^{\rm o},\bar a)\quot H^{\rm o}_{1})
$d(\bar a/ H)=d(\bar a/ H^{\rm o})$.
\end{itemize}
%Moreover 
\end{prop}
\begin{proof}
(i). By (fin) and \pref{dimadd} of Section \ref{qdim} and Remark \ref{finitedeltabase} above, we can find a finite subspace $H^{\rm o}_{1}\zsu{}H_{1}$
with $H^{\rm o}_{1}\nni H_{1}\cap\genp{\bar a}$, such that
$$d(\bar a\quot H_{1})=d(\bar a\quot H_{1}^{\rm o})=d(H^{\rm o}_{1},\bar a)-d(H^{\rm o}_{1})$$
and that
$$\delta(\bar a\quot H_{1})=\delta(\bar a\quot H_{1}^{\rm o})=\delta(H^{\rm o}_{1},\bar a)-\delta(H^{\rm o}_{1}).$$
Since $H^{\rm o}\zsu{}M$, the statement follows immediately by the relation between $\delta$ and $d$ for
finite subspaces of $M_{1}$.

\smallskip
(ii). Since $\delta(A/H)$ is non-negative for all finite subspace $A_{1}$ in $M_{1}$, take
a finite subspace $A_{1}$ containing $\bar a$ with a minimal value of $\delta(A/H)$.
It follows that for an arbitrary finite $C_{1}$ one has
$$\delta(C/H+A)=\delta(C+A/H)-\delta(A/H)\geq0.$$
This means $H+A$ is self-sufficient in $M$ and hence contains $\ssc(H,\bar a)$.

As a consequence,
%the generalised self-sufficient closure
%preserves the properties of her finite analogous, that is,
the second equality in (iii) holds: $\delta(\ssc(H_{1},\bar a)/H)=\min(\delta(A/H)\mid\bar a\inn A_{1}\inn M_{1},A_{1}\text{ finite})$.

\smallskip
(iii). Take a finite tuple $\bar b$ of $M_{1}$ linear independent over $H_{1}$,
such that $\genp{H_{1},\bar b}=\ssc(H_{1},\bar a)$.
Since $\bar b\inn\ssc(H_{1},\bar a)\inn\cl_{d}(\bar a\quot H_{1})$,
we have $d(\bar b\quot H_{1})=d(\bar a\quot H_{1})$.

As $\genp{H_{1},\bar b}$ is self-sufficient, we can find a finite strong subalgebra $H^{\rm o}$ of $H$,
such that $\genp{H_{1}^{\rm o},\bar b}\zsu{}M_{1}$ with $H_{1}^{\rm o}\nni H_{1}\cap\genp{\bar a}$ and
$d(\bar b/H^{\rm o})=d(\bar b/H)$.

Now by (i) and Lemma \ref{2transmogrifer} we obtain
$d(\bar b/H)\leq\delta(\bar b/H)\leq\delta(\bar b/H^{\rm o})=d(\bar b/H^{\rm o})$ and hence
$\delta(\ssc(H_{1},\bar a)/H)=\delta(\bar b/H)=d(\bar a/H)$. 

\smallskip
(iv). Follows from (iii). 

\smallskip
(v). Let $H^{\rm o}$ and $\bar b$ like in (iii).\,above, since $\ssc(H^{\rm o}_{1},\bar a)\inn\genp{H^{\rm o}_{1},\bar b}$,
we have
$$H^{\rm o}_{1}=\genp{H^{\rm o}_{1},\bar b}\cap H_{1}\nni\ssc(H^{\rm o}_{1},\bar a)\cap H_{1}\nni H^{\rm o}_{1}$$
that is $H^{\rm o}_{1}=\ssc(H^{\rm o}_{1},\bar a)\cap H_{1}$.

On the other hand, by applying submodularity \pref{submod} and (iii) above, we get
$$\delta(\bar b\quot H)\leq\delta(\ssc(H^{\rm o}_{1},\bar a)/H_{1})
\leq\delta(\ssc(H^{\rm o}_{1},\bar a)/H_{1}^{\rm o})=d(\bar b\quot H^{\rm o})=d(\bar b\quot H).$$
Thus by (iv), since  $\delta(\ssc(H^{\rm o}_{1},\bar a)/H)=d(\bar a/H)=d(\ssc(H^{\rm o}_{1},\bar a)/H)$, we have
$H+\ssc(H^{\rm o},\bar a)\zsu{}M$.
It follows $\genp{H_{1},\bar b}\inn H_{1}+\ssc(H^{\rm o}_{1},\bar a)$ and hence $\ssc(H_{1},\bar a)=H_{1}+\ssc(H_{1}^{\rm o},\bar a)$.
Moreover this yields also $\genp{H_{1}^{\rm o},\bar b}=\ssc(H^{\rm o}_{1},\bar a)$.
\end{proof}
From the last proposition it follows, for $H$ and $\bar a$ as above, that $\ssc(H_{1},\bar a)$ is
the intersection of all strong subspaces of $M_{1}$ containing $\genp{H_{1},\bar a}$. % and finite over $H_{1}$.
\begin{rem}\label{cielle2}
Let $\mathcal{B}$ be any set of $M_{1}$, then
%$$\cl_{d}(\mathcal{B})=\gen{\,\bigcup\{C_{1}\,\text{finite}\mid\delta(C_{1}\quot\ssc(\mathcal{B}))=0\}}.$$
$\cl_{d}(\mathcal{B})$ is the subspace of $M_{1}$ generated by all finite $C_{1}\inn M_{1}$ such that $\delta(C_{1}/
\ssc(\mathcal{B}))=0$.

In particular $\ssc(\mathcal{B})\inn\cl_{d}(\mathcal{B})$ for all sets $\mathcal{B}$.
\end{rem}
%\begin{proof}
%By definition $\delta(\ssc(\mathcal{B},a/\mathcal{B})=0$ for all $a\in\cl_{d}(\mathcal{B})$.
%
%On the contrary
%\end{proof}

\section{Amalgamation within $\nla{2}$}\label{amalga2}
Denote by $\Kl{2}$ the class of all finitely generated -- or equivalently, finite -- Lie algebras $M$ in $\nla{2}$,
which share property $\sig{2}{2}$ defined at page \pageref{sig22}. Then $\Kl{2}$ is a denumerable set.

At the end of this section we show properties (HP), (JEP) and (AP) for the class $\Kl{2}$
with respect to strong $\nla{2}$-%(rather than $\Lan{2}$-)
embeddings as described in Remark \ref{ModiFra}.\footnote{We tacitly perform two modifications of the
standard method: one changes $\Lan{2}$-embeddings into $\nla{2}$-ones, the second introduces strongness.}
The proof of Fact \ref{fraissteo}, to achieve a countable Fra\"iss\'e limit of $(\Kl{2},\zsu{})$ applies in this case as well and yields the same results.

In accordance to this, %and slightly abusing the notation of Section \ref{fraisse},
we rename by $\age(K)$ the collection of all finite $\nla{2}$-subalgebras of an algebra $K$ from $\nla{2}$.
If $\Klt{2}$ denotes the family of all $K$ in $\nla{2}$ with $\age(K)\inn\Kl{2}$, then $\Klt{2}$ is {\em almost} an elementary class\footnote
{$\nla{2}$ itself is not elementary as pointed out in Section \ref{nilgral}, but as a consequence of {\em richness}, the theory
of the Fra\"iss\'e limit in the next Section can express property \ref{genero}.\,of Definition \ref{lcp}} 
and we have $\Klt{2}=\{K\in\nla{2}\mid K\sat\sig{2}{2}\}$.

\smallskip
%Fix an algebra $K$ of $\Klt{2}$.
%We call\mn{\bf Check if ever used!} a quotient $M_{1}/N_{1}$ of vector subspaces of
%$K_{1}$, a {\em strong section of} $K$ if $N\zsu{}H$.
%In this case we shall also say that $M$ {\em strongly extends $N$ in $K$} and that
We say that $H\in\Klt{2}$ is a {\em finite} extension of $K=\gena{K_{1}}{H}$
if $H_{1}$ has finite $\Fp$-dimension over $K_{1}$ and $H$ is a {\em strong extension} of $K$ if $K\zsu{}H$.


\medskip
\begin{dfn}\label{amalgama}
Let $M$, $N$ and $K$ be algebras of $\nla{2}$. Assume we have $\nla{2}$-embeddings
$\phi$ of $N$ into $M$ and $\nu$ of $N$ into $K$. We say that an $\nla{2}$-algebra
$H$ {\em amalgamates $M$ and $K$ over $N$} if there exist $\nla{2}$-embeddings
$\mu$ of $M$ into $H$ and $\psi$ of $K$ into $H$ such that $\phi\mu=\nu\psi$. In this case we draw the following square.
\begin{labeq}{appi}
\begin{split}
\xymatrix@R-4mm{
&H\\
M\ar^(0.55){\mu}[ur]&&K\ar_(0.55){\psi}[ul]\\
&N\ar_(0.55){\phi}[ul]\ar^(0.55){\nu}[ur]
}\end{split}
\end{labeq}
\end{dfn}

\smallskip
It is always possible to build amalgams inside $\nla{2}$ as follows: assume $M$, $N$ and $K$ as above, we may consider $N$,
without loss of generality, as a common $\nla{2}$-subalgebra of $M$ and $K$, that is $\gena{N_{1}}{M}=N=\gena{N_{1}}{K}$.

We first build the $\Fp$-vector space amalgam $H_{1}=
M_{1}\oplus_{N_{1}}K_{1}$, %recall $H_{1}=M_{1}\oplus_{N_{1}}K_{1}$,
which is by definition $M_{1}\oplus K_{1}/\Delta(N_1)$ where $\Delta(N_{1})=\{(h,-h)\mid h\in N_{1}\}$.
In $H_{1}$, $M_{1}$ and $K_{1}$ meet exactly in $N_{1}$.

We now define
the {\em free amalgam of $M$ and $K$ over $N$} by
\begin{labeq}{fram}
\am{M}{N}{K}\defeq\frac{\fla{2}{H_{1}}}{\rd(M)+\rd(N)}=H_{1}\oplus\frac{\exs H_{1}}{\rd(M)+\rd(K)}.
\end{labeq}

By a matter of weight $\rd(\am{M}{N}{K})=\rd(M)+\rd(K)$ is an ideal of $\fla{2}{H_{1}}$
and hence the definition above is sound. Moreover $\am{M}{N}{K}=\gen{H_{1}}$ and lays in $\nla{2}$ and
$\rd(M)\cap\rd(K)=\rd(N)$. 

\smallskip
\begin{rem}
$\am{M}{N}{K}$ fits in the diagram \pref{appi} in the place of $H$, with the natural $\nla{2}$-embeddings. That is
$\am{M}{N}{K}$ amalgamates $M$ and $K$ over $N$.

Moreover $M\cap K=\gena{M_{1}}{\am{M}{N}{K}}\cap\gena{K_{1}}{\am{M}{N}{K}}=\gena{M_{1}\cap K_{1}}{\am{M}{N}{K}}=N$.
\end{rem}
\begin{proof}
Let $H$ denote $\am{M}{N}{K}$. Since we identify $\fla{2}{M_{1}}$ with an $\nla{2}$-subalgebra of $\fla{2}{H_{1}}$, we have to show $\rd(H)\cap\fla{2}{M_{1}}=\rd(M)$,
so that the map $w+\rd(M)\mto w+\rd(H)$ yields the desired $\nla{2}$-embedding $\mu$.
But this holds, since $(\rd(M)+\rd(K))\cap\fla{2}{M_{1}}=(\rd(M)+\rd(K))\cap\exs M_{1}=\rd(M)+(\rd(K)\cap\exs M_{1})
=\rd(M)+(\rd(K)\cap\exs N_{1})=\rd(M)+\rd(N)=\rd(M)$.

A symmetric argument for $K$ implies the statement and the {\em moreover} part follows by
$\rd(H)\inn\exs M_{1}+\exs K_{1}$ and $\exs M_{1}\cap\exs K_{1}=\exs N_{1}$.
\end{proof}

\smallskip
The following definitions provides a notion of {\em inner} free amalgam. 
\begin{dfn}\label{freeco}
Assume $M$ and $K$ are $\nla{2}$-extension of $N$ in a $\Klt{2}$-algebra $H$. % and $K_{1}\inn H_{1}$.
We say that $M$ is in {\em free composition with $K$ over $N$ in $H$} if $M+K(=\gena{M_{1}+K_{1}}{H})$ %\simeq
is isomorphic with $\am{M}{N}{K}$.
This is equivalent to require that
$M_{1}\cap K_{1}=N_{1}$ and %, that $M_{1}+K_{1}\simeq_{\Fp}\vam{M_{1}}{N_{1}}{K_{1}}$ and
that $$\rd_{H}(H_{1}+M_{1})\simeq_{\Fp}\rd_{H}(M_{1})+\rd_{H}(K_{1}).$$
\end{dfn}

Deficiency calculus yields an easy criterion to check for free-compositions:
\begin{lem}\label{freecomp}
Let $M$, $N$ and $K$ be $\nla{2}$-subalgebras of $H\in\Klt{2}$, then the following conditions are equivalent:
\begin{itemize}
\punto{i}$M$ is in free-composition with $K$ over $N$,
\punto{ii}$M_{1}\cap K_{1}=N_{1}$ and $\delta(A/N)=\delta(A/K)$ for any finite subspace $A_{1}$ of $M_{1}$.
%\punto{iii}$M_{1}\cap K_{1}=N_{1}$ and $\rd(K+M)\non\rd(K)\inn\exs M_{1}$.
\end{itemize}
\end{lem}
\begin{proof}
If $M_{1}\cap K_{1}=N_{1}$ holds, then %$\rd(M+K)=\rd(M)+\rd(K)$ exactly if $\delta(M/N)=\delta(M/K)$.
$\dfp(M_{1}/N_{1})=\dfp(M_{1}/K_{1})$ and by \pref{exsmod} we have
$\rd(M)\cap\rd(K)=\rd(M_{1}\cap K_{1})=\rd(N)$.

One has in fact a canonical linear embedding of $\rd(M/N)$ into $\rd(M/K)$. This embedding is onto
if and only if $\rd(K+M)=\rd(K)+\rd(M)$ but also iff
for any finite $A_{1}\inn M_{1}$ the corresponding mapping of $\rd(A/N)$ in $\rd(A/K)$ is onto.
This is true exactly if $\dfp(\rd(A/N))=\dfp(\rd(A/K))$ and hence exactly when
$\delta(A/N)=\delta(A/K)$ for all finite subspaces $A_{1}\inn M_{1}$.
\end{proof}

\begin{rem}
For the free amalgam $\am{M}{N}{K}$ with a finite dimensional
{\em side} $M_{1}/N_{1}$, one has $\delta(M/N)=\delta(M/K)$.
\end{rem}

At the end of the chapter we will see that
{\em composing} free-compositions turns out to be {\em transitivity} of forking in the theory of the $\Kl{2}$-rich structure.
This lemma will be helping.
\begin{lem}\label{fctrans}
Assume $H\nni M\nni N\inn K$ are $\nla{2}$-extensions. Then
$$\am{H}{N}{K}\simeq\am{H}{M}{(\am{M}{N}{K})}.$$
%\item[ii)]\mn{bl\"od}Let $H/N$ be a finite section of $L$ and let $N_{1}\inn M_{1}\inn K_{1}\inn L_{1}$.
%Then $H$ is in free composition with $K$ over $N$ iff $H$ is in free composition with
%$M$ over $N$ and $H+M$ is in free composition with $K$ over $M$.
\end{lem}
\begin{proof}
The statement essentially follows because it is true of vector space amalgams, that is
$\vam{H_{1}}{N_{1}}{K_{1}}\simeq_{\Fp}\vam{H_{1}}{M_{1}}{(\vam{M_{1}}{N_{1}}{K_{1}})}$.

Since $\am{M}{N}{K}$ $\nla{2}$-embeds into $\am{H}{N}{K}$, to conclude we have to show that
$H$ is in free composition with $\am{M}{N}{K}$ over $M$ in $\am{H}{N}{K}$. If deficiencies are computed inside $\am{H}{N}{K}$,
we have in fact
\begin{multline*}
\delta(H/M+K)=\delta(H+M/K)-\delta(M/K)=\delta(H/K)-\delta(M/N)=\\
=\delta(H/N)-\delta(M/N)=\delta(H/M).
\end{multline*}
Now Lemma \ref{freecomp} applies.
\end{proof}

\begin{rem*}
$H=\am{M}{N}{K}$ with the morphism in \pref{appi}, represents the {\em amalgamated coproduct} in the category $\nla{2}$.
This means,
for any $Z$ in $\nla{2}$ and $\nla{2}$-morphisms $\map{\alpha}{M}{Z}$ and $\map{\beta}{K}{Z}$ with
$\phi\alpha=\nu\beta$, there exists a unique morphism $\map{\zeta}{\am{M}{N}{K}}{Z}$ with
$\mu\zeta=\alpha$ and $\psi\zeta=\beta$.
\end{rem*}

\medskip
The next lemma shows that the free amalgam \pref{fram} preserves self-sufficient extensions.
\begin{lem}\label{asymam2}
In the notation of Definition \ref{2strong}, for any $k<\omega$, $N\zsu{k} K$ holds if and only if $M\zsu{k} \am{M}{N}{K}$
does.
In particular $N\zsu{}K$ iff $M\zsu{}\am{M}{N}{K}$.
\end{lem}
\begin{proof}
%\mn{we give two argmts cfr. strategy nil-$3$ case}\\
%{\sl (Argument I)}\quad Assume $E_{1}\inn L_{1}$ is a finite subspace.
%
%Find a basis 
%$\mathcal{E}^{m}\mathcal{E}^{h}\mathcal{E}^{k}(u_{i}+v_{i}:i=1,\dots,n)$ like {\bf ($\flat$)} for $E_{1}$.
%
%We have to show $\delta(E/ M)\geq0$.
%
%By definition $\delta(E_{1}/ M_{1})=\dfp(E_{1}/ M_{1})-\dfp(\rd(E/ M))$
%and $\rd(E/ M)$ equals $\rd(M+E)/\rd(M)$.
%
%We are going to build an $\Fp$-isomorphism of $\rd(M_{1}+E_{1})/\rd(M_{1})$ into
%$\rd(N_{1}+E_{1}\cap K_{1}+\gen{\mathcal{V}})/\rd(N_{1})$ as follows.
%Since $(M_{1}+E_{1})\cap K_{1}=N_{1}+E_{1}\cap K_{1}+\gen{\mathcal{V}}$, any $\phi$ in
%$\rd(M_{1}+E_{1})$ decomposes, by Lemma \ref{reldue} into $\phi^{M}+\phi^{K}$ where $\phi^{M}\in\exs M_{1}$, $\phi^{K}\in\exs (N_{1}+E_{1}\cap K_{1}+\gen{\mathcal{V}})$ and there exists $I\in\exs N_{1}$ such that $I+\phi^{K}\in \rd(K)\cap\exs(N_{1}+E_{1}\cap K_{1}+\gen{\mathcal{V}})$.
%
%Now map $\overline{\phi^{M}+\phi^{K}} $ to $\overline{I+\phi^{K}}$. This is independent of the choice
%of $I$ as two such $I$'s differ by an element of $\rd(K)\cap\exs N_{1}$.
%
%The map is obviously onto and mono, since for $I+\phi^{K}$ to be in $\rd(N)$ means
%$\phi^{M}-I+I+\phi^{K}$ is in $\rd(M)$.
%
%\smallskip
%On the other hand $\dfp(E_{1}/ M_{1})=\dfp(E_{1}\cap K_{1}+\gen{\mathcal{V}}/ N_{1})$ and
%thus $\delta(E_{1}/ M_{1})=\delta(E_{1}\cap K_{1}+\gen{\mathcal{V}}/ N_{1})\geq0$
%as desired. 
%
%\bigskip
%{\sl (Argument II)}\quad
Consider a subspace $D_{1}\nni M_{1}$ of $\vam{M_{1}}{N_{1}}{K_{1}}$. %with $\dfp(D_{1}/ M_{1})\leq k$.
Since $D_{1}=M_{1}+(D_{1}\cap K_{1})$ one has $D_{1}/ M_{1}\simeq_{\Fp}D_{1}\cap K_{1}/ N_{1}$.

On the other hand, since $\rd(K)=\rd(\am{M}{N}{K})\cap\exs K_{1}$, we have by \pref{exsmod},
$\rd(D)=(\rd(M)+\rd(K))\cap\exs D_{1}=\rd(M)+\rd(D_{1}\cap K_{1})$.
Now this yields $\rd(D/M)=\rd(D_{1}\cap K_{1}/N_{1})$.

Hence we may conclude
\begin{labeq}{deltamalgam}
\delta(D/M)=\delta(D_{1}\cap K_{1}/ N_{1})
\end{labeq}
%where, of course, $\dfp(D_{1}\cap K_{1}/ N_{1})\leq k$. T
and the statement of the lemma follows.
\end{proof}
\begin{cor}\label{parapa}
Let $M\nni N\inn K$ as in the previous lemma.
If $A$ denotes $\am{M}{N}{K}$ and $A_{1}\nni D_{1}\nni M_{1}$, let $D$ be $\gena{D_{1}}{A}$ and %$D\scap K$ 
$I$ denote $\gena{D_{1}\cap K_{1}}{A}$.
Then
$$D\simeq\am{M}{N}{\gena{D_{1}\cap K_{1}}{K}}\quad\text{and}\quad A\simeq\am{D}{I%D\cap K
}{K}.$$
\end{cor}
\begin{proof}
We assume for simplicity, that $D_{1}$ is finite over $M_{1}$. As observed above $D_{1}=M_{1}+(D_{1}\cap K_{1})$ and
by \pref{deltamalgam} we have
$$\delta(D_{1}\cap K_{1}/ M_{1})=\delta(D/M)=\delta(D_{1}\cap K_{1}/N_{1})$$
and thus, Lemma \ref{freecomp} gives the first statement. The second follows by the facts $A_{1}\simeq_{\Fp}
\vam{D_{1}}{I_{1}%D_{1}\cap K_{1}
}{K_{1}}$ and $\rd(A)=\rd(M)+\rd(K)=\rd(D)+\rd(K)$.
\end{proof}

\medskip
We now introduce {\em minimal strong extensions} of $\Klt{2}$-algebras. This is the main tool to
compute the rank of types in the rich $\Klt{2}$-structures.
\begin{dfn}\label{minimalext}
We say that a proper strong $\nla{2}$-extension $K\zsu{}H$ is {\em minimal} if
there is no subspace $V_{1}$ strictly in-between $H_{1}$ and $K_{1}$ such that $V$ is strong in $H$.
\end{dfn}

By Lemma \ref{fincharssc},(ii) minimal extensions are necessarily finite, moreover a finite
extension $H$ of $K$ is minimal exactly if $\delta(H/K^{\prime})<0$ for all $K_{1}\subsetneq
K_{1}^{\prime}\subsetneq H_{1}$.

It turns out that there are only three
types of minimal strong extensions, this is the content of the next proposition
\begin{prop}
Assume $H$ is a minimal extension of $K$, then only one of the following
three situation may occur.
\begin{itemize}
\punto{i} $H$ is a {\em free} or {\em transcendental} extension of $K$, that is $\dfp(H_{1}/K_{1})=\delta(H/K)=1$ and $\rd(H)=\rd(K)$.

\punto{ii} $H$ is an {\em algebraic} extension of $K$: $\dfp(H_{1}/K_{1})=1$ and $\delta(H/K)=0$.

\punto{iii} $H$ is a {\em prealgebraic} extension of $K$: $\dfp(H_{1}/K_{1})\geq2$, $\delta(H/K)=0$ and
for any finite $E_{1}\subsetneq H_{1}$ not entirely contained in $K_{1}$ one has $\delta(E/K)>0$. 
\end{itemize}
\end{prop}
\begin{proof}
We know $H_{1}$ is finite over $K_{1}$ and assume first
$d^{H}(H_{1}/K_{1})=\delta(H/K)=0$.

If $\delta(h/K)=0$ for some $h$ in $H_{1}$,
then $\gen{K_{1},h}$ is strong in $H$ by Lemma \ref{samedelta2} and by minimality $H=\gena{K_{1},h}{H}$. We are in (ii).

\smallskip
If there is no $h$ with \lqq saldo null\rqq over $K$, then $\dfp(H_{1}/K_{1})>1$ and by minimality
for any proper subspace $E_{1}$ of $H_{1}$ not entirely contained in $K_{1}$, we must have $\delta(E/K)>0$.
This gives a prealgebraic extension.

\medskip
On the other hand if $d^{H}(H/K)>0$, then there must be an $a$ of $H_{1}$ $\cl_{d}$-independent of
$K_{1}$. This implies $d^{H}(a/K)=1$ and $\gena{K_{1},a}{H}\zsu{}H$, hence $H=\gena{K_{1},a}{H}$ and
also $\delta(a/K)=1$. In particular $\rd(H)=\rd(K)$.
\end{proof}

An algebraic extension is associated to a divisor element according to the following remark.
\begin{rem}\label{divelement}
For any $\nla{2}$-subalgebra $K$ of $H\in\Klt{2}$, a {\em divisor} of $K$ is an element $a$ of $H_{1}\non K_{1}$
with $\delta(a/M)\leq0$. This is equivalent to require, that $[a,x]\in K_{2}$ for some non-trivial element $x$ of $K_{1}$.
If in addition $K$ is ($1$-)strong in $H$ and $a$ is a divisor of $K$, then $$\rd_{H}(K_{1},a)=\rd(K)\oplus\genp{[a,x]-\kappa}$$
for some $x\in K_{1}$ and $\kappa\in\exs K_{1}$. In particular $\gen{K,a}$ is a minimal algebraic extension of $K$.
\end{rem}

\begin{rem}\label{prealgchain}{\ }
\begin{itemize}
\item[1.]For any $B\in\Klt{2}$, any fixed $b$ of $B_{1}$ and $w$ in $B_{2}$, assume there is no $x\in B_{1}$ with $[x,b]=w$, then there
is a minimal algebraic strong extension $A=\gen{B_{1},a}$ of $B$ in $\Klt{2}$ such that $[a,b]=w$.

\item[2.]For any positive integer $n$ and any  $M\in\Kl{2}$,
if $\dfp(M_{1})$ is large enough ($\geq2+2n$), it is possible to find a chain $$M\zsu{}M^{1}\zsu{}M^{2}\zsu{}\dots\zsu{}M^{n}$$
in which $M^{i+1}$ is a minimal prealgebraic extension of $M^{i}$ for all $i$ and $M^{n}$ is in $\Kl{2}$.
\end{itemize}
\end{rem}
\begin{proof}
1. Define an extension $A$ of $B$ as follows: set first $A_{1}:=B_{1}\oplus\Fp$ and let $a\in A_{1}$ generate $A_{1}$ over $B_{1}$.
Then set $\rd(A):=\rd(B)\oplus\gen{[a,b]-\beta}$, where
$\beta$ is an element of $\exs B_{1}$ which represents $w$ modulo
$\rd(B)$. Hence $[a,b]-\beta$ is an element of $\exs A_{1}$.

Since in $\delta(A/B)=0$, $B$ is self-sufficient in $A$. We show
next that $A$ is in $\Klt{2}$. Let $E_{1}$ be a finite subspace of $A_{1}$,
then $E_{1}$ has dimension at most $1$ over $E_{1}\cap B_{1}$. Thus in a nontrivial
case, there exists $b^{\prime}\in B_{1}$ such that $E_{1}=\gen{a+b^{\prime},E_{1}\cap B_{1}}$.

As by submodularity \pref{submod} $\delta(E)=\delta(E_{1}\cap B_{1})+\delta(a+b^{\prime}/ E_{1}\cap B_{1})$ and by Lemma \ref{2cut}
$E_{1}\cap B_{1}\zsu{}E_{1}$, if $\dfp(E_{1}\cap B_{1})\geq2$ we have $\delta(E)\geq2$.

The other only case to be considered is when $\dfp(E_{1})=2$ and
$E_{1}=\gen{a+b^{\prime},u}$ for $b^{\prime},u$ in $B_{1}$.
If $\rd(E)\neq\triv$, then we may assume the equality $[a+b^{\prime},u]=[a,b]-\beta+\eta$ holds in $\exs A_{1}$,
for some $\eta$ in $\rd(B)$. This translates into $[a,u-b]=[u,b^{\prime}]-\beta+\eta\in\exs B_{1}$.

If we take any $\Fp$-basis $(b_{i}\mid i<n)$ of $B_{1}$
for some $n<\omega$, then the set $([a,b_{i}]\mid i<n)$ is a basis for
$\exs A_{1}$ {\em over} $\exs B_{1}$ (cfr.\,Fact \ref{ubc}). This yields that $u=b$ and that
$[u,b^{\prime}]-\beta$ belongs to $\rd(B)$. Thus the element $-b^{\prime}$ of $B_{1}$ solves the equation $[-b^{\prime},b]=w$ in $B$, contradicting our assumption.

\medskip
For 2. it is sufficient to prove the first step, assume hence $M$ is in $\Kl{2}$ with
%infinite dimensional $M_{1}$ and choose
at least four linearly independent element $b_{1},b_{2},c_{1},c_{2}$ in $M_{1}$.

Define an $\nla{2}$-algebra $K$ by means of the following presentation
$$K=\gen{a_{1},a_{2},M_{1}\mid[a_{1},b_{1}]+[a_{2},b_{2}],[a_{1},c_{1}]+[a_{2},c_{2}]}.$$
It is clear that $\delta(K/M)=0$ and that $K$ is a prealgebraic strong extension of $M$.

We have to show that $K$ lays in $\Kl{2}$ as well: for any finite $E_{1}\inn K_{1}$ we must prove
$\delta(E)\geq\min(2,\dfp(E_{1}))$.

By \pref{submod} we have
$\delta(E)\geq\delta(E_{1}\cap M_{1})+\delta(E/M)$. Moreover,
for any element $u$ of $K_{1}\non M_{1}$, then $\delta(u/M)>0$. For, since $u$ is without loss $s a_{1}+t a_{2}$
for some $s,t\in\Fp$, if an element $\rho$ of $\rd(K)$ lays in $\exs\genp{M_{1},u}$, then
$$\rho=[sa_{1}+ta_{2},m]+\mu=
u([a_{1},b_{1}]+[a_{2},b_{2}])+v([a_{1},c_{1}]+[a_{2},c_{2}])+\eta$$
for some $u,v\in\Fp$, $m\in M_{1}$, $\eta\in\rd(M)$ and some $\mu\in\exs M_{1}$.

As a consequence we obtain
\begin{labeq}{contrad}
[a_{1}, sm-ub_{1}-vc_{1}]+[a_{2},tm-ub_{2}-vc_{2}]\in\exs M_{1}
\end{labeq}
which is impossible unless $b_{1},b_{2},c_{1},c_{2}$ are linearly dependent.

Now if every $2$-generated $\nla{2}$-subalgebra of $K$ is free,
then the same is true of all its $3$-generated subalgebras (cfr.\,\cite[Lemma 4.5]{bad}).

Hence by the above inequality, since $M$ has $\sig{2}{2}$ we only have to prove this property in the case $\dfp(E_{1})=2$ and
$E_{1}\cap M_{1}=\triv$.
In which case, with no loss of generality $E_{1}=\genp{a_{1}+l,a_{2}+m}$
for $l,m$ elements of $M_{1}$. Now if some element of $\exs E_{1}$ meets $\rd(K)$, then
a contradiction like \pref{contrad} would follow. Thus $\rd(E)=\triv$ in this case and $\sig{2}{2}$ holds in general for $K$.
%the following equality must be true in $\exs K_{1}$,
%$$[a_{1}+u,a_{2}+v]=s([a_{1},b_{1}]+[a_{2},b_{2}])+t([a_{1},c_{1}]+[a_{2},c_{2}])+\rho$$
%for $s,t\in\Fp$ and $\rho\in\rd(M)$. This implies
%$$[a_{1},a_{2}]+[a_{1}, v-sb_{1}-tc_{1}]-[a_{2},u+sb_{2}+tc_{2}]=\rho-[u,v]\in\exs M_{1}$$
%which can never be the case.
\end{proof}

\begin{dfn}\label{mindecomp}
Let $K\in\Klt{2}$ be a finite strong extension of $M$, a {\em minimal decomposition} of $K$ over $M$ is a
sequence of minimal self-sufficient extensions %$M^{i}_{1}\nni M^{i-1}_{1}$ for $i=1,\dots n$ such that
\begin{labeq}{mindec}
M=M^{0}\zsu{}M^{1}\zsu{}\cdots\zsu{}M^{n}=K.
\end{labeq}
such that the following two conditions are satisfied:
\begin{itemize}
\punto{1}If $d^{K}(K/M)=d$, then $M^{i}\nni M^{i-1}$ is transcendental for $i\leq d$,
\punto{2}for all $i>d$, if $M^{i}$ {\em is not} a minimal algebraic extension of  $M^{i-1}$,
then there is no divisor $a$ of $M^{i-1}$ in $K$.
\end{itemize}
\end{dfn}
Since the notion of self-sufficiency is transitive, by Lemma \ref{samedelta2}, it is always possible to find
a minimal decomposition of $K$ over $M$ {\em for any} finite strong extension $K$ of $M$.
We first exhaust all transcendental steps and obtain $M^{d}$ like in (1), so that $d^{K}(K/M^{d})=\delta(K/M^{d})=0$.

Then (2) follows, by letting algebraic extensions take precedence over pr\ae{}lgebraic ones in the sequence.

With Proposition \ref{dizero} of the next section follows that the number of prealgebraic steps in a minimal decomposition
is an invariant of the elementary type of the extension.

\smallskip
Minimal decompositions commute with free amalgamation:
\begin{lem}\label{mindecamalg}
Let $M\zso{}M^{\sss 0}\inn H^{\sss 0}$ be $\nla{2}$-algebras.
Then $$M^{\sss 0}\zsu{}M^{1}\zsu{}\cdots\zsu{}M^{n}=M$$ is a minimal decomposition of $M$ over $M^{\sss 0}$ if and only if
$$H^{\sss 0}\zsu{}\am{M^{1}}{M^{\sss 0}}{H^{\sss 0}}\zsu{}\am{M^{2}}{M^{\sss 0}}{H^{\sss 0}}\zsu{}\cdots\zsu{}\am{M^{n}}{M^{\sss 0}}{H^{\sss 0}}$$%=\am{M}{M^{\sss 0}}{H^{\sss 0}}$$
is a minimal decomposition of $\am{M}{M^{\sss 0}}{H^{\sss 0}}$ over $H^{\sss 0}$.

In any of the two cases above, each extension $$\am{M^{i}}{M^{\sss 0}}{H^{\sss 0}}\nni \am{M^{i-1}}{M^{\sss 0}}{H^{\sss 0}}$$
is exactly of the same kind of $M^{i}\nni M^{i-1}$, for all $1\leq i\leq n$.
\end{lem}
\begin{proof}
As for all $i$, Lemmas \ref{fctrans} and \ref{asymam2} imply
$$\am{M^{i}}{M^{\sss 0}}{H^{\sss 0}}\simeq\am{M^{i}}{M^{i-1}}{(\am{M^{i-1}}{M^{\sss 0}}{H^{\sss 0}})}\zso{}\am{M^{i-1}}{M^{\sss 0}}{H^{\sss 0}},$$
both statements of the Lemma follow by %induction,
considering $M$ minimal over $M^{\sss 0}$.

\smallskip
Let then $H$ denote the free amalgam $\am{M}{M^{\sss 0}}{H^{\sss 0}}$. For any subspace $K_{1}$ of $H_{1}$ with $K_{1}\nni H^{\sss 0}_{1}$, since
$K_{1}=H^{\sss 0}_{1}+(K_{1}\cap M_{1})$, we have
$$H_{1}\supsetneq K_{1}\supsetneq H^{\sss 0}_{1}\iff M_{1}\supsetneq K_{1}\cap M_{1}\supsetneq M^{\sss 0}_{1}.$$
If now $K$ denote $\gena{K_{1}}{H}$ and $K^{\prime}=\gena{K_{1}\cap M_{1}}{M}$, then by Corollary \ref{parapa},
$K\simeq\am{K^{\prime}}{M^{\sss 0}}{H^{\sss 0}}$. Hence by Lemma \ref{asymam2} and Proposition \ref{fctrans}
$$K^{\prime}\zsu{}M\iff K\zsu{}\am{M}{K^{\prime}}{K}=\am{M}{K^{\prime}}{(\am{K^{\prime}}{M^{\sss 0}}{H^{\sss 0}})}=H.$$

This means $H\nni H^{\sss 0}$ is minimal exactly if $M\nni M^{\sss 0}$ is a minimal extension and
proves the first statement, while the second, follows by equality \pref{deltamalgam} -- here
$\delta(K/H^{\sss 0})=\delta(K^{\prime}/M^{\sss 0})$ -- of Lemma \ref{asymam2}.%, where $K$ and $K^{\prime}$ are as above.
In particular for any $h\in H_{1}$ there is an $m$ in $M_{1}$ such that $\delta(h/H^{\sss 0})=\delta(m/M^{\sss 0})$ and the lemma
follows.
\end{proof}

\medskip
Before we prove the next step toward amalgamation, we need to analyse in detail the space of relators $\rd$ in the free amalgam.
To do that, we have to find suitable bases for the subspaces of the vector-space amalgam, which
ease the treatment of basic monomials. This is Lemma 4.2 in \cite{bad}.
\begin{lem}\label{basE}
Assume $H$ is the free amalgam $\am{M}{N}{K}$ and let $E_{1}$ be a finite subspace of $H_{1}=\vam{M_{1}}{N_{1}}{K_{1}}$.

Assume there exists $n<\omega$ and subsets $\mathcal{U}=(u_{i})_{i=1}^{n}$ and $\mathcal{V}=(v_{i})_{i=1}^{n}$
of $M_{1}$ and $K_{1}$ respectively, such that $(u_{i}+v_{i})$ is a $\Fp$-basis
of $E_{1}$ over $E_{1}\cap M_{1}+E_{1}\cap K_{1}$.

%[EXT:] This yields that the set $(u_{i}+v_{i})$ is linearly
%independent over $E_{1}\cap M_{1}+E_{1}\cap K_{1}+H_{1}$ as well.
%[EXT:] \lambda\cdot u+v \in E_{1}\cap M_{1}+E_{1}\cap K_{1}+H_{1}
% implies, since it is in E_{1} too, that it is actually in E_{1}\cap M_{1}+
%E_{1}\cap K_{1}+(E_{1}\cap H_{1})=E_{1}\cap M_{1}+E_{1}\cap K_{1} !]
Then the subset $\mathcal{UV}$ of $H_{1}$ is linearly independent over $E_{1}\cap M_{1}+E_{1}\cap K_{1}+N_{1}$.
\end{lem}
\begin{proof}
We follow an inductive argument over $n<\omega$. Assume the assertion holds for $1\leq k\leq n-1$ and
$\mathcal{U}=(u_{i})$ and $\mathcal{V}=(v_{i})$ for $1\leq i\leq n$ are the sets mentioned in the statement.
If we set $\hat{\mathcal{U}}=\{u_{i}\mid i<n\}$ and $\hat{\mathcal{V}}=\{v_{i}\mid i<n\}$, then
%set $\hat{E}_{1}:=\gen{E_{1}\cap M_{1}+E_{1}\cap K_{1},u_{i}+v_{i}\mid i<n}$ so that
$\hat{\mathcal{U}}\hat{\mathcal{V}}$ is linearly independent over $E_{1}\cap M_{1}+E_{1}\cap K_{1}+N_{1}$.

Set $\tilde{E}_{1}:=\genp{E_{1}\cap M_{1},E_{1}\cap K_{1},\hat{\mathcal{U}},\hat{\mathcal{V}},u_{n}+v_{n}}$ and notice
%so that $\tilde{E}_{1}\cap M_{1}+\tilde{E}_{1}\cap K_{1}=$\\
that $u_{n}+v_{n}$ generates $\tilde{E}_{1}$ over $\tilde{E}_{1}\cap M_{1}+\tilde{E}_{1}\cap K_{1}$ hence, by induction,
$\{u_{n},v_{n}\}$ is linearly independent over $\tilde{E}_{1}\cap M_{1}+\tilde{E}_{1}\cap K_{1}+N_{1}=
\genp{E_{1}\cap M_{1},E_{1}\cap K_{1},N_{1},\hat{\mathcal{U}},\hat{\mathcal{V}}}$ and
the set $\mathcal{UV}$ is independent over $E_{1}\cap M_{1}+E_{1}\cap K_{1}+N_{1}$.

The assertion is therefore to be proven in the case $n=1$.
Let then $E_{1}$ be generated by a sum $u+v$ over
$E_{1}\cap M_{1}+E_{1}\cap K_{1}$ for $u\in M_{1}$ and $v\in K_{1}$.


It follows $u+v$ is not in $E_{1}\cap M_{1}+E_{1}\cap K_{1}+N_{1}$.
%otherwise
%$u+v\in E_{1}\cap M_{1}+E_{1}\cap K_{1}+(E_{1}\cap N_{1})=E_{1}\cap M_{1}+E_{1}\cap K_{1}$.
If now $su+tv \in E_{1}\cap M_{1}+E_{1}\cap K_{1}+N_{1}$ for some $s$ and $t$ in $\Fp$ and
say $s\neq0$, we have then $%\ambda(u+v)+  $\lambda\neq\mu$
(t-s)v\in K_{1}\cap (E_{1} +N_{1})=N_{1}+(E_{1}\cap K_{1})$ and thus $s(u+v)\in E_{1}\cap M_{1}+E_{1}\cap K_{1}+N_{1}$ which is a contradiction.
\end{proof}

\begin{rem}\label{basee}
Let $H$ be the free amlagam above. In the previous notation, for every finite $E_{1}\inn H_{1}$, we can find a $\Fp$-base of $E_{1}$
in the form
\begin{labeq}{flat}
\mathcal{E}_{N}\mathcal{E}_{M}\mathcal{E}_{K}(u_{i}+v_{i}\mid u_{i}\in\mathcal{U},v_{i}\in\mathcal{V})_{i=1,\dots,n}
\end{labeq}
where $\mathcal{E}_{N}$ is a base of $E_{1}\cap N_{1}$, and $\mathcal{E}_{M}$ and $\mathcal{E}_{K}$ complete
$\mathcal{E}_{N}$ to a basis of $E_{1}\cap M_{1}$ and $E_{1}\cap K_{1}$ respectively.
\end{rem}

A $\Fp$-basis $\mathcal{H}=\mathcal{H}_{M}\mathcal{H}_{N}\mathcal{H}_{K}$ of $H_{1}$ is said
{\em compatible} with the base \pref{flat}, if $\mathcal{E}_{N}\inn\mathcal{H}_{N}$ and $\mathcal{H}_{N}$ is a $\Fp$-basis of $N_{1}$;
if $\mathcal{U},\mathcal{E}_{M}\inn\mathcal{H}_{M}$ and $\mathcal{H}_{M}\mathcal{H}_{N}$ is a basis for $M_{1}$ and lastly
if $\mathcal{V},\mathcal{E}_{K}\inn\mathcal{H}_{K}$ and $\mathcal{H}_{N}\mathcal{H}_{K}$ is a basis for $K_{1}$.
%described above, we can always complete the $\Fp$-independent set $\mathcal{E}_{M}\mathcal{E}_{N}\mathcal{E}_{K}
%\mathcal{U}\mathcal{V}$ to a basis $\mathcal{D}$ of $D_{1}$ as follows: begin by finding a basis $\mathcal{E}_{N}$ of $E_{1}\cap H_{1}$, this is to be extended with $\mathcal{E}_{M}$ and $\mathcal{E}_{K}$ to a basis of $E_{1}\cap M_{1}$ and $E_{1}\cap K_{1}$ respectively. In this way $\mathcal{E}_{M}\mathcal{E}_{N}\mathcal{E}_{K}$ forms a basis of
%$E_{1}\cap M_{1}+E_{1}\cap K_{1}$.
%Let $\mathcal{U}$ and $\mathcal{V}$ be as above, according to Lemma \ref{basee}, we can first complete $\mathcal{E}_{N}$ to a basis $\mathcal{D}^{h}$ of $H_{1}$, and then 
%find  $\mathcal{D}^{m}$ containing $\mathcal{E}_{M}\mathcal{U}$ such that $\mathcal{D}^{h}\mathcal{D}^{m}$ is a basis of $M_{1}$.
%On the other hand we can complete  $\mathcal{D}^{h}$ to a basis $\mathcal{D}^{h}\mathcal{D}^{k}$ of $K_{1}$ with $\mathcal{D}^{k}\nni
%\mathcal{E}_{K}\mathcal{V}$.
%We call $\mathcal{D}=\mathcal{D}^{m}\mathcal{D}^{h}\mathcal{D}^{k}$
%a {\em compatible basis} of $D_{1}$ \emph{with} {\bf $(\flat)$}. %$E_{1}$.

\medskip
This way of extending bases to $H_{1}$ leads to the following description of $\rd(E)$ for any given finite $E$.

Recall with Fact \ref{basisgen}, Corollary \ref{ubc} and Definitions \ref{basicommutators} and \ref{supp}, that for 
a base $\mathcal{H}$ of $H_{1}$, any set $\mathscr{B}=\mathscr{B}_{\leq2}$ of basic monomials over $\mathcal{H}$
of weight $\leq2$, constitutes a basis of $\fla{2}{H_{1}}=H_{1}\oplus\exs H_{1}$. In particular chosen an order $(\mathcal{H},<)$,
the set $\mathscr{B}_{2}=\{[b,c]\mid b>c\in\mathcal{H}\}$, is a basis of $\exs H_{1}$ and $\mathscr{B}=\{\mathcal{H}<
\mathscr{B}_{2}\}$ a basis of $\fla{2}{H_{1}}$. The following is borrowed from \cite[Lemma 4.3]{bad}.
\begin{lem}\label{reldue}
Let $E=\gena{E_{1}}{H}$ be a finite subalgebra of the free amalgam $H=\am{M}{N}{K}$ for $M\nni N\inn K$ algebras
in $\nla{2}$.

Let  $\mathcal{E}=\mathcal{E}_{N}\mathcal{E}_{M}\mathcal{E}_{K}(u_{i}+v_{i}:i=1,\dots,n)$ be a basis of $E_{1}$ as in \pref{flat}.
%, and
%choose a linear order on $\mathcal{E}$.

We order a base $\mathcal{H}=\mathcal{H}_{N}>\mathcal{H}_{M}>\mathcal{H}_{K}$ of $H_{1}$ compatible with $\mathcal{E}$,
in such a way that $\mathcal{E}_{N}>\mathcal{E}_{M}>\mathcal{U}>\mathcal{E}_{K}>\mathcal{V}$.
%extending the order chosen above\mn{adjust indices here}.

Then each element $\Phi$ of $\rd(E)$ has the form
\begin{labeq}{2relator}
\Phi_{M}+\Phi_{K}+\phi_{u}+\phi_{v}
\end{labeq}
where
\begin{itemize}
\item[]$\Phi_{M}$ and $\Phi_{K}$ are linear combination of basic $\mathcal{H}$-commutators with support
in $\mathcal{E}_{N}\mathcal{E}_{M}$ and $\mathcal{E}_{N}\mathcal{E}_{K}$ respectively.
\item[]$\phi_{u}$ is a linear combination of basic $\mathcal{H}$-commutators $[h,u_{i}]$ where $h$ belongs to $\mathcal{E}_{N}$ and
$u_{i}$ is in $\mathcal{U}$
\item[]$\phi_{v}$ is obtained by replacing each instance of $u_{i}$ in $\phi_{u}$ by the corresponding $v_{i}$ from $\mathcal{V}$.
\end{itemize}
Finally there exists $\eta$ in $\exs N_{1}$ such that $\Phi_{M}+\phi_{u}+\eta$ belongs to $\rd(M)$ and $\Phi_{K}+\phi_{v}-\eta$
is in $\rd(K)$.
\end{lem}

\begin{proof}
Let $\Phi$ be an element of $\rd(E)$, which is by definition $\rd(H)\cap\exs E_{1}$. Now $\rd(H)=\rd(M)+\rd(K)\inn
\exs M_{1}+\exs K_{1}$, hence there exist $\rho_{M}$ in $\rd(M)$ and $\rho_{K}\in \rd(K)$ such that $\Phi=\rho_{M}+\rho_{K}$.

Write $\rho_{M}$ and $\rho_{K}$ as linear combinations of basic $\mathcal{H}$-monomials with support respectively in $\mathcal{H}_{N}\mathcal{H}_{M}$ and $\mathcal{H}_{N}\mathcal{H}_{K}$, and call $\Phi_{\mathcal{H}}$ the resulting unique expression
of basic $\mathcal{H}$-commutators which equals $\Phi$ after Corollary \ref{ubc}.

\smallskip
On the other hand, consider the linear order on $\mathcal{E}$ given by $\mathcal{E}_{N}>\mathcal{E}_{M}>\mathcal{E}_{K}>
\{u_{1}+v_{1}>\dots>u_{n}+v_{n}\}$, where the first three fragments inherit the order of $\mathcal{H}$ above.
Now write $\Phi\in\exs E_{1}$ as a linear combination $\Phi_{\mathcal{E}}$ of basic $\mathcal{E}$-commutators.
%We have $\Phi_{\mathcal{H}}=\Phi=\Phi_{\mathcal{E}}$.
By linearity, each monomial %$\mathcal{E}$-basic monomial in $\Phi_{\mathcal{E}}$
involving entries $u_{i}+v_{i}$ expands into a sum of 
basic monomials over $\mathcal{H}$: just transpose -- and accordingly change the sign of -- the entries which are in the wrong order. 
%of the form $[b,u_{i}+v_{i}]$ for $b$ in $\mathcal{E}$ equals
%indeed $\mathcal{H}$-basic terms $[b,u_{i}]+[b,v_{i}]$ (with transposed entries and
%inverse sign, if $b$ happens to be greater than $u_{i}$ or $v_{i}$).
This means $\Phi_{\mathcal{E}}$ is actually equal to a linear combination $\Phi^{\prime}$ of
basic $\mathcal{H}$-monomials as well.

Now comparing expressions $\Phi_{\mathcal{H}}=\Phi=\Phi^{\prime}$, by Corrollary \ref{ubc},
%we have in particular $\supp_{\mathcal{H}}(\Phi_{\mathcal{H}})=\supp_{\mathcal{H}}(\Phi^{\prime})$.
exactly the same monomials must appear in $\Phi_{\mathcal{H}}$ and $\Phi^{\prime}$.

It follows, that terms of the kind
$[b,u_{i}+v_{i}]$ with $b\in\mathcal{E}_{M}\mathcal{E}_{K}$ or $b=u_{j}+v_{j}$ for any $j$, are not allowed
in the expression $\Phi_{\mathcal{E}}$. Following the same
argument, basic monomials $[m,k]$ with $m\in\mathcal{E}_{M}$ and $k\in\mathcal{E}_{K}$ are excluded from $\supp_{\mathcal{E}}
(\Phi_{\mathcal{E}})$ as well.

We can conclude $\Phi$ consists of the sum $\Phi_{M}+\Phi_{K}+\phi_{u}+\phi_{v}$ described in the statement of the lemma.

To obtained $\eta$, consider equality $\Phi_{M}+\Phi_{K}+\phi_{u}+\phi_{v}=\Phi=\rho_{M}+\rho_{K}$ and set
$\eta:=\rho_{M}-\Phi_{M}-\phi_{u}=\Phi_{K}+\phi_{v}-\rho_{K}
\in\exs M_{1}\cap\exs K_{1}=\exs N_{1}$.
\end{proof}

With the above description of relators, we can prove the following lemma, which
shows, the only obstruction for the free amalgam to inherit property $\sig{2}{2}$ from
its components are the divisor elements of the base.

\smallskip
If some algebra $K$ extends $N$ and $N\inn H$, a divisor (cfr.\,Remark \ref{divelement}) $a\in H_{1}$ of $N$ is {\em realised} %by $b$
in $K$ over $N$, if there exists an element $b$ of $K_{1}$ and an isomorphism
of $\gena{N_{1},a}{H}$ onto $\gena{N_{1},b}{K}$ which fixes $N$ and maps $a$ onto $b$.

If $N$ is strong in both $H$ and $K$, according to Remark \ref{divelement},
for some $x\in N_{1}$ and $\eta\in\exs N_{1}$, $[a,x]-\eta$ generates $\rd(N,a)$ over $\rd(N)$.
In order to realise $a$ in $K$, it is sufficient to find $b\in K_{1}$ with $[b,x]-\eta\in\rd(K)$.
\begin{prop}\label{amalsigma2}
Assume $M\zso{k}N\zsu{}K$ for $\Klt{2}$-algebras $M$, $N$ and $K$, %with $\sig{2}{2}$,
where $K$ is a finite extension of $N$,
and the integer $k$ is not smaller than $\dfp(K_{1}/N_{1})$.

Assume also that for any divisor $a$ of $N$ in $K$, $a$ {\em is not realised in $M$ over $N$}.
Then $\am{M}{N}{K}$ satisfies $\sig{2}{2}$.
\end{prop}
\begin{proof}
Let $H$ denote $\am{M}{N}{K}$, then by Lemma \ref{asymam2} we have $M\zsu{}H\zso{k}K$.

Let $E_{1}$ be a finite subspace of $\vam{M_{1}}{H_{1}}{K_{1}}$ and choose a $\Fp$-basis
$\mathcal{E}=\mathcal{E}_{N}\mathcal{E}_{M}\mathcal{E}_{K}(u_{i}+v_{i}\mid i=1,\dots,n)$ of $E_{1}$ for suitable
$u_{i}$'s in $M_{1}$ and $v_{i}$'s in $K_{1}$ as described in Remark \ref{basee}.
We have to show $\delta(E)\geq\min(2,\dfp(E_{1}))$.

Applying submodularity \pref{submod} of $\delta$, we find $\delta(E)\geq\delta(E/ M)+\delta(E_{1}\cap M_{1})$.
Since $M$ is self-sufficient and satisfies $\sig{2}{2}$, if $\dfp(E_{1}\cap M_{1})\geq2$ we are done.
We might then assume $\dfp(E_{1}\cap M_{1})<2$.

\smallskip
If $E_{1}\cap M_{1}=\triv$,
then $\mathcal{E}=\mathcal{E}_{K}(u_{i}+v_{i}\mid i=1,\dots, n)$ and by Lemma \ref{reldue} we have $\rd(E)=
%\mn{check! Here and below}
\rd(E_{1}\cap K_{1})$. It follows $\dfp(E_{1})=\dfp(E_{1}\cap K_{1})+n$ and hence $\delta(E)=\delta(E_{1}\cap K_{1})+n$.
This yields $\delta(E)\geq\min(2,\dfp(E_{1}))$ since $\delta(\gena{E_{1}\cap K_{1}}{K})$ does.

\smallskip
Assume $E_{1}\cap M_{1}$ %\mn{too many cases considered?}
has dimension $1$. If $E_{1}\cap N_{1}=\triv$ then by Lemma \ref{reldue} again, $\rd(E)=\rd(E_{1}\cap
K_{1})$ because $\mathcal{E}=\{m\}\mathcal{E}_{K}(u_{i}+v_{i}\mid i=1,\dots,n)$ with $\{m\}$ as $\mathcal{E}_{M}$.
We can conclude as above: this time $\delta(E)=\delta(E_{1}\cap K_{1})+n+1$.

\smallskip
Assume now $E_{1}\cap M_{1}=E_{1}\cap N_{1}=\gen{h}$ has dimension $1$, so that
$E_{1}=\gen{E_{1}\cap K_{1},u_{i}+v_{i}\mid i=1,\dots,n}$ and $\mathcal{E}=\{h\}\mathcal{E}_{K}(u_{i}+v_{i}\mid i=1,\dots,n)$.

If $E_{1}\cap K_{1}$ is $\gen{h}$ as well ($\mathcal{E}_{K}=\vac$), then $E_{1}=\gen{h,u_{i}+v_{i}\mid i=1,\dots,n}$. If we assume that 
$\rd(E)$ is nontrivial %-- the converse would give us the desired conclusion --
then by Lemma \ref{reldue} a nonzero element
$\Phi$ of $\rd(E)$ is equal to a sum $\sum_{i=1}^{n}s_{i}[h,u_{i}+v_{i}]$ for some scalars $s_{i}$ in $\Fp$. Moreover for a suitable
$\eta$ in $\exs N_{1}$, $[h,\sum_{i=1}^{n}s_{i}u_{i}]+\eta$ lies in $\rd(M)$ and $[h,\sum_{i=1}^{n}s_{i}v_{i}]-\eta$ in $\rd(K)$.

If we now set $v:=\sum_{i=1}^{n}s_{i}v_{i}\in K_{1}\non N_{1}$, then we get $\delta(v/N)=0$ and
$v$ is a divisor of $K$. Since $N$ is at least $1$-self-sufficient in $M$, if we set $u:=-\sum_{i=1}^{n}s_{i}u_{i}$,
then $[h,u]-\eta\in\rd(M)$ and $u$ realises $v$ in $M$ over $N$. The hypotheses now imply that
this sitution cannot occur and in this case $\rd(E)=\triv$.

\smallskip
For the very last step we assume, as in the previous case, that
$E_{1}\cap M_{1}=E_{1}\cap H_{1}=\gen{h}$, but this time
$\dfp(E_{1}\cap K_{1})\geq2$ and thus $\delta(E_{1}\cap K_{1})\geq2$.
%By Lemma \pref{reldue} again, any $\phi$ in $\rd(E)$ equals
%a sum $\phi^{u}+\phi^{K}+\phi^{v}$, where $\phi^{u}=\sum_{i=1}^{n} s_{i}[h,u_{i}]$, $\phi^{v}=\sum_{i=1}^{n}s_{i}[h,v_{i}]$
%for $s_{i}\in\Fp$ and $\phi^{K}$ in $\exs E_{1}\cap K_{1}$.
%
%Moreover to any such $\phi$, an element $\beta$ of
%$\exs H_{1}$ is given, such that $\phi^{u}+\beta$ is in $\rd(M)$ and $\phi^{K}+\phi^{v}-\beta$ is in $\rd(K)$.
%%, where $\phi^{K}$ may be assumed non-trivial, to avoid the previous forbidden situation.
%
%Notice that the map $\phi\mapsto\phi^{u}+\beta$ can be factorised to a well defined linear morphism
%of $\rd(E_{1}/ E_{1}\cap K_{1})$ to $\rd(\gen{u_{i}\mid i=1,\dots,n}/ H_{1})$. If
%now $\phi^{u}+\beta$ is in $\rd(H_{1})$ then $\phi$ belongs to $\rd(E)\cap\exs K_{1}=\rd(E_{1}\cap K_{1})$,
%and therefore $\phi$ is a linear embedding of $\rd(E_{1}/ E_{1}\cap K_{1})$ into $\rd(\gen{u_{i}\mid i=1,\dots,n}/ H_{1})$.

Now by submodularity over $K$ we obtain
$\delta(E)\geq\delta(E/K)+\delta(E_{1}\cap K_{1})$
%=\delta(E_{1}\cap K_{1})+n-\dfp(\rd(E_{1}/ E_{1}\cap K_{1}))\geq
%\delta(E_{1}\cap K_{1})+\delta(\gen{u_{i}\mid i=1,\dots,n}/N)$ and 
and $\delta(E/K)=\delta(\gen{u_{i}+v_{i}\mid i=1,\dots,n}/K)\geq0$ since $H$ is $k$-self-sufficient in $M$ and by Lemma \ref{basE},
$n\leq\min(\dfp(M_{1}/N_{1}),\dfp(K_{1}/N_{1}))\leq k$.

\smallskip
As there is no more cases left, $H$ has $\sig{2}{2}$ and the proof is complete.
\end{proof}

The following {\em asymmetric amalgamation} both proves amalgamation in $\Klt{2}$ and makes it possible to
axiomatise the theory of the rich Fra\"iss\'e limit of $\Kl{2}$.
\begin{lem}[Asymmetric Amalgam]\label{asymalgadue}
Let $M$, $N$ and $K$ be algebras of $\Klt{2}$ %with $\sig{2}{2}$
such that $K$ is a finite self-sufficient extension of $N$, and $N$ is $n+\dfp(K_{1}/ N_{1})$-self-sufficiently
embedded in $M$, for some $n<\omega$.

Then there exists $H$ in $\Klt{2}$, %$\nla{2}$  satisfying axiom $\sig{2}{2}$,
which amalgamates $M$ and $K$ over $N$ as in Definition \ref{amalgama}, under which embeddings,
$M$ is strong and $K$ is $n$-strong in $H$.
\end{lem}
\begin{proof}
Fix an integer $n$ and assume $M$, $N$ and $K$ as above. We prove the
statement by induction on $l=\dfp(K_{1}/ N_{1})$. So let
$\widetilde{N}=\gena{\widetilde{N}_{1}}{K}$ be a self-sufficient
subalgebra of $K$ such that $K\nni\widetilde{N}$ is a minimal strong extension.
\footnote{consider the last step of a minimal decomposition of $K$ over $N$.}
Denote by $\widetilde{\nu}$ the strong embedding of
$N$ into $\widetilde{N}$, by $\nu^{\prime}$ the strong embedding of $\widetilde{N}$ into $K$
and by $\phi$ the embedding of $N$ into $M$.

If $\widetilde{l}=\dfp(\widetilde{N}_{1}/ N_{1})$ and
$l^{\prime}=\dfp(K/\widetilde{N}_{1})$, then by the inductive hypothesis there exists an
algebra $\widetilde{M}$ in $\Klt{2}$ which amalgamates $M$ and $\widetilde{N}$ over
$N$ by virtue of a strong embedding $\widetilde{\mu}$ of $M$ into $\widetilde{M}$
and of a $n+l^{\prime}$-strong embedding $\psi^{\prime}$ of $\widetilde{N}$ into $\widetilde{M}$ such that $\phi\widetilde{\mu}=\widetilde{\nu}\psi^{\prime}$.
$$
\xymatrix@C-1mm@R-2mm{
&&&H\ar@{-}[dl]_{\mu^{\prime}}\ar@{--}[drr]^{\psi}&&\\
&&\widetilde{M}\ar@{--}[drr]^{\psi^{\prime}}&&&K\ar@{-}[dl]_{\nu^{\prime}}\\
&&&&\widetilde{N}&\\
M\ar@{-}[uurr]^{\widetilde{\mu}}\ar@{--}[drr]^{\phi}&&&&&\\
&&N\ar@{-}[uurr]^{\widetilde{\nu}}&&&}
$$

Now we distinguish two cases: in the first one, $K$ is an algebraic extension of $\widetilde{N}$
with $K=\gen{\widetilde{N}_{1},a}$ for some $a$ in $K_{1}$ and
we assume that $K$ is realised in $\widetilde{M}$ over $\widetilde{N}$ by means of
an element $m$ of $\widetilde{M}_{1}$. We set in this case $H=\widetilde{M}$ and
let $\psi$ denote the $\nla{2}$-Lie isomorphism which fixes $\widetilde{N}$ and
maps $a$ onto $m$. Then clearly $H$ has $\sig{2}{2}$.
Now since $\widetilde{N}$ is $n+1$-selfsufficient
in $H$ and $\delta(m/\widetilde{N})=0$, it follows that
$\psi(K)$ is $n$-self-sufficient in $H$. This is clear since,
for any finite subspace $E_{1}$ of $H_{1}$ with $\dfp(E_{1}/\psi(K_{1}))\leq n$
one has $\delta(E/\psi(K))=\delta(E/\widetilde{N}_{1},m)=
\delta(E_{1},m/\widetilde{N}) %+\delta(m/\widetilde{N}_{1})
\geq0$.

\medskip
In the second case we consider algebraic extensions which {\em are not} realised
in $H$, as well as free or pre-algebraic extensions $K/\widetilde{N}$.

By Proposition \ref{amalsigma2} the free amalgam $\am{\widetilde{M}}{\widetilde{N}}{K}=:H$
satisfies property $\sig{2}{2}$. Denote with $\psi$ the canonical embedding of $K$
into $H$, as $\psi^{\prime}$ is $n+l^{\prime}$-strong, Lemma \ref{asymam2} implies that $\psi$ is
$n+l^{\prime}$-strong as well, and in particular $n$-strong. %\am{\widetilde{M}}{\widetilde{N}}{K}$.

\smallskip
In both of the cases considered above, denote with
$\mu^{\prime}$ the strong embedding of $\widetilde{M}$ into $H$ and put
$\mu:=\widetilde{\mu}\mu^{\prime}$ and $\nu:=\widetilde{\nu}\nu^{\prime}$.
Then $\mu$ is a strong embedding, and $\phi\mu=\nu\psi$ as required
by Definition \ref{amalgama}.
\end{proof}

\begin{cor}\label{amalgadue}
The countable class $\Kl{2}$ has the properties {\rm(HP)}, {\rm(JEP)} and {\rm(AP)} defined in Section \ref{fraisse},
with respect to strong $\nla{2}$-embeddings.
\end{cor}
\begin{proof}
As $\sig{2}{2}$ is expressed by universal sentences, then in particular $\nla{2}$-subalgebras of an object $A$ of $\Kl{2}$
still satisfy it. Hence we have (HP).

That $\Kl{2}$ satisfies (AP) is Corollary \ref{asymalgadue},
with $M,N$ and $K$ finite and strong embeddings on both sides.

Moreover, since the trivial algebra $\triv$ is self-sufficient in every structure of $\Kl{2}$,
if we apply Corollary \ref{asymalgadue} to pairs of algebras (with $N=\triv$) we obtain
the joint embedding property (JEP).
\end{proof}
Now Fact \ref{fraissteo} applies to the countable class $\Kl{2}$ with respect to strong $\nla{2}$-embeddings.
The Fra\"iss\'e limit $\K$ of $(\Kl{2},\zsu{})$ obtained in this way, is a countable $\Kl{2}$-{\em rich} algebra of $\Klt{2}$. This means
by definition
\begin{itemize}
\item[-]$\age(\K)=\Kl{2}$
\item[-]for any finite strong $\nla{2}$-extension $A$ of $B$
in $\Kl{2}$, if $\map{\beta}{B}{\K}$ is a self-sufficient embedding, there exists a strong $\nla{2}$-embedding $\alpha$ of $A$ into $\K$
with $\res{\alpha}{B}=\beta$.
\end{itemize}
\section{The theory $T^{2}$}\label{t2axioms}
%\begin{presection}
%The axiom-schema we propose here is the natural and obligatory one in a setting
%in which strongness is given by predimension and in which the class supports
%a nice amalgamation. This is also the line of \cite{jbg}, for an extensive account of
%various versions of amalgamation constructions one can see \cite{wag}.
%
%It has to be pointed out that this approach of approximating richness, by means of axioms
%$\sig{2}{3}$ below, works because of the {\em asymmetric} amalgam (Lemma \ref{asymalg}).
%\end{presection}
In this last section, we axiomatise the $\Lan{2}$-theory of the Fra\"iss\'e limit $\K$ of $(\Kl{2},\zsu{})$. We prove
it is $\omega$-stable and calculate its Morley rank.

\medskip
Nilpotency of class $2$ can be expressed universally in $\Lan{2}$,
in terms of simple commutators, by requiring $[x,y,z]=[[x,y],z]=\triv$ for all $x,y,z$.
Altough the language $\Lan{2}$ can naturally express the grading on each $\nla{2}$-algebra, by means of $M=P_{1}(M)+P_{2}(M)$,
$P_{1}(M)\cap P_{2}(M)=\triv$
and $(\forall xy)P_{2}([x,y])$, in general there is no first-order bound to the length of homogeneous sums of weight $2$, which could
express $\gena{P_{1}(M)}{}=M$.

In the axiom system chosen below for $\K$, this is sorted out in the strongest way possible: we require
each weight $2$ element to be the Lie bracket of exactly two elements (from $P_{1}$). This feature may be compared with
a corollary to Zilber's {\em Indecomposability Theorem}: in any group $G$ of finite Morley rank, there exists an integer $n$
such that  any $g\in G^{\prime}$ is the product of $n$ commutators $[x_{i},y_{i}]$. 

As a consequence, it will be true that elementary $\Lan{2}$-extensions are $\nla{2}$-extensions.

\smallskip
If $M$ is an $\Lan{2}$-structure, we define the theory
$T^{2}$ by means of the following denumerable first order schema of $\Lan{2}$-axioms, expressed in terms of $M$:
\begin{itemize}
\punto{$\sig{2}{1}$}$M$ is a graded nil-$2$ Lie algebra over the field $\Fp$. This corresponds to write properties
1.{\,}and 2.{\,}of Definition \ref{lcp} in $\Lan{2}$ as described above.
\punto{$\sig{2}{2}$}For any finite subspace $H_{1}$ of $M_{1}$ $\delta(H)\geq\min(\dfp(H_{1}),\,2)$. 
\punto{$\sig{2}{3}$}for any finite strong extension $A\nni B$ of $\Kl{2}$-algebras and any $n<\omega$,
if $B$ is $(\dfp(A_{1}/ B_{1})+n)$-selfsufficient in $M$,
then there exists an isomorphic copy of $A$ in $M$ over $B$, which is $n$-self-sufficient.
\punto{$\sig{2}{4}$} for all $y\in M$ with $P_{2}(y)$ and all $\triv\neq z\in M_{1}$ there is $x\in M_{1}$ such that $y=[z,x]$
\end{itemize}
We can first observe that axioms $\sig{2}{2}$ and $\sig{3}{3}$ imply that a model of $T^{2}$ cannot be finite.

\begin{teo}\label{Cazzuola}
%The rich %countable structures in 
An $\La_{2}$-structure $K$ is a rich algebra of $\Klt{2}$
%are exactly the countable $\omega$-saturated models of $T^{2}$.
if and only if $K$ is an $\omega$-saturated model of $T^{2}$.
\end{teo}
\begin{proof}
We start proving that a rich algebra $K$ of $\Klt{2}$ is also a model of $T^{2}$ which is henceforth consistent: the
Fra\"iss\'e limit $\K$ of $(\Kl{2},\zsu{})$ exhibits a countable model. The second part of the proof shows that an $\omega$-saturated model of $T^{2}$ is a rich $\Klt{2}$-algebra, now since rich
structures are $\Lan{2}_{\infty,\omega}$-equivalent, because of Fact \ref{fraissteo} ($\nla{2}$-embeddings are in particular
$\Lan{2}$-embeddings),
it follows that rich $\Klt{2}$-structures are $\omega$-saturated.

\smallskip
So let first $K$ be such a rich algebra in $\Klt{2}$, then axioms $\sig{2}{1}$ and $\sig{2}{2}$ are
satisfied automatically.

To prove $\sig{2}{3}$, assume $A$ is a finite strong extension in $\Kl{2}$ of a finite subalgebra
$B$ of ${K}$ and
$B$ is $(\dfp(A_{1}/ B_{1})+n)$-selfsufficient in ${K}$. Take a finite strong subalgebra $\widetilde{B}$ of ${K}$ containing $B$ (the selfsufficient closure of $B$ for instance).
We use the asymmetric amalgamation Lemma \ref{asymalgadue} to obtain a strong extension $\widetilde{A}$
of $\widetilde{B}$ in $\Kl{2}$, such that $A$ is $n$-self-sufficient in $\widetilde{A}$.

Now since ${K}$ is rich, $\widetilde{A}$ strongly embeds into ${K}$ over $\widetilde{B}$.
As a consequence of transitivity (Lemma \ref{2trans}), $A$ embeds into ${K}$
$n$-selfsufficiently over $B$.

\smallskip
To prove $\sig{2}{4}$, pick an element $w\in K_{2}$ and $m\in K_{1}$. Since $K=\gen{K_{1}}$, there exists a finite subspace
$B_{1}$ of $K_{1}$ with $m\in B_{1}$ and $w\in B_{2}$. We may
clearly assume that $B$ is self-sufficient in $K$.

If there exists $c$ in $B_{1}$ such that $[c,m]=w$ we are done, if not, then we can apply Remark \ref{prealgchain},
and find a minimal algebraic strong extension $A$ of $B$, such that $A$ is in $\Kl{2}$ and $A_{1}=\genp{B_{1},a}$ where
$[a,m]=w$. Now since $B\zsu{}K$ and $K$ is rich, $a$ is realised in $K$ over $B$. In particular 
there is $a^{\prime}$ in $K_{1}$ with $[a^{\prime},m]=w$ as desired.

\medskip
For the reverse implication suppose $M$ is an $\omega$-saturated model of $T^{2}$, then by $\sig{2}{4}$
the $\Lan{2}$-structure $M$ is in particular an object of $\nla{2}$.

\smallskip
Now let $A\nni B$ be a finite strong extension of $\Kl{2}$-algebras, where $B$ is a finite strong
$\nla{2}$-subalgebra of $M$. We may assume, without loss
of generality, that $A$ is a minimal extension of $B$; otherwise we decompose it
in a chain of minimal strong sections like \pref{mindec} and strongly embed each subalgebra stepwise in $M$, over the predecessor.

Assume first $A\nni B$ is a free extension and $B$ is a finite strong substructure in $M$,
by Proposition \ref{minimalext} and Lemma \ref{samedelta2} we are done if we find $a\in M_{1}$ which is $\cl_{d}$-independent of $B_{1}$, as $d(a/ B_{1})=1$ implies that $\gena{B_{1},a}{M}$ is strong in $M$

If we can prove that $d^{M}(M_{1})$ is infinite, the desired condition will follow.
To do so, denote by $\fla{2,n}{\bar x}$ the $\Lan{2}$-formula, which describes %quantifier-free diagram of
the finite free nil-$2$ Lie algebra %$\fla{2}{\bar x}$
in the following way: %generated by an $n$-element tuple $\tpl{x}{n}$.
for any $n$-tuple $\bar a$ in $M_{1}$, $M\sat\fla{2,n}{\bar a}$ means $\gena{\tpl{a}{n}}{M}\simeq\fla{2}{\bar a}$.

We show, with an inductive argument, that we can {\em strongly} embed $\fla{2,n}{\bar x}$ in $M$ for any $n<\omega$.
Axiom $\sig{2}{2}$ ensure  that for any independent pair $m_{1},m_{2}$ of $M_{1}$,
$\gena{m_{1},m_{2}}{M}$ is a selfsufficient subalgebra of $M$ isomorphic to
the free nilpotent algebra $\fla{2}{m_{1},m_{2}}$; this will be our inductive base.

Assume now $M\sat\fla{2,n}{\bar b}$ and $\gena{\bar b}{M}$ is strong in $M$. Consider
the collection $\Phi^{n+1}(x,\bar b)$ of all formulae %\mn{better $\Phi^{n+1}(x,\bar b)\wedge \Sigma(\bar b)$ to say $\bar b$ str?}
$\phi^{n+1}_{k}(x,\bar b)$ for $k<\omega$, where a $\Lan{2}$-structure $L$ %an $\nla{2}$-algebra $L$
satisfies $\phi^{l}_{k}(\bar y)$ in $\bar m$ exactly if $\bar m\inn L_{1}$, $L\sat\fla{2,l}{\bar m}$ and $\gena{\bar m}{L}$ is $k$-strong in $L$.

Now a finite portion of $\Phi^{n+1}$ is implied by a single formula $\phi^{n+1}_{k}(x,\bar b)$ with a sufficiently
large $k$. %, here $k$ may also be assumed to be bigger than. Now use axiom
Now since $\gena{\bar b}{M}$ is strong, we have $M\sat\phi^{n}_{k+1}(\bar b)$ and hence by $\sig{2}{3}$ there exists
$a$ in $M_{1}$ such that $M\sat\phi^{n+1}_{k}(a,\bar b)$.

We showed $\Phi^{n+1}(x,\bar b)$ is consistent with $T^{2}_{\bar b}$ and hence realized in $M_{1}$
by $\omega$-saturation for some $m\in M_{1}$. It follows $\gena{m,\bar b}{M}$ is
selfsufficient and hence $d^{M}(m,\bar b)=\delta(m,\bar b)=n+1$.
By induction, $M$ has infinite $d$-dimension.
\smallskip

If $A\nni B$ is a minimal strong extension with $\delta(A/B)=0$ (hence
algebraic or prealgebraic). Since $B$ is strong in $M$, (anyone among) axioms $\sig{2}{3}$ ensure
the existence of an isomorphic copy $A^{\prime}$ of $A$ in $M$ over $B$.
Note that algebraic extension may be sorted out with $\sig{2}{4}$ as well.
%\footnote{and hence exclude
%minimal algebraic extensions from the axioms $\sig{2}{3}$}.

Because of $\delta(A^{\prime}/ B)=0$, we have that $A^{\prime}$ is self-sufficient in $M$ as
well. This proves that $M$ is a rich Lie algebra in $\Kl{2}$.
\end{proof}

The proof of the theorem also shows that if $M$ is a $\kappa$-saturated model of $T^{2}$,
then its $\cl_{d}$ dimension $d(M)$ is not smaller than $\kappa$.


Note that the Fra�ss� limit $\K$ of $\Kl{2}$ in the last section, is ``the'' countable saturated model of $T^{2}$.
%\begin{lem}\label{alg-ex-cls}
%Define a sentence $\sig{2}{4}$ by the property
%\begin{itemize}
%\end{itemize}
%then $T^{2,\omega}\models\sig{2}{4}$.
%\end{lem}
%\begin{proof}
%\end{proof}

\begin{lem}\label{strongelm}
Elementary $\nla{2}$-embedding are strong. In particular, elementary $\Lan{2}$-extension
of models of $T^{2}$ are strong $\nla{2}$-extensions.
\end{lem}
\begin{proof}
Let $M$ be an elementary $\nla{2}$-subalgebra of $N$.

If $M$ is not strong in $N$, $\delta(A/M)<0$ for
some finite subspace $A_{1}$ of $N_{1}$. By Remark \ref{finitedeltabase} there is a finite strong $C_{1}$
in $M_{1}$ such that $\delta(A/M)=\delta(A/C)$. But then, since now $\delta(A/C)$ is expressible through
a formula over $C$, %-- the $\La$-diagram of $C+A$ over $C$,
for some finite subspace $A^{\prime}_{1}$ of $M_{1}$ we have $\delta(A^{\prime}_{1}/ C_{1})$
contradicting self-sufficiency of $C$ in $M$.
\end{proof}

\begin{prop}\label{bafo}
Assume $M$ and $M^{\prime}$ are two models of $T^{2}$. Let $\bar a$ and $\bar a^{\prime}$
be tuples of $M_{1}$ and $M^{\prime}_{1}$ respectively.

Then $\mathrm{tp}(\bar a)=\mathrm{tp}(\bar a^{\prime})$ %\mn{\"aqv:$\gena{a}{M}\equiv \gena{a^{\prime}}{M^{\prime}}$?}
if and only if the selfsufficient closure $\gena{\ssc_{2}(\bar a)}{M}$ is $\nla{2}$-isomorphic to
$\gena{\ssc_{2}(\bar a^{\prime})}{M^{\prime}}$
via a Lie isomorphism mapping $\bar a$ onto $\bar a^{\prime}$.
\end{prop}
\begin{proof}
If we assume $\mathrm{tp}(\bar a)=\mathrm{tp}({\bar a}^{\prime})$, then
we have $d^{M}(\bar a)=d^{M^{\prime}}({\bar a}^{\prime})$ and we can
find a finite subspace $A_{1}$ of $M_{1}$ containing $\bar a$ and
isomorphic to $\ssc^{M^{\prime}}({\bar a}^{\prime})$. This yields
$A_{1}=\ssc^{A}(\bar a)$. Now since $\delta(A)=d^{M}(\bar a ^{\prime})=d^{M}(\bar a)\leq d^{M}(A_{1})$,
$A$ is strong in $M$, it follows $A_{1}=\ssc^{M}(\bar a)$ by Lemma \ref{samed2}.

\medskip
For the other direction
we may assume that $M$ and $M^{\prime}$ are $\omega$-saturated,
since by Lemma \ref{strongelm} the self-sufficient closure of a subspace of $M_{1}$
will remain the same if computed in any elementary (saturated) extension of $M$.

Assume tuples $\bar b\inn M_{1}$ and $\bar b ^{\prime}\inn M_{1}^{\prime}$ generate isomorphic strong subalgebras
in $M$ and $M^{\prime}$ respectively, we show that $\bar b$ and $\bar b ^{\prime}$
can be matched up by an Ehrenfeucht-Fra\"iss\'e game of lenght $\omega$.
This implies that an isomorphism between $\gena{\bar b}{M}$ and $\gena{\bar b ^{\prime}}{M^{\prime}}$ preserves $\La_{\infty,\omega}$-formulas, hence $\mathrm{tp}(\bar b)=\mathrm{tp}(\bar b ^{\prime})$.

Assume one player chooses an element $m$ of $M$ -- say -- outside $\gena{\bar b}{M}$. Then the other player
first adds a linear independent tuple $\bar c$ over $\bar b$ such that
$m\in\gena{\bar b,\bar c}{M}$ and such that $\gen{\bar b,\bar c}\zsu{}M_{1}$.
Since $\gen{\bar b^{\prime}}$ is strong embeddable into $\gen{\bar b,\bar c}$ and $M^{\prime}$ is a rich $\Klt{2}$-structure, one can
respond with a tuple $\bar c^{\prime}$ of $M_{1}^{\prime}$
with $\gena{\bar b,\bar c}{M}\simeq
\gena{\bar b^{\prime},\bar c^{\prime}}{M}$ and $\gen{\bar b^{\prime},\bar c^{\prime}}\zsu{}M_{1}^{\prime}$.
We can play $\omega$ rounds in this way, back-and-forth between $M$ and $M^{\prime}$.
%loss of generality we may assume both $\bar a ^{i}$ to be contained in $K^{i}_{1}$,
%for assume for instance, $w=w_{1}+w_{2}$ is in $\bar a ^{1}$ for $w_{2}\neq\triv$.
%If $w_{2}$ is an homogeneous sum of Lie products from $K^{1}_{1}\cap\gena{\bar a ^{1}}{K^{1}}$ then replace $w$ with $w_{1}$ (and do the same for the corresponding
%element of $\bar a ^{2}$), otherwise pick $m$ in $K_{1}^{1}$ with $[m,e^{1}]=w_{2}$ for
%some $e^{1}\in K^{1}_{1}\cap\gena{\bar a ^{1}}{K^{1}}$ (this space may be also
%assumed to be non trivial). We have $\bar a ^{1} m$ is still strong in $K^{1}$.\mn{\bf need strong for non $\nla{2}$-subalg?}
%-------MAYBE THIS?-------------
%We say that a tuple $\bar a$ in $M\in\Kl{2}$ is a {\em strong tuple}\mn{{?}see ``constructions'' in AddColl} if there exists
%a strong finite subspace $H_{1}$ in $M_{1}$ such that $\bar a\inn \gena{H_{1}}{M}$
%and such that the set of elements from $H_{1}$ which are not in $\bar a$ are $d$-independent
%over $M_{1}\cap\bar a$.
\end{proof}

\begin{rem*}
Proposition \ref{bafo} allows a {\em converse} statement of Lemma \ref{strongelm}:
any self-sufficient extension $N$ of a model $M$ of $T^{2}$ is elementary.
\end{rem*}

Since $\triv$ is self-sufficient in every model, by the lemma above
we obtain that the theory $T^{2}$ is complete and in general
any two algebras $H\zsu{}M$ and $H^{\prime}\zsu{}M^{\prime}$ which are self-sufficient in models $M$ and $M^{\prime}$ of $T^{2}$
do have the same elementary type if and only if they are isomorphic.

\medskip
For the rest of the chapter, we assume a large saturated model $\mathbb{M}$ has been fixed, as monster model of $T^{2}$.
By the above remarks, any model $M$ of $T^{2}$ is a self-sufficient $\nla{2}$-subalgebra of $\mathbb{M}$ with $\card{M}<\mathbb{M}$
%.It follows by Lemma \ref{strongelm}, that $M$ is strong in $\mathbb{M}$
and in particular, by Lemma
\ref{samed2} $d^{M}=d^{\mathbb{M}}$ on $M_{1}$ for any model $M$. Since the theory will be proved
to be $\omega$-stable, for the most purposes the countable saturated model $\K$ will be enough.

\medskip
As an immediate corollary of the previous proposition and Lemma \ref{samedelta2} we have
\begin{rem}\label{indtypes}
For any strong $H$ in $\mathbb{M}$, any $a,a^{\prime}$ in $\mathbb{M}_{1}$ are
$\cl_{d}$-independent of $H$ -- that is $d(a/H_{1})=d(a^{\prime}/H_{1})=1$ -- exactly if
$\tp{a}{H}=\tp{a^{\prime}}{H}$.
\end{rem}
%\begin{proof}
%If $a$ is $\cl_{d}$-independent of $H_{1}$, then $a$ is linearly independent of $H_{1}$ and
%there is no {\em link} between $a$ and $H$, i.{}e. $\rd(H_{1},a)=\rd(H_{1})$. Therefore, under
%the above assumptions, $\gena{H_{1},a}{\mathbb{M}}\simeq_{H}\gena{H_{1},a^{\prime}}{\mathbb{M}}$ follows.
%Moreover by Lemma \ref{samedelta2} both substructures are self-sufficient, hence the previous proposition yields
%the desired statement.
%\end{proof}

On the other hand by Proposition \ref{samedelta2} and \ref{bafo} we obtain
\begin{cor}\label{isola}
Let $B$ be a finite
strong subalgebra of %$M_{1}$ for some %$\omega$-saturated
a model $M$ of $T^{2}$.

Assume $\bar a$ is a tuple
in $M_{1}$ such that $d(\bar a/ B_{1})=0$.
Let the $\Lan{2}_{B}$-formula $\Delta(\bar x,\bar{y})$
describe the quantifier-free diagram of $\ssc(B,\bar a)$ in such a way that
for any tuple $\bar c$ of $M_{1}$, for $M$ to satisfy $\Delta(\bar a,\bar c)$ means that
$\gena{B_{1},\bar a,\bar c}{M}\simeq\ssc(B,\bar a)$. %$\gena{B_{1},a,\bar c}{M}\simeq\gena{\ssc(B,a)}{M}$.
Then the formula $\exists\bar y\Delta(\bar x,\bar y)$ isolates $\tp{\bar a}{B_{1}}$.
\end{cor}

\smallskip
We will now prove that our theory $T^{2}$ is totally transcendental.
The outline of the proof below is borrowed from Wagner's \cite{wag}.
\begin{prop}\label{omegastab}
$T^{2}$ is $\omega$-stable.
\end{prop}
\begin{proof}{Proposition \ref{omegastab}}
Since $\mathbb{M}=\gena{\mathbb{M}_{1}}{\mathbb{M}}$,
it is sufficient to count types $\tp{\bar m}{H}$ for tuples $\bar m$ in $\mathbb{M}_{1}$
and countable sets $H\inn\mathbb{M}$ (cfr.\,\ref{tuttuno}). Moreover
without loss of generality we might assume that $H=\gena{H_{1}}{\mathbb{M}}$ is
a self-sufficient subalgebra of (or a countable model in) $\mathbb{M}$.

The type of $\bar m$ over $H$ is fully determined by the quantifier-free type
of $\gena{\ssc(H_{1},\bar m)}{\mathbb{M}}$. By Lemma \ref{fincharssc} we have
$\ssc(H_{1},\bar m)=\genp{H_{1},\bar a}$ for a finite tuple $\bar a$ of $\mathbb{M}_{1}$, linearly independent over $H_{1}$.
Moreover -- still by Lemma \ref{fincharssc} -- we can find a finite subalgebra $A$ and $B$ with $B\zsu{}H$ and
$A_{1}=\genp{B_{1},\bar a}$ such that $\delta(A/B)=\delta(A/H)$ and $A_{1}\cap H_{1}=B_{1}$.

By Lemma \ref{freecomp} $H$ and $A$ are in free composition over $B$, that is
$$\gena{\ssc(H_{1},\bar m)}{\mathbb{M}}=H+A\simeq\am{H}{B}{A}.$$

Since the isomorphism type of the free amalgam is fully determined by its components,
the type of $H+A$ is determined by $\tp{A_{1}}{B}$ and by $\tp{B_{1}}{H}$ -- that is
by the choice of $B_{1}$ into $H_{1}$.

\smallskip
Since we have a countable saturated model, namely the Fra�ss� limit $\K$
of $\Kl{2}$, the theory $T^{2}$ is small, this gives only countably many choices for $\tp{A_{1}}{B}$.
Altogether we have $\aleph_{\sss 0}\cdot\card{H_{1}}^{<\aleph_{\sss 0}}=\aleph_{\sss 0}$ possibilities for $\tp{\bar a}{H}$
and in particular for $\tp{\bar m}{H}$.
\end{proof}

\begin{rem}\label{tildarich}
Any $\omega$-saturated model $M$ of $T^{2}$, satisfies a stronger version of richness over $\Klt{2}$.
That is for {\em any} self-sufficient $\nla{2}$-subalgebra $N$ of $M$, if $H$ is a finite strong extension of $N$ in $\Klt{2}$,
then $M$ embeds $H$ self-sufficiently over $N$.
\end{rem}
\begin{proof}
Split $H_{1}/N_{1}$ into two strong sections $H_{1}/K_{1}$ and $K_{1}/N_{1}$ (cfr. Definition \ref{mindecomp}),
such that $\delta(H/K)=0$ and $d(H/N)=d(K/N)=\dfp(K_{1}/N_{1})$.

Now by saturation of $M$, iterating Remark \ref{indtypes} above, we first find a strong $\nla{2}$-subalgebra $\widetilde{K}$ of $M$
with $N\inn\widetilde{K}$ and $\widetilde{K}\simeq_{N}K$.

Secondly we consider the strong embedding $\widetilde{K}\into H$ and find -- by Proposition \ref{fincharssc} and the arguments
of the previous Proposition -- a finite $K^{\rm o}\zsu{}\widetilde{K}$ such that $H\simeq\am{H^{\rm o}}{K^{\rm o}}{\widetilde{K}}$ for a suitable finite $H^{\rm o}\inn H$ such that
$H=K+H^{\rm o}$.

With richness of $M$, find a strong embedding $\alpha$ of $H^{\rm o}$ into $M$ over $K^{\rm o}$.
Now since $\delta(\alpha(H^{\rm o})/\widetilde{K})=\delta(H^{\rm o}/\widetilde{K})
=\delta(H^{\rm o}/K^{\rm o})=0$, we obtained the desired strong embedding of $H$ into $M$ as $\gena{\widetilde{K}+\alpha(H^{\rm o})}
{M}$.
\end{proof}

\bigskip
The next paragraphs are devoted to describe the algebraic closure of sets
of $\mathbb{M}_{1}$.

First observe that axioms $\sig{2}{2}$ imply that $\aut(\mathbb{M})$ is $2$-transitive on $\mathbb{M}_{1}$ as a group
of $\Fp$-linear automorphisms, that is to say,
transitive on the set of linearly independent ordered pairs from $\mathbb{M}$.
In particular $\acl(\triv)=\triv$, and $\acl(a,b)=\gena{a,b}{\mathbb{M}}$ for any pair of elements $a,b\in\mathbb{M}_{1}$.

Now take a finite subspace $C_{1}$ of $\mathbb{M}_{1}$, since by saturation $d(\mathbb{M}_{1}/C)$ is infinite,
Remark \ref{indtypes} implies that for any , $\acl(C_{1})\cap\mathbb{M}_{1}$ is contained in $\cl_{d}(C_{1})$.

It is also straightforward to see that $A=\ssc(C)$ is contained in the algebraic closure of $C$:
if $A$ has infinitely many conjugates in $\mathbb{M}$ over $C$, then we can find a strong copy $A^{\prime}$ of $A$ such that
$C\inn A\cap A^{\prime}\subsetneq A$ but his contradicts minimality of self-sufficient closure.
With Lemma \ref{finchar}, we may also conclude that $\acl(C_{1})\cap\mathbb{M}_{1}$ is self-sufficient.
One has then
\begin{labeq}{inclusures}
ssc(C_{1})\zsu{}\acl(C_{1})\cap\mathbb{M}_{1}\zsu{}\cl_{d}(C_{1}).
\end{labeq}
As opposed to amalgamation constructions in relational languages, here the
self-sufficient closure does not equal algebraic closure (see \cite{wag}). On the other hand
in our theory $T^{2}$ the algebraic closure does not coincide with the
geometric closure $\cl_{d}$, as actually happens in the collapsed case.

\smallskip
We also have
\begin{labeq}{acluno}
\acl(C_{1})=\gena{\acl(C_{1})\cap\mathbb{M}_{1}}{\mathbb{M}}
\end{labeq}
for if
$C_{1}$ is not trivial, for any element $m=m_{1}+m_{2}$ of $\acl(C_{1})$, property $\sig{2}{4}$ implies
$m=m_{1}+[h,x]$ for some $h\in C_{1}$ and some $x$ in $\mathbb{M}_{1}$.

Now $m_{1}\in\acl(C_{1})\cap\mathbb{M}_{1}$ and $x$ is algebraic over $m_{1},h_{1}$, by axiom
$\sig{2}{2}$ (cfr.\,Remark \ref{tuttuno}).

\smallskip
We can actually fully characterise $\acl(C_{1})$ for a given $C_{1}\inn\mathbb{M}_{1}$, in terms of the divisor elements
defined in Remark \ref{divelement}.

Call a self-sufficient subalgebra $C$ of $\mathbb{M}$ {\em divisibly closed} if
whenever $\delta(a/C)=0$ for $a\in\mathbb{M}_{1}$, then $a\in C_{1}$.
%there is no divisor of $C$ in $\mathbb{M}_{1}$,
%which is linearly independent over $C_{1}$.

By the remarks above, $\acl(C_{1})$ is divisibly closed. Moreover if $U$ and $V$ are divisibly closed $\nla{2}$-algebras 
then $W=\gena{U_{1}\cap V_{1}}{\mathbb{M}}$ is also divisibly closed, for %as a consequence of axiom $\sig{2}{2}$. This is for,
if $\delta(x/U_{1}\cap V_{1})=0$ then
%there is $u\in C_{1}\cap C_{1}$ such that $[x,u]\in U_{2}$, hence there are 
$\delta(x/U)=\delta(x/V)=0$ and $x\in U_{1}\cap V_{1}$.

Since meet-closed classes give rise to closure operators, we let $\mathscr{D}_{C}$ denote the collection
of all subspaces $H_{1}$ containing $C_{1}$, which generate divisibly closed self-sufficient algebras in $\mathbb{M}$,
then set
$$\div(C_{1})=\bigcap\mathscr{D}_{C}$$
and consistently to our terminology $\div(C)=\gena{\div(C_{1})}{\mathbb{M}}$.

\begin{lem}\label{acldiv}
For any subspace $C_{1}$ of $\mathbb{M}_{1}$ we have
$$\acl(C_{1})%\cap\mathbb{M}_{1}
=\div(\ssc(C))$$
\end{lem}
\begin{proof}
%We can of course reduce ourself to the case in which $C$
Since $\ssc(C_{1})\inn\acl(C_{1})$, we may actually assume $C$ to be self-sufficient and finite.
As $\acl(C_{1})=\gena{\acl(C_{1})\cap\mathbb{M}_{1}}{\mathbb{M}}$ is divisibly closed,
it is enough to show that $\div(C_{1})$ contains $\acl(C_{1})\cap\mathbb{M}_{1}$.

Assume an element  $a$ of $\mathbb{M}_{1}$ is in $\acl(C_{1})$, let $A_{1}$ be $\ssc^{\mathbb{M}}(C_{1},a)$ and $B_{1}$ denote
$\ssc(C_{1},a)\cap\div(C_{1})$.
Suppose by contradiction $A$ is not included in $\div(C_{1})$, then by \pref{inclusures} we have a non-trivial finite strong extension $A$
of $B$ such that $d(A/B)=\delta(A/B)=0$.

Take distinct $B$-isomorphic copies $A=A^{1}$, $A^{2}$, \dots, $A^{n}$ of $A$ for $n<\omega$; set $\circledast^{0}_{B}A=B$ and
$\circledast^{1}_{B}A=A^{1}$. For all $1\leq n<\omega$, also define inductively
$\circledast^{n}_{B}A=\am{(\circledast^{n-1}_{B}A)}{B}{A^{n}}$.

Then $\circledast^{n}_{B}A$ is in $\Kl{2}$ for all $n$. This follows by Lemma \ref{amalsigma2}:
since $B$ is divisibly closed in $A^{n}$, there is no divisor of $B$ in $A_{1}$ to prevent the free amalgam of $\circledast^{n-1}_{B}A$
and $A^{n}$ over $B$ from satisfying property $\sig{2}{2}$.

Since $\mathbb{M}$ is $\Kl{2}$-rich, we can
strongly embed $\circledast^{n}_{B}A$ into $\mathbb{M}$ over $\circledast^{n-1}_{B}A$ for all $n<\omega$.
We obtain thus arbitrarily many distinct self-sufficient copies $A^{i}$ of $A$ over $B$ and hence infinitely many $C$-conjugates of
$A$ in $\mathbb{M}$ against algebraicity over $C$.
\end{proof}
By Definition \ref{mindecomp} now follows
\begin{rem}\label{aclssc}
Let $A$ finitely extend $B$ in $\mathbb{M}$. 
Assume $$B=B^{0}
\zsu{}B^{1}\zsu{}\dots\zsu{}B^{n}=A$$
is a minimal decomposition
of $A$ over $B$. Then $\acl(B_{1})\cap A_{1}=B^{k}_{1}$, for some $1\leq k\leq n$
such that $B^{i+1}\nni B^{i}$ is a minimal algebraic extension for all $i=1,\dots,k$
and $k$ is maximal with respect to this property.
\end{rem}
%\begin{proof}
%As observed before, $\acl(B_{1})$ is self-sufficient in $\mathbb{M}$,
%hence $\acl(B_{1})\cap A_{1}\zsu{}A_{1}$. By minimality of the extensions $B^{i}\nni B^{i-1}$
%then $\acl(B_{1})\cap A_{1}=B^{k}$ for some $k$. The rest of the claim follows by
%the previous proposition.
%\end{proof}
\subsection{Rank computations}\label{rango}
Recall from Section \ref{stab}, that between tuples $\bar a$ and small sets $\mathcal{A},\mathcal{B}$ of $\mathbb{M}$, forking independence relation $\ffin{\bar a}{\mathcal{B}}{\mathcal{A}}$ holds whenever $\mr(\bar a/\mathcal{B})=\mr(\bar a/\mathcal{AB})$.

\medskip
We will use the following notation and facts in the sequel. 
\medskip
\begin{rem}\label{tuttuno}{\ }
For a fixed non-trivial element $m$ of $\mathbb{M}_{1}$ we define:
\begin{align} %\tag{$\Gamma$} %_{m}$} %\label{derivem}
\lmap{\vartheta_{m}}{{\mathbb{M}_{1}}\times\mathbb{M}_{1}&}{\mathbb{M}}\label{tuttheta}\\\notag
(a_{1},a_{2})&\longmapsto a_{1}+[m,a_{2}]
\end{align}
for all $a_{1},a_{2}$ in $\mathbb{M}_{1}$. Note that $\theta_{m}$ is
a $\Fp$-linear (non-bilinear) morphism. We have
\begin{itemize}
\item[1.]For any fixed $m\neq\triv$ of $\mathbb{M}_{1}$ the map $\theta_{m}$ defined in above is surjective and 
its fibres are all isomorphic to $\Fp$.
\item[2.]For any tuple $\bar a$ of $\mathbb{M}$ and small sets $\mathcal{A},\mathcal{B}$ there exists
a tuple $\bar a^{\prime}$ of $\mathbb{M}_{1}$ and {\em subspaces} $A_{1}, B_{1}$ of $\mathbb{M}_{1}$ such that
$\ffin{\bar a}{\mathcal{B}}{\mathcal{A}}$ iff $\ffin{\bar a^{\prime}}{{B_{1}}}{{A_{1}}}$.
\item[3.]For any tuple $\bar a\inn\mathbb{M}$ and set $\mathcal{A}\inn\mathbb{M}$ there exists
an $\nla{2}$-subalgebra $A$ of $\mathbb{M}$ interalgebraic with $\mathcal{A}$ such that $\mr(\bar a/\mathcal{A})=\mr(\bar a/A)=\mr(\bar a/A_{1})$.
\end{itemize}
\end{rem}
\begin{proof}
Statement 1. follows by Axiom $\sig{2}{4}$ for the surjectivity, while by $\sig{2}{2}$ its fibres have all
size exactly $p$: as pointed out in Section \ref{deltadue}, $[m,u]=[m,v]$ implies necessarily $u-v$ linearly depends of $m$.

\medskip\noindent
For 2. choose an element $m$ in $\mathbb{M}_{1}$ with $\ffin{m}{\mathcal{B}}{\,\mathcal{A},\bar a}$.
The properties of the definable map $\vartheta_{m}$ above,
allow us to find subspaces $B_{1}\inn A_{1}\inn\mathbb{M}_{1}$ with $m\in B_{1}$ and a tuple $\bar a^{\prime}\inn\mathbb{M}_{1}$, such that
$B_{1}\inn\acl(m,\mathcal{B})$, $\mathcal{B}\inn\dcl(B_{1})$, $A_{1}\inn\acl(m,\mathcal{B},\mathcal{A})$, $\mathcal{A}\inn
\dcl(A)$, $\bar a^{\prime}\in\acl(m,\bar a)$ and $\bar a\inn\dcl(m,\bar a^{\prime})$.

Some forking calculus (Remark \ref{re:extrafking}) now yields
$$\ffin{\bar a}{\mathcal{B}}{\,\mathcal{A}}\iff\ffin{\bar a}{m,\mathcal{B}}{\,\mathcal{A}}\iff\ffin{\bar a^{\prime}}{B}{A}$$ and hence the desired equivalence.

\medskip\noindent
3. is proven by similar same arguments.
%assume we want to compute $\mr(a/\bf m)$ for $a$ in $\mathbb{M}_{1}$ and an arbitrary finite set ${\bf m}$ of $\mathbb{M}$.
%Assume $b$ is %a finite set of $\mathbb{M}_{1}\backslash\{\triv\}$ 
%non-zero element of $\mathbb{M}_{1}$ which is forking-independent of
%$a$ over ${\bf m}$. Now choose -- with $\sig{2}{4}$ -- a finite set ${\bf c}$ of $\mathbb{M}$ %of the same length of ${\bf m}$, such that
%such that for any $m\in{\bf m}$, $m_{1}\in{\bf c}$ and $m_{2}=[b,c]$ for some $c\in{\bf c}$ and such that all $c$ in ${\bf c}$ are employed in
%such a task. %If we denote by ${\bf m}_{1}$ the set of all $P_{1}$-components of elements in ${\bf m}$,
%$\sig{2}{2}$ now yields
%${\bf c}\inn\acl(b,{\bf m})$ and ${\bf m}\inn\dcl({\bf c},b)$ and this is enough to infer
%\begin{labeq}{forkuno}
%\mr(a/{\bf m})=\mr(a/b\,{\bf m})=\mr(a/b\,{\bf m}{\bf c})=\mr(a/b\,{\bf c}).
%\end{labeq}
\end{proof}
%{\center \rule{\textwidth}{0.5pt}}

With a fine description of types in $T_{2}$ {\em � la John B. Goode} (\cite{jbg}) it will be possible
to calculate Morley rank of $\mathbb{M}$.
\begin{teo}\label{titi}
$T^{2}$ has Morley Rank $\omega\cdot2$ and Morley degree $1$.
\end{teo}
The crucial step in the proof relies in the following proposition. With $S(B)$ for $B\inn\mathbb{M}$ we denote the union
of all $\ssp{n}{B}$ as $n$ ranges in $\omega$. %, while with $P(\bar x)$ we denote the formula
%$P_{1}(x_{1})\wedge\dots\wedge P_{1}(x_{n})$.
\begin{prop}\label{dizero}
Let $B$ a finite self-sufficient subalgebra of $\mathbb{M}$.

Consider the following set of types in $\ssp{}{B}$
\begin{labeq}{tpdizero}
{\mathfrak X}=\{\tp{\bar a}{B}\mid B\zsu{}\mathbb{M},\bar a\inn\mathbb{M}_{1}, d(\bar a/B)=0\},
\end{labeq}
then $\mr(p)$ and $U(p)$ are finite and coincide for all type $p$ in ${\mathfrak X}$.

The finite rank of $\tp{\bar a}{B}$ coincide with the
number of prealgebraic steps in a minimal decomposition of $\ssc(B,\bar a)$ over $B$.
\end{prop}
\begin{proof}
As the $\cl_{2}$-dimension over $B$ of a tuple $\bar a$ is an invariant of the type of $\bar a$ over $B$, the family ${\mathfrak X}$ is well defined. For a fixed $B$ and length $n$, $\ssp{n}{B}\cap{\mathfrak X}$ is a closed set. % in $\gen{P_{1}(\bar x)}$.

By Lemma \ref{isola} each type of $\mathfrak{X}$ is isolated. Moreover if $q\in\ssp{}{C}$ extends a type $p$ of $\mathfrak{X}$ over $B$,
for some finite set $C$ above $B$, then by $\ssc$ we find a finite strong subspace $D_{1}\zsu{}\mathbb{M}_{1}$ such that
$B\inn D$ and $D\inn\acl(C)$. Of course $d(\bar a/D)=0$ for any realisation $\bar a\inn\mathbb{M}_{1}$ of $q$. % and $D\inn\acl(C)$.

Therefore, up to algebraicity, the assumptions of Lemma \ref{RMU} with respect to $\mathfrak{X}$ are fulfilled. Morley rank
and $U$-rank do coincide on $\mathfrak{X}$.

\medskip
We are now to prove that types in $\mathfrak{X}$ have finite rank, to do this assume $d(\bar a/ B)=0$
for some tuple $\bar a$ of $\mathbb{M}_{1}$ and set $A$ equal to $\ssc(B,\bar a)$.

Assume first that $A$ is a {\em minimal extension} of $B$. Then since
$d(A/ B)=d(\bar a/ B)=0$, then $A$ is either algebraic or pre-algebraic over $B$.

In the former case, then clearly $\mr(\bar a/B)=0$. Next we show that the type
of a pre-algebraic extension $A$ of a self-sufficient algebra $B$ in $\mathbb{M}$,
is {\em minimal} in the sense that it admits a unique non-algebraic extension to
every set $C$ containing $B$, this is equivalent for such types to have Lascar rank $1$.
In our case, since Morley and Lascar rank coincide, these types are actually {\em strongly} minimal.

That $A$ isn't algebraic over $B$ is Lemma \ref{acldiv}.
We may then take without loss, a subspace $C_{1}$ of $\mathbb{M}_{1}$ containing $B_{1}$.
Since $A$ is minimal over $B$ and the intersection $A_{1}\cap\ssc(C_{1})$ is strong in $A_{1}$ after Lemma \ref{2cut},
then either $A$ {\em is contained} in $\ssc(C)$ -- hence algebraic over $C$ -- or $A_{1}\cap\ssc(C_{1})=B_{1}$.
In the latter case we have $0\leq\delta(A/\ssc(C))\leq\delta(A/B)=0$, which implies that $A$ and $\ssc(C)$
are in free composition over $B$ (Lemma \ref{freecomp}) and that $A+\ssc(C)$ is self-sufficient in $\mathbb{M}$
(Lemma \ref{samedelta2}).

Since by Proposition \ref{bafo} the isomorphism type of
$$\ssc(A+C)=\gena{A_{1}+\ssc(C_{1})}{M}\simeq\am{A}{B}{\ssc(C)}$$
fully determines the type of $A$ over $C$, this gives but only one non-algebraic type over $C$ extending
$\tp{A_{1}}{B}$. That is $\mrd(\bar a/B)=(1,1)$.

\smallskip
For the case in which $A$ is not minimal over $B$ let
$$B=A^{0}\zsu{}A^{1}\zsu{}\,\cdots\,\zsu{} A^{n}=A$$
be a minimal decomposition of $A$ over $B$ as in \pref{mindec}.

Since again $d(A^{i+1}/ A^{i})=0$ for each $i$, each section $A_{1}^{i+1}/ A_{1}^{i}$ is of algebraic
or pre-algebraic kind.

We may now use additivity of Lascar Rank (Fact \ref{uddi}) and obtain
\begin{align*}
\mr(\bar a/B)=U(\bar a/ B)=U(A/ B)
=U(A^{n}/ A^{n-1})+\,\dots\,+U(A^{1}/ A^{0}) %=\\
\end{align*}
and conclude $\mr(\bar a/ B)\leq n$.

\smallskip
We have shown that,
types $\tp{\bar a}{B}$ of tuples $\bar a$ of $\mathbb{M}_{1}$,
over a strong $B_{1}\inn\mathbb{M}_{1}$, such that
$d(\bar a/B)=0$, do have finite Morley Rank, and
this rank coincides with the number of pre-algebraic steps in a minimal decomposition
of $\ssc(B,\bar a)$ over $B$, which is a posteriori an invariant of types in ${\mathfrak X}$.
\end{proof}

\medskip
\begin{proofof}{Theorem \ref{titi}}

\noindent
({\sl 1$^{\text{\uline{st}}}$ Claim})\quad$\mr(\mathbb{M})=\mr(\mathbb{M}_{1}\times\mathbb{M}_{1})$
and $\md(\mathbb{M})\leq\md(\mathbb{M}_{1}\times\mathbb{M}_{1})$.

\smallskip
Considering the definable map $\vartheta_{m}$ of Remark \ref{tuttuno}.
With Fact \ref{ziemr}({\it 1.}) we obtain $\mr(\mathbb{M}_{1}\times\mathbb{M}_{1})=\mr(\mathbb{M})$. The statement
about degrees is also trivial.

\medskip\noindent
({\sl 2$^{\,\text{\uline{nd}}}$ Claim})\quad $\mrd(\mathbb{M}_{1})=(\omega,1)$.
\smallskip

The claim will follow by showing that there is a unique generic type
in the group $(\mathbb{M}_{1},+)$, such type having Morley rank $\omega$.

By the {\em finite} local character (cfr. Fact \ref{stableforking}) of non-forking in totally transcendental theories,
in order to compute the rank of types in $T^{2}$, it is enough to consider finite sets of parameters.

We can therefore restrict our analysis to the clopen sets
$$\ssp{{P}_{1}}{B}:=\{p\in\ssp{1}{B}\mid P_{1}(x)\in p\}$$
for finite %\mn{actually arbitrary}
sets of parameters $B$ in $\mathbb{M}$. 

Moreover by Remark \ref{tuttuno} and algebraicity of $\ssc$, the sets $B$ above may always be assumed to be
finite strong $\nla{2}$-subalgebras of $\mathbb{M}_{1}$.

By Remark \ref{indtypes}, all the elements of $\mathbb{M}_{1}$ which are $\cl_{2}$-independent of $B$
have all the same type over $B$, which we denote by $p_{B}$.

Denote by $\mathfrak{X}_{B}$ %=\{%p\in\ssp{{P}_{1}}{B}
the set of all types $\tp{m}{B}$ %\mid m\ind(m/ B)=0\,\text{whenever $m\sat p$}\}$,
of elements $m$ of $\mathbb{M}_{1}$ with $d(m/ B)=0$.

We have then
\begin{labeq}{stoned}
\ssp{P_{1}}{B}=\mathfrak{X}_{B}\cup\{p_{B}\}
\end{labeq}
and Morley rank of types in ${\mathfrak X}_{B}$ is finite by Proposition \ref{dizero}.

On the other hand, by Remark \ref{prealgchain} the rich model $\mathbb{M}$ can embed
arbitrarily long chains of prealgebraic extensions. This implies by Proposition \ref{dizero},
Morley rank of types in $\mathfrak{X}_{B}$ is not bounded.

\smallskip
As a result, we have $\mr(p_{B})\geq\omega$ and, by Remark \ref{belrango} for any formula $\psi(x)$ over $B$,
either $\mr(\psi(x))=0$ or $\mr(\neg\psi(x))=0$. Hence $\mr(p_{B})=\omega=\mr(\mathbb{M}_{1})$ for any strong finite $B$.

When $B=\triv$, the unique generic type $p_{\triv}$ in $\mathbb{M}_{1}$ over $\vac$ is the type of any non-trivial element, it follows
$\mathbb{M}_{1}$ is connected and the claim is proved.

In particular since Lascar rank is connected, then $U(p_{B})$ must also be equal to $\omega$.
That is for complete types in $P_{1}$ Lascar rank and Morley rank do coincide.

\bigskip\noindent
({\sl 3$^{\,\text{\uline{rd}}}$ Claim})\quad $\mrd(\mathbb{M}_{1}\times\mathbb{M}_{1})=(\omega\cdot2,1)$.

It suffices once again to discuss rank of types of couples of elements in $\mathbb{M}$ over finite strong subspaces.
Once again, we use $\cl_{2}$ dimension, to discern kind of types.

Let $B$ be a fixed finite
self-sufficient algebra in $\mathbb{M}$, for arbitrary elements
$a,b$ of $\mathbb{M}_{1}$, we have $d(a,b/B)=d(a/B,b)+d(b/B)\leq2$.

We may therefore assume without loss of generality, one of the following
three cases holds:
\begin{itemize}
\punto{1}$d(a,b/B)=2$
\punto{2}$d(a/B,b)=0$ and $d(b/B)=1$
\punto{3}$d(a,b/B)=0$
\end{itemize}
%Also, these three situations exhaust all kind of types of pairs in $\mathbb{M}_{1}$.

In the first case, $a$ and $b$ are in particular linearly independent over $B_{1}$ and
by Lemma \ref{samedelta2} we have $B_{1}\zsu{}\gen{B_{1},a}\zsu{}\gen{B_{1},a,b}\zsu{}\mathbb{M}_{1}$.
Remark \ref{indtypes} implies that such pairs have all the same type over $B$.
This type will be denoted $q_{B}$.

\smallskip
On the other hand, by Proposition \ref{dizero}, all types in (3) with $d(a,b/B)=0$
have finite -- unbounded -- Morley (=Lascar) rank.

\smallskip
Now we have to deal with case (2) Types. We show for such
types Morley rank is bounded by $\omega\cdot2$.

Let $\bar c$ a tuple in $\mathbb{M}_{1}$ such that $\ssc(B,a,b)=\gena{B_{1},a,b,\bar c}{\mathbb{M}}$, we have
$$
\mr(a,b/B)=\mr(a,b,\bar c/B)\leq\mr(\varphi(x,\bar y,z))
$$
where $\varphi(x,\bar y,b)$ describes the quantifier-free $\Lan{2}_{Bb}$-type of $A\defeq\ssc(B,a,b)$ like in
Corollary \ref{isola}. The variables $x,\bar y$ take the places of $a,\bar c$.

Moreover since by (2) $d(a,\bar c/B,b)=0$, let $\mr(a,\bar c/B,b)=r<\omega$. Note also that $\gen{B_{1},b}$ is strong,
and that $r$ coincides with the number of pre-algebraic extensions in a minimal decomposition of $A$ over $\gena{B_{1},b}{\mathbb{M}}$.

We want to apply Fact \ref{ziemr}.({\it 2.}) to the definable map $\map{\pi}{\mathscr{D}}{\mathscr{E}}$
where $\mathscr{D}$ denotes $\varphi(\mathbb{M}_{1})$ and $\mathscr{E}$ stands for $((\exists x\exists\bar y)\varphi)(\mathbb{M}_{1})$ and $\pi$ is just the projection $(x,\bar y,z)\mapsto z$.

By the second claim above $\mr(\mathscr{E})$ is at most $\omega$ and if $e$ is an element of $\mathscr{E}$,
we will prove $\mr(\pi^{-1}(e))=\mr(\varphi(x,\bar y,e))$ is smaller than $r$.

By Remark \ref{belrango} it suffices to prove $\mr(p(x,\bar y))\leq r$ for types $p$ in $\ssp{x,\bar y}{C}$ whenever
$p$ implies $\varphi(x,\bar y,e)$ and $C$ is a finite self-sufficient subalgebra of $\mathbb{M}$ which contains $B$ and $e$.

Let $u,\bar v$ realise $p(x,\bar y)$ over $C$ and let $U$ denote $\gena{B_{1},e,u,\bar v}{\mathbb{M}}$.

Now $\varphi(u,\bar v,e)$ witness $\gena{B_{1},e}{\mathbb{M}}\zsu{}U$ and a minimal decomposition %of $A$ over $\gena{B,e}{\mathbb{M}}$,
$$\gena{B_{1},e}{\mathbb{M}}\zsu{}U^{1}\zsu{}\cdots\zsu{}U^{n}=U$$
with at most $r$ pre-algebraic steps and $\delta(U^{i+1}/U^{i})=0$ for all $i$.

Since Lemma \ref{2cut} implies $C_{1}\cap U_{1}\zsu{}U_{1}$, then $C_{1}$ meets $U_{1}$ necessarily in some $U_{1}^{k}$ for
$1\leq k\leq n$. Moreover $\delta(U/C)\leq\delta(U/U^{k})=0$ and then $C+U$ is self-sufficient. This yields
$$\ssc(C,u,\bar v)=C+U=\am{U}{U^{k}}{C}$$ %\am{\gena{U_{1}^{k},u,\bar v}{\mathbb{M}}}{U^{k}}{C}$$
and $d(u,\bar v/C)=0$. Now by Lemma \ref{mindecamalg}
\begin{labeq}{decot}
C\zsu{}C+U^{k+1}\zsu{}\cdots\zsu{}C+U^{n}=C+U
\end{labeq}
is a minimal decomposition of $C+U$ over $C$ with
$$C+U^{i+1}\simeq\am{U^{i+1}}{U^{i}}{(C+U^{i})}$$
for all $i\geq k$ and where the minimal extension $C+U^{i+1}$ of $C+U^{i}$ is exactly
of the same type as $U^{i+1}$ over $U^{i}$. This means that the minimal decomposition \pref{decot} contains
at most $r$ pre-algebraic steps.

Since as observed before $\mr(u,\bar v/C)=\mr(\ssc(C,u,\bar v)/C)$ coincides
with the prealgebraic steps of a minimal decomposition between $C$ and $\ssc(C,u,\bar v)$, this rank is bounded
by $r$.
This yields $\mr(\varphi(x,\bar y,e))\leq r$ and Fact \ref{ziemr} now implies
$\mr(\phi(x,\bar y,z))\leq n\cdot(\omega+1)=n\cdot\omega+n=\omega+n$.

As already pointed out in the previous claim, on the other hand $\mathbb{M}$
is rich enough to embed diagrams with rank exactly $\omega+n$ with
unboundedly large $n<\omega$. As a result, by the same arguments we used above, the type $q_{B}$
is the unique generic of Morley rank $\omega\cdot2$.
It follows $\mr(\mathbb{M}_{1}\times\mathbb{M}_{1})=\omega\cdot2$.

\medskip
Putting the three claims together we obtain $\mrd(\mathbb{M})=(\omega\cdot2,1)$ and  the theorem is proven.
\end{proofof}
\begin{rem}
Let $\mathbb{G}=G(\mathbb{M})$ be the $\ngb{2}{p}$-group interpretable in $\mathbb{M}$ with Corollary \ref{co:interp}.
Then $\mathbb{G}$ is a connected $\omega$-stable group of Morley rank $\omega\cdot2$ with $Z(\mathbb{G})=\mathbb{G}^{\prime}$
and $\mr(\mathbb{G}_{\rm ab})=\mr(G^{\prime})=\omega$.
\end{rem}



\subsection{Characterisation of Forking Independence}\label{forking}
We conclude the chapter with a complete picture of forking in $T^{2}$ in terms of $\cl_{d}$-independence and free amalgamation.

\smallskip
Recall that the self-sufficient closure $\ssc^{\mathbb{M}}$
is defined on {\em sets} $S$ of $\mathbb{M}$, by composing $\ssc(\genp{S})$.

We now introduce a ternary relation among sets $A,B$ and tuples $\bar a$ of $\mathbb{M}_{1}$ 
$\fin{\bar a}{B}{A}$ which will turn out to be a {\em irreflexive independence relation} in the sense of \cite{ad}.
\begin{dfn}\label{indepdef}
For any tuple $\bar a$ of $\mathbb{M}_{1}$ and any small {\em sets} $A$ and $B$ of $\mathbb{M}_{1}$ {\em define}
$$
\fin{\bar a}{B}{A}\quad\text{if}\quad
\begin{cases}
d(\bar a/B)=d(\bar a/AB)\\
\ssc(B,\bar a)\cap\ssc(AB)\inn\acl(B).%\ssc(B_{1},\bar a)\cap
\end{cases}
$$
Each time $\fin{\bar a}{B}{A}$ holds, we say that $\bar a$ is {\em $\ind$-independent}
of $A$ {\em over} $B$.

We extend this relation to {\em $\nla{2}$-subalgebras} $A,B$ of $\mathbb{M}$ by writing
$\fin{\bar a}{B}{A}$ whenever $\fin{\bar a}{B_{1}}{A_{1}}$.
\end{dfn}

Notice that  $\ind$ satisfies {\small\sc Invariance} as defined in Section \ref{stab}, that is
for all $A,B$ and all $\sigma\in\aut(\mathbb{M})$, $\fin{\bar a}{B}{A}$ iff $\fin{\bar a^{\sigma}}{B^{\sigma}}{A^{\sigma}}$.

We forget for the moment that our theory is totally-transcendental and
prove indeed that the properties in Fact \ref{stableforking} are satisfied by $\ind$-independence and,
as a result, that $\ind$ is non-forking independence among tuples and sets of $\mathbb{M}_{1}$.
On the other hand, Remark \ref{tuttuno} imply that forking is witnessed entirely by tuples and sets of $\mathbb{M}_{1}$.

\smallskip
Note for the moment, that $\cl_{d}$-{\em independece} alone cannot coincide with non-forking, for assume
$B$ is a strong algebra in $\mathbb{M}$ and $\bar a$ is a tuple of $\mathbb{M}_{1}$, assume that
$\ssc(B,\bar a)$ decomposes into $B\zsu{}A\zsu{}\ssc(B,\bar a)$ where
$\ssc(B,\bar a)$ is minimal algebraic over $A$ and $A$ is a pre-algebraic extension of $B$, then $d(\bar a/A)=d(\bar a/B)=0$
but, as proven in Theorem \ref{titi},  $\mr(\bar a/A)=0<\mr(\bar a/B)=1$.

This is essentially the only obstruction, since in the collapsed structure
prealgebraic extensions are forced to be algebraic: and the geometric closure and the algebraic coincide.

\medskip
The following proposition shows that $\ind$-independence is expressible by means
of free composition of algebras.
\begin{prop}\label{forkingchar}
For any tuple $\bar a$ of $\mathbb{M}_{1}$ and subalgebras $A$, $B$ of $\mathbb{M}$ if $C$ denotes
$\gena{\ssc(B_{1},\bar a)\cap\ssc(A_{1}+B_{1})}{\mathbb{M}}$, then the following holds:
\begin{itemize}
\punto{i.}$d(\bar a/B)=d(\bar a/AB)$ implies $\ssc(A+B,\bar a)\simeq\am{\ssc(B,\bar a)}{C}{\ssc(A+B)}$
\punto{ii.}assume both that $\ssc(A+B,\bar a)\simeq\am{\ssc(B,\bar a)}{C}{\ssc(A+B)}$ and
%$\ssc(B_{1},\bar a)\cap\ssc(B_{1}+A_{1})
$C_{1}\inn\acl(B_{1})$ hold, then $\fin{\bar a}{B}{A}$.
%\ssc(B_{1},\bar a)\cap
%$\ssc(A,\bar a)$ is the sum of $\ssc(B,\bar a)$ and $\ssc(A+B)$ and 
%these two algebras are in free composition inside $\mathbb{M}$, in symbols
%$$\ssc(A,\bar a)\simeq{\ssc(B,\bar a)}\!\!\underset{\ssc(B,\bar a)\cap\ssc(A+B)}{\circledast} \!\!\!\ssc(A+B)$$
\end{itemize}
\end{prop}
\begin{proof}
(i.) For sake of simplicity we will assume,
that $B\zsu{}A\zsu{}\mathbb{M}$. Let also $C_{1}$ denote $\ssc(B_{1},\bar a)\cap A_{1}$. Then we have to show
$\ssc(A,\bar a)\simeq\am{\ssc(B,\bar a)}{C}{A}$ whenever $d(\bar a/A)=d(\bar a/B)$.

\smallskip
We first prove that $\delta_{2}(\ssc(B,\bar a)/C)=\delta_{2}(\ssc(B,\bar a)/A )$.
We have in fact, by Lemma \ref{2transmogrifer} and \ref{fincharssc}.{}(i),(iii)
\begin{multline}\label{diseq}
d(\bar a/ A_{1})
	\leq d(\ssc(B_{1},\bar a)/A_{1})\leq\\
   \leq\delta_{2}(\ssc(B_{1},\bar a)/A_{1})\leq\delta_{2}(\ssc(B_{1},\bar a)/C_{1})=\\
=\delta_{2}(\ssc(C_{1},\bar a)/C_{1})=d(\bar a/ C_{1})\leq d(\bar a/ B_{1}).
\end{multline}

This implies on one side, by the hypothesis and Lemma $\ref{freecomp}$
that $\ssc(B,\bar a)$ is in free composition with $A$ (over $C$). This means just
that $\ssc(B,\bar a)+A=\gena{\ssc(B_{1},\bar a)+A_{1}}{\mathbb{M}}\simeq \am{\ssc(B,\bar a)}{B}{A}$.

\smallskip
On the other hand, since by \pref{diseq},
$d(\ssc(B_{1},\bar a)/A_{1})=\delta_{2}(\ssc(B_{1},\bar a)/A_{1})$,
Lemma \ref{fincharssc}.{}(iv) again, implies that $\ssc(B_{1},\bar a)+A_{1}$ is self-sufficient in $\mathbb{M}_{1}$, and
therefore $\ssc(B_{1},\bar a)+A_{1}=\ssc(A_{1},\bar a)$.

\medskip\noindent
(ii.) Assume now $\ssc(A,\bar a)\simeq\am{\ssc(B,\bar a)}{C}{A}$.
Then in particular %$\ssc(A_{1},\bar a)\simeq\vam{\ssc(B_{1},\bar a)}{B_{1}}{A_{1}}$, hence
%$\ssc(B_{1},\bar a)\cap A_{1}=B_{1}$. Moreover
$\ssc(B_{1},\bar a)+A_{1}=\ssc(A_{1},\bar a)\zsu{}\mathbb{M}_{1}$. The second hypothesis in (ii.) implies, with Remark
\ref{aclssc}, $\delta_{2}(\ssc(B_{1},\bar a)/B_{1})=\delta_{2}(\ssc(B_{1},\bar a)/C_{1})$, hence by
Lemmas \ref{freecomp} and \ref{fincharssc}.{}(iv) we have %inequality \pref{diseq} above, we deduce
\begin{multline*}
d(\bar a/B_{1})=\delta_{2}(\ssc_{2}(B_{1},\bar a)/B_{1})=\\
=\delta_{2}(\ssc_{2}(B_{1},\bar a)/C_{1})=\delta_{2}(\ssc_{2}(B_{1},\bar a)/A_{1})=\\
=d(\ssc_{2}(B_{1},\bar a)/A_{1})=d(\bar a/ A).
%=\delta_{2}(\ssc_{2}(B_{1},m)/ A_{1})=\delta_{2}(\ssc_{2}(A_{1},\bar a)/ A)=
%d(\bar a/A_{1})$ and hence $\fin{\bar a}{B}{A}$.
\end{multline*}
\end{proof}
\begin{rem}
It is clear that we may extend the $\ind$-relation to sets or subalgebras in the first entry by defining
$\fin{A}{B}{C}$ to hold, whenever all tuples $\bar a$ from $A$ are $\ind$-independent of $C$ over $B$.

By 
\end{rem}

\medskip
The correspondence between the relation $\ind$ and free amalgams allows us to prove
that $\ind$-independence satisfies a {\em finite} instance of {\small\sc Boundedness} property, described
in section \ref{stab}. This is shown in the next proposition, along with {\em finite} {\small\sc Local Character} for $\ind$ (cfr.{\ }Fact \ref{stableforking}).
%, as $\omega$-stability of $T^{2}$ actually requires 

\medskip
We call a type $p\in\ssp{}{A}$ with $A\nni B$, a $\ind$-{\em independendent} extension of $\res{p}{B}$ if for any $\bar a$ realising $p$ over $A$,
$\fin{\bar a}{B}{A}$ holds.

\begin{lem}\label{finitelcb}
For any $\nla{2}$-subalgebra $B$ of $\mathbb{M}$ and $\bar a$ in $\mathbb{M}_{1}$.
\begin{itemize}
\punto{i.}There is a finite subset $B^{\sss 0}\inn B$
such that $\fin{\bar a}{B^{\sss 0}}{B}$.
%Also $B^{\sss 0}$ is such that $\tp{\bar a}{B}$ is the unique $\ind$-independent
%extension of $\tp{\bar a}{B^{\sss 0}}$ to $B$,
\punto{ii.}For any $A\nni B$, there are at most finitely many distinct orbits under $\aut_{A}(\mathbb{M})$
in the set of all $\bar a^{\prime}$ with $\fin{\bar a^{\prime}}{B}{A}$ and $\bar a^{\prime}\equiv_{B}\bar a$. That is, at most finitely many
$\ind$-independent extensions of $\tp{\bar a}{B}$ to $A$.
\end{itemize}
\end{lem}
\begin{proof}
(i.) That $\ind$ satisfies (i.) is precisely statement (v) of Lemma \ref{fincharssc}.

\smallskip
(ii.) Assume
$\fin{\bar a^{\prime}}{B}{A}$, for some $\bar a^{\prime}$ in $\mathbb{M}$. This gives by Proposition \ref{forkingchar}
that $\ssc(A,\bar a^{\prime})\simeq\am{\ssc(B,\bar a^{\prime})}{C}{\ssc(A)}$,
where $C_{1}=\ssc(B_{1},\bar a^{\prime})\cap\ssc(A_{1})\inn\acl(B_{1})$.

On the other hand  $\tp{\bar a^{\prime}}{B}=
\tp{\bar a}{B}$ implies with Proposition \ref{bafo}, that $\ssc(B,\bar a^{\prime})\simeq\ssc(B,\bar a)$ via an isomorphism which maps
$\bar a$ onto $\bar a^{\prime}$ and fixes $B$.

This implies that the quantifier-free type of $\ssc(A,\bar a^{\prime})$, depends only on the choices for the subspace $C_{1}$
between $B_{1}$ and $\ssc(B_{1},\bar a^{\prime})\cap\acl(B_{1})$, and these are just in a finite number.
%More precisely,
%Corollary \ref{aclssc} implies the choices for such a $C_{1}$ are restricted to $B^{i}_{1}$ for $1\leq i\leq k$, where
%$B=B^{0}\zsu{}B^{1}\zsu{}\dots\zsu{}B^{k}$ are the first consecutive minimal algebraic steps
%of any minimal decomposition $(B^{i})_{i\leq n}$ of $\ssc(B,\bar a)$ over $B$.
 
Proposition \ref{bafo} again, give only finitely many representatives $\bar a^{\prime}$ in $\mathbb{M}_{1}$
modulo $A$-conjugacy, which are $\ind$-independent of $A$ over $B$.
\end{proof}

%Assume $B\zsu{}A\zsu{}\mathbb{M}$ and $\bar a$ is a tuple $\ind$-independent of $A$ over $B$ and 
%$B=B^{0}\zsu{}B^{1}\zsu{}\dots\zsu{}B^{n}=\ssc(B,\bar a)$ is a minimal decomposition
%of $\ssc(B,\bar a)$ over $B$. Observe first, that by minimality  $A_{1}$ necessarily cuts
%$\ssc(B,\bar a)$ in $B^{k}$ for some $k\leq n$. Now $\fin{\bar a}{B}{A}$ and Corollary \ref{aclssc} imply that
%for all $i$ with $0\leq i\leq k$, the extension $B^{i}_{1}/B^{i-1}_{1}$ is of {\em algebraic} type.
%Thus, by the same argument of Proposition \ref{finitelcb} follows that
As expected, a type over a algebraically closed set $B$ is $\ind$-{\em stationary}
in the sense of the following 
\begin{cor}\label{stationary}
Assume a tuple $\bar a\inn \mathbb{M}_{1}$ and a subalgebra $B$ are given.
A type $p=\tp{\bar a}{B}$ is $\ind$-{\em stationary} {\rm(that is, it admits
a unique $\ind$-independent extension $q$ to any algebra $A\nni B$ of $\mathbb{M}$)} % with $\fin{\bar a}{B}{A}$)}, if and only
whenever $\ssc(B_{1},\bar a)\cap\acl(B_{1})=B_{1}$.
\end{cor}

\medskip
We prove the remaining non-forking properties in the following proposition.
\begin{prop}\label{indepnotion}
The relation $\fin{}{}{}$ introduced by Definition \ref{indepdef} satisfies the following properties
\begin{description}
%\item{\small{\sc (Transitivity)}}
\punto{\small\sc Transitivity}
for all $C\inn B\inn A$, from $\fin{\bar a}{C}{B}$ and $\fin{\bar a}{B}{A}$ follows $\fin{\bar a}{C}{A}$,
\punto{\small\sc %Weak
Monotony}if $\fin{\bar a}{C}{A}$ and $C\inn B\inn A$ then $\fin{\bar a}{C}{B}$.
\punto{\small\sc Existence}for any $\bar a$ and $B\nni C$ there exists a tuple $\bar a^{\prime}$ in $\mathbb{M}_{1}$ with $\bar a^{\prime}\equiv_{C}\bar a$
such that $\fin{\bar a^{\prime}}{C}{B}$.
\end{description}
\end{prop}
\begin{proof}
To prove {\small\sc Transitivity}, let $A\nni B\nni C$ be (strong) subalgebras of $\mathbb{M}$, and $\bar a$ is a tuple of $\mathbb{M}_{1}$ with both $\fin{\bar a}{B}{A}$ and $\fin{\bar a}{C}{B}$. This gives $d(\bar a/A)=d(\bar a/B)=d(\bar a/C)$ and hence we have to
show $\ssc(C_{1},\bar a)\cap A_{1}\inn\acl(C_{1})$.

Since $\bar a$ is $\ind$-independent of $A$ over $B$
$$\ssc(C_{1},\bar a)\cap A_{1}=\ssc(C_{1},\bar a)\cap\ssc(B_{1},\bar a)\cap A_{1}\inn\ssc(C_{1},\bar a)
%\cap\ssc(B_{1},\bar a)
\cap\acl(B_{1}).$$
If $D_{1}$ denotes $\ssc(C_{1},\bar a)\cap B_{1}$, then $\ssc(C_{1},\bar a)=\ssc(D_{1},\bar a)$ and $\fin{\bar a}{C}{B}$
implies $D_{1}\inn\acl(C_{1})$ and
with Proposition \ref{forkingchar}, that $\ssc(D,\bar a)$ is in free composition with $B$ over $D$,
moreover $\ssc(B,\bar a)\simeq\am{\ssc(D,\bar a)}{D}{B}$.

Lemma \ref{mindecamalg} and Remark \ref{aclssc} now imply $\acl(B_{1})\cap\ssc(B_{1},\bar a)$ is forced to meet any minimal decomposition
of $\ssc(D,\bar a)$ over $D$ necessarily within the first adjacent minimal algebraic extensions of $D$.

This means $\ssc(D_{1},\bar a)\cap\acl(B_{1})%\cap\ssc(B_{1},\bar a)
\inn\acl(D_{1})$ and since $D_{1}\inn\acl(C_{1})$, we may
conclude $\ssc(C_{1},\bar a)\cap A_{1}\inn\acl(C_{1})$ as desired.

\medskip
While {\small\sc %Weak
Monotony} is trivial to prove, to show {\small\sc Existence} let $\bar a$ be a tuple and
$B\nni C$ algebras, which might -- without loss of generality be assumed strong in $\mathbb{M}$. Denote
by $A$ the self-sufficient closure $\ssc(C,\bar a)$.

By collecting all divisors\footnote{
or simply by taking the relative algebraic closure $\acl(C_{1})\cap A_{1}$}
of $C$ in $A_{1}$ which are realised (cfr.\,Definition \ref{divelement}) in $B$, we can find $\widetilde{C}$, such that
$C\inn\widetilde{C}\inn\acl(C_{1})$, $A\zso{}\widetilde{C}\zsu{}B$ and there is no divisor of $\widetilde{C}$ in $A_{1}$ which
is realised in $B$. %It also follows $A=\ssc(\widetilde{C},\bar a)$.

Take an isomorphic copy $\widetilde{A}$ of $A$ and denote by $\tilde a$ the
image of $\bar a$ inside $\widetilde{A}$. Now denote by $\widetilde{B}$, the free amalgam $\am{B}{\widetilde{C}}{\widetilde{A}}$ of $B$ and $\widetilde{A}$ over $\widetilde{C}$.
By Lemma \ref{amalsigma2} follows, that $\widetilde{B}$ inherits $\sig{2}{2}$ and hence $\widetilde{B}$ is a finite strong
extension of $B$ which is in $\Klt{2}$. By richness of $\mathbb{M}$ after Remark \ref{tildarich} we can find an embedding $\sigma$
of $\widetilde{B}$ into $\mathbb{M}$ over $B$ such that $\widetilde{B}^{\sigma}\zsu{}\mathbb{M}$.

Now since $\widetilde{A}\zsu{}\widetilde{B}$, $\widetilde{A}$ coincides with $\ssc^{\widetilde{B}}(C,\tilde a)$ and hence
if $\bar a^{\prime}$ denotes the image in $\mathbb{M}_{1}$ of $\tilde a$ under $\sigma$, we have
$\widetilde{A}^{\sigma}=\ssc^{\mathbb{M}}(C,\bar a^{\prime})$.

On the other hand $\widetilde{B}$ must coincide with the the self-sufficient closure (in $\widetilde{B}$) of $B$ and $\tilde a$.
%$\ssc^{\widetilde{B}}(B,\tilde a)$
This gives analogously
$\widetilde{B}^{\sigma}=\ssc^{\mathbb{M}}(B,\bar a^{\prime})\simeq\am{B}{\widetilde{C}}{\ssc^{\mathbb{M}}(C,\bar a^{\prime})}$.
With (ii.)\,of Proposition \ref{forkingchar} we obtain $\fin{\bar a^{\prime}}{C}{B}$. Then of course $\bar a^{\prime}\equiv_{C}\bar a$, since
$\ssc^{\mathbb{M}}(C,\bar a^{\prime})\simeq\widetilde{A}\simeq\ssc^{\mathbb{M}}(C,\bar a)$.
\end{proof}

\medskip
Putting Lemma \ref{finitelcb} and Proposition \ref{indepnotion} together with Fact \ref{stableforking} we reobtain
($\omega$-) stability of $T_{2}$ and in particular:
\begin{cor}
On the sets of $\mathbb{M}_{1}$ forking independence and $\ind$-independence coincide.
This is $$\fin{\bar a}{B}{A}\iff\ffin{\bar a}{B}{A}$$
for all $\bar a,A,B\inn\mathbb{M}_{1}$.
\end{cor}
\begin{rem*}
For any given {\em set} $B$ in $\mathbb{M}_{1}$, the forking geometry on the generic type $p=p_{\ssc(B)}$ in $\mathbb{M}_{1}$
over $\ssc(B)$ defined
in Theorem \ref{titi}, is exactly the $\cl_{d}$-geometry. 
That is to say, for any $a,\bar b$ in $p(\mathbb{M}_{1}\!)$ we have
$$a\in\cl_{d}(\bar b/B) \iff\nfin{a}{B}{\bar b}$$
\end{rem*}
\begin{proof}
By Remark \ref{cielle2} and the definition of $\ind$, we may assume $B$ is a self-sufficient subspace of $\mathbb{M}_{1}$.

For the left-to-right implication, a routine induction argument with monotonicity of forking let us assume $\bar b$ is a singleton $b\in p_{B}(\mathbb{M}_{1})$. Now $a\in\cl_{d}(B,b)\non\cl_{d}(B)$ gives -- by exchange -- $b\in\cl_{d}(B,a)$, hence
$d(a/B)=1>0=d(a/B,\bar b)$.

\medskip
Now if $\nfin{a}{B}{\bar b}$, then either $d(a/B,\bar b)=0$ or $\ssc(B,a)=\genp{B,a}\inn\ssc(B,\bar b)\inn\cl_{d}(B,\bar b)$.
\end{proof}
\subsection{Around weak elimination of imaginaries and $CM$-Triviality}\label{cmt}
We conclude this chapter with some results about identifying
canonical bases for types over models of $T^{2}$.

In this section, a distinguished non-trivial element of $\mathbb{M}_{1}$ will be added to the signature as parameter.
With Remark \ref{tuttuno}, this implies we can actually work with tuples from $\mathbb{M}_{1}$ only.

The result which follows
provides a notion of {\em weak} canonical base for types of self-sufficient tuples over models
and inspired a program about showing $CM$-triviality for $T^{2}$. This is now
a work in progress which is not accomplished in this thesis.

\medskip
For any strong $\nla{2}$-subalgebra $H$ of $\mathbb{M}$, we call a tuple $\bar a\inn\mathbb{M}_{1}$ a {\em strong tuple over $H$},
if $\bar a$ is linearly independent of $H_{1}$ and if $\genp{H_{1},\bar a}\zsu{}\mathbb{M}_{1}$. For the
definition of canonical bases we refer to Section \ref{stab}.

\begin{lem}\label{teowei}
We fix a constant $c$ to the language $\Lan{2}$, interpreted by an element of $\mathbb{M}_{1}$ different from $\triv$.

For any model $M$ of $T^{2}_{c}$, and any strong tuple $\bar a$ over $M$,
there is a finite strong $\nla{2}$-subalgebra $C$ of $M$,
such that
$$C\inn\acl^{eq}(Cb(\bar a/M))\text{ and }Cb(\bar a/M)\inn\dcl^{eq}(C).$$
\end{lem}
\begin{proof}
Let $p$ denote $\tp{\bar a}{M}$ and $\bar a$ be $\tupl{a}{0}{n-1}$.

Let $\aut_{\{\!M\!\}}(\mathbb{M})$ denote the group of all automorphism of the monster $\mathbb{M}$,
which leave $M$ invariant.

If $\sigma$ is an automorphism of $\mathbb{M}$, whose restriction to ${M}$ fixes the type $p$,
then $\bar a^{\sigma}\equiv_{M}\bar a$.
%This means, by Proposition \ref{bafo}, that the application fixing $M$ and sending $\bar m$ to $\bar m^{\sigma}$
%may be extended to an isomorphism $\phi$ of $\gena{M,\bar a}{\mathbb{M}}$ onto $\ssc(M,\bar m^{\sigma})=
%%:\gena{M,\bar b}{\mathbb{M}}$.
%\gena{M,\bar a^{\sigma}}{\mathbb{M}}=:\gena{M,\bar b}{\mathbb{M}}$\mn{Careful! $\bar a^{\phi}$ {\bf may not} be $\bar a^{\sigma}$!!!}
%for some $\bar b\inn\mathbb{M}_{1}$ with $\phi\colon \bar a\stackrel[M]{\simeq}{\longmapsto}\bar b$.
%
%Proposition \ref{bafo} again now implies that $\phi$ is actually elementary, that is $\phi\colon\bar a
%\stackrel[M]{\equiv}{\longmapsto}
%\bar b$.
Since $M$ is small, the strong homogeneity of the monster implies, that
the action {\em on} $M$ of the stabiliser of the type $p$ %with respect of the action of
in $\aut(M)$ %on $\ssp{}{M}$
coincides with the action on $M$ %orbit of $\bar c$ %$C$
under the pointwise stabiliser of %$\bar m$ {\em and } %$\bar a$ (and 
$\bar a$ in $\aut_{\{\!M\!\}}(\mathbb{M})$. % which we denote by $\aut_{\{\!M\!\}}(\mathbb{M})$.%_{\bar m}$.
We will first find a finite subspace of $M$ which is invariant under all $\sigma\in\aut_{\{\!M\!\}}(\mathbb{M})$ with
$\bar a^{\sigma}=\bar a$.

%On the other hand if $\sigma\in\aut_{\{\!M\!\}}(\mathbb{M})$ and fixes $\bar m$, then $\gen{M_{1},\bar a}^{\sigma}=
%\ssc(M_{1},\bar m)^{\sigma}=\ssc(M_{1},\bar m^{\sigma})=\gen{M_{1},\bar a}$ and hence $\gen{\bar a}^{\sigma}=\gen{\bar a}$.

%Since the self-sufficient closure commutes with automorphisms,
%one has
%$$\gena{M,\bar a^{\sigma}}{\mathbb{M}}=\ssc(M,\bar m)^{\sigma}=\ssc(M^{\sigma},\bar m^{\sigma})%{\gena{M_{1},\bar a}{\mathbb{M}}}^{\sigma}=\gena{M_{1},\bar a^{\sigma_{1}}}{\mathbb{M}}
%=\ssc(M,\bar m)=\gena{M,\bar a}{\mathbb{M}}$$
%and in particular $\gen{\bar a^{\sigma}}=\gen{\bar a}$ in $\mathbb{M}_{1}$.

%For any automorphism $\sigma$ of $\aut_{\{\!M\!\}}(\mathbb{M})$ %fixes $\bar a$ pointwise and let
%let $\sigma_{1}$ denote the restriction of $\sigma$ to $M_{1}$: $\sigma_{1}$ is a linear isomorphism in
%$\mathit{GL}(\mathbb{M}_{1})$ which leaves the subspace $M_{1}$ invariant.
%Let also $\widehat\sigma$ denote the graded Lie isomorphism induced by $\sigma_{1}$ on the free graded algebra
%$\fla{2}{\mathbb{M}_{1}}=\mathbb{M}_{1}\oplus\exs\mathbb{M}_{1}$.
%Remark that $\widehat\sigma$ leave the
%subspace $\exs\gen{M_{1},\bar a}$ of $\exs\mathbb{M}_{1}$ invariant.

\medskip
Let %$\mathcal{B}$ an ordered basis of $M_{1}$ and
$(\rho_{i})_{i<m}$ be a set in $\exs\genp{M_{1},\bar a}$ linearly independent over $\exs M_{1}$ which is a basis
of $\rd_{\mathbb{M}}(M,\bar a)$ over $\rd(M)$. Since $M$ is self-sufficient, then $m\leq n$.%\dfp(\bar a/M_{1})$.
%After having expressed each $\rho_{i}$ as linear combinations of basic monomials
%over the ordered basis $\mathcal{A}=\{\bar a>\mathcal{B}\}$ of $\gen{M_{1},\bar a}$, we may assume that

\smallskip
For all $i<m$, we find a tuple $\bar b_{i}=(\utpl{b_{i}}{0}{n-1})$ of length $n$, of not necessarily linearly independent elements of $M_{1}$
such that 
\begin{labeq}{rhodi}
\rho_{i}=\alpha_{i}+\beta_{i}+\gamma_{i}
%\rho_{i}=\alpha_{i}(\bar a)+\beta_{i}(\bar a,\bar c)+\gamma_{i}(\bar c)
\end{labeq}
%where $\alpha_{i}\in\exs\gen{\bar a}$, $\beta_{i}\in\exs\gen{\bar a,\bar c}\non\exs\gen{\bar a}+\exs\gen{\bar c}$ and
%$\gamma_{i}\in\exs\gen{\bar c}\non\rd(M)$ are supposed to be
%where the terms
%$\alpha_{i}$, $\beta_{i}$ and $\gamma_{i}$ are linear combinations of basic monomials
%over the ordered set $\{\bar a>\bar c\}$ and the ordering on $\bar c$ is the one inherited from $\mathcal{B}$. We can also
%require
where each $\alpha_{i}=\alpha_{i}(\bar a)%\neq\triv <-------WHY??????
$ is in $\exs\genp{\bar a}$
and every $\gamma_{i}$ %(\bar c)\in
-- when nontrivial -- lays in $\exs M_{1}\non\rd(M)$ for all $i<m$.
%\sout{This yields\mn{Try: get the $c$'s right now}, since $M$ is a model,
%that we can find a $c_{i}\in M_{1}$, such that $[c,c_{i}]=\bar\gamma_{i}$ for all $i<m$. Here $\bar\gamma_{i}$ denotes
%the element of $M$ $\gamma_{i}+\rd(M)$.}
Moreover, all nontrivial $\beta_{i}$ are given by
$$\beta_{i}=\beta(\bar a,\bar b_{i})=\sum_{k<n}[a_{k},{b_{i}}^{k}]\in\exs\genp{M_{1},\bar a}\non\exs\genp{\bar a}+\exs M_{1}.$$

\smallskip
%Now take a tuple $\bar b$ of $\mathcal{B}$ %be a tuple of $M$,
%minimal with the properties:
%\begin{itemize}
%\item[-]$\{\bar a,\bar b\}\nni\supp_{\mathcal{A}}(\beta_{i})$ for all $i$ and
%\item[-]$\gen{\bar a,\bar b}\nni\bar m$.\mn{or, embed this feature somewherelse!!!}
%\end{itemize}


\smallskip
Let now $\sigma$ be an automorphism  in
$\aut_{\{\!M\!\}}(\mathbb{M})$ which fixes %$\bar a$ and
$\bar a$ pointwise. Let $\widehat\sigma$ be the graded Lie isomorphism induced by $\res{\sigma}{\mathbb{M}_{1}}$ on the free graded algebra $\fla{2}{\mathbb{M}_{1}}=\mathbb{M}_{1}\oplus\exs\mathbb{M}_{1}$ like in Lemma \ref{commufreeno}.

With this notation, as $\sigma(\gena{M,\bar a}{\mathbb{M}})=\gena{M,\bar a}{\mathbb{M}}$, it follows $\widehat\sigma\left(\rd(M_{1},\bar a)\right)=\rd(M_{1},\bar a)$.
Moreover since $(\rho_{i})_{i<m}$ is a basis of $\rd(\bar a/M)$, for all $i$ we have %,${\rho_{i}}^{\widehat\sigma}\in\rd(M,\bar a)$, it follows
\begin{labeq}{rhosig}
{\rho_{i}}^{\widehat\sigma}-\sum_{j<m} s_{j}\rho_{j}=\mu
\end{labeq}
%for all $i$ and
for some $\mu$ in $\exs M_{1}$ and $s_{j}$ in $\Fp$. On the other side
$$
{\rho_{i}}^{\widehat\sigma}=\alpha_{i}(\bar a)%^{\widehat\sigma}
+\beta(\bar a,{\bar b_{i}})^{\widehat\sigma}+\gamma_{i}^{\widehat\sigma}
=
\alpha_{i}(\bar a)+\beta(\bar a,{\bar b_{i}}^{\,\,\sigma})+\gamma_{i}^{\,\widehat\sigma}
$$
where
\begin{labeq}{betasig}
\beta(\bar a,{\bar b_{i}}^{\,\sigma})=\sum_{k<n}[a_{k},\sigma({b_{i}}^{k})]
\end{labeq}

Now \pref{rhosig} becomes
\begin{labeq}{rhomuc}
\alpha_{i}-\sum_{j<m}s_{j}\alpha_{j}+\beta_{i}%(\bar a,{\bar b_{i}}^{\,\sigma})
^{\,\widehat\sigma}-\sum_{j<m}s_{j}\beta_{j}%(\bar a,{\bar b_{j}})
=\mu-\gamma_{i}^{\,\widehat\sigma}+\sum_{j<m}s_{j}\gamma_{j}\:\in\exs M_{1}
\end{labeq}
and since $\genp{\alpha_{i},\beta_{i},\beta_{i}^{\,\widehat\sigma}\mid i<m}\cap\exs M_{1}=\triv$, one has
$$\alpha_{i}+\beta(\bar a,{\bar b_{i}})^{\widehat\sigma}=\sum_{j<m}
s_{j}(\alpha_{j}+\beta(\bar a,{\bar b_{j}})).$$ Now by the same arguments, we get
$$\beta(\bar a,{\bar b_{i}}^{\,\sigma})=\sum_{j<m}s_{j}\beta(\bar a,{\bar b_{j}}).$$

Hence in particular
%On the other hand, since $\gen{\bar a}=\gen{\bar a^{\,\sigma}}$, we have $a_{k}=\sum_{l<n}r_{k}^{l}\sigma(a_{l})$ for
%some $r_{k}^{l}$ in $\Fp$ and
\begin{labeq}{beta}
%\beta_{i}(\bar a^{\,\sigma},{\bar b_{i}}^{\,\sigma})
\sum_{k<n}[a_{k},\sigma({b_{i}}^{k})]=
\sum_{j<m}s_{j}\beta(\bar a,{\bar b_{j}})=\sum_{k<n}[a_{k},\sum_{j<m}s_{j}{b_{j}}^{k}].
%=\sum_{l<n}[\sigma(a_{l}),\sum_{\substack{j<m\\k<n}}r_{k}^{l}s_{j}{b_{j}}^{k}].
\end{labeq}

Now by Fact \ref{ubc},
in $\exs\mathbb{M}_{1}$ we have $[a_{k},{M}_{1}]\cap\sum_{j\neq k}[a_{j},{M}_{1}]=\triv$.
%and hence $\exs\gen{M_{1},\bar a}=\bigoplus_{k<n}[a_{k},{M}_{1}]$.
Therefore \pref{betasig} and \pref{beta} imply for all $k$, $[a_{k},\sigma({b_{i}}^{k})]=
\sum_{k<n}[a_{k},\sum_{j<m}s_{j}{b_{j}}^{k}]$
in $\exs\mathbb{M}_{1}$.

This yields $\sigma({b_{i}}^{k})=\sum_{j<m}s_{j}{b_{j}}^{k}$ and hence
$\sigma(\genp{\bar b_{i}\mid i<m})\inn\genp{\bar b_{i}\mid i<m}$.

If we denote by $\bar b$ the tuple $\tupl{\bar b}{0}{m}$,
%since $\sigma$ fixes $\bar m$, we may also assume -- by making $\bar b$ bigger if necessary --
%that $\bar m\inn\gen{\bar a,\bar b}$ and again
then $\bar b\inn M_{1}$ and $(\genp{\bar b})^{\sigma}\inn\genp{\bar b}$.

\medskip
Let now $\Gamma_{i}$ denote the image of $\gamma_{i}$ in $M_{2}$ modulo $\rd(M)$, that is
$\Gamma_{i}=\gamma_{i}+\rd(M)$ for all $i$. Since, by \pref{rhosig} and \pref{rhomuc} we have
$$
\rd(M)\ni{\rho_{i}}^{\widehat\sigma}-\sum_{j<m} s_{j}\rho_{j}=\gamma_{i}^{\,\widehat\sigma}-\sum_{j<m}s_{j}\gamma_{j},$$
we deduce
$${\Gamma_{i}}^{\sigma}=\gamma_{i}^{\,\widehat\sigma}+\rd(M)=\sum s_{j}\gamma_{j}
+\rd(M)=\sum s_{j}\Gamma_{j}$$
and hence, if $\overline{\Gamma}$ denotes the tuple $(\tupl{\Gamma}{0}{m})$,
we get $\gen{\overline{\Gamma}}^{\sigma}\inn\gen{\overline{\Gamma}}$.

\smallskip
Till now we have shown, that for any automorphism $\sigma$ of $M$,
if $\sigma$ fixes the type $p$, then $\sigma$ leaves both spaces $\genp{\bar b}$ and $\genp{\overline{\Gamma}}$
invariant.

As $Cb(\bar a/M)$ doesn't change by passing to non-forking extensions of $p$,
we may assume $M$ is $\omega$-saturated. As $Cb(\bar a/M)$ is a finite imaginary,
and $\genp{\bar b}$ and $\genp{\overline{\Gamma}}$ are finite subspaces, they lay in $\acl^{eq}(Cb(p))$.

\medskip
Now since $M$ is a model, by means of $\sig{2}{4}$ we can find $c_{i}$ in $M_{1}$ such that
$[c,c_{i}]=\Gamma_{i}$ in $M$ for all $i<m$. This means $[c,c_{i}]$ can play the role of $\gamma_{i}$ in $\exs M_{1}$.
Moreover, by Remark \ref{tuttuno} if the tuple $\bar c$ collects all the $c_{i}$, %by above remarks we have
we have $\bar c\inn\acl(\bar\Gamma)$ -- $c$ is the constant added to the language.

Take $C=\gena{C_{1}}{M}$ with $C_{1}=\ssc(\bar b,\bar c,c)$,
then on one side $C\inn\acl(\bar b,\bar c)$ and hence $C\inn\acl^{eq}(Cb(p))$.

\medskip
On the other hand, by construction we have $\delta(\bar a/C)=\delta(\bar a/M)$. % and $\gen{C_{1},\bar a}\nni\bar m$,
Therefore, as $C$ is strong in $M$ and $\bar a$ is a strong tuple over $M$, Lemma \ref{fincharssc} implies
$$
d(\bar a/M)=\delta(\bar a/M)=\delta(\bar a/C)\geq d(\bar a/C)\geq d(\bar a/M)
$$
and hence $d(\bar a/C)=d(\bar a/M)$.

This also yields -- with Lemma \ref{fincharssc} again --  $M+\gen{C_{1},\bar a}\zsu{}\mathbb{M}$ and $\ssc(C_{1},\bar a)
=\genp{C_{1},\bar a}$. Now since $\bar a$ is linearly independent over $M_{1}$, we have
$\fin{\bar a}{C}{M}$ and hence $\ffin{\bar a}{C}{M}$.

\medskip
Now since $M$ is a model, % $\gen{C_{1},\bar a}$ meets
$\acl(C_{1})\inn M$ %necessarily in $C_{1}$
and this yields with Corollary \ref{stationary}, that $\tp{\bar a}{C}$
is stationary. Now Fact \ref{ziecb}\,(2.) implies $Cb(p)\inn\dcl^{eq}(C)$.

For such strong tuples $\bar a$ over $M$, %denote $C$ with $WCb(\bar a/M)$. W
we have shown
\begin{gather}\label{cibbi}
\begin{cases}C\inn\acl(Cb(\bar a/M))\\
Cb(\bar a/M)\inn\dcl^{eq}(C).\end{cases}
\end{gather}
\end{proof}
%%%--------------------------CASE 2---------------------------
%{\bf Case 2:}\quad{\sl $p=\tp{\bar d}{M}$ for $\bar d=\tupl{d}{0}{r-1}$ a linearly independent tuple in $\mathbb{M}_{1}$
%over $M_{1}$}
%
%\medskip
%Take a strong tuple $\bar a$ over $M$ such that $\gena{M,\bar a}{\mathbb{M}}$ is $\ssc(M,\bar d)$.
%We assume $\bar a$ is of length $n>r$ and that $a_{k}=d_{k}$ for all $i<r$.
%
%Take a {\em Morley sequence} $(\bar a^{i})_{i<\omega}$ in $\tp{\bar a}{M}$ and let $C$ be a finite strong subalgebra of $M$,
%as constructed in Case 1 with respect to $\bar a$ over $M$: properties \pref{cibbi} above hold for $C$. Recall
%$C=\ssc(\bar b,\bar c,c)$ for suitable tuples $\bar b$ and $\bar c$ of $M_{1}$.
%
%If $\bar a^{<i}$ denotes the sequence $\bar a=\bar a^{0},\bar a^{1},\dots,\bar a^{i-1}$, then we have by definition
%$\fin{\bar a^{i}}{M}{\bar a^{<i}}$ for all $i<\omega$ and since $\fin{\bar a}{C}{M}$, $(\bar a^{i})_{i<\omega}$ is a Morley sequence also in $\tp{\bar a}{C}$.
%
%By indiscernibility, if we let $\bar d^{i}$ denote the tuple $\tupl{a^{i}}{0}{r-1}$, we have in particular $\ssc(M,\bar d^{i})=\gen{M,\bar a^{i}}$
%and also $\fin{\bar d^{i}}{M}{\bar d^{<i}}$ for all $i<\omega$. That means, $(\bar d^{i})_{i<\omega}$ is a Morley sequence in $p$ and in $\tp{\bar d}
%{C}$ as well.
%Moreover, by Lemma \ref{fincharssc} $\gen{C,\bar a^{i}}=\ssc(C,\bar d^{i})$ for all $i$. 
%
%Let $m>\dfp(C_{1})$ and $Y$ denote $\gena{C,\bar a^{i}\mid i\leq m}{\mathbb{M}}=\underset{0\leq i\leq m}{{\circledast_{C}}}\gen{C,\bar a^{i}}$. The $\ind$-independence of the $\bar a^{i}$'s over $C$ implies $Y$ is self-sufficient in $\mathbb{M}$.
%
%Define $X$ to be the self-sufficient closure $\ssc(\gen{\bar d^{i}\mid i<m})\inn Y$.
%
%\smallskip
%We are to show, that $\bar a=\bar a^{0}$ is in $X_{1}$. Assume by contradiction, this is not the case.
%
%\smallskip
%As $T^{2}_{c}$ is stable, the sequence $\{\bar a^{i}\}_{i\leq m}$ is an indiscernible set over $C$ and any permutation of the $\bar d^{i}$'s
%(or of the $\bar a^{i}$'s) may be extended, in particular,
%to an automorphism of the algebra $Y$ which fixes $C$.
%By construction, the strong subspace $X$ is invariant under such automorphisms.
%
%This implies, we may actually find a number $s$ with $r\leq s<n-1$ and transform each $\bar a^{i}$ in such a way that, for every index $i$,
%a first segment $a^{i}_{0},\dots, a^{i}_{s-1}$ lays in $X_{1}$ and $a^{i}_{s},\dots,a^{i}_{n-1}$ is linearly independent over $C_{1}+X_{1}$.
%The length $s$ does not depend of $i$.
%
%\smallskip
%We claim, the set $\{a^{i}_{j}\mid0\leq i<m,s\leq j<n\}$ is linearly independent of $C_{1}+X_{1}$.
%This is done by induction over the length $\card{I}$ of sums $\sum_{i\in I}e_{i}$, where $e_{i}$ is a nontrivial linear combination
%in $\bar a^{i}$ and $I$ is a subset of $\{0,1,\dots,m-1\}$. We show $\sum_{i\in I}e_{i}$ cannot belong to $C_{1}+X_{1}$.
%
%The case $I=\{i\}$ is clear. The inductive basis will be given by $\card{I}=2$.
%
%Assume without loss of generality, the sum $e_{0}+e_{1}$ lays in $C_{1}+X_{1}$. Consider the automorphism $\sigma_{i}$
%of $Y$ over $C$, which leaves $\bar a^{0}$ fixed and transposes $\bar a^{1}$ with $\bar a^{i}$. Denote by $e_{2}$
%the image of $e_{1}$ under $\sigma$, then $e_{0}+e_{2}$ lays again inside $C_{1}+X_{1}$.
%This is then true also of $e_{1}-e_{2}$.
%
%
%
%Now consider the isomorphism $\tau$ which extends the $\Fp$-isomorphism we obtain -- as image of $e_{1}$ -- %$e^{3}\in\genp{\bar a^{3}$ such
%a linear combination $e_{3}$ in $\bar a^{3}_{\geq s}$ such that
%$e_{3}-e_{1}$ is in $C_{1}+X_{1}$.
%
%But this gives $e_{2}-e_{1}+(p-1)(e_{3}-e_{1})= \in C_{1}+X_{1}$.
%
%Assume we have a sum of length $\card{I}=k<m$, and the claim holds for all lengths less than $k$.
%
%Again by indiscernibility, we may test the case $e:=e^{0}+e^{1}+\dots+e^{k}\in C_{1}+X_{1}$. Consider a new automorphism $\sigma$
%of $Y$ over $C$, fixing all $\bar a^{i}$ for $i<k$ and moving $\bar a^{k}$ to $\bar a^{k+1}$. Denote by $e^{k+1}$ the image of
%$e^{k}$ under $\sigma$. We obtain $e^{\sigma}\in C_{1}+X_{1}$ and hence $e^{k}-e^{k+1}=e-e^{\sigma}\in X_{1}+C_{1}$.
%With the same arguments, now we can move $e^{k}-e^{k+1}$ to a linear combination in $\bar a^{0},\bar a^{1}$ which
%lays in $C_{1}+X_{1}$. This is a contradiction to the inductive basis and the claim follows.
%
%We rename $Y$ to be the free composition of only the first $m$ components: $Y=\circledast_{i<m}\gen{C,\bar a^{i}}$ and
%change $X_{1}$ into $X_{1}\cap Y_{1}$. Then rename $X$ accordingly.
%By the claim above follows $\dfp(Y_{1}/C_{1}+X_{1})=m(n-s)$.
%
%On the other hand, since $\genp{C_{1},\bar a^{i}}=\ssc(C_{1},\bar %a^{i}_{<s})$
%d^{i})=\ssc(C_{1},\bar a^{i}_{<s})$, for all $i$ we have $\delta(\bar a^{i}/C_{1},\bar a^{i}_{<s})
%<0$. This means $\dfp(\rd(\bar a^{i}/C_{1},\bar a^{i}_{<s}))>n-s$.
%
%Moreover, as $\genp{C_{1},\bar a^{i}}\cap X_{1}\inn\genp{C_{1},\bar a^{i}_{<s}}$, for all $i$ the
%quotient $\rd(\bar a^{i}/C_{1},\bar a^{i}_{<s})$ embeds into $\rd(Y/Z)$.
%
%If we now compute $\delta(Y/X)=m(n-s)+\dfp(C_{1}/X_{1})-\dfp(\rd(Y/X))$, since $\dfp(C_{1})<m$
%we obtain $\delta(Y/X)<0$ which is a contradiction. As a result we must have $s=n-1$ and
%all $\bar a^{i}$'s lay inside $X$. In particular $\bar a$ does, as claimed.
%
%\medskip
%We now show that $\bar b$ lays inside $X_{1}$ as well.
%
%Assume this is not the case, then for any $i$ we may find elements $\rho^{i}$ of $\rd(C,\bar a^{i})$,
%such that the subset $\{\rho^{i}\}_{i<m}\inn\rd(Y)$  is linearly independent over $\exs X_{1}$.
%It suffices for all $i$ to pick a suitable element $\rho^{i}=\rho_{t}$ like in the basis \pref{rhodi} at page \pageref{rhodi} with $\rho_{t}=
%\alpha_{t}(\bar a^{i})+\beta(\bar a^{i},\bar b_{t})+[c,c_{t}]$ and $\bar b_{t}\nsubseteq X_{1}$.
%
%Since $m$ is greater than $\dfp(C_{1})$ and %$\rd(Y)=\sum_{i<m}\rd(C,\bar a^{i})$
%$\bar a^{i}\inn X_{1}$ for all $i$, then $\delta(Y/X)=\dfp(C_{1}/X_{1})-\rd(Y/X)<0$. This is impossible unless
%the whole $\bar b$ lays in $X_{1}$.
%
%\smallskip
%We have in particular $\bar a,\bar b\inn X_{1}=\ssc(\bar d^{i}\mid i<m)\inn\acl(\bar d^{i}\mid i<\omega)$. Since the $\bar d^{i}$'s build a Morley
%sequence in $\tp{\bar d}{M}$, it follows that $\bar a,\bar b\inn\acl(Cb(\bar d/M))$.
%
%Now, by the shape of the relations $\rho_{t}$ above, we have $\overline{\alpha_{t}(\bar a)+\beta(\bar a,\bar b_{t})}=
%\overline{[c,c_{t}]}$ in $\mathbb{M}$.
%
%It follows $\overline{[c,c_{t}]}\in\dcl(\bar a,\bar b)$ and hence by Remark \ref{tuttuno}, $c_{t}\in\acl(\bar a, \bar b)$ for all $t$.
%
%In conclusion $\bar b,\bar c\inn\acl(Cb(\bar d/M))$ and hence $C\inn\acl(Cb(\bar d/M))$.
%
%On the other hand, since $\fin{\bar a}{C}{M}$ and $\bar d\inn%\genp{C_{1},
%\bar a$ by (Symmetry), %(Algebraicity)
%and (Weak Monotony), we have $\fin{\bar d}{C}{M}$. With the same arguments used in the previous case,
%follows $$Cb(\bar d/M)\inn\dcl^{eq}(C).$$
%
%We have shown that the same finite $C$ obtained for $\bar a$ over $M$ suits for $\bar d$ as well.
%%%--------------------------CASE---3---------------------------
%
%\medskip\noindent
%{\bf Case 3:}\quad{\sl $p$ is the type over $M$, of an arbitrary tuple $\bar e$ of $\mathbb{M}_{1}$}.
%
%\smallskip
%Let $\bar d$ be a subtuple of maximal length in $\bar e$ which is linearly independent over $M_{1}$.
%Then $\bar e$ is linearly inter-dependent with the tuple $\bar d\bar m$, for some tuple $\bar m$ of $M_{1}$.
%
%By Case 2, we can find a finite strong subalgebra $\widetilde{C}$ of $M$ as in the statement of the theorem corresponding
%to the type of $\bar d$ over $M$. That is $\widetilde{C}\inn\acl(Cb(\bar d/M))$ and $Cb(\bar d/M)\inn\dcl^{eq}(\widetilde{C})$.
%
%Then, on one side $\genp{\widetilde{C}_{1},\bar m}\inn\acl(Cb(\bar e/M))$.
%Now set $C=\ssc(\widetilde{C},\bar m)$. By base monotonicity and algebraicity (cfr.{\,}Section \ref{stab}) we have
%$$\fin{\bar d}{\widetilde{C}}{M}\Rightarrow\fin{\bar d}{C}{M}\Rightarrow\fin{\bar e}{C}{M}$$
%and $\tp{\bar e}{C}$ -- as in the previous cases -- is stationary.
%%[EXT] THIS FOLLOWS FROM $\fin{\bar d}{\widetilde{C}}{\bar m}$ WHICH IMPLIES
%%$\ssc(\widetilde{C},\bar m,\bar d)=\ssc(\widetilde{C},\bar d)+C$
%%AND HENCE $\ssc(C,\bar e)\cap\acl(C)\inn C$.

Compared to Lemma \ref{weimodels}, we have obtained the condition for the weak elimination only with respect
to types of strong tuples.

We denote with $C_{M,\bar a}$ the finite self-sufficient subalgebra of $M$ as in \pref{cibbi} above, relative to the tuple $\bar a$ and
the model $M$.

Remark that to obtain a finite subspace of $M$ with the first property of \pref{cibbi}, it is enough for the tuple $\bar a$ to be linearly independent over
$M_{1}$. On the contrary, by $\delta$-calculus alone (Lemma \ref{fincharssc}) a finite object with the second
property of \pref{cibbi} is found for any tuple over a model -- this is the finite character of forking,
common to all totally transcendental structures. Starting with a strong tuple over $M$ ensures the existence
of a finite subalgebra with both properties.

%Now for a tuple $\bar d$ of $\mathbb{M}_{1}$ and a model $M$, let $\bar a$ be a strong tuple
%over $M$ such that $\ssc(M,\bar d)=\gen{M,\bar a}$. The idea here, is to compare $C_{M,\bar a}$ with
%$Cb(\bar d/M)$ up to algebraicity.
%
%On one side we have -- with Fact \ref{pilcb}, for instance -- $Cb(\bar d/M)\inn\acl^{eq}(C_{M,\bar a})$.
%On the other, if $C_{M,\bar a}=\ssc(\bar b,\bar c,c)$ as in the proof of Lemma \ref{teowei},
%we can prove that $\bar b\inn\acl^{eq}(Cb(\bar d/M))$.
%
%We also conjecture that, modulo a suitable $\vac$-definable equivalence relation  $\epsilon$,
%${C_{M,\bar a}}_{/\large\epsilon}$ is eq-interalgebraic with $\acl^{eq}(Cb(\bar d/M))$.
\smallskip
This said, we can sketch a strategy for proving $CM$-triviality as follows.
By Fact \ref{pilcmt} one can test $CM$-triviality (Definition \ref{cmtdef}) by means of {\em real} tuples over {\em models}. Moreover
this property for $T^{2}_{c}$ implies the result for $T^{2}$.

\smallskip
So {\em take a tuple $\bar d$ from $\mathbb{M}_{1}$ and models $M\ess N$ with $\acl^{eq}(M,\bar d)\cap N=M$.
We have to show $Cb(\bar d/M)\inn\acl^{eq}(Cb(\bar d/N))$.}

\smallskip
Let $\bar a$ be a strong tuple over $M$ with $\ssc(M,\bar d)=\gen{M,\bar a}$ and $\bar a^{\prime}$, a strong tuple
over $N$ with $\ssc(N,\bar d)=\gen{N,\bar a^{\prime}}$.

We have firstly $Cb(\bar d/M)\inn\acl^{eq}(C_{M,\bar a})$, by Fact \ref{pilcb} and \pref{cibbi} above.

Now by the hypothesis follows $\genp{M_{1},\bar a}\cap N_{1}=M_{1}$ and hence we may assume $\bar a\inn\bar a^{\prime}$.
For the same reason we also have that $\rd(\bar a/M)$ embeds into $\rd(\bar a^{\prime}/N)$ and
-- by the proof of Lemma \ref{teowei} --
we may build $C_{N,\bar a^{\prime}}$ with $C_{M,\bar a}\inn C_{N,\bar a^{\prime}}$.

Up to now, we used indeed the definition of $CM$-triviality {\em relative to} $\ssc(~)$ {\em over the theory of
$\Fp$-vectorspaces}
contained in the paper of Wagner, Blossier and Martin-Pizarro \cite{cmtr}.

We now believe, there exists in general a
$\vac$-definable equivalence relation  $\epsilon$,
such that $C_{N,\bar a^{\prime}}/\epsilon$ is eq-interalgebraic with $Cb(\bar d/N)$\footnote{If $C_{N,\bar a^{\prime}}$ is
$\ssc(\bar b,\bar c,c)$ as in Theorem \ref{teowei},
then we can prove $\bar b\inn\acl^{eq}(Cb(\bar d/N))$ and the eq-sort $\epsilon$ should control the
orbits of the tuple $\bar c$ under automorphism which fix $\tp{\bar d}{N}$.}.

This should fill in the gap and get %$C_{N,\bar a^{\prime}}$
$Cb(\bar d/M)$ contained in $\acl^{eq}(Cb(\bar d/N))$.

%We may also think as more likely that the theory $T^{2}_{c}$ have {\em geometric elimination of imaginaries},
%that is: {\em to every imaginary element $e$ there exists a real one $a$ with $\acl^{eq}(e)=\acl^{eq}(a)$.}

%\begin{prop}\label{pr:cmtriv}
%$T^{2}_{c}$ is a $CM$-trivial theory. That is for any real tuple $\bar e$ of $\mathbb{M}$ and models $M\ess N$,
%if $\acl(M,\bar e)\cap N=M$, it follows $Cb(\bar e/M)\inn\acl^{eq}(Cb(\bar e/N))$.
%\end{prop}
%\crule
%By the following proof and Case 1 of Theorem \ref{teowei} it would be enough to show
%that for any tuple $\bar e$ like above, we may find a {\em strong} tuple $\bar a$ {\em over $M$} with
%%$\genp{M_{1},\bar a}\cap N_{1}=M_{1}$
%$\bar a\in\acl(M,\bar e)$ and such that
%$$
%\begin{cases}
%Cb(\bar e/M)\inn\acl^{eq}(Cb(\bar a/M))\\
%Cb(\bar a/N)\inn\acl^{eq}(Cb(\bar e/N))
%\end{cases}
%$$
%With this respect, could Fact \ref{pilcb} be helpful.
%\crule
%\begin{proof}
%By means of the definable map $\vartheta_{c}$ of Remark \ref{tuttuno}, each tuple of $\mathbb{M}$ is interalgebraic
%with some tuple from $\mathbb{M}_{1}$. We may -- by Fact \ref{pilcb} -- assume that $\bar e$ lays in $\mathbb{M}_{1}$.
% 
%On account of the discussion around Case 3 in the proof of Theorem \ref{teowei}, we may assume that
%$\bar e$ is linearly independent over $M_{1}$, and hence -- by the hypotheses -- over $N_{1}$.
%
%\smallskip
%%Let therefore $\bar m$ be a tuple in $\mathbb{M}_{1}$ and $M\ess N\ess\mathbb{M}$ with $\acl(M,\bar m)\cap N=M$. 
%By \pref{acluno} follows in particular $\ssc(M_{1},\bar e)\cap N_{1}=M_{1}$.
%As a consequence, if $\bar a$ is
%a strong tuple of $\mathbb{M}_{1}$ over $M_{1}$ such that $\ssc(M_{1},\bar e)=\genp{M_{1},\bar a}$,
%then $\bar a$ is linearly independent over $N_{1}$ as well.
%
%Now if $\ssc(N_{1},\bar e)$ is $\genp{N_{1},\bar c}$ with $\bar c$ linearly independent of $N_{1}$,
%then we may assume that $\bar a$ is a subtuple of $\bar c$. Since $\genp{M_{1},\bar a}\cap N_{1}=M_{1}$ we have
%by Corollary \ref{cor:2modker} $\rd(M,\bar a)\cap\rd(N)=\rd(M)$ and hence
%$$\rd_{\mathbb{M}}(\bar a/M)\inn\rd_{\mathbb{M}}(\bar c/N).$$
%That means, any basis of $\rd_{\mathbb{M}}(\bar a/M)$ can be extended to a basis of $\rd_{\mathbb{M}}(\bar c/N)$.
%
%Therefore, by the construction of a weak canonical base $C$ for %$\tp{\bar e}{M}$
%$\bar e$ over $M$ as in the proof of Theorem \ref{teowei},
%we may extend any such $C$ to a finite strong subalgebra $D$ of $N$ which is a weak canonical base for $\tp{\bar e}{N}$.
%
%We may now conclude with Theorem \ref{teowei}, that
%$Cb(\bar e/M)\inn\dcl^{eq}(C)$ and $D\inn\acl(Cb(\bar e/N))$.
%
%Since $C\inn D$, we get $Cb(\bar e/M)\inn\dcl^{eq}(\acl(Cb(\bar e/N)))=\acl^{eq}(Cb(\bar e/N))$ as desired.
%\end{proof}
%
%As mentioned in Section \ref{stab}, by \cite[Proposition 3.2]{pilcm}, we may conclude
%\begin{cor}
%No infinite field is interpretable in $T^{2}$.
%\end{cor}
\newpage
\thispagestyle{empty}
\cleardoublepage
%----------------------------------------CHAPTER THREE-------------------
\chapter{Deficiencies in Higher Class}\label{tre}
We are going to develop a new tool which permits an inductive approach to the construction
of deficiencies for $\nla{c}$-algebras in a nilpotency class $c$ higher than $2$.
The main results and details concern the case $c=3$, for which a predimension-like approach is adopted.
\section{A ``free lift'' Functor}\label{freelift}
%\medskip
Fix a prime $p$ grater than $c$ and
assume $M=M_{1}\oplus\cdots\oplus M_{c}$ is a graded Lie algebra of $\nla{c}$ as defined
in Definition \ref{lcp} of Section \ref{nilgral}, in particular $M=\gen{M_{1}}$.

The $c^{\textsl{th}}$-homogeneous component $M_{c}$ of $M$ coincides
with the ideal $\gamma_{c}(M)$ of $M$ -- the $c^{\textsl{th}}$ term of the lower central series.
If we denote by $M_{*}$ the quotient $M/M_{c}$, then $M_{*}$ is
again generated by $M_{1}$ (modulo $M_{c}$) and $M_{*}\simeq M_{1}\oplus\cdots\oplus M_{c-1}$. That is $M_{*}\in\nla{c-1}$ and we can refer to $M_{*}$ as the {\em truncation} of $M$ to $\nla{c-1}$. We denote by $\ast$ the resulting functor of $\nla{c}$ into $\nla{c-1}$.

%Since $M_{*}=M$ for all $M\in\nla{c-1}\inn\nla{c}$, $*$ is a retraction of $\nla{c}$ onto $\nla{c-1}$ and
For a given $M$ in $\nla{c}$, with
abuse of the above notation, we denote again by $\ast$ the canonical epimorphism of $M$ onto $M_{*}$.
Moreover, for any $m$ in $M$, we set $m_{*}=m\ast=m+M_{c}$.

\medskip
In Section \ref{nilgral}, we denoted by $\fla{n}{X}$ the free $n$-nilpotent Lie $\Fp$-algebra over the set $X$. This is
in particular a free object of $\nla{n}$.
Moreover any $A$ in $\nla{n}$, is an epimorphic image of $\fla{n}{A_{1}}$.

\smallskip
Now for an algebra $M$ in $\nla{c-1}$, let $R$ be the homogeneous ideal of $\fla{c-1}{M_{1}}$ which gives
the presentation $\gen{M_{1}\mid R}$ of $M$, as defined in section \ref{nilgral}.

We identify $\fla{c-1}{M_{1}}$ with the subspace $L_{1}\oplus\cdots\oplus L_{c-1}$ of $L=\fla{c}{M_{1}}$, and denote by
$\iota$ the $\Fp$-linear embedding of $\fla{c-1}{M_{1}}$ into $\fla{c}{M_{1}}$.

Let $\J(M)$ denote the ideal of $\fla{c}{M_{1}}$ generated by $\iota(R)$.
Since $R$ is an ideal of $\fla{c-1}{M_{1}}$, in the notation of section \ref{nilgral}, we have
$\J(M)=\genid{\iota(R)}{}=\genp{\iota(R),[\iota(R),\fla{c}{M_{1}}]}$. %in $\fla{c}{M_{1}}$.
Notice that $\J(M)$ is homogeneous with
$\J(M)_{1}=\triv$ (cfr.\,\pref{homojei} below).
\begin{dfn}\label{freelifting}
For any $M$ of $\nla{c}$, define $\fl M$ to be %${\rm F}(M)$ or by $F^{n}_{M}$
the quotient $\fla{c}{M_{1}}/\J(M)$ and call {\em free lift}, the map
\begin{align*}\tag{\sf fl}
\lmap{\frl}{\nla{c-1}&}{\nla{c}}\\
M&\longmapsto F_{M}=\gen{M_{1}\mid\J(M)}.
\end{align*}
\end{dfn}
\begin{prop}\label{morphifreelift}
For any algebra $M=\gen{M_{1}\mid R}$ of $\nla{c-1}$ one has $(\fl M)_{*}\simeq M$ and we adopt the convention to identify $(\fl M)_{*}$ with $M$,
coherently with the choice for $(\fl M)_{1}$ to be $M_{1}$.

\smallskip
For any $N$ in $\nla{c}$, if $\varphi$ is an $\nla{c-1}$-morphism
of $M$ into $N_{*}$, then there exists a unique $\nla{c}$-morphism $\widetilde\varphi$ of $\fl M$ into $N$
such that $\widetilde\varphi{}{*}={*}\varphi$.
\begin{labeq}{commufree}
\begin{split}
\xymatrix@C+4mm{
\fl{M}\ar[r]^{\widetilde\varphi}\ar[d]^{*}&N\ar[d]^{*}\\
M\ar[r]^{\varphi}&N_{*}
.}
\end{split}
\end{labeq}

Moreover, the map $\varphi\mapsto\widetilde\varphi$ yelds a bijection
\begin{labeq}{freeliftadjoint}
\Hom_{\nla{c-1}}(M,N_{*})\to\Hom_{\nla{c}}(\frl( M),N)
\end{labeq}
for any $M$ in $\nla{c-1}$ and $N$ in $\nla{c}$.
%then $\fl{M}$ has the following features:
%\begin{itemize}
%\item[-]$(\fl M)_{*}\simeq M$
%\item[-]for any $N\in\nla{c}$ with $N_{*}\simeq M$ (hence $N_{1}\simeq M_{1}$) we have a \sout{unique} canonical
%$\nla{c}$-epimorphism $\epi{\epsilon}{\fl M}{N}$ up to \dots.
%\end{itemize}
\end{prop}

%\vfill
%Once a prime number $p$ is fixed, we define $\nla{n,p}$, or
%short $\nla{c}$, to be the class of nilpotent Lie Algebras of nilpotency class $n$,
%over the finite field $\Fp$,
%with the additional property that each algebra $M$ of $\nla{c}$,
%is generated by some $\Fp$-vector space $M_{1}$, that is $M=\gen{M_{1}}$.
%In this way we obtain a natural graduation of $M$ according to the \emph{commutator weight} with respect to
%$M_{1}$, that is
%$M=\oplus_{i\leq n}M_{i}$, where $M_{i}$ denote the $i^{\mathrm{th}}$-\emph{homogeneous component}
%of $M$, the $\Fp$-vector subspace of $M$ generated by all Lie products of $M_{1}$-weight $i$. By convention $M_{0}=\triv$.
%
%If $M$ belongs to $\nla{c}$, we say that $H$ is an $\nla{c}$-subalgebra
%of $M$ if $H=\gena{H_{1}}{M}$ where $H_{1}$ is a vector
%subspace of $M_{1}$. Of course $H\in\nla{c}$ as well.
%In the future for a subalgebra of $M\in\nla{c}$ we will
%always mean a subalgebra of this special kind.
%
%For a morphism $\map{\phi}{L}{M}$ of $\nla{c}$-algebras we mean a \emph{graded} Lie morphism of $L$ to $M$, that
%is $\phi(L_{i})\inn M_{i}$ for all $i$.
%
%With $M^k$ we name the $k$-th term of the \emph{lower central chain} of $M$, namely the ideal
%$M^k=\sum_{k\leq i}M_i$.
%We define as $\map{\tr{n}}{\nla{c+1}}{\nla{c}}$ the map defined by quotienting out the last component:
%$\tr{n}A=A\quot A_{n+1}$, note that here $A_{n+1}=A^{n+1}$, in particular $A_{n+1}$ is an ideal.
%Moreover if $A=\oplus_{i\leq n+1}A_{i}$ then $\tr{n}A\simeq\oplus_{i\leq n}A_{i}$, therefore, since $\nla{c}\inn\nla{c+1}$, we can regard $\tr{n}$ as an $\nla{c+1}$ morphism which maps identically the homogeneous components up to the $n+1^{th}$,
%which is mapped to $\triv$. 
%
%\medskip
%$\fla{c}{X}$ will denote the \emph{free $n$-nilpotent Lie Algebra} over $\Fp$ with set of generators $X$.
%We know, see for example [Bh], that if $V$ is an $\Fp$-vector space with basis $X$, then $\fla{c}{V}=\fla{c}{X}$. $\fla{c}{X}$ is the $\mathfrak{N}_{n}$-free algebra with free generator set $X$, where $\mathfrak{N}_{n}^{p}$ denotes the variety\footnote{
%whose defining word is $[x_{1},\,\dots,\,x_{n+1}]$}
%of nilpotent Lie algebras of class $\leq n$ over the field $\Fp$. One sees easily that $L^{n}(X)$ lies in $\nla{c}$ as well. Note that $\nla{c}$ is not a subvariety of $\mathfrak{N}_{n}$.
%
%\bigskip
%If $A\in\nla{c}$ we can assume $A=\fla{c}{X}\quot R$, where $X$ is an $\Fp$ basis of the vector space $A_1$,
%and $R$ is an ideal of $\fla{c}{X}$ which contains what we call the \emph{relators} of $A$.
%According to the previous observation we have, for $\nla{c}$-algebras, a canonical \emph{presentation} $A=\fla{c}{A_{1}}\quot
%R$. The notation $A=\gen{x\in A_{1}\mid \rho\in R}$ or $A=\gen{A_{1}\mid\mathcal{R}}$ will also be used, where $\mathcal{R}$ is a set of words generating the verbal ideal $R$.
%By our assumpion on $\nla{c}$, we can assume that $R$ is
%a \emph{homogeneous} ideal\footnote{\dots be sure.},
%that is $R=\sum_{i\leq n}R\cap{\fla{c}{A_{1}}}_{i}=:\sum_{i\leq n}R_{i}$. Moreover, in this case we have $R_{1}=\triv$.
%
%If now $M$ is in $\nla{c}$ presented as $M=\gen{M_{1}\mid\mathcal{R}}$, where
%the set of words $\mathcal{R}$ are, by definition, objects of the (absolutely) free Lie algebra $L_{p}(M_{1})$ over the field $\Fp$. We define the subvariety $\mathcal{V}^{n+1}(M)$ of $\mathfrak{N}
%_{n+1}^{p}$ as follows
%$$\mathcal{V}^{n+1}(M)=\left\{L\in\mathfrak{N}_{n+1}^{p}\mid\mathcal{R}(L)=\triv\right\}=
%\mathfrak{N}_{n+1}^{p}\cap\var^{p}(M).$$
%Define now $\fl{M}$ to be the free object in the variety $\mathcal{V}^{n+1}(M)$
%with set of free generators $M_{1}$. We have then
%$\fl{M}=\fla{c+1}{M_{1}}\quot\J(M)$ where $\J(M)$ is the verbal ideal of $\fla{c+1}{M_{1}}$
%generated by the words $\mathcal{R}$. Also $\fl{M}\in\nla{c+1}$.

\medskip
\begin{proof}
Let $L$ denote $\fla{c}{M_{1}}$. Then $L_{c}=\gamma_{c}(L)$ and $\J(M)$ are ideals of $L$ such that
$\fl{M}=L/\J(M)$ and $L/L_{c}\simeq\fla{c-1}{M_{1}}$. We have the following isomorphisms of Lie algebras\footnote{
In the row below $+$ is the ordinary {\em sum} between subalgebras and ideals of a Lie algebra.}
$$
(\fl{M})_{*}\simeq\frac{ L}{L_{c}+\J(M)}\simeq\frac{L/L_{c}}{(L_{c}+\J(M))/L_{c}}%\simeq\frac{\fla{c-1}{M_{1}}}{\J(M)/(\J(M)\cap L_{c})}.
$$
Moreover, as $R$ is homogeneous and equals $R_{2}+\cdots+R_{c-1}$, we have
\begin{labeq}{homojei}
\J(M)=\iota(R)\oplus\sum_{i=2}^{c-1}[R_{i},L_{c-i}]%\gamma_{c-i}(L)]
\end{labeq}
where the right direct summand is contained in $L_{c}$.

Hence -- as algebras -- $(L_{c}+\J(M))/L_{c}\simeq\J(M)/(\J(M)\cap L_{c})\simeq R$ and therefore
$(\fl{M})_{*}\simeq_{\nla{c-1}}\fla{c-1}{M_{1}}/R=M$.
The first assertion is proved.

\medskip
Now by interpreting $M$ and $N_{*}$ as objects of $\nla{c}$, we obtain the presentations $\map{\mu}{\fla{c}{M_{1}}}{M}$ and
$\map{\eta}{\fla{c}{N_{1}}}{N_{*}}$. For any morphism $\varphi$ of $M$ into $N_{*}$, we obtain
a unique $\nla{c}$-morphism $\widehat\varphi$ of $\fla{c}{M_{1}}$ into $\fla{c}{N_{1}}$ by means of  Lemma \ref{commufreeno}, with
the property $\widehat\varphi\eta=\mu\varphi$.

Through the canonical $\Fp$-embedding $\iota$, we identify $\fla{c-1}{M_{1}}$ with a subspace of $\fla{c}{M_{1}}$ as described above.
If we denote by $R$ the %subspace of $\fla{c}{M_{1}}$ which is the 
kernel of the restriction of $\mu$ to $\fla{c-1}{M_{1}}$, we 
obtain $\ker(\mu)=R\oplus\gamma_{c}(\fla{c}{M_{1}})$ and $\fla{c-1}{M_{1}}$ presents $M$ modulo $R$.

It follows $\J(M)$ is the ideal of $\fla{c}{M_{1}}$ generated by $R$. In particular $\J(M)\inn\ker(\mu)$ and, if $\pi_{\J}$ presents
$\fl M$ as a quotient of $\fla{c}{M_{1}}$, then $\pi_{\J}{}*=\mu$.

On the other hand if $\epsilon_{N}$ presents $N$ from $\fla{c}{N_{1}}$, then $\epsilon_{N}{}\ast=\eta$.

\smallskip
If now $w\in R$, %\inn\fla{c}{M_{1}}$
then $(w\,\widehat\varphi\epsilon_{N})_{*}=w\,\widehat\varphi\eta=w\,\mu\varphi=\triv$. But this means $w\,\widehat\varphi\epsilon_{N}$
lays inside $N_{c}\cap R^{\,\widehat\varphi\epsilon_{N}}$ which is trivial by a matter of weight
and hence $w\,\widehat\varphi\epsilon_{N}=\triv$.

This yields that $\J(M)\inn\ker(\widehat\varphi\epsilon_{N})$ and hence we {\em define} $\widetilde\varphi$ as
the quotient of $\widehat\varphi\epsilon_{N}$ modulo $\J(M)$, that is $\bar w\mapsto w\,\widehat\varphi\epsilon_{N}$ for
$w\in\fla{c}{M_{1}}$.

Any other map $\varphi^{\prime}$ of $\fl M$ to $N$ with $\varphi^{\prime}\ast=\ast\varphi$ fits in the diagram below
in the place of $\widetilde\varphi$. In particular $\widehat\varphi\epsilon_{N}=\pi_{\J}\varphi^{\prime}$ and hence
$\varphi^{\prime}=\widetilde\varphi$.

$$\xymatrix@R-5mm@C+2mm{
&\fla{c}{M_{1}}\ar^(0.4){\widehat\varphi}[drr]\ar_{\mu}[dddl]\ar^{{\pi}_{\J}}[dd]&&\\
&&&\fla{c}{N_{1}}\ar_(0.45){\eta}[dddl]\ar^{\epsilon_{N}}[dd]\\
&\fl M\ar^(0.4){*}[dl]\ar^(0.4){\widetilde\varphi}[drr]|!{[rru];[rdd]}\hole&&\\
M\ar^(0.52){\varphi}[drr]&&&N\ar^(0.4){*}[dl]\\
&&N_{*}&
}$$
\end{proof}


Consider now a morphism $\map{\phi}{M}{N}$ of $\nla{c-1}$-algebras $M$ and $N$, by identifying $N$ with $(\fl N)_{*}$
we define the {\em free lift} of $\phi$ as the morphism
$\frl(\phi)\defeq\widetilde\phi$ of $\fl M$ into $\fl N$ given by Proposition \ref{morphifreelift}. In particular \pref{commufree} holds for $\phi$
and hence we have
\begin{cor}
The free lift mapping $\frl$ is a functor of the category $\nla{c-1}$ into $\nla{c}$, adjoint to $\ast$.
Moreover for any $\nla{c-1}$-morphism $\map{\phi}{M}{N}$, the square below
\begin{labeq}{morphistar}
\begin{split}
\xymatrix@C+5mm{
\fl{M}\ar[r]^{\frl(\phi)}\ar[d]^{*}&\fl{N}\ar[d]^{*}\\
M\ar[r]^{\phi}&N
}
\end{split}
\end{labeq}
commutes. Also by \pref{commufresco} and the construction of $\frl(\phi)$ we obtain.
\begin{labeq}{liftbild}
\im(\frl(\phi))=\gena{\phi(M_{1})}{\fl N}
\end{labeq}
\end{cor}
\subsection{Isolating essential maximal-weight Relators}\label{maxrels}
With the functor $\frl$ it is possible to isolate the {\em relevant maximal weight relators} in the sense of the following approach.

\medskip
Consider an object $M$ in $\nla{c}$ given by $M=\gen{M_{1}\mid R}$ where $R$ is a usual an homogeneous ideal
of $\fla{c}{M_{1}}$, that is $R=R_{2}\oplus\cdots\oplus R_{c}$. Then $M_{*}$ may be presented in $\nla{c-1}$ as
$\fla{c-1}{M_{1}}/(R_{2}+\dots+R_{c-1})$.

%We overwrite the above notation and set $\J(M):=\J(M_{*})$.
If we present $\frl(M_{*})=\fl{M_{*}}$ in $\nla{c}$ as $\gen{M_{1}\mid\J(M_{*})}$,
then by definition $\J(M_{*})$ is contained in $R$.

Denote by $\rc(M)$ the quotient $R/\J(M_{*})$, by $\pi_{\J}$ the canonical map modulo $\J(M_{*})$ and consider
the following morphism of exact sequences. The rightmost square is in $\nla{c}$.
\begin{labeq}{morphipres}
\begin{split}
\xymatrix@C-2mm{
{\,R\,}\ar[r]\ar ^{\pi_{\J}}[d]&\fla{c}{M_{1}}\ar[r]\ar^{\pi_{\J}}[d]&M\ar@{=}[d]\\
\rc(M)\ar[r]&\fl{M_{*}}\ar^{\pam{M}}[r]&M
}
\end{split}
\end{labeq}
where $\pam{M}$ is the natural map of $\fl{M_{*}}$ onto $M$ with kernel $\rc(M)$.

Notice (cfr.\,Remark \ref{sangalgano} for instance) that
$(\rc(M))_{i}=\triv$ for all $2\leq i<c$. In particular $M$ is $\nla{c}$-isomorphic to $\fl{M_{*}}/\rc(M)$. 

Notice that the map $\pam{M}$ may also be obtained with Proposition \ref{morphifreelift} as $\widetilde{\id}_{M_{*}}$. As such,
by \pref{commufree}, we have $\map{\pam{M}\ast=\ast\id_{M_{*}}}{\fl{M_{*}}}{M_{*}}$.

\smallskip
In this sense $\rc(M)$ isolates the {\em essential} relators of $M$ of maximal weight $c$. Those,
which do not arise from relators $R_{i}$ in lower weight ($i<c$) reaching the weight $c$ by means of Lie brackets.

In Section \ref{preditre}, we adopt the second row in diagram \pref{morphipres} above as a suitable presentation
of $M$ to perform the amalgamation process.

\medskip
The philosophy behind this definition is the inductive strategy described %on page \pageref{indunil} of
in the Introduction and will be tested in Section \ref{preditre} below, switching from $\nla{2}$ to $\nla{3}$.
\subsection{Embedding Issues for $\frl$}\label{embiss}
In the forthcoming sections the functor constructed above will essentially be applied to the following situation:
consider an $\nla{c-1}$-subalgebra $H$ of $M\in\nla{c-1}$ and the inclusion $i\colon H\inn M$.
Denote by $\gamma$ the lifted morphism $\frl(i)$.

We identify $\fla{c}{H_{1}}$ with the $\nla{c}$-subalgebra $\gena{H_{1}}{\fla{c}{M_{1}}}$ of $\fla{c}{M_{1}}$.

Now consider the $\nla{c}$-presentations $\gen{H_{1}\mid\J(H)}$ and $\gen{M_{1}\mid\J(M)}$ for $\fl H=\frl(H)$ and $\fl M=\frl(M)$
respectively.
Since $H$ is an $\nla{c-1}$-subalgebra of $M$, we may consider $\J(H)$ as a subspace of $\fla{c}{M_{1}}$ with
$\J(H)\inn\fla{c}{H_{1}}\cap\J(M)$. By Proposition \ref{morphifreelift} the map $\gamma$ coincides with
\begin{align}\label{gam}
\lmap{\gamma%{H}{M}
=\frl(i)}{\fl H}{&\fl M}\\\notag
w+\J(H)\longmapsto & \,w+\J(M)
\end{align}
%sending $w+\J(H)$ to $w+\J(M)$,
for all $w$ in $\fla{c}{H_{1}}$.

As a consequence we get
\begin{rem}\label{sangalgano} With the above considerations about notations
\begin{labeq}{kerfli}
\ker(\gamma)=(\fla{c}{H_{1}}\cap\J(M))/\J(H).
\end{labeq}

\medskip
Moreover if $M$ is $\fla{c-1}{M_{1}}/R$ then $H=\fla{c-1}{H_{1}}/R\cap\fla{c-1}{H_{1}}$ and hence for all $i<c$, by \pref{homojei}
we have $(\fla{c}{H_{1}}\cap\J(M))_{i}=\J(H)_{i}$. It follows
$\ker(\gamma)$ is a homogeneous ideal of total weight $c$, that is $\ker(\gamma)$ is contained in $(\fl H)_{c}$.
\end{rem}

\medskip
We illustrate below how the free-lift functor actually doesn't preserve -- in general -- embeddings. This example and
the result which follows concern the particular case of lifting $\nla{2}$-algebras.
\begin{rem}\label{nonmono}
There are extensions of $\nla{2}$-algebras $M\nni H$, such that the map $\gamma$ of $\fl H$ into $\fl M$, defined in
\pref{gam} is not injective.
\end{rem}
\begin{proof}
Consider an algebra $M$ of $\nla{2}$ given by the presentation $M=\gen{M_{1}\mid\rd(M)}$, where $M_{1}$ is freely generated
by the $\Fp$-base $$\mathcal{B}=\{a,u,x,y,z,h_{1},\dots,h_{4},e_{1},\dots,e_{4}\}$$
and $\rd(M)$ is the span in $\exs M_{1}$ of the following linearly independent relators:
\begin{gather*}
\begin{split}
[h_{1},h_{2}]+[x,a],\\
[h_{3},h_{4}]+[a,u],
\end{split}\quad
\begin{split}
[z,y]+[u,x],\\
[e_{1},e_{2}]+[y,a],\\
[e_{3},e_{4}]+[a,z].
\end{split}
\end{gather*}
Let $H_{1}$ denote the subspace of $M_{1}$ generated by
$\mathcal{B}\non\{a\}$ in $M_{1}$.
Now consider the following homogeneous sum of weight $3$ in $\fla{3}{M_{1}}$:
\begin{align*}
[[h_{1},h_{2}]&+\uline{[x,a],u]}+\\
[[h_{3},h_{4}]&+\uline{[a,u],x]}+\\
\uline{[[z,y],a]}&+\uline{[[u,x],a]}+\\
[[e_{1},e_{2}]+\uline{[y,a],z]}&+\\
[[e_{3},e_{4}]+\uline{[a,z],y]}.&
\end{align*}
By definition, this lays in $\J(M)$, but after deleting Jacobi sums it is also equal to
$$
w:=[h_{1},h_{2},u]+[h_{3},h_{4},x]+[e_{1},e_{2},z]+[e_{3},e_{4},y]
$$
and hence belongs to $\fla{3}{H_{1}}$. By $\pref{kerfli}$, now $\bar w\in \fl H$ is a non-zero element of $\ker(\gamma)=(\fla{3}{H_{1}}\cap\J(M))/
\J(H)$ since $w\notin\J(H)$.
\end{proof}
Notice that in the example above, $H$ {\em is not self-sufficient} in $M$ ($\delta(a/H)=-3$). This is actually
the only obstruction for extensions of $\Klt{2}$ not to be lifted to $\nla{3}$-embeddings by $\frl$:
\begin{prop}\label{bellemma}
Let $M$ be a $\Klt{2}$-algebra. Assume $i\colon H=\gena{H_{1}}{M}\inn M$ is a self-sufficient $\nla{2}$-embedding,
then the map $\map{\gamma=\frl(i)}{\fl H}{\fl M}$ is an $\nla{3}$-monomorphism.

By Remark \pref{sangalgano} this is equivalent for a strong extensions $H\zsu{}M$ to imply \[\J(H)=\J(M)\cap\fla{3}{H_{1}}\]
where as above we let $\fla{3}{H_{1}}$ coincide with $\gena{H_{1}}{\fla{3}{M_{1}}}$.
\end{prop}
%\sout{The quite long proof of the above proposition is deferred to the end of the section/Cha[ter}
\begin{proof}
%Let $\mathcal{M}$ be a base for $M_1$ and $\xoh\inn \mathcal{M}$ be a base for $H_1$.
%As in diagramm \pref{communoemezzo} we have:
As \emph{$\delta$-strongness} traduces into a local property by Proposition \ref{finchar}, we may assume without loss of generality, that $H_{1}$ is a {\em finite} subspace of $M_{1}$.

If $\gamma$ denotes $\frl(i)$ as above,
assuming $\gamma$ not injective, yields a nontrivial $w\in(\jei{M}\cap\fla{3}{H_{1}})\setminus\jei{H}$.

As $M$ lays in $\nla{2}$ and hence $M=\fla{2}{M_{1}}/\rd(M)$, we have by definition
$$
\J(M)=%\genid{N}{\fla{3}{M_1}}=
\genp{\mu,\,[\nu,z]\mid\mu,\nu\in\rd(M),\,z\in M_1}.
$$

Moreover, by the remarks above $w$ may be assumed homogeneous of weight $3$, hence a finite sum like
\begin{labeq}{summa}
w=\sum_{\al}[\nu_{\al},z_{\al}].
\end{labeq}
for some $\nu_{\al}\in\rd(M)$ and $z_{\al}\in M_{1}$.
On the other hand $w$ must be also identical to a linear combination of monomials of weight $3$ in elements of $H_{1}$.

\smallskip
Extract a maximal subset $\mathcal{Z}$ out of the $z_{\alpha}$'s above, which is linearly independent over $H_{1}$.
With bilinearity of the Lie bracket
and since $\rd(M)$ % and $\rd(H)$ are
is an additive subgroup of $\exs M_{1}$, we can actually give expression \pref{summa}
the form
\begin{labeq}{mustiola}
w=%\sum_{i=1}^{m_{1}}[\nuu{i}, u_i]+\sum_{j=1}^{n_{1}}[\nuw{j}, w_j]
\sum_{u\in\mathcal{U}_{0}}[\nuu{},u]+\sum_{z\in\mathcal{Z}}[\nu_{z},z]
\end{labeq}
where $\mathcal{U}_{0}$ %${\{u_i\}}_{i=1,\dots,m_{1}}$
is a linearly independent subset of $H_1$, $\mathcal{Z}\inn M_{1}$ is the above set independent over $H_{1}$ and $\nuu{},\nu_{z}$
belong to $\rd(M)$.
 \pref{mustiola}.

\smallskip
Now we claim that sum \pref{mustiola} can be transformed into
\begin{labeq}{sangennaro}
w=%\sum_{i=1}^m[\nuu{i}, u_i]+\sum_{j=1}^n[\nux{j}, x_j]+\sum_{k=1}^r[\lay{k},y_k]
\sum_{u\in\mathcal{U}}[\nuu{}, u]+\sum_{x\in\mathcal{X}}[\nux{}, x]+\sum_{y\in\mathcal{Y}}[\lay{},y]
\end{labeq}
where $\mathcal{U}$ %{\left\{u_i\right\}}_{i=1,\dots,m}$
is a $\Fp$-independent subset of $H_1$, $\mathcal{XY}$ %{\{x_j, y_k\}}^{k=1,\dots,r}_{j=1,\dots,n}$
is linearly independent over $H_1$ in $M_{1}$,
the set $\{\nuu{},\nux{}\mid u\in\mathcal{U},x\in\mathcal{X}\}$ %^{k=1,\dots,r}_{j=1,\dots,n}$
is linearly independent over $\exs H_1$ in $\rd(M)$, and $\{\lay{}\mid y\in\mathcal{Y}\}$ %_{k=1,\dots,r}$
is an independent subset of $\rd(H)$.

To obtain \pref{sangennaro} from \pref{mustiola} we adopt the following steps.
Let first $\mathcal{X}$ be a maximal subset of $\mathcal{Z}$ with the property that $\{\nux{}\mid x\in\mathcal{X}\}$ is an
$\exs H_{1}$-independent subset of the $\nu_{z}$'s. We transform by bilinearity of the Lie product,
the sum $\sum_{z\in\mathcal{Z}}[\nu_{z},z]$ %$\sum_{j=1}^{n_{1}}[\nuw{j}, w_j]$
of \pref{mustiola} into
%Assume
%%\pref{mustiola} has been trasformed, in such a way that a first segment of the sum
%%has the desired independence properties:$$
%we have
%$$
%\sum_{j=1}^{n_{1}}[\nuw{j}, w_j]=\sum_{j=1}^{n_0}[\nux{j}, x_j]+\sum_{k=1}^{r_0}[\lay{k},y_k]+[\nu,z]+\sum_{\beta}[\nu_{\beta},z_{\beta}]
%$$
%where ${\{\nux{j}\}}_{j}$ is a maximal set of $\nuw{}$'s which are independent over $\exs H_{1}$ and
%the $\lay{}$'s are independent inside $\rd(H)$.  
%
%If now $\{\nux{1},\dots,\nux{n_0},\nu\}$ is not independent over $\exs H_1$,
%%and $(x_{j},y_{k},z,z_{\beta})$ is independent over $H_1$.
%then ${\nu}=\summ{j}{s_{j}}{\nux{j}}+
%\summ{k}{\mu_k}{\lay{k}}+h$, with $h\in \rd(H)$, $h$ indepent over ${\{\lay{k}\}}_{k=1}^{r_0}$. %and  $[\nu,z]=$.
%
%Now put $\lay{r_0+1}=h$ and $y_{r_0+1}=z$, distribute $\nu$ by bilinearity, and obtain
%$$
%%=\sum_{i=1}^{m_0}[\nuu{i}, u_i]+
%\sum_{j=1}^{n_{1}}[\nuw{j}, w_j]=\sum_{j=1}^{n_0}[\nux{j}, x_j+s_{j}z]+\sum_{k=1}^{r_0}[\lay{k},y_k+\mu_kz]
%+[\lay{r_0+1},y_{r_0+1}]+\sum_{\beta}[\nu_{\beta},z_{\beta}].
%$$
%Note that the set ${\{x_j+s_{j}z,\, y_k+\mu_kz,\, y_{r_{0}+1},\,z_{\beta}\}}_{j,k,\beta}$
%is still independent over $H_1$.
%Changing names to the second entries in the Lie bracket we have
%$$
%\sum_{j=1}^{n_{1}}[\nuw{j}, w_j]=\sum_{j=1}^{n_0}[\nux{j}, x_j]+\sum_{k=1}^{r_{0}+1}[\lay{k},y_k]
%+\sum_{\beta}[\nu_{\beta},z_{\beta}].
%$$
%Repeated application of this procedure will lead to
$$ %\sum_{j=1}^{n_{1}}[\nuw{j}, w_j]=\sum_{j=1}^{n}[\nux{j}, x_j]+\sum_{k=1}^{r}[\lay{k},y_k]
\sum_{x\in\mathcal{X}}[\nux{}, x]+\sum_{y\in\mathcal{Y}}[\lay{},y]
$$ %with the $x_{j},y_{k}$'s independent over $H_{1}$, the $\nux{j}$'s independent over $\exs H_{1}$ and
%the $\lay{k}$'s linearly independent inside $\rd(H)$.
where $\mathcal{Y}$ is a subset of $\mathcal{Z}$ such that $\mathcal{XY}$ and the $\lay{}$'s have the properties claimed for \pref{sangennaro}.

To get \pref{sangennaro} we now modify the set $\{\nuu{},\nux{}\mid u\in\mathcal{U}_{0},x\in\mathcal{X}\}$ to get an
$\exs H_{1}$-independent one, this will possibly reduce the length $\card{\mathcal{U}_{0}}$ of the first segment of \pref{mustiola}.

Assume $\mathcal{U}$ is a maximal subset of $\mathcal{U}_{0}$ for which %the tuple $\overline{\nu}_{u}$ is a maximal subset of the
%$\nuu{i}$'s which 
$\{\nuu{}\mid u\in\mathcal{U}\}$ is linearly independent
over $\genp{\exs H_{1},\nux{}\mid x\in\mathcal{X}}$ in $\exs M_{1}$. Then for any $u_{0}\in\mathcal{U}_{0}\non\mathcal{U}$ we have
$$\nuu{0}=\sum_{u\in\mathcal{U}}t_{u}\nuu{}+\sum_{x\in\mathcal{X}}s_{x}\nux{}+\eta$$
where $t_{u}$ and $s_{u}$ are in $\Fp$ for all $u$ and $\eta$ is an element of $\rd(H)$. This yields
\begin{labeq}{sanfanulo}
[\nuu{0},u_{0}]=\sum_{u\in\mathcal{U}}[\nuu{},t_{u}u_{0}]+\sum_{x\in\mathcal{X}}[\nux{},s_{x}u_{0}]+[\eta,u_{0}].
\end{labeq}
Since $w$ is not in $\J(H)$, by replacing $w$ with $w-[\eta,u_{0}]$, we still obtain an element of $\fla{3}{H_{1}}\cap\J(M)$
which does not belong to $\J(H)$. On the other hand we merge\footnote
{This is actually how we reached expressions \pref{mustiola} and \pref{sangennaro}.}
the remaining terms of \pref{sanfanulo}
into
$$ %\begin{labeq}{sanfatucchio}
w=\!\!\!\!\sum_{%\substack{
u\in\mathcal{U}_{0}\non\mathcal{U}, u_{0}}\!\!\![\nuu{},u]+\sum_{u\in\mathcal{U}}
[\nuu{},u+t_{u}u_{0}]+\sum_{x\in\mathcal{X}}[\nux{},x+s_{x}u_{0}]+\sum_{y\in\mathcal{Y}}[\lay{},y]
$$ %\end{labeq}
%Now the maps
%$$\mathcal{U}\ni u\mapsto u+t_{u}u_{0}\quad\text{and}\quad\mathcal{X}\ni x\mapsto x+s_{x}u_{0}$$
%preserve the linear independence of $\mathcal{U}$ and the independence of $\mathcal{X}$ over $\genp{\mathcal{Y},H_{1}}$.
Now the sets $$\{u+t_{u}u_{0}\mid u\in\mathcal{U}\}\quad\text{and}\quad\{x+s_{x}u_{0}\mid x\in\mathcal{X}\}$$
are again respectively linearly independent and linearly independent over $\genp{\mathcal{Y},H_{1}}$.

Iterating this step for all $u\in\mathcal{U}_{0}\non\mathcal{U}$ -- each time renaming
the $u$'s, the $x$'s and $w$ -- we reach the desired expression \pref{sangennaro}. At the end
\pref{sangennaro} is not trivial if $w$ doesn't lay in $\J(H)$.

\medskip
Now arrange the above sets into a linearly ordered base $\mathcal{M}$ of $M_1$ according to the following hierarchy:
$$
\mathcal{M}=\{\bu>\hu>\mathcal{X}>\mathcal{Y}>\vu\}
$$
%with $\bu=\{u_i\}$,
where $\hu$ is a completion of $\bu$ to a base of $H_1$,
%$\mathcal{X}=\{x_j\}$, $\mathcal{Y}=\{y_k\}$ and
$\vu$ is a completion of $\bu\hu \mathcal{X} \mathcal{Y}$ to a base
of $M_{1}$ and each of the above subparts of $\mathcal{M}$ is ordered in some way.

According to Definitions \ref{basicommutators} and \ref{supp} and Fact \ref{ubc},
we write elements $\nuu{}$, $\nux{}$ and $\lay{}$ in \pref{sangennaro} as $\Fp$-linear combinations
of basic $\mathcal{M}$-monomials of weight $2$.
%with respect to the base $\mathcal{M}$, according to the ordering chosen above.

As a result, for suitable scalars $\theta_{\alpha}\in\Fp$, the sum \pref{sangennaro} becomes a linear combination
\begin{labeq}{santarita}
\summ{\al}{\theta_{\al}}{[a_{\al},b_{\al},z_{\al}]}
\end{labeq}
of left-normed commutators of weight $3$, where each monomial $[a_{\al},b_{\al},z_{\al}]=[a,b,z]$ has
$a,b$ laying in $\mathcal{M}$ with $a>b$ while $z$ belongs to $\bu\mathcal{X}\mathcal{Y}$.
If in addition $z\geq b$, the term $\trec{a}{b}{z}$ is a basic monomial of weight $3$.
If on the contrary $a>b>z$, then we call the monomial $\trec{a}{b}{z}$ a \emph{prebasic} monomial.

Applying the Jacobi Identity, every prebasic monomial $\trec{a}{b}{z}$ can be transformed (cfr.\,\cite[p.577]{mhalll})
in the sum of two basic commutators, namely
\begin{labeq}{prebi}
\trec{a}{b}{z}=\trec{a}{z}{b}-\trec{b}{z}{a}.
\end{labeq}

On the other hand, with Fact \ref{ubc} again, as an element of $\fla{3}{H_1}$, %\inn\fla{3}{\mathcal{M}}$,
$w$ admits a unique expression as a linear combination $\mathcal{B}^H$ of basic monomials over $\bu\hu$ of weight $3$.

We have then
\begin{labeq}{basipre}
\mathcal{B}^H=w=\mathcal{B}+p\mathcal{B}=\mathcal{B}+\mathcal{B}_*
\end{labeq}
where $\mathcal{B}$, $p\mathcal{B}$ are sums of respectively basic and prebasic commutators over $\mathcal{M}$
representing $\pref{santarita}$ and $\mathcal{B}_{*}$ is the sum of basic $\mathcal{M}$-monomials arising from $p\mathcal{B}$
by means of substitutions \pref{prebi}.

By abuse of notation, we let $\mathcal{B}$, $\mathcal{B}^{H}$, $p\mathcal{B}$ and $\mathcal{B}_{*}$ also denote the {\em sets} of
monomials which appear in the corresponding sum.

From a comparison of equality $\mathcal{B}^{H}=\mathcal{B}+\mathcal{B}_{*}$ %in $\pref{basipre}$
and by unicity in Fact \ref{ubc}, it follows that 
$\mathcal{B}^{H}\inn \mathcal{B}\mathcal{B}_{*}$ and
%in $\mathcal{B}+\mathcal{B}_*$ will be cancelled all the terms that do not appear in $\mathcal{B}^H$.
each basic $\mathcal{M}$-monomial in $\mathcal{B}\mathcal{B}_{*}$ which is not in $\mathcal{B}^{H}$, must be cancelled
from the sum $\mathcal{B}+\mathcal{B}_{*}$ by the same commutator with opposite coefficient,
the latter laying again in $\mathcal{B}\mathcal{B}_{*}$.
This happens in particular of all basic terms containing elements of $\mathcal{M}$ which are not in $H_1$.

Assume a term $\trec{a}{b}{z}$ appearing in \pref{santarita} as a $\mathcal{B}$-element is to be cancelled,
then the same commutator, with opposite sign will be necessarily found in $\mathcal{B}_{*}$ and not of course in $\mathcal{B}$ again.
The same holds with the roles of $\mathcal{B}$ and $\mathcal{B}_{*}$ exchanged.

Also notice that the elements of $\mathcal{M}$, appearing in the rightmost entry of the
Lie brackets in \pref{santarita} force the sum to be grouped after the labels $\bu$, $\mathcal{X}$ and $\mathcal{Y}$.
\begin{itemize}
\punto{Claim 1}Monomials $\trec{a}{b}{z}$ appearing in \pref{santarita} do not have entries from $\vu$.
\end{itemize}
Assume on the contrary, \pref{santarita} contains a commutator $\trec{a}{v}{z}$ with $v\in\vu$, $z\in\bu \mathcal{X}$.
Then necessarily $z\geq v$ and $\trec{a}{v}{z}$ is basic. It follows that, monomials whose support meets $\vu$
cannot appear in $p\mathcal{B}$ and then $\mathcal{B}_{*}$-basic terms will not contain $\vu$-elements.
By the above remarks, we conclude, there is no hope for  $\trec{a}{v}{z}$ to be cancelled from $\mathcal{B}$ and this implies such terms simply don't occur.

In particular, all $\nuu{}$ and $\nux{}$ have support contained in $\bu\hu \mathcal{X}\mathcal{Y}$.

%\medskip
\begin{itemize}
\punto{Claim 2}The sum \pref{sangennaro} contains terms $[\nu_{\ast}, \ast]$ with $\ast\in \mathcal{X}\mathcal{Y}$. Concisely
$\mathcal{X}\mathcal{Y}\neq\vac$. Moreover $\bu$ cannot be empty either.
\end{itemize}
Assume $w=\sum_{u\in\mathcal{U}}[\nuu{}, u]$ only. As the $\nuu{}$'s are independent over $\exs H_1$, then there is at least one
monomial $\trec{a}{b}{u}$ of \pref{santarita} with 
$[a,b]$ not entirely supported on $\bu\hu$. This would contradict (Claim 1).

We also have $\mathcal{U}\neq\vac$, for if in $\mathcal{B}+\mathcal{B}_*$ every monomial contains an element from $\mathcal{X}\mathcal{Y}$, %$w$ would not belong to $\fla{3}{H_{1}}$.
%nothing remains after cancellation of non-$H_1$ terms. 
the entire expression would cancel although $w$ is nontrivial.
\begin{itemize}
\punto{Claim 3}The support of each $\lay{}$ is contained in $\bu$.
\end{itemize}
Assume not. Then in \pref{santarita} appears a term $\trec{a}{b}{y}$ with $a$ or $b$ in $\mathcal{H}$.
As $a>b$ and $\bu>\hu>\mathcal{Y}$, it follows necessarily
$b\in\hu$ and $\trec{a}{b}{y}$ is prebasic, its transformation in two $\mathcal{B}_*$-elements produces basic terms
$\trec{a}{y}{b}$ and $\trec{b}{y}{a}$, both not in $\mathcal{B}^{H}$.
On the other hand, the commutator $\trec{a}{y}{b}$ cannot be found in part
$\mathcal{B}$ of \pref{basipre} and will not be cancelled. This is a contradiction

\medskip
We eventually prove the assertion of the lemma contradicting the self-sufficiency of $H$ in $M$.

Consider the subspace $C_{1}=\genp{H_1,\mathcal{X},\mathcal{Y}}$ of $M_{1}$.
On one hand by (Claim 2) $\mathcal{U}\neq\vac$ and $C\supsetneq H$, while (Claim 3) together with axiom $\sig{2}{2}$ imply $\card{\mathcal{Y}}<\card{\mathcal{U}}$ as
$\delta(\mathcal{U})$ must be positive and the $\lay{}$'s are in $\exs\genp{\bu}$.

On the other hand by (Claim 1), since the $\nuu{},\nux{}$'s are relators in $\rd(M)$, which are linearly independent over $\exs H_{1}$
and have support inside $C_{1}$, we have $\delta(C/H)=\dfp(C_{1}/H_{1})-\dfp(\rd(C/H))\leq\card{\mathcal{X}}+\card{\mathcal{Y}}-
(\card{\mathcal{X}}+\card{\mathcal{U}})<0$ which is impossible.
The proof is now complete.
\end{proof}
\section{Predimensions for the nil-$3$ Class}\label{preditre}
For the rest of the chapter we assume a prime $p$ has been fixed, greater than $3$.
We will consider Lie algebras of $\nla{3}$.

\smallskip
For a chosen $M$ in $\nla{3}$ and any $\nla{3}$-subalgebra $A$ of $M$,
the truncation $A_{*}$ of Section \ref{freelift}, is $\nla{2}$-isomorphic to $\gena{A_{1}}{M_{*}}$. These algebras will be identified in the
sequel.

This means, we can apply on $M_{1}$ the ``pregeometric machinery'' introduced in Chapter \ref{due},
associated to $M_{*}$.
With this purpose we rename by $\delta_{2}$, the nil-$2$ deficiency $\delta$ of Definition \ref{deficienzwei} and set, for finite subspaces
$A_{1}$ of $M_{1}$
$$\delta_{2}(A_{1})=\dfp(A_{1})-\rd(A_{*}).$$
Following our previous convention, the same integer will be equal to $\delta_{2}(A)$ for ease of notation.

On the same line, for any $\nla{3}$-extension $M$ of $N$, we will
write $N\zsu[2]{}M$ if $N_{*}$ is self-sufficient in $M_{*}$ according to Definition \ref{2strong}.
The same meaning is attributed to the expression $N_{1}\zsu[2]{}M_{1}$.

Moreover, for a given $M$ in $\nla{3}$, we denote by $d_{2}^{M}$ the
dimension function on $M_{1}$ obtained by the predimension $\delta_{2}$.

\medskip
On the other hand for an $\nla{3}$-algebra $A$, we have from \pref{morphipres} of Section \ref{maxrels} above,
the following {\em lifted} presentation:
\begin{labeq}{liftpres}
\rt(A)\lto\fl{A_{*}}\stackrel{\pam{A}}{\lto}A
\end{labeq}
and in particular
$$A\simeq_{\nla{3}}\frac{\fl{A_{*}}}{\rt(A)}=A_{*}\oplus\frac{(\fl{A_{*}})_{3}}{\rt(A)}.$$

This yields a new integer invariant attached to $\nla{3}$-objects, defined in the following
\begin{dfn}
For a finitely generated $\nla{3}$-algebra $A$ we define the
$\nla{3}$-deficiency as the integer
\begin{labeq}{deltre}
\delta_{3}(A)=\delta_{2}(A)-\dfp\left(\rt(A)\right).
\end{labeq}
%where $\rt(A)$ is the kernel of the canonical map $\map{\pam{A}}{\fl{A_{*}}}{A}$ defined in \pref{morphipres}.
In particular $\delta_{3}(A)$ depends only of the quantifier-free $\Lan{3}$-diagram
of $A$. As a consequence a possible lower bound to $\delta_{3}$ in $M_{1}$ is axiomatisable via $\Lan{3}$-sentences.
\end{dfn}

\smallskip
In the scope of section \ref{schur}, if we compare \pref{deltre} above with \pref{LieDef} on page \pageref{LieDef},
we obtain the same thing, just
differently organised. In fact if we consider \pref{LieSchur} we get
$$\dfp(H_{2}(A,\nla{3}))=\dfp(\rd(A_{*}))+\dfp(\rt(A)).$$

\medskip
This said, in view of a definition of self-sufficiency for $\nla{3}$-extensions, we need a different
notion of predimension, which emulates the local property \pref{erredi} of $\rd$ and ease computations for a future
notion of free $\nla{3}$-amalgam.

The point here is that for an arbitrary $A\inn M$ like above, $\rt(A)$ is not in general a subspace of $\rt(M)$.

\smallskip
Take an $\nla{3}$-inclusion $i\colon H\inn M$, this means as usual $H=\gena{H_{1}}{M}$ for $H_{1}\inn M_{1}$ and consider
the truncation to $\nla{2}$, $i_{*}\colon H_{*}\inn M_{*}$.
We denote by $\gam{H}{M}$ the map described in \pref{gam}
\begin{labeq}{gammalift}
\lmap{\gam{H}{M}:=\frl(i_{*})}{\fl{H_{*}}}{\fl{M_{*}}}.
\end{labeq}
By \pref{liftbild} and \pref{kerfli}, %if $\kerg{H}{M}$ denotes the kernel of $\gam{H}{M}$,
we have
\begin{labeq}{immaker}
\im(\gam{H}{M})=\gena{H_{1}}{\fl{M_{*}}}\quad\text{and}\quad%\kerg{H}{M}
\ker(\gam{H}{M})=\frac{\fla{3}{H_{1}}\cap\J(M_{*})}{\J(H_{*})}
\end{labeq}
while by Proposition \ref{bellemma} we obtain for all $\nla{3}$-subalgebras $H\inn M$ as above
\begin{cor}\label{corembel}
If $H\zsu[2]{}M$ then $\map{\gam{H}{M}}{\fl{H_{*}}}{\fl{M_{*}}}$ is injective.
\end{cor}

We want also, the new presentation obtained in \pref{liftpres} to interact with subalgebras, that is
\begin{lem}\label{gammap}
For an $\nla{3}$-subalgebra $H$ of $M$, we obtain the following commutative diagram with exact rows.
\begin{labeq}{liftpresmorph}
\begin{split}
\xymatrix@!C{
\rt(H)\ar^{\gam{H}{M}}[d]\ar[r]&\fl{H_{*}}\ar^{\gam{H}{M}}[d]\ar^{\pam{H}}[r]&\extracolsep{3cm} H\ar^{i}[d]\\
\rt(M)\ar[r]&\fl{M_{*}}\ar^{\pam{M}}[r]&\extracolsep{3cm}M}
\end{split}
\end{labeq}
In particular $\ker(\gam{H}{M})\inn\rt(H)$.
%\uwave{In general for a given $M$ we may consider the directed system $(\fl{H_{*}},\gam{H}{K}\mid H_{1}\inn K_{1}\inn M_{1})$.}
\end{lem}
\begin{proof}
We show that the rightmost square in \pref{liftpresmorph} commutes. This follows by proposition \ref{morphifreelift}
and the fact $(\gam{H}{M}\pam{M})\ast=\ast\,i_{*}=(\pam{H}i)\ast$ applied to the diagrams
$$\xymatrix@C+7mm{
\fl{H_{*}}\ar^{\ast}[d]\ar^{\gam{H}{M}\pam{M}}@<+3pt>[r]\ar_{\pam{H}i}@<-3pt>[r]&M\ar^{\ast}[d]\\
H_{*}\ar_{i_{*}}[r]&M_{*}}$$
\end{proof}

This allows us to define a more {\em adaptive} deficiency, which depends
of the embedding in the ambient structure $M$.
\begin{dfn}\label{ded}
Let $M$ be an $\nla{3}$-algebra. For any $H_{1}\inn M_{1}$. We set
$$\rt_{M}(H)=\rt_{M}(H_{1})=\gam{H}{M}\left(\rt(H)\right)$$
%as the space of the {\em relative relators} for $H$ in $M$.
and define for finitely generated $H$% the {\em relative deficiency} as
\begin{labeq}{dedef}
\ded^{M}(H)=d^{M}_{2}(H_{1})-\dfp(\rt_{M}(H))
\end{labeq}
and for any $\nla{3}$-subalgebra $N$ of $M$ and finite $H_{1}\inn M_{1}$
\begin{labeq}{reldedef}
\ded^{M}(H/N)=d^{M}_{2}(H/N)-\dfp(\rt_{M}(H/N))
\end{labeq}
where
$\rt_{M}(H/N)$ is the quotient $\Fp$-vector space $%\frac{
\rt_{M}(N+H)/\rt_{M}(N)$.
As before for $N+H$ is meant $\gena{N_{1}+H_{1}}{M}$ and in general any of the expressions above are allowed to carry
indices $_{1}$. In fact as in the nil-$2$ case,
we are searching for a notion of predimension -- and eventually a pregeometry -- which is concerned with sets from the {\em sort}
$M_{1}$. 
\end{dfn}
\begin{rem}\label{vecchialenza}
By \pref{immaker} and \pref{liftpresmorph} we have
\begin{itemize}
\item[1.]for $H_{1}\inn N_{1}\inn M_{1}$ we have
\begin{labeq}{dedim}
\rt_{M}(H)=\rt(M)\cap\im(\gam{H}{M})=\rt(M)\cap\gena{H_{1}}{\fl M}
\end{labeq}
\item[2.]since $\gam{H}{N}\gam{N}{M}=\gam{H}{M}$, $\rt_{N}(H_{1})$ maps {\em onto} $\rt_{M}(H_{1})$ via $\gam{N}{M}$. 
\end{itemize}
\end{rem}
In particular we obtain a form of the local relators, which is a lot similar to $\rd_{M}(H)$ in \pref{erredi} of Chapter \ref{due}.

\medskip
Assume $A$ is a finite $\nla{3}$-subalgebra of $M$ with $A\zsu[2]{}M$, by Corollary \ref{corembel} above follows,
that $\rt(A)\simeq_{\Fp}\rt_{M}(A)$ and hence, as $d_{2}^{M}(A)=\delta_{2}(A)$, we have $\delta_{3}(A)=\ded^{M}(A)$.
In particular, by Lemma \ref{samed2} and Remark \ref{vecchialenza}.(2.)\,we have.
\begin{lem}\label{sameded}
For a given $\nla{3}$-algebra $M$, the integers $\ded^{M}(A)$ and $\delta_{3}(A)$ do coincide on all finitely generated $\nla{3}$-subalgebras $A$ of $M$ when $A_{*}$ is self-sufficient in $M_{*}$ with respect to $\delta_{2}$.

Moreover $\ded^{M}$ coincides with $\ded^{N}$ on the subspaces of $N_{1}$, for all $\delta_{2}$-strong extensions $N\zsu[2]{}M$.
\end{lem}
In Section \ref{classetre} below we actually show that $\delta_{3}$ and $\ded$ are always comparable, in the direction
$\ded^{M}\leq\delta_{3}$ for all $M$.

\smallskip
It is worth to mention here, that for a given $M$ and subspaces $H_{1}$, $K_{1}$ of $M_{1}$, it is not in general the case that
\begin{labeq}{notmodular}
\gena{H_{1}}{\fl M}\cap\gena{K_{1}}{\fl M}\quad\text{and}\quad\gena{H_{1}\cap K_{1}}{\fl M}
\end{labeq}
coincide.

It is also not true in general
that $\rt_{M}(H_{1})\cap\rt_{M}(K)$ equals $\rt_{M}(H_{1}\cap K_{1})$ and the analogous of {\em submodularity} \pref{submod}
of Section \ref{sec:deltadue} for
$\delta_{3}$ and $\ded$ may fail. In fact we have
\begin{rem}\label{nossummo}
$\ded^{M}$ and $\delta_{3}$ are not in general submodular.
\end{rem}
\begin{proof}
Consider the $\nla{3}$-algebra given by the presentation $M=\gen{M_{1}\mid R}$, where $M_{1}$ is the vector space over $\Fp$ with
basis $\{a,b_{1},\dots,b_{4},m_{1},m_{2}\}$ and $R$ is the ideal of $\fla{3}{M_{1}}$ generated by the relators
\begin{gather}
\begin{split}
\rho=[a,b_{1}]-[b_{2},b_{3}]
\end{split}\qquad
\begin{split}
\alpha=&[b_{2},b_{3},b_{4}]-[m_{1},m_{2},m_{2}],\\
\beta=&[b_{2},b_{3},m_{1}]-[b_{4},m_{2},m_{2}].
\end{split}
\end{gather}
In this algebra, $\J(M)$ is the ideal of $\fla{3}{M_{1}}$ generated by $\rho$.

Set now $N_{1}$ the subspace of $M_{1}$ generated by $b_{1},\dots,b_{4},m_{1},m_{2}$ and $E_{1}=\genp{a,b_{1},b_{4},m_{1},
m_{2}}$. We have $\alpha\equiv[a,b_{1},b_{4}]-[m_{1},m_{2},m_{2}]$ and $\beta\equiv[a,b_{1},m_{1}]-[b_{4},m_{2},m_{2}]$ modulo
$\J(M)$, and hence $\bar\alpha\in\rt_{M}(E_{1})\non\rt_{M}(E_{1}\cap N_{1})$.

As $\gen{N_{1},a}=M$, we have $d_{2}^{M}(E/N)=d_{2}^{M}(a/N)=\delta_{2}(a/N)=0$ while $d_{2}^{M}(E_{1}/E_{1}\cap N_{1})=
\delta_{2}(E_{1}/E_{1}\cap N_{1})=1$. This means
$$0=\ded^{M}(E/N)>\ded^{M}(E_{1}/E_{1}\cap N_{1})=-1$$
and, as $E,N\zsu[2]{}M$, $\delta_{3}(E_{1}+N_{1})+\delta_{3}(E_{1}\cap N_{1})>\delta_{3}(E)+\delta_{3}(N)$.
\end{proof}

\medskip
We now define self-sufficiency on extensions of $\nla{3}$-algebras.
\begin{dfn}\label{trestrongness}
We say that an $\nla{3}$-subalgebra $H$ of $M$ is self-sufficient and write $H\zsu[3]{}M$ if
\begin{itemize}
\item[-] $H\zsu[2]{}M$ and
\item[-] $\ded^{M}(E\quot H)\geq0$ for all finite subspaces $E_{1}\inn M_{1}$.
\end{itemize}
By the first condition, and the following Lemma \ref{deltaded}, it is possible to express
$H\dsu M$ in terms of $\delta_{3}$ as well\footnote{
One defines of course $\delta_{3}(E/H)$ as $\delta_{2}(E/H)-\dfp(\rt(H+E)/\rt(H))$.}.
For a finite $H$ -- say -- this property is
actually part of the elementary type of $H$.
\end{dfn}
\subsection{A first asymmetric Amalgamation in $\nla{3}$}
\documentclass[a4paper,11pt,german,english]{article}
\usepackage{babel}
\usepackage[latin1]{inputenc}
\usepackage{amsmath,amsfonts,amssymb,amsthm}
\usepackage{ModNet}
\usepackage{FHL}
\usepackage{pdfsync}
%\usepackage{syntonly}
\usepackage[all]{xy}
%\linespread{1.3}
%\usepackage[mathscr]{euscript}
\title{Asymmetric Amalgam}
%\renewcommand{\thesection}{\arabic{section}} 
%\setcounter{tocdepth}{3}
%\syntaxonly
%\includeonly{}
\begin{document}
\maketitle\noindent
%\tableofcontents
Starting with $M\nni B\dsu A$ nil-$3$ algebras, we use the following

\smallskip
{\bf Notation and Setting:}
\begin{itemize}
\item[-]Asterisks in the subscript substitute the old $\tr{2}$ i.e. $M_{\ast}=\tr{2}M$
\item[-]$L_{*}:=M_{*}\amalg_{B_{*}}A_{*}$ {\em the $\Kl^{2}$ free (graded) amalgam}
\item[-]$\ftr{L}$ {\em is the ``free lift''} of $L$ or of $L_{*}$ i.e. $\fr{3}L$ or $\fr{3}\tr{2}{L}$ in the old notation, according to whether $L$ is in $\nla{2}$ or $\nla{3}$.
\item[-]if $L_{*}^{\sss +}:=\ta L_{*}$ {\em the $\ta$-closure of $L_{*}$} then we set $\ftr{L^{\sss +}}=\fr{3}{(L_{*}^{\sss +})}$
\item[-]$\gam{X}{Y}$ is the canonical map of $F_{X}$ into $F_{Y}$ with image $\gena{X_{1}}
{F_{Y}}$.
\end{itemize}

\medskip
We first prove a result concerning intersection of subalgebras under free lifting
%!TEX encoding = UTF-8 Unicodestrengthen an older result,
which will be crucial in the following.

Assume $L\in\nla{2}$ and $H=\gena{H_{1}}{L}$ and
$K=\gena{K_{1}}{L}$ are subalgebras of $L$,
%such that $\gena{H_{1}}{{L}}\cap\gena{K_{1}}{{L}}=\gena{H_{1}\cap K_{1}}{{L}}$.
then we ask whether
\begin{itemize}
\item[(d)]\quad\quad$\gena{H_{1}}{F_{L}}\cap\gena{K_{1}}{F_{L}}=\gena{H_{1}\cap K_{1}}{F_{L}}.$

A necessary condition to get (d) is first to get this intersection in weight $2$. Observe
that if $M\nni B\inn A$ are $\Kl^{2}$-algebras and if $L$ denotes their amalgam as above,
then for any $E_{1}\nni M_{1}$ and $D_{1}\nni A_{1}$ one has $\gena{E_{1}}{L}\cap\gena{D_{1}}{L}=\gena{E_{1}\cap D_{1}}{L}$. This is for if $w_{E}-w_{D}\in N^{2}(L)=N^{2}(M)+N^{2}(A)$ where $w_{E}$ and $w_{D}$ are homogeneous polynomials of weight $2$, then $w_{E}-v_{M}=w_{D}+v_{A}\in\exs E_{1}\cap D_{1}$ for some $v_{A}$ and $v_{M}$.

%Naturally a symmetric result holds
We now prove (d) in a very special case.
\end{itemize}

\begin{lem*}
Assume $L,M,A,B$ as above and $E_{1}\nni M_{1}$. Then
\begin{itemize}
\item[$(*)$]\quad\quad$\gena{E_{1}}{F_L}\cap\gena{A_{1}}{F_L}=\gena{E_{1}\cap A_{1}}{F_L}$
\end{itemize}
\end{lem*}
\begin{proof}
Suppose $w_{E}-w_{A}\in\J(L)$ where we
may assume $w_{E}$ and $w_{A}$ are homogeneous polynomial of weight $3$ lying
in $\fla{3}{E_{1}}$ and $\fla{3}{A_{1}}$ respectively. Both these algebras are considered
free subalgebras of $\fla{3}{L_{1}}$.

We arrange a basis $X$ for $L_{1}$ as follows $X=\{X^{a}>X^{e}>X^{B}>X^{m}\}$, where
$X^{B}$ is a basis for $B_{1}$, $X^{B}X^{m}$ is a basis for $M_{1}$, $X^{m}X^{B}X^{e}$ is
a basis for $E_{1}$ and $X^{e}X^{B}$ is a basis for $E_{1}\cap A_{1}$. With $X^{a}$ we complete
$X^{e}X^{B}X^{m}$ to a basis for $L_{1}$ and $X^{a}X^{e}X^{B}$ is a basis of $A_{1}$.


Now  since $\J(L)_{3}=\gen{\J(A),\J(M),[N^{2}(A),M_{1}],[N^{2}(M),A_{1}]}$ as an $\Fp$-subspace,
without loss of generality $w_{E}-w_{A}$, may be written as a sum of terms like
$[\nu_{M},x]$ and %$[\nu_{M},y]$ and
$[\nu_{A},y]$. Here and below, we consider $x\in X^{a}$ % $y\in X^{e}$
and $y\in X^{m}$, moreover $\nu_{A}\in N^{2}(A)$ and $\nu_{M}\in N^{2}(M)$.

Once each $\nu_{A}$ and $\nu_{M}$ is expressed as sums of basic monomials over $X$, we obtain an equality in $\fla{3}{L_{1}}$:

$$w_{E}-w_{A}=\mathcal{B}^{M,a}+%\mathcal{B}^{M,e}+
p\mathcal{B}^{A,m}$$

where 
\begin{itemize}
\item[$\mathcal{B}^{M,a}$] is a sum of basic terms $[m_{1},m_{2},x]$ for $m_{i}\in X^{B}X^{m}$
%\item[$\mathcal{B}^{M,e}$] is a sum of basic terms $[m_{1},m_{2},y]$ for $m_{i}\in X^{B}X^{m}$
\item[$p\mathcal{B}^{A,m}$]is a sum of prebasic terms $[a_{1},a_{2},y]$ for $a_{i}\in X^{a}X^{e}X^{B}$.
\end{itemize}
We now transform all prebasic monomials above into $[a_{1},y,a_{2}]-[a_{2},y,a_{1}]$ which are basic
and whose sum we denote by $\mathcal{B}^{A,m}_{*}$.

We obtain a sum of basic commutators
$$w_{E}-w_{A}=\mathcal{B}^{M,a}+
\mathcal{B}^{A,m}_{*}.$$

Since an expression in basic monomials for $w_{E}-w_{A}$ does not involve
{\em mixed} terms (i.{}e.{\,}monomials supported on both $X^{a}$ and $X^{m}$), by uniqueness all mixed terms must cancel each other from the sum $\mathcal{B}^{M,a}+%\mathcal{B}^{M,e}+
\mathcal{B}^{A,m}_{*}$. Cancellations do not arise within the same group, the only possibility
instead is that mixed $\mathcal{B}^{A,m}_{*}$-monomials cancel mixed $\mathcal{B}^{M,a}$-monomials and vice versa.

Consider indeed a term $[m_{1},m_{2},x]$ above with $m_{2}\in X^{m}$, %(note that such a term has to exist in $\mathcal{B}^{M,a}$),
%the corresponding prebasic which is meant to cancel it will be 
this is to be neutralised by the prebasic commutator
$[x,m_{1},m_{2}]$ of $p\mathcal{B}^{A,m}$, hence $m_{1}\in X^{B}$ and $\mathcal{B}^{A,m}_{*}$ contains the basic commutator $[x,m_{2},m_{1}]$ which differs from any $\mathcal{B}^{M,a}$-term. We deduce that no mixed $\mathcal{B}^{M,a}$-term is present in the sum above, and, with much similar arguments no
mixed $p\mathcal{B}^{A,m}$-term shows up as well.

We showed $w_{E}$ and $w_{A}$ lies, modulo $\J(L)$, in $\fla{3}{E_{1}\cap A_{1}}$.
\end{proof}

%\medskip
%Denote $D_{1}=H_{1}\cap K_{1}$, and set $\mathcal{H}:=(\gena{H_{1}}{F_{L}})_{3}$, $\mathcal{K}:=(\gena{K_{1}}{F_{L}})_{3}$, $\mathcal{D}:=(\gena{D_{1}}{F_{L}})_{3}$.

%\smallskip
%We artificially induce a graded Lie algebra structure on
%$$X=L\oplus\left( \mathcal{H}\oplus_{\mathcal{D}}\mathcal{K}\right)$$
%if we define Lie brackets to be nontrivial
%only for products of weight not greater than $3$, which arise from $\gena{H_{1}}{L}$
%or $\gena{K_{1}}{L}$.\footnote{this means $[w,y]\neq\triv$ if both $w$ and $y$
%belong to $\gena{H_{1}}{L}$ or $\gena{K_{1}}{L}$, moreover the value
%of $[w,y]$ is the one in $\mathcal{H}$ or in $\mathcal{K}$. This makes sense since
%$\gena{H_{1}}{F_{L}}=\gena{H_{1}}{L}\oplus\mathcal{H}$, and the same holds for $K$ and $D$}

%Now since $X_{*}=L$, we have an epimorphismus $\pam{X}$ of $\ftr{L}$ onto
%$X$. Moreover it {\em should be clear} that $\pam{X}$ maps $\gena{H_{1}}{\ftr{L}}$
%isomorphic onto $\gena{H_{1}}{X}$, and the same holds with $H_{1}$ and $K_{1}$.

%Observe that, since at the first two levels we have a good intersection, in $X$ we have
%$\gena{H_{1}}{X}\cap\gena{K_{1}}{X}=\gena{D_{1}}{X}$. Now,
%considering preimages through $\pam{X}$ we get $(\ast)$.

\bigskip\noindent
We proceed amalgamating the triple $M_{*}\nni B_{*}\zsu A_{*}$ to get $M_{*}\zsu L_{*}\nni A_{*}$, so that
$\gam{B}{A}$ and $\gam{M}{L}$ are monomorphisms.

Then set $\ke{B}{}=\ker\gam{B}{M}$ and $\ke{A}{}=\ker\gam{A}{L}$. Since $\gam{B}{M}\gam{M}{L}=\gam{B}{A}\gam{A}{L}$, we have $$\gam{B}{A}\ke{B}{}=\ke{A}{}\cap\gena{B_{1}}{\ftr{A}}$$
({\em by a basic commutator argument, similar to those used before, we can actually show that $\ke{A}{}\inn\gena{B_{1}}{\ftr{A}}$ and hence
$\gam{B}{A}\ke{B}{}=\ke{A}{}$ but what we stated should be enough for what follows}).

%\bigskip
%Thes i sthe argument \dots We choose an ordered base $X_{1}^{L}=\{X_{1}^{a}>X_{1}^{B}>X_{1}^{m}\}$ of $L_{1}$ where $X_{1}^{B}$ is a basis for $B_{1}$,
%$X_{1}^{A}=X_{1}^{a}X_{1}^{B}$ is a basis for $A_{1}$ and $X_{1}^{m}$ completes $X_{1}^{B}$ to a basis $X_{1}^{M}$ of $M_{1}$. Each of the three segments of $X_{1}^{L}$ is ordered arbitrarily.

%We consider the surjective map
%$\map{\lambda_{A}}{\fr{3}\tr{2}A}{\gena{A_{1}}{\fr{3}\lda}}$ as in \pref{lollipop}, where
%$\fr{3}\lda$ is presented by $\fla{3}{X_{1}^{L}}\quot\jei{L}$ and $\fr{3}\tr{2}A$ by $\fla{3}{X_{1}^{A}}\quot\jei{A}$.

%%To prove that $\lambda_{A}$ is an injection we have to show that
%We have $$\ker(\lambda_{A})=\frac{\fla{3}{X_{1}^{A}}\cap\jei{L}}{\jei{A}}%=\mathbf{0}
%$$
%moreover $\ker(\lambda_{A})$ is homogeneous of weight $3$.

%We note that
%\begin{labeq}{bulaba}
%\jei{L}_{3}=\gen{\jei{M}_{3},\,\jei{A}_{3},\,[\nu^{A},y],\,[\nu^{M},x]}_{+}^{\fla{3}{X_{1}^{L}}}
%\end{labeq}
%$$\text{where}\quad x\in X_{1}^{a},\,y\in X_{1}^{m},\,\nu^{A}\in N^{2}(A),\,\nu^{M}\in N^{2}(M)$$
%and we can assume both the $\nu^{A}$'s and the $\nu^{M}$'s to be independent over $\exs B_{1}$.

%Following the reasoning which led to expression \pref{basipre}, an element $w$ in $\ker(\lambda_{A})$
%%which is not zero
%may be written modulo $\jei{A}$ as
%\begin{labeq}{baluba}
%B_{A}=w=B^{M}+pB^{M}+B^{Ma}+pB^{aM}
%\end{labeq}
%where $B^{M}$, $pB^{M}$ are respectively sums of basic and prebasic commutators
%over $X_{1}^{B}X_{1}^{m}$ which cover the $\jei{M}$-part of $w$. $B^{Ma}$ is a sum of commutators
%arising from terms of type $[\nu^{M},x]$ in $w$, these are necessarily basic after the order of $X_{1}^{L}$
%we chose. $pB^{aM}$ denotes the sum of prebasic commutators obtained by the terms $[\nu^{A},y]$ of $\jei{L}$.
%$B_{A}$ is a linear combination of basic commutators over $X_{1}^{A}$ which write the word $w$
%as member of $\fla{3}{X_{1}^{A}}$.

%We transform the prebasic commutators in the right term of \pref{baluba} in basic ones,  by means of substitutions \pref{prebi}, thus we have
%$$B_{A}=B^{M}+B^{M}_{*}+B^{Ma}+B^{aM}_{*}$$
%where now both terms of the equality concern basic commutators over $X_{1}^{L}$.

%By uniqueness all
%the terms on the right which do not appear in $B_{A}$ must be cancelled by the sum.

%By the shape of the generators for $\jei{M}$ in \pref{bulaba}, we see that we fall in contradiction if
%some term of type $B^{Ma}$ or $pB^{aM}$ is present in \pref{baluba},
%because, after basic transformations, these cannot cancel each other, nor be cancelled by terms of $B^{M}+B_{*}^{M}$.

%As a consequence $B_{A}=w=B^{M}+pB^{M}\in\jei{M}$, and this forces $w$ to lay in
%$\fla{3}{X_{1}^{A}}\cap\fla{3}{X_{1}^{M}}=\fla{3}{X_{1}^{M}\cap X_{1}^{A}}=\fla{3}{X_{1}^{B}}$.

%We have
%\begin{labeq}{kerlam}
%K:=\ker(\lambda_{A})=\frac{(\jei{M}\cap
%\fla{3}{X_{1}^{B}})+\jei{A}}
%{\jei{A}}\inn\gena{B_{1}}{\fr{3}\tr{2}A}.
%\end{labeq}

%Now because $B_{1}\zsu A_{1}$, on account of lemma \ref{bellemma} we have $\jei{B}=\jei{A}\cap\fla{3}{X_{1}^{B}}$. Hence if we look at the analog of $\lambda_{A}$ for $B$ $\map{\lambda_{B}}{\fr{3}\tr{2}B}{\fr{3}\tr{2}M}$, it holds
%\begin{multline*}
%\ker\lambda_{B}=
%\frac{\jei{M}\cap\fla{3}{X_{1}^{B}}}{\jei{B}}=\\
%=\frac{\jei{M}\cap\fla{3}{X_{1}^{B}}}{\jei{A}\cap\fla{3}{X_{1}^{B}}}
%=\frac{\jei{M}\cap\fla{3}{X_{1}^{B}}}{\jei{A}\cap(\jei{M}\cap\fla{3}{X_{1}^{B}})}
%\simeq K
%\end{multline*}


\medskip
So if we write 
$\overline{F}_{\sss A}$ for $\ftr{A}\quot_{\!\ke{A}{}}$ and
$\overline{F}_{\sss B}$ for $\ftr{B}\quot_{\!\ke{B}{}}$ and we bar the maps
we take the quotients of,
%and call the respective quotient maps $
we obtain the following
{\em injective} commutative arrows
\begin{labeq}{pream}
\xymatrix{
&{\ftr{L}}&\\
{\ftr{M}}\ar%@{^{(}->}
[ur]^{\gam{M}{L}}&&{\overline{F}_{\sss A}%\quot_{\ke{A}{}}
}\ar%@{_{(}->}
[ul]_{\bgam{A}{L}}\\
&{\overline{F}_{\sss B}%\quot_{\ke{B}{}}
}\ar%@{_{(}->}
[ul]^{\bgam{B}{M}}\ar%@{^{(}->}
[ur]_{\bgam{B}{A}}&}
\end{labeq}
Recall we have $\gena{M_{1}}{\ftr{L}}=\gam{M}{L}\ftr{M}$ and $\gena{A_{1}}{\ftr{L}}=
\gam{A}{L}F_{\sss A}=\bgam{A}{L} \overline{F}_{\sss A}$.

Now since $L_{*}$ is the free amalgam of $M_{*}$ and $A_{*}$ over $B_{*}$, we have
 $\gena{M_{1}}{{L}_{*}}\cap\gena{A_{1}}{{L}_{*}}=\gena{M_{1}\cap A_{1}}{{L}_{*}}$
hence by $(\ast)$ above, $\gena{M_{1}}{\ftr{L}}\cap\gena{A_{1}}{\ftr{L}}=\gena{M_{1}\cap A_{1}}{\ftr{L}}=\gena{B_{1}}{\ftr{L}}$.

%\begin{center}
%\begin{itemize}
%\item[($\ast$)] we artificially induce a graded Lie algebra structure on
%$$X=L_{*}\oplus\left( (F_{\sss M})_{3} \oplus_{(\overline{F}_{B})_{3}}
%(\overline{F}_{\sss A})_{3} \right)$$
%if we define Lie brackets to be nontrivial
%for products of weight not greater than $3$, which arises from $\gena{M_{1}}{L_{*}}$
%or $\gena{A_{1}}{L_{*}}$ only\footnote{this means $[w,y]\neq\triv$ if both $w$ and $y$
%belong to $\gena{M_{1}}{L_{*}}$ or $\gena{A_{1}}{L_{*}}$, moreover the value
%of $[w,y]$ is the one in $\ftr{M}$ or in $\overline{F}_{\sss A}$. This makes sense since,
%$F_{\sss M}=\gena{M_{1}}{L_{*}}\oplus(F_{\sss M})_{3}$, $\overline{F}_{\sss A}=\gena{A_{1}}
%{L_{\ast}}\oplus(\overline{F}_{A})_{3}$ and $\overline{F}_{\sss B}=\gena{B_{1}}
%{L_{*}}\oplus(\overline{F}_{\sss B})_{3}$.
%}.

%Now since $X_{*}=L_{*}$, we have an epimorphismus $\pam{X}$ of $\ftr{L}$ onto
%$X$. Moreover $\pam{X}$ is injective on $\gena{M_{1}}{\ftr{L}}\simeq\ftr{M}$ and maps $\gena{B_{1}}{\ftr{L}}$
%onto $\gena{B_{1}}{X}$.

%Observe that since at the first two levels we are in a free amalgam we have
%$\gena{M_{1}}{X}\cap\gena{A_{1}}{X}=\gena{B_{1}%M_{1}\cap A_{1}
%}{X}$, now
%considering preimages through $\pam{X}$ we get the desired intersection.
%\end{itemize}
%\end{center}

\bigskip 
We now define $N_{A}:=\gam{A}{L}N^{3}(A)$
%=\bgam{A}{L}\overline{N}^{3}(A)$.
and $N_{M}:=\gam{M}{L}N^{3}(M)\simeq
N^{3}(M)$ and also set
%\begin{multline*}
$N_{B}:=\gam{A}{L}N^{3}_{A}(B_{1})=
%[EXP]
%\bgam{A}{L}( ( N^{3}_{A}(B_{1}) + \ke{A}{} )\quot \ke{A}{} )
\bgam{A}{L}(\bgam{B}{A} (N^{3}(B)\quot\ke{B}{}) )
=\gam{M}{L}(\bgam{B}{M} (N^{3}(B)\quot\ke{B}{}) )
=\gam{M}{L}N^{3}_{M}(B_{1})=N_{M}\cap\gena{B_{1}}{F_{L}}$.
%\end{multline*}

\medskip
Moreover all inclusions and arrows above preserve in $\ftr{L^{\sss +}}$ since
$\gam{L}{L^{\sss +}}$ is a monomorphism ($L_{1}\zsu L^{+}_{1}$), this allows us to define:
$$L:=\frac{\ftr{L}}{N_{M}+N_{A}}\quad\text{and}\quad L^{\sss +}:=\frac{\ftr{L^{\sss +}}}{N_{M}+N_{A}}$$
where $L\simeq\gena{L_{1}}{L^{+}}$ for $L_{1}=M_{1}\oplus_{B_{1}}A_{1}$
and a weight argument yields first that $N_{M}+N_{A}$ is an ideal of $\ftr{L^{\sss +}}$ and then
shows $N^{3}(L)=N^{3}(L^{+})$ is actually $N_{M}+N_{A}$.

We have now to show $N^{3}_{L^{+}}(A_{1})=N^{3}_{L}(A_{1})=N_{A}$, and this holds since
\begin{multline*}
N^{3}_{L^{+}}(A_{1})=
N^{3}(L^{+})
\cap\gena{A_{1}}{\ftr{ L^{\sss +} } }=(N_{M}+N_{A})\cap\gena{A_{1}}{\ftr{ L}}=\\
=N_{A}+( N_{M} \cap \gena{A_{1}}{\ftr{ L}} )
=N_{A}+( N_{M}\cap\gena{M_{1}}{\ftr{L}} \cap \gena{A_{1}}{\ftr{ L}} )=\\
=N_{A}+( N_{M}\cap\gena{B_{1}}{\ftr{L}})=N_{A}+N_{B}=N_{A}
\end{multline*}

Since $\ke{A}{}\inn N^{3}(A)$, this also implies $A\simeq\ftr{A}\quot_{\!N^{3}(A)} \simeq\overline{F}_{\sss A}\quot_{\!N_{A}}\simeq\gena{A_{1}}{L^{\sss +}}(\simeq\gena{A_{1}}{L})$ and the same results hold for $M$ of course.

\bigskip\noindent
Now $M\dsu L$.

\smallskip
Assume $M_{1}\inn E_{1}\zsu L_{1}$.
%If we assume $A$ to be $\ta$-closed as well\footnote{
%or if we have $\gena{E_{1}}{L_{*}}\cap\gena{A_{1}}{L_{*}}=\gena{E_{1}\cap A_{1}}{L_{*}}$
%considering $L$ and {\em not} $L^{+}$. This is however
%to be discussed with the proof of rich/model theorem},
By $(\ast)$ we have $\gena{E_{1}}{\ftr{L}}\cap\gena{A_{1}}{\ftr{L}}=\gena{E_{1}\cap A_{1}}{\ftr{L}}$, hence if
%and to get this we essentially repeat the argument $(\ast)$ above,\footnote{{\bf Note:} To make this work here we have to assume that
%$\gena{E_{1}}{L^{+}_{*}}\cap\gena{A_{1}}{L^{+}_{*}}=
%\gena{E_{1}\cap A_{1}}{L^{+}_{*}}$ and this is achieved if we assume, say,
%$A$ to be $\ta$-closed as well.}
%this time with
%$$
%X:=L_{*}^{+}\oplus\left( (F_{\sss E})_{3} \oplus_{(\overline{F}_{D})_{3}}
%(\overline{F}_{\sss A})_{3} \right)
%$$
%where $D=\gena{E_{1}\cap A_{1}}{F_{L^{+}}}$.
we compute $N^{3}_{L^{+}}(E_{1})$ we get
$$
(N_{M}+N_{A})\cap\gena{E_{1}}{F_{L}}=N_{M}+(N_{A}\cap
\gena{E_{1}}{F_{L}})=N_{M}+N_{L}^{3}(E_{1}\cap A_{1}).
$$

%On the other hand $N_{M}\cap N_{L^{+}}^{3}(E_{1}\cap A_{1})=
%N^{3}_{L^{+}}(M_{1})
Therefore $N^{3}_{L}(E_{1})\quot N^{3}_{L}(M_{1})$ is isomorphic to
$N^{3}_{L}(E_{1}\cap A_{1})\quot N^{3}_{L}(B_{1})$ which is epimorphic image
of  $N^{3}_{A}(E_{1}\cap A_{1})\quot N^{3}_{A}(B_{1})$ through $\gam{A}{L}$.

\medskip
Since on the other hand we have $d_{2}^{L}(E_{1}\quot M_{1})\geq d_{2}^{A}(E_{1}\cap
A_{1}\quot B_{1})$,\footnote{follow arguments already used in the symmetric amalgam}
we get in the end $\ded^{L}(E_{1}\quot M_{1})\geq\ded^{A}(E_{1}\cap A_{1}\quot
B_{1})\geq0$ since $B\dsu A$.

\bigskip
If we wish $M\dsu L^{+}$ as well, we should prove (d) for
$$\gena{E_{1}}{\ftr{L^{+}}}\cap\gena{A_{1}}{\ftr{L^{+}}}=\gena{E_{1}\cap A_{1}}{\ftr{L^{+}}}$$
where now $M_{1}\inn E_{1}\inn L_{1}^{+}$ and conclude with the above arguments.

Since we do not have (d) in the case above, even for a couple of $2$-strong subalgebras,
we actually cannot prove $M\dsu L^{+}$ even if $L$ is a {\em symmetric} amalgam.
\end{document}

\subsection{Toward an Amalgamation Class}\label{classetre}
Despite Remark \ref{nossummo} prevent us from plainly recasting the proof of Lemma \ref{amalsigma2} (and
Lemma \ref{2trans}) for $\nla{3}$-objects,
various attempts were made to prove that a non-negative lower bound for $\delta_{3}$ is preserved under the $\nla{3}$-amalgamation
\pref{amalgatre}.

The strategy is to search (locally) for a free composition at level of $\nla{2}$-algebras inside the $\nla{3}$-amalgam.
%and the finite {\em test space} $E_{1}$
This allows to apply Lemma \ref{moduliftlem} and hence obtain submodularity for $\ded^{M}$. The same procedure
could help also to decide whether $\dsu{}$ is transitive.

The above ideas will be discussed at the end of the section, but first we need to compare $\delta_{3}$ and $\ded^{M}$.

\smallskip
To this end we prove the following crucial result.
\begin{cor}\label{deltaded}
For any $M\in\nla{3}$ such that $M_{*}\in\Klt{2}$ and any finite $\nla{3}$-subalgebra $B\inn M$ %with $M=\gena{M_{1}}{L}$,
we have $\ded^{M}(B)\leq\delta_{3}(B)$.
\end{cor}
The proof of the statement above relies on the following result, which will be proved after Lemma \ref{crocolemma} below.
\begin{teo}\label{crocotheorem}
Let $A\nni B$ be an extension of  finite $\nla{2}$-algebras, for $A$ in $\Kl{2}$. Assume $A=\ssc^{A}(B)$, then
%$k_{B}^{A}:=\dfp(\ker\gam{B}{A})
$\dfp(\ker(\gam{B}{A}))\leq-\delta_{2}(A\quot B)$.
\end{teo}
Remark that the above statement doesn't replace Proposition \ref{bellemma}.

\begin{proofof}{Corollary \ref{deltaded}}
Consider the map $\map{\gam{B}{M}}{\fl{B_{*}}}{\fl{M_{*}}}$ defined above.
%and set $\dkerg{B}{M}=\dfp(\ker(\gam{B}{M}))$.

Since $\pam{B}=\gam{B}{M}\,\pam{M}$ by Lemma \ref{gammap}, now $\gam{B}{M}$ maps
$\ker(\pam{B})=\rt(B)$ onto $\ker(\pam{M})\cap\gam{B}{M}(\fl B)=\rt_{M}(B_{1})$
with kernel $\ker(\gam{B}{M})\,(\inn\rt(B))$.

Therefore $\dfp(\rt_{M}(B_{1}))=\dfp(\rt(B))-\dfp(\ker(\gam{B}{M}))$ and if $A_{1}$ is the self-sufficient closure
of $B_{1}$ in $M_{*}$, then by Corollary \ref{corembel} and Remark \ref{samed2}, we have
$\delta_{3}(B)-\ded^{M}(B)=\delta_{2}(B)-d_{2}^{M}(B)-\dfp(\ker(\gam{B}{M}))=\delta_{2}(B)-d_{2}^{A}(B)-\dfp(\ker(\gam{B}{A}))$.

By theorem \ref{crocotheorem} $\delta_{2}(B)-d_{2}^{A}(B)-\dkerg{B}{A}\geq0$
\end{proofof}
%%\section{A Construction? of $\delta_{3}$}
Remember that for each $n<\omega$, any $A\in\nla{n}$ is of the form
$A=\gena{A_{1}}{A}=\oplus_{i\leq n}A_{i}$.
Now for all $M\in\nla{n}$ we define with abuse $\tr{2}M=M\quot M^{3}\simeq{M}_{1}\oplus{M}_{2}$.
The monolinear part $M_{1}=({\tr{2}M})_{1}$ can be once more endowed with a quasidimension function
$\map{\delta_{2}}{\fpe{M}}{\Z}$
considering relations from $N^{2}(\tr{2}M)\inn_{\sm{id}}{M}_{2}$. As usual
$\fpe{M}$ denotes the collection of all finite dimensional subspaces of $M_{1}$.
%, coincides with $\fpe{\tr{2}M}$.

Again the notion of \emph{strong spaces} with respect to $\delta _{2}$ is well defined.
The dimension function $d_{2}$, obtained from $\delta_{2}$, gives rise to the pr\"{a}geometry
$(M_{1},\cl_{2})$, where $\cl_{2}=\cl_{d_{2}}$.
%For any $M\in\nla{n}$, we define
%\tojoris{$\fsz{M}$ as the family of all the vector $\Fp$-subspaces $H_{1}$ of $M_{1}$
%for which $d_{1}(H_{1})<\omega$ and such that $H_{1}\zsu M_{1}$.}

\medskip
%Let $T_{2}$ be the elementary theory of the limit structure $\mathfrak{C}_{2}$ of the class $\mathcal{K}_{2}$.
%Let $\fhlim{3}$ be the class of structures $M$ of $\nla{3}$ such that $\tr{2}M\models T_{2}$.
%\smallskip
For each algebra $M$ in $\nla{3}$ let $\map{\pi}{\fr{3}\tr{2}M}{M}$ be the canonical map as in \pref{communo} above,
$\pi$ is defined by
$\bar w\;(\mathit{mod}\:\jei{\tr{2}M})\longmapsto\bar w \;({mod}\:R)$. Where $M=\fla{3}{X_{1}}\quot R$.
Then define $N^{3}(M)=\ker\pi$, we have $N^{3}(M)\inn(\fr{3}\tr{2}M)_{3}$.

Let now $H=\gena{H_{1}}{M}$ be a subalgebra of $M\in\nla{3}$ and assume
$H_{1}\zsu M_{1}$ as defined above.
%$\tr{2}H$ is a selfsufficient substructure in $\tr{2}M$.
%we will confuse equality with isomorphismand write 
we have $\fr{3}\tr{2}H\simeq\gena{H_{1}}{\fr{3}\tr{2}M}$.

For $H\inn M\in\nla{3}$ such that $H=\gena{H_{1}}{M}$ and
$H_{1}\zsu M_{1}$, we define
$$N^{3}(H_{1})=N^{3}(M)\cap\gena{H_{1}}{\fr{3}\tr{2}M}.$$%=N^{3}(M)\cap\fr{3}\tr{2}H

Because $H_{1}$ is selfsufficient in $M_{1}$ and $\tr{2}\gena{H_{1}}{M}=\gena{H_{1}}{\tr{2}M}$,
on account of Lemma \ref{bellemma},
$N^{3}(H_{1})$ will be isomorphic with the kernel of the
canonical projection $\map{\pi^{H}}{\fr{3}\tr{2}H}{H}$. If no confusion arises
this ideal will be also denoted $N^{3}(H)$.
%in this situation our definition does not depend on $M$, but only on the linear part $H_{1}$.
%\medskip
%is a $\cl_{2}$-closed space.}
%If $H_{1}\in\fpz{M}$ then it is also $d_{2}H_{1}<\omega$.

\smallskip
For any $M$ in $\nla{3}$ we define a new map
$$\map{\delta_{3}}{\fpe{M}}{\Z}$$
by means of
\begin{labeq}{deltatre}
\delta_{3}H_{1}=d_{2}H_{1}-\dim_{\Fp}(N^{3}(\ssc_{2}H_{1}))\quad\forall H_{1}\in\fpe{M}
\end{labeq}
%Here of course $H=\gena{H_{1}}{M}$, %when this is clear from the context
%as this will be almost everywhere the case
If $H=\gena{H_{1}}{M}$ we will often write simply $\delta_{3}H$ instead of $\delta_{3}H_{1}$.

%The definition of $\delta_{3}$ does make sense because a $\clz$-space is a $\delta_{2}$-strong space, as we proved in \ref{Kapzwei}, moreover
As the structures in argument of $\delta_{3}$ are of finite character, $\dfp(N^{3}(H))$ is an integer.

The definition of $\delta_{3}$ depends only on $H_{1}$. This means that if $M\inn L\in\nla{3}$ and $M_{1}\zsu L_{1}$ and we define respectively 
$\delta_{3}^{M}$ and $\delta_{3}^{L}$ as above,
%its values on $\fpe{M}\inn\fpe{L}$ will coincide with those
%of $\map{\delta_{3}}{\fpe{M}}{\Z}$
then ${\delta_{3}^{L}}_{|_{\fpe{M}}}\!=\delta_{3}^{M}$
because $\fr{3}\tr{2}M\simeq\gena{M_{1}}{\fr{3}\tr{2}L}$ and
$N^{3}(M)\simeq\gena{M_{1}}{\fr{3}\tr{2}L}\cap N^{3}(L)$.

Therefore working with strong $\delta_{2}$-extensions there won't be need to specify
each time which $\delta_{3}$ we are referring to.

\medskip
For general $M\in\nla{3}$, $\delta_{3}$ will \emph{not} be a $\cl_{1}$ (nor a $\cl_{2}$) quasidimension function
on $M_{1}$.
We have however $\delta_{3}(\ssc_{2}(H_{1},m))\leq\delta_{3}(H)+1.$

Subadditivity holds only in special cases, as the next Lemma shows.
%As done in the $2$-nilpotent case, when $U$ and $V$ are subalgebras of $M\in\nla{3}$,
%with $U+V$ we refer to$\gena{U_{1}+V_{1}}{M}$.
\begin{lem}\label{presubatre}
Let $M\in\nla{3}$ and $U_{1}$, $V_{1}$ are $\delta_{2}$-strong spaces in $M_{1}$ such that
%$\tr{2}U\cap\tr{2}V=\gena{U_{1}\cap V_{1}}{\tr{2}M}$, then
$\gena{U_{1}}{\tr{2}M}\cap\gena{V_{1}}{\tr{2}M}=\gena{U_{1}\cap V_{1}}{\tr{2}M}$.
We have
$$\delta_3\left(\ssc_{2}(U_{1}+V_{1})\right)+\delta_{3}(U_{1}\cap V_{1})\leq\delta_{3}U+\delta_{3}V.$$
\end{lem}
\begin{proof}
%As $\delta_{2}$ is a $\cle$-quasidimension, on one hand we have
%$$\delta_2(U+V)+\delta_{2}(U\cap V)\leq\delta_{2}U+\delta_{2}V.$$
As $d_{2}$ is a dimensionfunction and $\clz$ extends $\cle$, on one hand we have
$$d_{2}(U_{1}+V_{1})=d_{2}(\cle(U_{1}\cup V_{1}))=
d_{2}(U_{1}\cup V_{1})\leq
d_{2}U+d_{2}V-d_{2}(U\cap V).$$
Since $\delta_{3}(U+V)=d_{2}(U+V)-\dfp(N^{3}(U+V))$ we only have to show
$$\dfp(N^{3}(\ssc_{2}(U_{1}+V_{1}))\geq\dfp(N^{3}(U))+\dfp(N^{3}(V))-\dfp(N^{3}(U\cap V)).$$
Let $W=\fr{3}\tr{2}M$ and $N^{3}=N^{3}(M)$, we have
%\begin{multline*}
%N^{3}(\ssc_{2}(U+V))\nni
%N^{3}((U+V)=
$$
N^{3}\cap\gena{\ssc_{2}(U_{1}+V_{1})}{W}\supseteq
%N^{3}\cap\left({\gena{U_{1}+V_{1}}{W}}\right)_{3}\supseteq\\
%\supseteqN^{3}\cap\left({\gena{U_{1}}{W}}_{3}+{\gena{V_{1}}{W}}_{3}\right)\supseteq
N^{3}\cap\left(\gena{U_{1}}{W}\right)+N^{3}\cap\left(\gena{V_{1}}{W}\right).
$$
%\end{multline*}
The previous inclusion, has to be regarded as
between vector spaces in the weight $3$ part of $W$.
Our assumptions and Lemma \pref{bellemmino} now imply
$\gena{U_{1}}{W}\cap\gena{V_{1}}{W}=\gena{U_{1}\cap V_{1}}{W}$, therefore
\begin{multline*}
\dfp(N^{3}(\ssc_{2}(U_{1}+V_{1})))\geq\dfp(N^{3}\cap\gena{U_{1}}{W}+N^{3}\cap\gena{V_{1}}{W})=\\
=\dfp(N^{3}\cap\gena{U_{1}}{W})+\dfp(N^{3}\cap\gena{V_{1}}{W})-\dfp(\gena{U_{1}}{W}\cap\gena{V_{1}}{W}
\cap N^{3})\geq\\
\geq\dfp(N^{3}(U))+\dfp(N^{3}(V))-\dfp(N^{3}(U\cap V)).
\end{multline*}
This concludes the proof.
\end{proof}
%In what follows we make the ground hypothesis that \tojoris{$\fpz{\fhlim{2}}$ is (upward) directed,}
%although it seems not too sound.
%\bigskip
 %Define now $\mathcal{K}_{3}=\{A\in\nla{3}\mid\tr{2}A\in\fpz{\mathfrak{C}_{2}},\,\delta_{3}H\geq0\;\forall H\in
%\fpz{A}\}$. In particular for each $M\in\Kl^{3,\dots}$, we have $\tr{2}M\in\Kl^{2,\dots}$.
%We observe that $\mathcal{K}_{3}^{\prime}$ and $\mathcal{K}_{3}$ are related by the following
%\begin{lem}
%Assume $M\in\mathcal{K}_{3}^{\prime}$ then $\fpz{M}\inn\mathcal{K}_{3}$.
%\end{lem}
%We use now the properties of the 2-nilpotent limit to prove that 2-spaces in $\fhlim{2}$ have
%nice intersections.
%\begin{lem}
%Assume $U_{1},V_{1}\in\fpz{\fhlim{2}}$
%and $U=\gena{U_{1}}{\fhlim{2}}$, $V=\gena{V_{1}}{\fhlim{2}}$ then $U\cap V=\gena{U_{1}\cap V_{1}}{\fhlim{2}}$.
%\end{lem}
% 
%Consider now $M\in\Kl^{3,\dots}$ and $\map{\delta_{3}}{\fpz{M}}{\mathbb{N}}$.
%We want to prove that $\delta_{3}$ is a $\clz$-quasidimension, to do so
%we have to show (qd'1), \dots , (qd'3) are true on such structures $M$.
%So we start proving $\delta_{3}\clz(\vac)=0$. We have that $\clz(\vac)=(0)$ because if $a$ is a nontrivial
%element of $M_{1}$ which lays in $\clz(\vac)\nni\cle(\vac)=(0)$ then we'd have $d_{2}(0)=
%d_{2}(0,a)=d_{2}(a)=1$, which is nonsense as obviously $\delta_{2}(0)=0$.
%%$\tr{2}M\models T_{2}$ and so each space of the form
%%$\gen{a}_{1}$ is $2$-strong for each element $a\in M_{1}$; in particular $d_{2}(0)=\delta_{2}(0)=0$.
%Then by computation $\delta_{3}(0)=0$ and so we get (qd'1).
%\smallskip
%To prove (qd'2) we must find $\delta_{3}\clz(a)\leq1$ for each $a\in M_{1}$.
%In general for an arbitrary subset $X$ of $M_{1}$,
%if $\dim_{2}X=\dim_{2}\clz(X)<\omega$, then $d_{2}X=d_{2}\clz(X)$ as $d_{2}$ and $\dim_{2}$ do agree on finite
%dimensional sets in the sense of the $\clz$-praegeometry.
%In our case $\dim_{2}(a)=1$ so $d_{2}\clz(a)=d_{2}(a)$.
%%As $\tr{2}M\models T_{2}$ it follows that each space of the form $\gen{a}_{1}$ is $2$-strong,
%%for each element
%%$a\in M_{1}$, but this implies $d_{2}(a)=\delta_{2}(a)\leq1$.
%We conclude $\delta_{3}\clz(a)=d_{2}(a)-\dfp(N^{3}(a))=\delta_{2}(a)-\dfp(N^{3}(a))\leq1$.
%\smallskip
%Finally, also (qd'3) holds. Let $U,V\in\fpz{M}$ .
%Since $M\in\Kl^{3,\dots}$, on account of the previous lemma, we first have $\tr{2}U\cap\tr{2}V=\gena{U_{1}\cap V_{1}}{\tr{2}M}$.
%We then observe that $\omega>\dim_{2}(U\cup V)=\dim_{2}(\clz(U\cup V))$,
%%a consequence the values of $d_{2}$ and $\dim_{2}$ agrees on both the sets $U\cup V$ and $\clz(U\cup V)$.
%therefore $d_{2}\clz(U\cup V)=d_{2}(U\cup V)=d_{2}\cle(U\cup V)=d_{2}(U+V)$.
%%$d_{2}\clz(U\cup V)=\dim_{2}(\clz(U\cup V))=\dim_{2}(U\cup V)=d_{2}(U\cup V)$. Therefore we have
%On the other hand $N^{3}(U+V)=N^{3}(\cle(U_{1}\cup V_{1}))\inn N^{3}(\clz(U_{1}\cup V_{1}))$.
%At the end we have
%\begin{multline*}
%\delta_{3}\clz(U\cup V)=d_{2}\clz(U_{1}\cup V_{1})-\dfp(N^{3}(\clz(U_{1}\cup V_{1})))\leq\\
%\leq d_{2}(U+V)-\dfp(N^{3}(U+V))=\delta_{3}(U+V).
%\end{multline*}
%Now applying Lemma \pref{presubatre} we get
%$\delta_{3}\cl_{2}(U\cup V)+\delta_{3}(U\cap V)\leq\delta_{3}U+\delta_{3}V$ as we wanted.
%We have indeed proved the following
%\begin{cor}
%$\delta_{3}$ is a $\clz$-quasidimensionfunction, provided $M$ belongs to $\mathcal{K}_{3}$.
%\end{cor}

\begin{dfn}
Let $M\in\nla{3}$ a finite dimensional $H_{1}\inn M_{1}$ is a $\delta_{3}$-strong substructure of $M_{1}$
if
\begin{itemize}
\item[-]$H_{1}\zsu M_{1}$
\item[-]for any $C_{1}\in\fpe{M}$ such that $C_{1}\nni H_{1}$ we have $\delta_{3}H_{1}\leq\delta_{3}C_{1}$.
\end{itemize}
We write ambiguously $H_{1}\dsu M_{1}$ or $H\dsu M$, provided $H=\gena{H_{1}}{M}$.
\end{dfn}
%and $\delta_{3}$-strong embedding.

We define minimal $\delta_{3}$-extensions.
N\"{a}mlich, $B\dsu A$ is a minimal extension if for each $H_{1}\zsu A_{1}$ such that
$A_{1}\supsetneq H_{1}\supsetneq B_{1}$, we have $\delta_{3}A<\delta_{3}H$.

We write $\delta_{3}(a\quot B)$ to denote $\delta_{3}\gen{B_{1},a}-\delta_{3}B$ where $a,B_{1}\inn M_{1}$.

\medskip
In what follows we amalgamate objects of a subclass $\Kl^{\dots}$ of $\nla{3}$ with respect to $\delta_{3}$
strong embeddings.
Let $$\K^{\dots}=\left\{A=\gen{A_{1}}\in\nla{3}\mid\tr{2}A\in\Kl^{2,\dots},\;(\triv)\dsu A,\;A\sat(\iota)_{3}\right\}$$
with
%Objects of $\Kl^{3,\dots}$ will be algebras $A=\gen{A_{1}}\in\nla{3}$ such that
%$A_{1}$ can be identified with a strong, finite dimensional subspace of the limit $\fhlim{2}$, moreover we require
%$\tr{2}A\simeq\gena{A_{1}}{\fhlim{2}}$ with $A_{1}\zsu{\fhlim{2}}_{1}$ and $(\triv)\dsu A$ and $A\sat(\iota)_{3}$.
$$(\iota)_{3}\colon\quad(\forall x,\,P_{1}(x))(\forall y,\,P_{2}(y))([x,y]\neq 0).$$
As $\delta_{3}(\triv)=0$, with $(\triv)\dsu A$ we mean that $\delta_{3}H_{1}\geq0$ for each $H_{1}\in\fpe{A}$.

%Note that for each $M\in\Kl^{3,\dots}$, we have $\tr{2}M\in\Kl^{2,\dots}$.
Remember that $\Kl^{2,\dots}$ is the class of $2$-nilpotent finitely generated
Lie Algebras with a non negative $\delta_{2}$
on the subspaces and which satisfy
$(\iota)_{2}$ where
$$(\iota)_{2}\colon\quad(\forall x,\,P_{1}(x))(\forall y,\,P_{1}(y))([x,y]= 0)\rightarrow
(\textsl{$x$ lin.{}depends on $y$}).$$

\begin{lem}
Let $A,B,C$ be $\Kl^{3,\dots}$-structures, such that $A\dso B\dsu C$, where $A\quot B$ is a minimal extension,
then there exists $D\in\Kl^{3,\dots}$ da{\ss} alles amalgamiert.
\end{lem}
\begin{proof}
%From the axioms for $\Kl^{3,\dots}$ we have $B=\gena{B_{1}}{A}$, in particular $\tr{2}B=\gena{B_{1}}{\tr{2}A}$.
%So we can assume that $B_{1}\zsu A_{1}\zsu{\fhlim{2}}_{1}$ and $\tr{2}B\simeq\gena{B_{1}}{\fhlim{2}}$.
%The same is done with $C$, so $B=\gena{B_{1}}{\fhlim{2}}$ for some $B_{1}\zsu C_{1}$.
We start describing a \emph{na\"{i}ve amalgamation} procedure which first freely amalgamates at level 2, and then raises the structure to a free amalgam in weight 3.
We call it {\bf (NAM)} and it will be recalled in the rest of the proof when suitable ground hypotheses are attained.

If we denote with $A^{\prime}$, $B^{\prime}$ and $C^{\prime}$, respectively $\tr{2}A$, $\tr{2}B$ and
$\tr{2}C$ then
%$A^{\prime}$ and $C^{\prime}$ two algebras
%in $\Kl^{2,\dots}$ isomorphic to $\gena{A_{1}}{\fhlim{2}}$ and $\gena{C_{1}}{\fhlim{2}}$ respectively.
%We can say 
we have $A^{\prime}\zso B^{\prime}\zsu C^{\prime}$.
We build the free amalgam $D^{\prime}=\fram{A^{\prime}}{B^{\prime}}{C^{\prime}}$ of $A^{\prime}$ and $C^{\prime}$ over $B^{\prime}$ following the construction in Chapter $2$.
\begin{description}
\item[(NAM)]Assume that $D^{\prime}$ is in $\Kl^{2,\dots}$.
%Now, because $\fhlim{2}\zso B^{\prime}\zsu D^{\prime}\in\Kl^{2,\dots}$ and
%$\fhlim{2}$ is \emph{reich} with respect to strong embeddings,
%we can strong embed $D^{\prime}$ in $\fhlim{2}$ over $B^{\prime}$.
%With abusive notation we name $\abu$, $\cbu$ and $\dbu$, the monolinear parts in the embedded
%amalgam $D^{\prime}$, no new names for the images in $\fhlim{2}$.
%So far we have $B_{1}$ strong inside both $\abu$ and $\cbu$, these two strong in $\dbu$ and this last
%strong in ${\fhlim{2}}_{1}$ (here strong is strong substructure).
As $B$ coincides both with $\gena{B_{1}}{A}$ and $\gena{B_{1}}{C}$, then $\gena{B_{1}}{A^{\prime}}=B^{\prime}=
\gena{B_{1}}{C^{\prime}}$;
therefore this setting implies $B_{1}=\abu\cap\cbu$ in $D^{\prime}$ and
\begin{labeq}{inter}
B^{\prime}=\gena{B_{1}}{D^{\prime}}=\gena{\abu}{D^{\prime}}\cap\gena{\cbu}{D^{\prime}}.
\end{labeq}
Now because both $A^{\prime}$ and $C^{\prime}$ are selfsufficient in $D^{\prime}$,
on account of Lemma \pref{bellemma},
there exist two embeddings $j_{A}$ and $j_{C}$ of $\fr{3}\tr{2}A$ and $\fr{3}\tr{2}C$
into $\fr{3}D^{\prime}$ their images being $\gena{\abu}{\fr{3}D^{\prime}}$ and $\gena{\cbu}
{\fr{3}D^{\prime}}$ respectively. From \pref{inter} and lemma \ref{bellemmino} instead, we get
\begin{labeq}{freeint}
\gena{B_{1}}{\fr{3}D^{\prime}}=\gena{\abu}{\fr{3}D^{\prime}}\cap\gena{\cbu}{\fr{3}D^{\prime}}
\end{labeq}
here $B_{1}$, $\abu$ and $\cbu$ are identified with its images modulo $j_{A}$ and $j_{C}$. 

We set $N^{3}(A)^{\bullet}=j_{A}(N^{3}(A))$ and $N^{3}(C)^{\bullet}=j_{C}(N^{3}(C))$.
And we build
$$D=\fr{3}D^{\prime}\quot N^{3}(A)^{\bullet}+N^{3}(C)^{\bullet}.$$
As $N^{3}(A)^{\bullet}+N^{3}(C)^{\bullet}$ is an ideal consisting only of weight $3$ elements,
$\fr{3}\tr{2}D=\fr{3}(\tr{2}\fr{3}D^{\prime})%\simeq
=\fr{3}D^{\prime}$, hence we obtain $N^{3}(D)=N^{3}(A)^{\bullet}+N^{3}(C)^{\bullet}$.

From \pref{freeint} we have that
$$N^{3}(B)^{\bullet}:=j_{A}(N^{3}(B))=N^{3}(A)^{\bullet}\cap N^{3}(C)^{\bullet}=j_{C}(N^{3}(B))$$
and the following compatibility equations
\begin{labeq}{compa}
N^{3}(A)^{\bullet}=N^{3}(D)\cap\gena{\abu}{\fr{3}D^{\prime}}
\end{labeq}and
$$
N^{3}(C)^{\bullet}=N^{3}(D)\cap\gena{\cbu}{\fr{3}D^{\prime}}.
$$

So far, are $A$ and $C$ embeddable in $D$ via Lie Algebra monomorphisms,
just regard $A\simeq\fr{3}\tr{2}A\quot N^{3}(A)$ and take the quotient of $j_{A}$:
\begin{eqnarray}
\map{\bar j _{A}}{&\fr{3}\tr{2}A\quot N^{3}(A)}{D}\\
&\bar w\longmapsto \bar w
\end{eqnarray}
on account of \pref{compa} is $\bar j_{A}$ one-to-one, in particular its image is again
$\gena{\abu}{D}=(\gena{\abu}{\fr{3}D^{\prime}}+N^{3}(D))\quot N^{3}(D)$.\quad Do the same for $C$.

Next we show that these are $\delta_{3}$-strong embeddings.

We test it for $C$. From the definition of $\delta_{3}$ it is sufficient to prove $\delta_{3}C_{1}\leq\delta_{3}E_{1}$
for each $E_{1}\zsu\dbu$ such that $E_{1}\nni\cbu$.
As $E_{1}=C_{1}+(E_{1}\cap \abu)$ we finish once we show
$\delta_{3}(E\quot C)=\delta_{3}(E_{1}\cap\abu\quot B)$ because
%$E_{1}\cap \abu$ is a $\delta_{2}$ strong subspace of $\abu$ and 
$B\dsu A$ and $\delta_{3}(E_{1}\cap \abu\quot B)\geq 0$.

We have
$$\delta_{3}(E\quot C)=\delta_{3}(E_{1})-\delta_{3}(C_{1})=d_{2}(E_{1}\quot C_{1})
-\big(\dfp(N^{3}(E))-\dfp(N^{3}(C))\big).$$
Now as $\tr{2}D=D^{\prime}$, in the $2$ amalgam everything behaves good for $\delta_{2}$,
and we have $d_{2}(E_{1})-d_{2}(C_{1})=\delta_{2}(E_{1})-\delta_{2}(C_{1})=\delta_{2}(E_{1}\cap\abu\quot
B_{1})$, we've already seen this in chapter $2$.

To conclude we have to show
$$N^{3}(E_{1}\cap \abu)\quot N^{3}(B)^{\bullet}\simeq
N^{3}(C_{1}+(E_{1}\cap \abu))\quot N^{3}(C)^{\bullet}.$$
Consider the map
\begin{eqnarray*}
N^{3}(E_{1}\cap \abu)\quot N^{3}(B)^{\bullet}&\longrightarrow&N^{3}(E_{1})\quot N^{3}(C)^{\bullet}\\
\bar\eta&\longmapsto&\bar\eta\quad\quad\forall\eta\in N^{3}(E_{1}\cap\abu).
\end{eqnarray*}
Soundness is immediate. Moreover our map is injective because
\begin{multline*}
N^{3}(C)^{\bullet}\cap\gena{E_{1}\cap \abu}{\fr{3}D^{\prime}}=\\
=N^{3}(D)\cap\gena{\cbu}{\fr{3}D^{\prime}}\cap\gena{E_{1}\cap \abu}{\fr{3}D^{\prime}}=\\
=N^{3}(D)\cap\gena{\abu\cap C_{1}}{\fr{3}D^{\prime}}=N^{3}(B)^{\bullet}.
\end{multline*}
Here we used $\gena{\cbu}{\fr{3}D^{\prime}}\cap\gena{E_{1}\cap\abu}{\fr{3}D^{\prime}}=
\gena{\cbu\cap\abu}{\fr{3}D^{\prime}}$ because both $\cbu$ and $E_{1}\cap\abu$ are $\delta_{2}$ strong in $D_{1}$ and because $\gena{\cbu}{D^{\prime}}\cap\gena{E_{1}\cap\abu}{D^{\prime}}=\gena{\bbu}{D^{\prime}}$.

On the other hand, since again $\gena{\abu}{D^{\prime}}\cap\gena{E_{1}}{D^{\prime}}=\gena{E_{1}\cap\abu}{D^{\prime}}$ we have
\begin{multline*}
N^{3}(E_{1})=N^{3}(D)\cap\gena{E_{1}}{\fr{3}D^{\prime}}=\\
=(N^{3}(A)^{\bullet}+N^{3}(C)^{\bullet})\cap\gena{E_{1}}{\fr{3}D^{\prime}}=\\
=(N^{3}(A)^{\bullet}\cap\gena{E_{1}}{\fr{3}D^{\prime}})+N^{3}(C)^{\bullet}=\\
=(N^{3}(D)\cap\gena{E_{1}\cap A_{1}}{\fr{3}D^{\prime}})+N^{3}(C)^{\bullet}
\end{multline*}
and this gives that our map is onto.

We see now that $\dfp(N^{3}(E))-\dfp(N^{3}(C))=\dfp(N^{3}(E_{1}\cap\abu))-\dfp(N^{3}(B)^{\bullet})$.
We have shown that $C$ is $\delta_{3}$-strong embeddable in $D$.

A completely analogous argument proves that $A$ can be found in $D$ as a $\delta_{3}$ strong structure.

Now to prove $(\triv)\dsu D_{1}$ we pick an arbitrary $E_{1}\zsu D_{1}$, since
$\gena{\abu}{D^{\prime}}\cap\gena{E_{1}}{D^{\prime}}=\gena{E_{1}\cap\abu}{D^{\prime}}$,
in this case we can apply Lemma \ref{presubatre} and find
\begin{multline*}
\delta_{3}\ssc_{2}(A_{1}+E_{1})+\delta_{3}(A_{1}\cap E_{1})\leq\\
\leq\delta_{3}A+\delta_{3}E\leq\delta_{3}\ssc_{2}(A_{1}+E_{1})+\delta_{3}E.
\end{multline*}
We used $\delta_{3}A\leq\delta_{3}\ssc_{2}(A_{1}+E_{1})$ because $A\dsu D$.

As $E_{1}\cap A_{1}\zsu A_{1}$ und $A\in\Kl^{3,\dots}$, it follows $0\leq\delta_{3}(E_{1}\cap A_{1})\leq\delta_{3}E$.

So far we have constructed a $D\in\nla{3}$ such that $A\dsu D\dso C$ (up to $\delta_{3}$-strong
embedding) with $\tr{2}D\in\Kl^{2,\dots}$ and
$\delta_{3}$ is non negative on the $\delta_{2}$-strong subspaces of $D_{1}$.
\end{description}

We proceed with a discussion of minimal $\delta_{3}$ extensions, in order to decide whether or not a {\bf (NAM)}
construction is possible, and whether it leads to a structure $D$ which lays in $\Kl^{3,\dots}$.
We will actually provide axiom $(\iota)_{3}$ for the amalgam $D$.

Assume $A_{1}\quot B_{1}$ is minimal with respect to $\delta_{3}$,
then only the following cases can occur:
\begin{itemize}
\item$\delta_{3}(A\quot B)>0$ (\emph{free minimal extension})
\item%$\delta_{3}(A\quot B)=0$ and 
$A\quot B$ is a \emph{divisor extension}:\\
there exists an element $a\in A_{1}\non B_{1}$ such that\end{itemize}
$$ %\begin{labeq}{dreidiv}
\text{($3$-div.)}\quad
(\exists\beta,\,P_{2}(\beta))\:\beta\in\gena{a,B_{1}}{\fr{3}\tr{2}A}\;\exists\Phi\in\gena{B_{1}}{\fr{3}\tr{2}A}\:\left([a,\beta]-\Phi\,\in N^{3}(A)\right).
%\end{labeq}
$$\begin{itemize}
%\item$\delta_{3}(A\quot B)=0$ and $\exists\alpha\in A_{2}$ such that
%($3$-div.{}-II)\quad\dots{\sl to be defined}

\item$\delta_{3}(A\quot B)=0$ and no ($3$-div.) divisor (\emph{pr\"{a}algebraic extension}).
\end{itemize}

Assume first $A_{1}\quot B_{1}$ is free minimal. %and $\delta_{3}(A\quot B)>0$.
Then we claim $d_{2}A=d_{2}B+1$ and $A_{1}\quot B_{1}$ is also $\delta_{2}$ minimal.

We must however have $d_{2}A>d_{2}B$, assume $d_{2}A>d_{2}B+1$, then there exist an element $a\in A_{1}$,
such that $A_{1}\supsetneq\ssc_{2}(B_{1},a)\supsetneq B_{1}$. Now from minimality it follows
$\delta_{3}\ssc_{2}(B_{1},a)>\delta_{3}A>\delta_{3}B$. But for each $a\in A_{1}$ we have $\delta_{3}\ssc_{2}
(B_{1},a)\leq\delta_{3}B+1$. And this cannot happen.
In particular it follows $\delta_{3}A=\delta_{3}B+1$ and $\dfp(N^{3}(A))=\dfp(N^{3}(B))$.

Now if there exists $A_{1}\supsetneq H_{1}\supsetneq B_{1}$ with $A_{1}\zso H_{1}$, since
$N^{3}(A)=N^{3}(B)$, then $\delta_{3}H$ is either equal to $\delta_{3}A$ or to $\delta_{3}B$. In both cases
that is against minimality. This proves minimality of $A_{1}$ over $B_{1}$ with respect to $\delta_{2}$,
therefore $A_{1}=\gen{B_{1},a}$ for some $a$ in $A_{1}$ free from $B_{1}$.

In this case the free amalgam of $\tr{2}A$ and $\tr{2}C$ over $\tr{2}B$ gives an element of $\Kl^{2,\dots}$ and
applying {\bf (NAM)} we obtain a structure $D$ which has still $(\iota)_{3}$ since $N^{3}(D)=N^{3}(C)^{\bullet}$.
Thus $D$ is in $\Kl^{3,\dots}$ and amalgamates $A$ and $C$ over $B$.

\medskip
Assume now $A_{1}\quot B_{1}$ is a divisor $\delta_{3}$-extension and suppose ($3$-div.) holds for an
element $a\in A_{1}\non B_{1}$.
By definition of ($3$-div.) there exists a weight $2$ element $\beta\in\gena{B_{1},a}{\fr{3}\tr{2}A}$ and a linear combination of weight $3$ commutators
$\Phi\in\gena{B_{1}}{\fr{3}\tr{2}A}$ such that $[a,\beta]-\Phi\,\in N^{3}(A)$.

Since $B\dsu A$ and $\dfp(N^{3}(\ssc_{2}(a,B_{1})))\geq\dfp(N^{3}(B))+1$,
we obtain $\delta_{3}(a\quot B_{1})=0$ and $d_{2}(a\quot B_{1})=1$.

By $\delta_{3}$-minimality we conclude
that $A_{1}=\ssc_{2}(B_{1},a)$ and $N^{3}(A)$ is generated
in $\fr{3}\tr{2}A$ by $N^{3}(B)$ and $[a,\beta]-\Phi$.

On the other hand, we can show that if there is $a$ in $A_{1}\non B_{1}$ such that $\delta_{3}(a\quot B)=0$
and $d_{2}(a\quot B)=1$ then $a$ satisfies ($3$-div.).
%Now we show that $A_{1}\quot B_{1}$ is also a $\delta_{2}$ minimal extension, were indeed $H_{1}$ a $\delta_{2}$ strong proper refinement,
%\emph{then $\delta_{2}H_{1}=d_{2}A$ if and only if $H\ni a$.}\\
%({\bf Note}: if this argument does not hold, we can use again \pref{cloclo} and freely
%amalgamate anyway).
%So if $d_{2}H_{1}=d_{2}A$ then $[a,\beta]-\Phi\,\in N^{3}(H)$ and $\delta_{3}H=\delta_{3}A$.
%Analogously if $d_{2}H_{1}=d_{2}B$, $\delta_{3}H=\delta_{3}B$.
%In both cases we found a contradiction to $\delta_{3}$ minimality.

We prove next $\gen{B_{1},a}\zsu A_{1}$.
Let then $H_{1}$ be an arbitrary subspace of $A_{1}$ containing both $B_{1}$ and $a$,
if we compare $\delta_{3}H_{1}$ and $\delta_{3}A$,
since $N^{3}(\ssc_{2}H_{1})=N^{3}(A)$, then
$d_{2}H=d_{2}A$.
So $\delta_{2}\gen{B_{1},a}\leq\delta_{2}B_{1}+1=d_{2}A_{1}=d_{2}H_{1}\leq\delta_{2}H_{1}$.

As a result $A_{1}=\gen{B_{1},a}$ is $\delta_{2}$-free (miniimal) over $B_{1}$, therefore it can be with $C_{1}$
freely amalgamated in $\Kl^{2,\dots}$ over $B_{1}$.
Now if $a$ is not realised in $C$ over $B$ we can apply {\bf (NAM)} and conclude.

Otherwise, there exists a $c\in C_{1}\non B_{1}$ such that $[c,\beta^{\prime}]-\Phi\,\in N^{3}(C)$
where $\beta^{\prime}$ is obtained replacing $c$ in all occurrences of $a$ appearing in the
commutators which form $\beta$. Here $\Phi$ must be regarded as an element of $\gena{B_{1}}{\fr{3}\tr{2}C}$.
Since $\delta_{3}(c\quot B_{1})=0$ again, we have $d_{2}(c\quot B_{1})=1$ and
$c$ results free from $B_{1}$ with respect to weight $2$ relations.
It follows, we can map $a$ to $c$ fixing $B_{1}$, and obtain a Lie morphism of $A$ into $C$.
We take $C$ therefore, as the desired amalgam, in order to keep axiom $(\iota)_{3}$.

\medskip
Assume now $A\quot B$ is a minimal prealgebric $\delta_{3}$ extension.

%We treat two subcases:\begin{itemize}
%\item[-](Pr\"{a}al.{}I) for all $a\in A_{1}\non B_{1}$, $\delta_{3}(a\quot B)>0$
%\item[-](Pr\"{a}al.{}II) there exists an element $a\in  A_{1}\non B_{1}$ such that $\delta_{3}(a\quot B)=0$

%({\bf Note}: there's no need to discuss case (Pr\"{a}al.{}II) if $\delta_{3}(a\quot B)=0$ and $d_{2}(a\quot B)=1$ are \"{a}quivalent to $a\sat$($3$-div.). This seems to be obstructed by
%an eventual inequality between $\ssc_{2}(B,a)$ and $\gen{B_{1},a}$ unless we show
%$A\quot B$ is also $\delta_{2}$ minimal).
%\end{itemize}
As we are in presence of no $3$-divisor,
we have only to assure that the level $2$ admits a free amalgamation which leads to
an object in $\Kl^{2,\dots}$, afterwards we proceed applying {\bf (NAM)} to get a structure
$D$ which of course has $(\iota)_{3}$, and therefore belongs to $\Kl^{3,\dots}$.
%\smallskip
%Assume first (Pr\"{a}al.{}I).\\

Assume first $d_{2}A=d_{2}B$. Since $\delta_{3}A=\delta_{3}B$, we have $N^{3}(A)$=$N^{3}(B)$.
It follows $A_{1}$ is a minimal $\delta_{2}$ extension of $B_{1}$, because a strong proper refinement of it
would produce a $\delta_{3}$ strong refinement strictly between $B_{1}$ and $A_{1}$.

Is $A_{1}\quot B_{1}$ a prealgebraic $\delta_{2}$ extension or a divisor extension ($2$-div.{}),
which is not realized in $C_{1}$, then proceed applying {\bf (NAM)}.
If it is a $\delta_{2}$ divisor extension, $A_{1}=\gen{B_{1},a}$ for some $a\in A_{1}$ and there is a morphism over $B_{1}$ mapping $a$ to some $c\in C_{1}$, then, because $A=\gena{B_{1},a}{A}$, the
weight $3$ structure of $A$ can be reconstructed in $C_{3}$, by means of commutators
with occurrences of $c$. So $A$ is realized in $C$ over $B$.

%If $A_{1}\supsetneq H_{1}\supsetneq B_{1}$ and $A_{1}\zso H_{1}$, then by minimality in
%$\delta_{3}$ we get $\delta_{3}H>\delta_{3}A=\delta_{3}B$, so $d_{2}(H\quot B)>\dfp(N^{3}(H))-\dfp(N^{3}(B))\geq 0$, therefore $\delta_{2}H_{1}>\delta_{2}A_{1}$ and $A_{1}\quot B_{1}$ is minimal as a $\delta_{2}$
%extension too.

Assume then $d_{2}A>d_{2}B$. 
In this case $\delta_{3}(A\quot B)=0$ assures that
\begin{labeq}{cloclo}
B_{1}=\cl_{2}(B_{1})\cap A_{1}
\end{labeq}
in fact this is equivalent to $d_{2}H>d_{2}B$, for each $\delta_{2}$-strong proper refinement $H$ of $A_{1}\quot B_{1}$. If we consider such an $H$ then by minimality $\delta_{3}H>\delta_{3}A=\delta_{3}B$, so $d_{2}(H\quot B)>\dfp(N^{3}(H))-\dfp(N^{3}(B))\geq 0$.

%We induce on $m=d_{2}(A\quot B)$ that $\tr{2}A$ and $\tr{2}A$, when freely amalgamated
%over $\tr{2}A$, give still an object of $\Kl^{2,\dots}$.
%Let $m=1$, if $A\quot B$ is minimal it's easy,
%if not we split $A_{1}\quot B_{1}$ with a first minimal $\delta_{2}$-extension, $F_{1}
%\quot B_{1}$, then $F$ must be of free type over $B_{1}$. This also implies
%It follows there's no divisor $a\in A_{1}$ over $B_{1}$

On account of \pref{cloclo}, we have
$\delta_{2}(a\quot B_{1})>0$ for each $a\in A_{1}\non B_{1}$, therefore
no element $a$ of $A_{1}\non B_{1}$ can be a divisor in $B$. We remind that this means,
there's no $\beta\in\exs B_{1}$ such that $[a,b]-\beta\,\in N^{2}(B)$ for some $b\in B_{1}$.
In particular no element of $a$ can be realised in $C_{1}$ over $B_{1}$.
%$A_{1}$ may be joint to $F_{1}$ via a chain of minimal extensions none of which is frei.
This leads to conclude $\fram{A}{B}{C}$ is in $\Kl^{2,\dots}$, we can now apply {\bf (NAM)} as desired.\\
({\bf Note}: if it is somehow too ambitious to consider the entire $A_{1}\quot B_{1}$ we can
adopt an inductive argument on $m=d_{2}(A_{1}\quot B_{1})$, so we can split the extension in
$A\quot B^{*}$ and $B^{*}\quot B$, wobei $B^{*}=\clz(F_{1})\cap A_{1}$ where $F_{1}$ is
a minimal (frei!) extension of $B_{1}$ in $A_{1}$. Naturally $d_{2}(B\quot B^{*})=1$).
%>>>>>>>>>>>>>>>>>>>>>>>>>>>>>>>>>>>>>>>>>>>>>>>>>>>>>>>>>>>>>>>>>>>
%��������������������������������������������������������������������������������������������������
%--------------------WE NEED THIS IF WE DON'T USE (PREAL II)-------------------------------------
%.............................................................................................................................................
%We proceed on induction (?) on $m=d_{2}A-d_{2}B$.
%It remain to discuss the divisor case (cases?).
%So assume there is an $a\in A_{1}\non B_{1}$ such that 
%Assume then (pr\"{a}al.{}II).\\
%This gives first $A_{1}=\ssc_{2}(B_{1},a)$.
%Then if $d_{2}A=d_{2}B$ we have that $A_{1}\quot B_{1}$ is a minimal
%$\delta_{2}$-extension. Is $A_{1}\quot B_{1}$ a prealgebraic $\delta_{2}$ extension or a divisor extension ($2$-div.{}),
%which is not realized in $C_{1}$, then proceed applying {\bf (NAM)}.

%If it is a $\delta_{2}$ divisor extension, $A_{1}=\gen{B_{1},a}$ for some $a\in A_{1}$ and there is a morphism over $B_{1}$ mapping $a$ to some $c\in C_{1}$, then, because $A=\gena{B_{1},a}{A}$, the weight $3$ structure of $A$ can be reconstructed in $C_{3}$, by means of commutators
%with occurrences of $c$. So $A$ is realized in $C$ over $B$.

%Assume now $d_{2}(A)=d_{2}(B)+1$, then as before we have $B_{1}=\cl_{2}(B_{1})\cap A_{1}$.
%We conclude, as before, applying {\bf(NAM)}.
%Bislang hoffe ich, Fall ($3$-div.{}-II) \"{u}berhaupt unm\"{o}glich sei!

This concludes the proof, as there's no other case to be considered.
\end{proof}

%%\section{A Construction of $\delta_{3}$}
So far we have constructed $\K^{2}$ the countable  infinite-rank limit of the class $\Kl^{2}_{0}$ consisting of
finite dimensional $2$-nilpotent algebras of $\nla{2}$.
$\K^{2}$ is a saturated model of the theory
$T^{2}$ axiomatised by $\sig{2}{1},\,\dots\,,\sig{2}{5}$.

We are on the way to construct an analog generic structure $\K^{3}$ rich in a suitable (pseudo) elementary
class $\Kl^{3}$ of $\nla{3}$-algebras which will be an $\omega$-saturated model of a theory $T^{3}$
axiomatised by properties $\sig{3}{1},\,\dots\,,\sig{3}{5}$.

Objects of $\K^{3}$ will be generated by $\delta_{2}$-strong subspaces of $\K^{2}_{1}$.
A new predimension function $\delta_{3}$ wird introduced on the monolinear component of the
$3$-nilpotent structures, its positive part consisting of the $d_{2}$ dimension obtained
on $\K^{2}_{1}$, the super modular negative part will be counting relation of commutator wieght $3$.

\medskip
We need a definition concerning subspaces of $\K^{2}_{1}$.
Recall that our language contains constant names $a,\,b$, the corresponding elements
will have a central role in the rest of this work.

%\begin{dfn}
Let $H_{1}$ be an $\Fp$ subspace of $\K^{2}_{1}$ containing
$a$ and $H=\gena{H_{1}}{\K^{2}}$.
We say that $H_{1}$ is $\ta$-closed (in $\K^{2}_{1}$) if for each $w\in H$, such that
$P_{2}(w)$ then there is $x\in H_{1}$ such that $[a,x]=w$. $H$ will be also called
$\ta$-closed (in $\K^{2}$).
In what follows it will be assumed that \emph{subspaces} of $\K^{2}_{1}$
always contain the element $a$.
%\end{dfn}

We recall here

\begin{itemize}
\punto{$\sig{2}{3}$}$\quad(\forall x,y)([x,y]=0\rightarrow x\,\text{``lin. depends on''}\,y)$
\punto{$\sig{2}{4}$}$\quad(\forall y,\,P_{2}(y))(\forall z,\,P_{1}(z))(\exists x,\,P_{1}(x))\,[z,x]=y$
\end{itemize}

There's a canonical way of constructing a $\ta$-closure of a subspace $H_{1}$ of $\K^{2}_{1}$.
Consider the linear map $\map{\gamma_{a}}{\K^{2}_{1}}{\K^{2}_{2}}$ defined by $x\mapsto[a,x]$.
On account of axioms $\sig{2}{3}$ and $\sig{2}{4}$ is $\gamma_{a}$ surjective and of finite fiber.
As usual $H$ is the algebra generated in $\K^{2}$ by $H_{1}$, define $H^{1}_{1}=H_{1}$ and
$H^{2}_{1}=\gamma_{a}^{-1}(H_{2})$ and recursively
$H^{n+1}_{1}=\gamma_{a}^{-1}(H^{n}_{2})$,
where $H^{n}=\gena{H^{n}_{1}}{\K^{2}}$.
We have $H^{n}_{1}\inn H^{n+1}_{1}$, so take $\ta H_{1}=\cup_{n<\omega}H^{n}_{1}$ and define $\ta H=\gena{\ta H_{1}}{\K^{2}}$.

If we start with an $H_{1}$ of finite linear dimension such that $H_{1}\zsu\K^{2}_{1}$ then we can see that at each step $\delta_{2}H^{k}_{1}=\delta_{2}H_{1}^{k+1}$. Then we get $d_{2}H=d_{2}\ta H$ and
$\ta H\zsu\K^{2}$.
If $H_{1}$ is unendlich with endlich $d_{2}$ dimension and still selfsufficient
we can adopt a similar argument to build its $\ta$-closure which is still selfsufficient and of the same $d_{2}$.

With the use of axiom $\sig{2}{3}$ we see that $\ta$-closed spaces are closed under intersection
and prove the following
\begin{lem}\label{ta-schnitt}
Let $H_{1}$ and $L_{1}$ be $\ta$-closed subspaces of $\K^{2}_{1}$, then $H\cap L=\gena{H_{1}\cap K_{1}}{\K^{2}}$.
\end{lem}
\begin{proof}
Inclusion right-to-left is clear.
Consider instead $w\in\gena{H_{1}}{\K^{2}}\cap\gena{L_{1}}{\K^{2}}$ of weight $2$. Then there exist $x\in H_{1}$ and $y\in L_{1}$ such that
$[a,x]=w=[a,y]$, but then $x-y$ must be a scalar multiple of $a\in H_{1}\cap L_{1}$, thus both $x$ and $y$
lay in $H_{1}\cap L_{1}$. 

\end{proof}

\medskip
For each algebra $M$ in $\nla{3}$ let $\map{\pi}{\fr{3}\tr{2}M}{M}$ be the canonical map as in \pref{communo} above,
$\pi$ is defined by
$\bar w\;(\mathit{mod}\:\jei{\tr{2}M})\longmapsto\bar w \;({mod}\:R)$. Where $M=\fla{3}{X_{1}}\quot R$.
Then define
\begin{labeq}{3ker}
N^{3}(M)=\ker\pi.
\end{labeq}
We have of course $N^{3}(M)\inn(\fr{3}\tr{2}M)_{3}$.

For each subalgebra $H=\gena{H_{1}}{M}$ of $M$ set
$$N^{3}(H_{1})=N^{3}(M)\cap\gena{H_{1}}{\fr{3}\tr{2}M}.$$

Let now $M=\gen{M_{1}}$ be an algebra of $\nla{3}$ such that $\tr{2}M\inn\K^{2}$
%can be $\delta_{2}$-strongly embedded in $\K^{2}$
in particular $M_{1}$ can be identified with a subspace
%selfsufficient space
of $\K^{2}_{1}$.

For each $H_{1}$ subspace of $M_{1}$ such that $d_{2}H_{1}<\omega$ define
(when it make sense)
\begin{labeq}{deltatre}
\delta_{3}H_{1}=d_{2}H_{1}-\dim_{\Fp}(N^{3}(H_{1}))
\end{labeq}
If $H=\gena{H_{1}}{M}$ we will often write simply $\delta_{3}H$ instead of $\delta_{3}H_{1}$.

Assume now $M\inn L\in\nla{3}$ with $\tr{2}L\inn\K^{2}$, $d_{2}L<\omega$ and $M_{1}\zsu L_{1}$
(writing so we will always mean that $\tr{2}M\zsu\tr{2}L$).
If we define $\delta_{3}^{M}$ and $\delta_{3}^{L}$ as above, computing dimensions respectively
in $N^{3}(M)$ and in $N^{3}(L)$,
then for each $H_{1}\inn M_{1}$ we have $\delta_{3}^{M}(H_{1})=\delta_{3}^{M}(H_{1})$.
This because on account of lemma \ref{bellemma} $\fr{3}\tr{2}M\simeq\gena{M_{1}}{\fr{3}\tr{2}L}$ and
$N^{3}(M)\simeq\gena{M_{1}}{\fr{3}\tr{2}L}\cap N^{3}(L)$.

Therefore working with strong $\delta_{2}$-extensions there won't be need to specify
each time which $\delta_{3}$ we are referring to.

\medskip
It's time to define the subclass of $\nla{3}$ that we wish to amalgamate once we give
a notion of $\delta_{3}$-strong embedding.

Assume $M\in\nla{3}$ with $M_{1}\inn\K^{2}_{1}$, we say that $M$ has the
property %$\sig{3}{2}^{\prime}$ if
\begin{flushleft}
%(\sig{3}{2}^{\prime})\textsl{for each finitely generated $\Fp$-subspace }H_{1}\zsu M_{1},\quad\delta_{3}H_{1}\geq2.
$(\sig{3}{2}%^{\prime}
)$\quad\textsl{if for each finitely generated $\Fp$-subspace $H_{1}\inn%\zsu
M_{1}$ follows
$\delta_{3}H_{1}\geq2$}. (\dots es st\"ort nicht, aber vielleich reicht beliebige $H_{1}$ \dots)
\end{flushleft}

Now consider the following first order sentence,
$$(\sig{3}{3})\quad(\forall z,\,P_{2}(z))(\forall x,y,\,P_{1}(x)\wedge P_{1}(y))([z,x]=[z,y]\,\rightarrow
%\textsl{``$x$ lin.{}depends on $y$''}
\:x=y)$$
and then define
$$\Kl^{3}=\left\{
M=\gen{M_{1}}\in\nla{3}\mid
\tr{2}M\zsu\K^{2},\,M_{1}\, \ta\text{-closed},\,M\,\text{has}
\,\sig{3}{2}^{\prime}\,\text{and}\,\sig{3}{3}
\right\}.$$
We isolate objects in the class that have finite $d_{2}$-dimension in
$$\Klf=\{M\in\Kl^{3}\mid d_{2}M_{1}<\omega\}.$$
We note that $\delta_{3}$ is always defined on the objects of $\Kl^{3}_\textsf{f}$ and
that the value of $\delta_{3}$ is always bigger than $2$ on each subspace of
every structure in the class.

\begin{dfn}
Let $M\in\Kl^{3}$ and $H_{1}$ a subspace of $M_{1}$ at which $\delta_{3}$ is defined, $H=\gena{H_{1}}{M}\in\Klf$ %mit $d_{2}H<\omega$, %\in\Klf$
we say that $H_{1}$ is a \emph{$\delta_{3}$-strong
subspace} of $M_{1}$ if 
\begin{itemize}
\item[-]$H_{1}\zsu M_{1}$\quad(\"uberflussig, falls $H$ in $\Klf$ liegt \dots)
\item[-]for any $C\in\Klf$ such that $C_{1}\nni H_{1}$ we have $\delta_{3}H_{1}\leq\delta_{3}C_{1}$.
\end{itemize}
We write ambiguously $H_{1}\dsu M_{1}$ or $H\dsu M$, provided $H=\gena{H_{1}}{M}$.
\end{dfn}
If $M\in\Kl^{3}$ and $H_{1}$ is a subspace of  $M_{1}$, the $\delta_{3}$ constructed
with respect to $M$ yields $\delta_{3}H_{1}\geq\delta_{3}\ssc^{M}_{2}H_{1}\geq\delta_{3}\ta\ssc_{2}^{M}H_{1}$. We note therefore that if $H_{1}\dsu M_{1}$ then \emph{for all} $C_{1}\nni H_{1}$ it holds
$\delta_{3}C_{1}\geq\delta_{3}H_{1}$.

Thanks to the $\ta$-closure we can prove subadditivity of $\delta_{3}$ on $\Klf$-structures.
\begin{lem}\label{presubatre}
Let $U$ and $V$ be algebras in $\Klf$ both contained in $M\in\Kl^{3}$. We have
$$\delta_3\left(U_{1}+V_{1}\right)+\delta_{3}(U_{1}\cap V_{1})\leq\delta_{3}U+\delta_{3}V.$$
\end{lem}
\begin{proof}
As $d_{2}$ is a dimensionfunction, on one hand we have
$$d_{2}(U_{1}+V_{1})=d_{2}(\cle(U_{1}\cup V_{1}))=d_{2}(U_{1}\cup V_{1})\leq
d_{2}U+d_{2}V-d_{2}(U\cap V).$$
Since $\delta_{3}(U+V)=d_{2}(U+V)-\dfp(N^{3}(U+V))$ we only have to show
$$\dfp(N^{3}(U_{1}+V_{1})\geq\dfp(N^{3}(U))+\dfp(N^{3}(V))-\dfp(N^{3}(U\cap V)).$$
Let $W=\fr{3}\tr{2}M$ and $N^{3}=N^{3}(M)$, we have
$$
N^{3}\cap\gena{U_{1}+V_{1}}{W}\supseteq
N^{3}\cap\left(\gena{U_{1}}{W}\right)+N^{3}\cap\left(\gena{V_{1}}{W}\right).
$$
The previous inclusion, has to be regarded as
between vector spaces in the weight $3$ part of $W$.

On account of lemma \ref{ta-schnitt}, we have that $\tr{2}U\cap\tr{2}V=\gena{U_{1}\cap V_{1}}{\tr{2}M}$,
moreover the definition of $\Klf$ implies that $\tr{2}U$ and $\tr{2}V$ are $\delta_{2}$-strong
embeddable in $\tr{2}M$, thus we can apply lemma \ref{bellemmino} and obtain
$\gena{U_{1}}{W}\cap\gena{V_{1}}{W}=\gena{U_{1}\cap V_{1}}{W}$.
Therefore we have
\begin{multline*}
\dfp(N^{3}(U_{1}+V_{1}))\geq\dfp(N^{3}\cap\gena{U_{1}}{W}+N^{3}\cap\gena{V_{1}}{W})=\\
=\dfp(N^{3}\cap\gena{U_{1}}{W})+\dfp(N^{3}\cap\gena{V_{1}}{W})-\dfp(\gena{U_{1}}{W}\cap\gena{V_{1}}{W}
\cap N^{3})\geq\\
\geq\dfp(N^{3}(U))+\dfp(N^{3}(V))-\dfp(N^{3}(U\cap V)).
\end{multline*}
This concludes the proof.
\end{proof}

The next lemmas lead to the construction of a $\delta_{3}$-selfsufficient closure.
\begin{lem}
Let $A$ and $B$ be subalgebras of $M\in\Kl^{3}$, $A,B\in\Klf$, and $E_{1}$, be a $\ta$-closed subspace of $M_{1}$,
if $A\dsu B$ then $A_{1}\cap E_{1}\,\dsu\,B_{1}\cap E_{1}$.

The same holds replacing $B$ with $M$.
\end{lem}
\begin {proof}
Take $V_{1}\inn M_{1}$ such that $A_{1}\cap E_{1}\inn V_{1}\inn B_{1}\cap E_{1}$ and $V\in\Klf$.
Observe that $E_{1}\cap A_{1}=V_{1}\cap A_{1}$ and use subadditivity of $\delta_{3}$ to compute
%\begin{multline*}
%\delta_{3}\ta(\ssc_{2}(A_{1}+V_{1}))+\delta_{3}(A_{1}\cap E_{1})\leq\\\leq
$$
\delta_{3}(A_{1}+V_{1})+\delta_{3}(A_{1}\cap E_{1})\leq\delta_{3}A_{1}+\delta_{3}V_{1}\leq\\
\leq\delta_{3}%\ta(\ssc_{2}(A_{1}+V_{1}))
(A_{1}+V_{1})+\delta_{3}V_{1}.
$$
The last equality uses $\delta_{3}A_{1}\leq\delta_{3}(A_{1}+V_{1})$ from
the fact $A\dsu B$. %and $\gena{\ta(\ssc_{2}(A_{1}+V_{1}))}{B}\in\Klf$,

Conclude $\delta_{3}(A_{1}\cap E_{1})\leq\delta_{3}V_{1}$ as desired.
\end{proof}
\begin{cor}
$M\in\Kl^{3}$, $A\dsu B$ an extension of $\Klf$-structures contained in $M$.
If $B\dsu M$ then $A\dsu M$.
\end{cor}
\begin{cor}
Let $A$ and $B$ in $\Klf$ be $\delta_{3}$-strong subalgebras of $M\in\Kl^{3}$,
then  $A_{1}\cap B_{1}\dsu M_{1}$.
\end{cor}

Let $M$ be a $\Kl^{3}$. Consider now a subspace $H_{1}$ of $M_{1}$ with $d_{2}H_{1}<\omega$, then define
$$\ssc_{3}H_{1}\!\!=\bigcap_{\substack{H_{1}\inn F_{1}\dsu M_{1}\\ F\in\Klf}}\!F_{1}$$
and as usual $\ssc_{3}H=\gena{\ssc_{3}H_{1}}{M}$.

We observe that $\ssc_{3}H$ is the smallest $\Klf$-algebra above $H$ which is $\delta_{3}$-strong
in $M$.

\subsection{The theory $T^{3}$}
To deal with the axioms of $T^{3}$ that we are going to define, we have to introduce a new auxiliary
class in which we consider structures of finite linear dimension.

Define
$$\Kl=\{A=\gen{A_{1}}\mid\tr{2}A\in\Kl^{2}_\textsf{0},A\,\text{has}\,\sig{3}{2}^{\prime}\,\text{and}\,\sig{3}{3}\}.$$

If $A\in\Kl$ and $A\inn B$ with $B$ in $\Kl$ or in $\Kl^{3}$, we say that $A_{1}$
is $\delta_{3}$-strong in $B_{1}$ if $\tr{2}A\zsu{}B_{*}$ and if for each  $C_{1}\inn_{\omega}
B_{1}$ with $C_{1}\nni A_{1}$, we have $\delta_{3}A_{1}\leq\delta_{3}C_{1}$.
We write equivalently $A_{1}\dsu B_{1}$ or $A\dsu B$. If $B\in\Klf$ we call
$A\quot B$ a $\Klf$-extension.

\medskip
A crucial definition
\begin{dfn}
A structure $K$ in $\Kl^{3}$ is \emph{rich} if for each $\delta_{3}$-strong extension $N\dsu M$ of
$\Klf$-algebras, if $N\dsu K$ then there is a strong embedding of $M$ in $K$ over $N$.
\end{dfn}

As the trivial subalgebra $\triv$ is $\delta_{3}$-strong in each $\Klf$-algebra,
a rich structure of $\Kl^{3}$ is also $\Klf$-universal. A rich structure $K\in\Kl^{3}$ is also homogeneous
with respect to $\delta_{3}$-strong $\Klf$-subalgebras.

We prove amalgamation property for the class $\Klf$, this yields the existence
of a rich structure $\K^{3}$ in $\Kl^{3}$.

\medskip
A $\delta_{3}$-strong extension $M\quot N$ of algebras in $\Klf$ is \emph{minimal} if there is no algebra
$M^{\prime}\dsu M$ in $\Klf$ such that $N\subsetneq M^{\prime}\subsetneq M$ or equivalently if for
each $N_{1}\subsetneq M^{\prime}_{1}\subsetneq M_{1}$ which is $\ta$-closed and $\delta_{2}$-strong in $M_{1}$,
we have $\delta_{3}M_{1}^{\prime}>\delta_{3}M_{1}$.

Minimal extensions of $\Klz$-algebras are similarly defined.

\begin{lem}[Asymmetric-Cripple Amalgam]
Let $A,B\in\Kl$ and $M$ in $\Kl^{3}$. If $B=\gena{B_{1}}{M}$ is a subalgebra of $M$ and $B\dsu A$, % and $B\dsu C$ then
%Let $B=\gena{B_{1}}{C}$ and $B\dsu A$ then there exists a $D$ in
%$\Klf$ such that $C\dsu D$ and $A=\gena{A_{1}}{D}$.
then there exists $L\in\nla{3}$ with the following properties
%as well as
%Lie monomorphisms $\map{e}{A}{D}$ and $\map{f}{C}{D}$
%such that
\begin{itemize}
%\item[-]
\item[{\rm(i)}]%=\gena{A_{1}}{L}$ and $
$M\dsu L\nni A$ and $B$ is preserved in $L$ by both inclusions
\item[{\rm(ii)}]$\tr{2}L\in\bar{\Kl}^{2}$ and $d_{2}L<\omega$
\item[{\rm(iii)}]$L$ has $\sig{3}{2}$ and $\sig{3}{3}$.
\end{itemize}
\end{lem}

\begin{proof}
Without loss of generality we can assume that $B\dsu A$ is a $\Kl$-minimal extension and
that the algebra $A$ is not realised in $M$ over $B$.

%We start with a \emph{na\"{i}ve amalgamation} procedure {\bf (NAM)}, that yields an $\nla{3}$-algebra $L$
%in which $A$ is contained as a subalgebra, $M$ figures as a $\delta_{3}$-strong substructure and
%to which belong the following properties 

%Secondly, a discussion on the different type of minimal extensions $A\quot B$
%will provide axiom $\sig{3}{3}$ for $D$.

%\begin{itemize}
%\item[
\bigskip
%{\bf (NAM)} %]
%If we denote with $A^{\prime}$, $B^{\prime}$ and $C^{\prime}$, respectively $\tr{2}A$, $\tr{2}B$ and
%$\tr{2}C$ then, after the definition of $\Klf$, we have $A^{\prime}\zso B^{\prime}\zsu C^{\prime}$.
According to the hypothesis the truncations $\tr{2}A$ and $\tr{2}B$ lies in $\Kl^{2}$, while
$\tr{2}M\in\bar{\Kl}^{2}$.

%and because by our assumptions assuming that $\tr{2}A$ is not realised in $\tr{2}M$,
We construct the free $\nla{2}$-amalgam $\tr{2}M\star_{\tr{2}B}\tr{2}A=:\lda$ %lies still in $\bar{\Kl}^{2}$.
recalling that
$$
L^{\downarrow}=\left(M_{1}\oplus_{B_{1}}A_{1}\right)\oplus
\frac{\exs(M_{1}\oplus_{B_{1}}A_{1})}{N^{2}(M)+N^{2}(A)}.
$$

By our assumptions $\tr{2}M\nni\tr{2}B\zsu\tr{2}A$, hence $\tr{2}M\zsu L^{\downarrow}\nni\tr{2}A$. Moreover asymmetric amalgamation in $\nla{2}$ assures axiom $\sig{2}{2}$ to $\lda$. Approximating $M_{1}$ with finitely
generated strong subspaces, we also obtain $d_{2}\lda=d_{2}M_{1}+d_{2}A_{1}-d_{2}B_{1}$, which is
finite as $d_{2}M_{1}<\omega$. Set $L_{1}=M_{1}\oplus_{B_{1}}A_{1}=(\lda)_{1}$.

We also have
\begin{labeq}{inter}
\tr{2}B=\gena{B_{1}}{\lda}=\gena{M_{1}}{\lda}\cap\gena{A_{1}}{\lda}.
\end{labeq}

\medskip
%Suppose at this point that $\lda$ belongs to the class $\bar{\Kl}^{2}$, hence $\lda$ can be assumed to
%be a $\delta_{2}$-strong
%subspace of the rich $\Kl^{2}$-algebra $\K^{2}$. Therefore $\lda$ is a self-sufficient subalgebra
%of a $\ta$-closed space $\ta\lda\zsu\K^{2}$.

%We still have
%Build the free $\nla{2}$-amalgam $D^{\prime}=\fram{A^{\prime}}{B^{\prime}}{C^{\prime}}$ of $A^{\prime}$ and $C^{\prime}$ over $B^{\prime}$. Assume $D^{\prime}$ is in $\Kl^{2}$ then, because
%$B^{\prime}\zsu D^{\prime}$,
%we $\delta_{2}$-strongly embed $D^{\prime}$ in $\K^{2}$ over $B^{\prime}$,
%%we can find a copy of $D^{\prime}$ in 
%we keep calling $D^{\prime}$ the isomorphic image in the generic model of $T^{2}$.

%Consider $\ta D^{\prime}$ the $\ta$-closure of $D^{\prime}$ in $\K^{2}$,
%because both $A^{\prime}$ and $C^{\prime}$ are selfsufficient in $D^{\prime}$
%and $D^{\prime}\zsu\ta D^{\prime}$, we have that $A^{\prime}$ and $C^{\prime}$ are both $\delta_{2}$-strong
%in $\ta D^{\prime}$. Moreover $\ta$-closure implies
%\begin{labeq}{inter}
%\tr{2}B=\gena{B_{1}}{\ta\lda}=\gena{M_{1}}{\ta\lda}\cap\gena{A_{1}}{\ta\lda}.
%\end{labeq}

Now as $\tr{2}M\zsu\lda$, on account of Lemma \ref{bellemma},
there exists an embedding $j_{M}$ of $\fr{3}\tr{2}M$ %and $\fr{3}\tr{2}C$
into $\fr{3}\lda$ with image $\gena{M_{1}}{\fr{3}\lda}$.

\medskip
The same does not hold for $\fr{3}\tr{2}A$, we therefore deform it a little to
map it monomorphically into  $\fr{3}\lda$.

%Some more lines must be spent on proving that $\fr{3}\tr{2}A$ embeds in $\fr{3}\lda$.
%This is achieved once we show $\fr{3}\tr{2}\simeq\gena{A_{1}}{\fr{3}\lda}$, because
%$\fr{3}\lda\simeq\gena{L_{1}}{\fr{3}\ta\lda}$.
%To see this we use notations
In what follows, we use terminology and definitions of section \ref{emblem} and lemma \ref{bellemma}.

We choose an ordered base $X_{1}^{L}=\{X_{1}^{a}>X_{1}^{B}>X_{1}^{m}\}$ of $L_{1}$ where $X_{1}^{B}$ is a basis for $B_{1}$,
$X_{1}^{A}=X_{1}^{a}X_{1}^{B}$ is a basis for $A_{1}$ and $X_{1}^{m}$ completes $X_{1}^{B}$ to a basis $X_{1}^{M}$ of $M_{1}$. Each of the three segments of $X_{1}^{L}$ is ordered arbitrarily.

We consider the surjective map
$\map{\lambda_{A}}{\fr{3}\tr{2}A}{\gena{A_{1}}{\fr{3}\lda}}$ as in \pref{lollipop}, where
$\fr{3}\lda$ is presented by $\fla{3}{X_{1}^{L}}\quot\jei{L}$ and $\fr{3}\tr{2}A$ by $\fla{3}{X_{1}^{A}}\quot\jei{A}$.

%To prove that $\lambda_{A}$ is an injection we have to show that
We have $$\ker(\lambda_{A})=\frac{\fla{3}{X_{1}^{A}}\cap\jei{L}}{\jei{A}}%=\mathbf{0}
$$
moreover $\ker(\lambda_{A})$ is homogeneous of weight $3$.

We note that
\begin{labeq}{bulaba}
\jei{L}_{3}=\gen{\jei{M}_{3},\,\jei{A}_{3},\,[\nu^{A},y],\,[\nu^{M},x]}_{+}^{\fla{3}{X_{1}^{L}}}
\end{labeq}
$$\text{where}\quad x\in X_{1}^{a},\,y\in X_{1}^{m},\,\nu^{A}\in N^{2}(A),\,\nu^{M}\in N^{2}(M)$$
and we can assume both the $\nu^{A}$'s and the $\nu^{M}$'s to be independent over $\exs B_{1}$.

Following the reasoning which led to expression \pref{basipre}, an element $w$ in $\ker(\lambda_{A})$
%which is not zero
may be written modulo $\jei{A}$ as
\begin{labeq}{baluba}
B_{A}=w=B^{M}+pB^{M}+B^{Ma}+pB^{aM}
\end{labeq}
where $B^{M}$, $pB^{M}$ are respectively sums of basic and prebasic commutators
over $X_{1}^{B}X_{1}^{m}$ which cover the $\jei{M}$-part of $w$. $B^{Ma}$ is a sum of commutators
arising from terms of type $[\nu^{M},x]$ in $w$, these are necessarily basic after the order of $X_{1}^{L}$
we chose. $pB^{aM}$ denotes the sum of prebasic commutators obtained by the terms $[\nu^{A},y]$ of $\jei{L}$.
$B_{A}$ is a linear combination of basic commutators over $X_{1}^{A}$ which write the word $w$
as member of $\fla{3}{X_{1}^{A}}$.

We transform the prebasic commutators in the right term of \pref{baluba} in basic ones,  by means of substitutions \pref{prebi}, thus we have
$$B_{A}=B^{M}+B^{M}_{*}+B^{Ma}+B^{aM}_{*}$$
where now both terms of the equality concern basic commutators over $X_{1}^{L}$.

By uniqueness all
the terms on the right which do not appear in $B_{A}$ must be cancelled by the sum.

By the shape of the generators for $\jei{M}$ in \pref{bulaba}, we see that we fall in contradiction if
some term of type $B^{Ma}$ or $pB^{aM}$ is present in \pref{baluba},
because, after basic transformations, these cannot cancel each other, nor be cancelled by terms of $B^{M}+B_{*}^{M}$.

As a consequence $B_{A}=w=B^{M}+pB^{M}\in\jei{M}$, and this forces $w$ to lay in
$\fla{3}{X_{1}^{A}}\cap\fla{3}{X_{1}^{M}}=\fla{3}{X_{1}^{M}\cap X_{1}^{A}}=\fla{3}{X_{1}^{B}}$.

We have
\begin{labeq}{kerlam}
K:=\ker(\lambda_{A})=\frac{(\jei{M}\cap
\fla{3}{X_{1}^{B}})+\jei{A}}
{\jei{A}}\inn\gena{B_{1}}{\fr{3}\tr{2}A}.
\end{labeq}

Now because $B_{1}\zsu A_{1}$, on account of lemma \ref{bellemma} we have $\jei{B}=\jei{A}\cap\fla{3}{X_{1}^{B}}$. Hence if we look at the analog of $\lambda_{A}$ for $B$ $\map{\lambda_{B}}{\fr{3}\tr{2}B}{\fr{3}\tr{2}M}$, it holds
\begin{multline*}
\ker\lambda_{B}=
\frac{\jei{M}\cap\fla{3}{X_{1}^{B}}}{\jei{B}}=\\
=\frac{\jei{M}\cap\fla{3}{X_{1}^{B}}}{\jei{A}\cap\fla{3}{X_{1}^{B}}}
=\frac{\jei{M}\cap\fla{3}{X_{1}^{B}}}{\jei{A}\cap(\jei{M}\cap\fla{3}{X_{1}^{B}})}
\simeq K
\end{multline*}

If we consider the quotient maps of $\lambda_{A}$ and $\lambda_{B}$ modulo $K$, and name them
$\bar\lambda_{A}$ and $\bar\lambda_{B}$, we have the following injective commutative diagram:
\begin{labeq}{pream}
\xymatrix{
&{\fr{3}\lda}&\\
{\fr{3}\tr{2}M}\ar%@{^{(}->}
[ur]^{j_{M}}&&{\fr{3}\tr{2}A\quot K}\ar%@{_{(}->}
[ul]_{\bar\lambda_{A}}\\
&{\fr{3}\tr{2}B\quot K}\ar%@{_{(}->}
[ul]_{\bar\lambda_{B}}\ar%@{^{(}->}
[ur]^{j_{B}}&}.\end{labeq}
Now %by \pref{kerlam}
we note $\jei{M}\cap\fla{3}{X_{1}^{B}}$ are relators for $B$, hence $K\inn N^{3}(B)\inn N^{3}(A)$.
This allows us to write diagram \pref{pream} modulo $N^{3}(M)$, $N^{3}(B)\quot K$ and $N^{3}(A)\quot K$.
We keep on calling $N^{3}(M)$, its image under $j_{M}$ and we set $N_{A}=\bar\lambda_{A}(N^{3}(A)\quot K)$.
We get
$$
\xymatrix{
&\frac{\fr{3}\lda}{N^{3}(M)+N_{A}}&\\
{M}\ar[ur]^{\bar j_{M}}&&{A}\ar[ul]_{\bar{\bar\lambda}_{A}}\\
&{B}\ar[ul]_{\bar{\bar\lambda}_{B}}\ar[ur]^{\bar j_{B}}&}
$$
%-----------------------------------------------------------------------------------------------------------------

{\bf This was percented \dots\dots }
\begin{flushright}
Therefore $w$ is in $\jei{B}\inn\jei{A}$ as we desired.
With similar arguments we can show that
$\gena{B_{1}}{\fr{3}\lda}=\gena{M_{1}}{\fr{3}\lda}\cap\gena{A_{1}}{\fr{3}\lda}$,
therefore
and $\gena{\cbu}{\fr{3}\ta D^{\prime}}$ respectively.
\bigskip
Using similar arguments and \pref{inter}
%and lemma \ref{bellemmino}, 
we get
\begin{labeq}{freeint}
\gena{B_{1}}{\fr{3}\lda}=\gena{M_{1}}{\fr{3}\lda}\cap\gena{A_{1}}{\fr{3}\lda}.
\end{labeq}
here $B_{1}$, $\abu$ and $\cbu$ are identified with its images modulo $j_{A}$ and $j_{C}$. 

With abuse in notation we keep calling respectively
$N^{3}(M)$ and $N^{3}(A)$ the images of these ideals in $\fr{3}\lda$ under the
maps $j_{M}$ and $j_{A}$.
\end{flushright}
%-----------------------------------------------------------------------------------------

Now build the $\nla{3}$-algebra

\bigskip
Set
$$L=\frac{\fr{3}\lda}{N^{3}(M)+N_{A}}.$$
%$L$ is our candidate for the proof of Lemma.

The maps $\bar j_{M}$ and $\bar{\bar{\lambda}}_{A}$ are one-to-one because
$j_{M}^{-1}(N^{3}(M)^{j_{M}}+N_{A})=N^{3}(M)$ and similarly $\bar\lambda_{A}^{-1}(N^{3}(M)+N_{A})=N^{3}(A)\quot K$.
This follows from $\gena{M_1}{\fr{3}\lda}\cap\gena{A_1}{\fr{3}\lda}=\gena{B_1}{\fr{3}\lda}$.

As $N^{3}(M)+N_{A}$ is an ideal of $\fr{3}\lda$ consisting only of weight $3$ elements, we have
$\fr{3}\tr{2}L=\fr{3}(\tr{2}\fr{3}\lda)=\fr{3}\lda$, hence we obtain that $N^{3}(M)+N_{A}$ is exactly
$N^{3}(L)$ of definition \pref{3ker}.

%With abuse, we set $N^{3}(B):=j_{M}(N^{3}(B))$ which, because
%$$N^{3}(M)\cap\gena{B_{1}}{\fr{3}\tr{2}M}=N^{3}(B)=N^{3}(A)\cap\gena{B_{1}}{\fr{3}\tr{2}A}$$
%%and because of lemma \ref{bellemmino},
%is also equal to $j_{A}(N^{3}(B))$ and to $N^{3}(M)\cap N^{3}(A)$.

The following compatibility equations also hold
\begin{labeq}{compa}
N^{3}(M)=N^{3}(L)\cap\gena{M_{1}}{\fr{3}\lda}
\end{labeq}and
$$
N^{3}(A)=N^{3}(L)\cap\gena{A_{1}}{\fr{3}\lda}.
$$

%So far, are $M$ and $A$ embeddable in $L$ via Lie monomorphisms,
%just regard for example $M\simeq\fr{3}\tr{2}M\quot N^{3}(M)$ and take the quotient of $j_{M}$:
%\begin{eqnarray}
%\map{\bar j _{M}}{&\fr{3}\tr{2}M\quot N^{3}(M)}{L}\\
%&\bar w\longmapsto \bar w
%\end{eqnarray}
%on account of \pref{compa} is $\bar j_{M}$ one-to-one, in particular its image is again
%$\gena{M_{1}}{L}=(\gena{M_{1}}{\fr{3}\ta\lda}+N^{3}(L))\quot N^{3}(L)$.

%An embedding of $A$ in $L$ is constructed in the very same way.
%-------------------------------------------------------------------------------------------------------------------

\medskip
We show next that
%if $B_{1}\dsu A_{1}$
the embedding of $M$ into $L$ is $\delta_{3}$-strong. In what follows the $N^{3}$-parts are calculated with
respect of $L$.

It suffices to prove $\delta_{3}E_{1}\geq\delta_{3}M_{1}$ for each $E_{1}\zsu L_{1}$ such that
$E_{1}\nni M_{1}$.%and $E_{1}$ $$-closed.

%We first observe $\delta_{3}E_{1}\geq\delta_{3}(E_{1}\cap D_{1})$. On one hand we have $d_{2}E_{1}\geq
%d_{2}(E_{1}\cap D_{1})$. On the other hand,

%and in particular $\gena{E_{1}}{\fr{3}\ta\lda}\cap\gena{A_{1}}{\fr{3}\ta\lda}=\gena{E_{1}\cap A_{1}}{\fr{3}\ta\lda}$.
%Therefore
%\begin{multline}\label{ntreonto}
%N^{3}(E_{1})=\\
%=N^{3}(L)\cap\gena{E_{1}}{\fr{3}\ta D^{\prime}}=
%(N^{3}(A)+N^{3}(C))\cap\gena{E_{1}}{\fr{3}\ta D^{\prime}}=\\
%=(N^{3}(A)\cap\gena{E_{1}}{\fr{3}\ta D^{\prime}})+N^{3}(C)=\\
%=(N^{3}(L)\cap\gena{A_{1}}{\fr{3}\ta D^{\prime}}\cap\gena{E_{1}}{\fr{3}\ta D^{\prime}})+N^{3}(C)=\\
%=(N^{3}(L)\cap\gena{A_{1}\cap E_{1}}{\fr{3}\ta D^{\prime}})+N^{3}(C)\inn\\
%\inn N^{3}(E_{1}\cap A_{1}+M_{1})\inn N^{3}(E_{1}\cap L_{1})
%\end{multline}
%Therefore we have to show $\delta_{3}(E_{1}\cap L_{1})\geq\delta_{3}M_{1}$.

%As $E_{1}\cap D_{1}=M_{1}+(E_{1}\cap \abu)$ w
Because
%$E_{1}\cap \abu$ is a $\delta_{2}$ strong subspace of $\abu$ and 
$B\dsu A$, %and $\delta_{3}(E_{1}\cap \abu\quot B)\geq 0W
we finish once we show
$\delta_{3}E_{1}-\delta_{3}M\geq\delta_{3}(E_{1}\cap A_{1})-\delta_{3}B$.

Approximating $E_{1}$ with a sequence of strong sets in $\lda$, on one side we have
%$$\delta_{3}(E_{1}\cap D_{1}\quot C)
%$$\delta_{3}(E_{1}\cap D_{1})-\delta_{3}(M_{1})=
%d_{2}(E_{1}\cap D_{1}\quot M_{1})
%-\big(\dfp(N^{3}(E_{1}\cap D_{1}))-\dfp(N^{3}(C))\big).$$
%Everything behaves good for $d_{2}$, we have 
$d_{2}E_{1}-d_{2}(M_{1})=d_{2}(E_{1}\cap A_{1})-d_{2}B_{1}$.
%(\dots eigentlich stimmt das f`\"ur $\delta_{2}$ aber man kann approximieren \dots)

%To conclude we have to show
It remains to show that
$$\dfp(N^{3}(E_{1})\quot N^{3}(M))\leq\dfp(N^{3}(E_{1}\cap A_{1})\quot N^{3}(B)).$$
So assume $\Psi_{1},\dots,\Psi_{m}$ are in $N^{3}(E_{1})=N^{3}(L)\cap\gena{E_{1}}{\fr{3}\lda}$ independent
over $\gena{M_{1}}{\fr{3}\lda}$. By definition of $N^{3}(L)$, for each $i$ we have
$\Psi_{i}=\Psi_{i}^{M}+\Psi_{i}^{A}$, where $\Psi_{i}^{M}\in N^{3}(M)$ and $\Psi_{i}^{A}\in N^{3}(A)$.

Now we see that, because $\lda$ is the free amalgam of $\tr{2}M$ and $\tr{2}A$ and $E_{1}\nni M_{1}$,
it holds $\gena{E_{1}}{\lda}\cap\gena{A_{1}}{\lda}=\gena{E_{1}\cap A_{1}}{\lda}$,
therefore $$\gena{E_{1}}{\fr{3}\lda}\cap\gena{A_{1}}{\fr{3}\lda}=\gena{E_{1}\cap A_{1}}{\fr{3}\lda}.$$
%and in particular $N^{3}(E_{1}\cap A_{1})=N^{3}(E_{1})\cap N^{3}(A_{1})$.

In our case then it follows that the set $\{\Psi_{i}^{A}\}_{i=1}^{m}$ lies in $N^{3}(E_{1}\cap A_{1})$ and it is easy
to see that it is independent
over $\gena{B_{1}}{\fr{3}\lda}$. This gives the desired inequality and proves $M_{1}\dsu L_{1}$. 

%Consider the map
%\begin{eqnarray*}
%N^{3}(E_{1}\cap \abu)\quot N^{3}(B)^{\bullet}&\longrightarrow&N^{3}(E_{1}\cap D_{1})\quot N^{3}(C)^{\bullet}\\
%\bar\eta&\longmapsto&\bar\eta\quad\quad\forall\eta\in N^{3}(E_{1}\cap\abu).
%\end{eqnarray*}
%Soundness is immediate. Moreover our map is injective because
%\begin{multline*}
%N^{3}(C)^{\bullet}\cap\gena{E_{1}\cap \abu}{\fr{3}\ta D^{\prime}}=\\
%=N^{3}(D)\cap\gena{\cbu}{\fr{3}\ta D^{\prime}}\cap\gena{E_{1}\cap \abu}{\fr{3}\ta D^{\prime}}=\\
%=N^{3}(D)\cap\gena{\abu\cap C_{1}}{\fr{3}\ta D^{\prime}}=N^{3}(B)^{\bullet}.
%\end{multline*}
%Here we used $\gena{\cbu}{\fr{3}\ta D^{\prime}}\cap\gena{E_{1}\cap\abu}{\fr{3}\ta D^{\prime}}=
%\gena{\cbu\cap\abu}{\fr{3}\ta D^{\prime}}$ because both $\cbu$ and $E_{1}\cap\abu$ are $\delta_{2}$ strong in $D_{1}$ and $\ta$ closed and because $\gena{\cbu}{\ta D^{\prime}}\cap\gena{E_{1}\cap\abu}{\ta D^{\prime}}=\gena{\bbu}{\ta D^{\prime}}$.

%To see that this map is onto use the arguments in \pref{ntreonto} again to get
%$$N^{3}(E_{1}\cap D_{1})=(
%N^{3}(D)\cap\gena{E_{1}\cap A_{1}}{\fr{3}\ta D^{\prime}})+N^{3}(C)^{\bullet}.$$
%On the other hand, since again $\gena{\abu}{\ta D^{\prime}}\cap\gena{E_{1}}{\ta D^{\prime}}=\gena{E_{1}\cap\abu}{\ta D^{\prime}}$ we have
%\begin{multline*}
%N^{3}(E_{1})=N^{3}(D)\cap\gena{E_{1}}{\fr{3}\ta D^{\prime}}=\\
%=(N^{3}(A)^{\bullet}+N^{3}(C)^{\bullet})\cap\gena{E_{1}}{\fr{3}\ta D^{\prime}}=\\
%=(N^{3}(A)^{\bullet}\cap\gena{E_{1}}{\fr{3}\ta D^{\prime}})+N^{3}(C)^{\bullet}=\\
%=(N^{3}(D)\cap\gena{E_{1}\cap A_{1}}{\fr{3}\ta D^{\prime}})+N^{3}(C)^{\bullet}
%\end{multline*}
%and this gives that our map is onto.

%We have shown that
%$$\dfp(N^{3}(E))-\dfp(N^{3}(C))=\dfp(N^{3}(E_{1}\cap\abu))-\dfp(N^{3}(B)^{\bullet})$$
%thus $C$ is $\delta_{3}$-strong embeddable in $D$.

%A completely analogous argument proves that $A$ is $\delta_{3}$-strong in $D$.
%%\end{itemize}

\medskip
We show that $L$ has $\sig{3}{2}$.
It suffices to show $\delta_{3}H_{1}\geq2$ for each  $H_{1}\zsu L_{1}$ of finite linear dimension. 

Consider the relative $\ta$-closure $H_{1}^{+}$ of $H_{1}$ in $L_{1}$, then because $M_{1}$ is
$\ta$-closed, it holds
$\gena{H_{1}^{+}}{\lda}\cap\gena{M_{1}}{\lda}=\gena{H_{1}^{+}\cap M_{1}}{\lda}$.

Now by lemma \ref{presubatre} and since $M_{1}\dsu L_{1}$, we have
$$
\delta_{3}(H_{1}^{+}+M_{1})+\delta_{3}(H_{1}^{+}\cap M_{1})
\leq\delta_{3}H_{1}^{+}+\delta_{3}M_{1}\leq\delta_{3}H_{1}^{+}+\delta_{3}(H_{1}^{+}+M_{1}).
$$
Now as $M\in\Kl^{was\dots}$, we conclude
$\delta_{3}H_{1}\geq\delta_{3}H_{1}^{+}\geq\delta_{3}(H_{1}^{+}\cap M_{1})\geq2$.

\bigskip
%So far we have exhibited an $L\in\nla{3}$
To conclude the proof we have to provide axiom $\sig{3}{3}$ for $L$
and show that $\tr{2}L=\lda$ lies in $\Kl^{2}$, that is $\lda$ must satisfy axiom $\sig{2}{3}$.

This is done discussing the kind of minmal extension $A\quot B$ which can actually occur.
We recall here axiom $\sig{3}{3}$
$$(\sig{3}{3})\quad(\forall z,\,P_{2}(z))(\forall x,y,\,P_{1}(x)\wedge P_{1}(y))([z,x]=[z,y]\,\rightarrow
%\textsl{``$x$ lin.{}depends on $y$''}
\:x=y)$$
and note that, because both $M$ and $A$ satisfy $\sig{3}{3}$ the only case for it to fail in $L$ is in which
there exist elements $e\in A_{1}\non B_{1}$ and $m\in M_{1}\non B_{1}$ and $0\neq\beta\in B_{2}$ such
that
\begin{labeq}{sigtre}
[\beta,m]=[\beta,e]\in B_{3}.
\end{labeq}

\begin{itemize}
\item[-]\emph{Algebraic extension.}
%(the only case in which the axiom could fail)
Assume $\sig{3}{3}$ fails in $L$, then there exists $\Phi\in\gena{B_{1}}{\fr{3}\tr{2}A}$ such that  $\Phi-[\beta,e]\in N^{3}(A)$
for $e\in A_{1}\non B_{1}$.

It follows $\delta_{3}(e\quot B_{1})=0$ and $d_{2}(e\quot B_{1})=1$ so
$\gen{B_{1},e}\zsu A_{1}$ and by minimality $A_{1}=\gen{B_{1},e}_{1}$.
$N^{3}(A)=\gena{N^{3}(B),\Phi-[\beta,e]}{\fr{3}\tr{2}A}$.

Moreover \pref{sigtre} implies that we can realise $A$ in $M$ over $B$.
This is forbidden by our hypothesis.  

We can also prove $\tr{2}A$ is free minimal over $\tr{2}B$ and $\lda\in\bar{\Kl}^{2}$.

\item[-]\emph{Free extension.} Assume $\delta_{3}(A\quot B)>0$ it follows $d_{2}A>d_{2}B$ and
$\tr{2}A$ is free minimal over $\tr{2}B$, therefore $\lda$ is
of the desired kind. We have $A_{1}=\gen{B_{1},e}_{1}$ and $N^{3}(A)=N^{3}(B)$, hence
$\sig{3}{3}$ holds in $L$ for $[\beta,e]\notin B_{3}$ for all $\beta$ and $e\in A_{1}\non
B_{1}$.

\item[-]\emph{Prealgebraic extension.} We are in the case $\delta_{3}(A\quot B)=0$ but 
%$\tr{2}A_{1}$ is not realised in $\tr{2}M$
$\delta_{3}(e\quot B_{1})>0$ for each $e\in A_{1}\non B_{1}$; in particular $\sig{3}{3}$ cannot
fail in $L$ and $\delta_{2}(e\quot B_{1})>0$ for all $e\in A_{1}\non B_{1}$.

This assures that $\lda\sat\sig{2}{3}$ therefore $\lda\in\bar{\Kl}^{2}$.
As we've seen axiom $\sig{3}{3}$ does not fail in this case in $L$.
\end{itemize}
This concludes the proof as there's no other case to consider.

\end{proof}
%\begin{lem}[Asymmetric Amalgamation Lemma]
Let $A,B\in\Klz$ and $M$ in $\Kl^{3}$. If $B$ is a subalgebra of $M$ ($B=\gena{B_{1}}{M}$) and $B\dsu A$, % and $B\dsu C$ then
%Let $B=\gena{B_{1}}{C}$ and $B\dsu A$ then there exists a $D$ in
%$\Klf$ such that $C\dsu D$ and $A=\gena{A_{1}}{D}$.
then there exists $L\in\K^{3}$ such that
%as well as
%Lie monomorphisms $\map{e}{A}{D}$ and $\map{f}{C}{D}$
%such that
$A=\gena{A_{1}}{L}$ and $M\dsu L$.
%and $B^{e}=B^{f}$ pointwise.
\end{lem}

\begin{proof}
Without loss of generality we can assume that $B\dsu A$ is a $\Klz$-minimal extension.

%We start with a \emph{na\"{i}ve amalgamation} procedure {\bf (NAM)}, that yields an $\nla{3}$-algebra $L$
%in which $A$ is contained as a subalgebra, $M$ figures as a $\delta_{3}$-strong substructure and
%to which belong the following properties 

%Secondly, a discussion on the different type of minimal extensions $A\quot B$
%will provide axiom $\sig{3}{3}$ for $D$.

%\begin{itemize}
%\item[
\bigskip
{\bf (NAM)} %]
%If we denote with $A^{\prime}$, $B^{\prime}$ and $C^{\prime}$, respectively $\tr{2}A$, $\tr{2}B$ and
%$\tr{2}C$ then, after the definition of $\Klf$, we have $A^{\prime}\zso B^{\prime}\zsu C^{\prime}$.
Consider the truncations
$$\tr{2}M\nni\tr{2}B\zsu\tr{2}A$$ and
construct the free $\nla{2}$-amalgam $\tr{2}M\star_{\tr{2}B}\tr{2}A=:\lda$.

We recall that
$$
L^{\downarrow}=\left(M_{1}\oplus_{B_{1}}A_{1}\right)\oplus
\frac{\exs(M_{1}\oplus_{B_{1}}A_{1})}{N^{2}(M)+N^{2}(A)}
$$
and that $\tr{2}M\zsu L^{\downarrow}\nni\tr{2}A$, moreover, approximating $M_{1}$ with finitely
generated strong subspaces, we obtain $d_{2}\lda=d_{2}M_{1}+d_{2}A_{1}-d_{2}B_{1}$, which is
finite as $d_{2}M_{1}<\omega$. Set $L_{1}=M_{1}\oplus_{B_{1}}A_{1}=(\lda)_{1}$.

We also have $\tr{2}B=\gena{B_{1}}{\lda}=\gena{M_{1}}{\lda}\cap\gena{A_{1}}{\lda}$.

\medskip
Suppose at this point that $\lda$ belongs to the class $\bar{\Kl}^{2}$, hence $\lda$ can be assumed to
be a $\delta_{2}$-strong
subspace of the rich $\Kl^{2}$-algebra $\K^{2}$. Therefore $\lda$ is a self-sufficient subalgebra
of a $\ta$-closed space $\ta\lda\zsu\K^{2}$.

We still have
%Build the free $\nla{2}$-amalgam $D^{\prime}=\fram{A^{\prime}}{B^{\prime}}{C^{\prime}}$ of $A^{\prime}$ and $C^{\prime}$ over $B^{\prime}$. Assume $D^{\prime}$ is in $\Kl^{2}$ then, because
%$B^{\prime}\zsu D^{\prime}$,
%we $\delta_{2}$-strongly embed $D^{\prime}$ in $\K^{2}$ over $B^{\prime}$,
%%we can find a copy of $D^{\prime}$ in 
%we keep calling $D^{\prime}$ the isomorphic image in the generic model of $T^{2}$.

%Consider $\ta D^{\prime}$ the $\ta$-closure of $D^{\prime}$ in $\K^{2}$,
%because both $A^{\prime}$ and $C^{\prime}$ are selfsufficient in $D^{\prime}$
%and $D^{\prime}\zsu\ta D^{\prime}$, we have that $A^{\prime}$ and $C^{\prime}$ are both $\delta_{2}$-strong
%in $\ta D^{\prime}$. Moreover $\ta$-closure implies
\begin{labeq}{inter}
\tr{2}B=\gena{B_{1}}{\ta\lda}=\gena{M_{1}}{\ta\lda}\cap\gena{A_{1}}{\ta\lda}.
\end{labeq}

Now as $\tr{2}M\zsu(\lda\zsu)\,\ta\lda$, on account of Lemma \ref{bellemma},
there exists an embedding $j_{M}$ of $\fr{3}\tr{2}M$ %and $\fr{3}\tr{2}C$
into $\fr{3}\ta\lda$ with image $\gena{M_{1}}{\fr{3}\ta\lda}$.

Some more lines must be spent on proving that $\fr{3}\tr{2}A$ embeds in $\fr{3}\ta\lda$.
This is achieved once we show $\fr{3}\tr{2}\simeq\gena{A_{1}}{\fr{3}\lda}$, because
$\fr{3}\lda\simeq\gena{L_{1}}{\fr{3}\ta\lda}$.
To see this let \dots

With similar arguments we can show that
$\gena{B_{1}}{\fr{3}\lda}=\gena{M_{1}}{\fr{3}\lda}\cap\gena{A_{1}}{\fr{3}\lda}$,
therefore
%and $\gena{\cbu}
%{\fr{3}\ta D^{\prime}}$ respectively.
%From \pref{inter} and lemma \ref{bellemmino} instead, we get
\begin{labeq}{freeint}
\gena{B_{1}}{\fr{3}\ta\lda}=\gena{M_{1}}{\fr{3}\ta\lda}\cap\gena{A_{1}}{\fr{3}\ta\lda}.
\end{labeq}
%here $B_{1}$, $\abu$ and $\cbu$ are identified with its images modulo $j_{A}$ and $j_{C}$. 

With abuse in notation we keep calling respectively
$N^{3}(M)$ and $N^{3}(A)$ the images of these ideals in $\fr{3}\ta\lda$ under the
maps $j_{M}$ and $j_{A}$.

Now build the $\nla{3}$-algebra
$$L=\frac{\fr{3}\ta\lda}{N^{3}(M)+N^{3}(A)}.$$

As $N^{3}(M)+N^{3}(A)$ is an ideal of $\fr{3}\ta\lda$ consisting only of weight $3$ elements, we have
$\fr{3}\tr{2}L=\fr{3}(\tr{2}\fr{3}\ta\lda)=\fr{3}\ta\lda$, hence we obtain that $N^{3}(M)+N^{3}(A)$ is exactly
$N^{3}(L)$ of definition \pref{3ker}.

With abuse, from \pref{freeint} we set
$$N^{3}(B):=j_{M}(N^{3}(B))=N^{3}(M)\cap N^{3}(A)=j_{A}(N^{3}(B))$$
and the following compatibility equations
\begin{labeq}{compa}
N^{3}(M)=N^{3}(L)\cap\gena{M_{1}}{\fr{3}\ta\lda}
\end{labeq}and
$$
N^{3}(A)=N^{3}(L)\cap\gena{A_{1}}{\fr{3}\ta\lda}.
$$

So far, are $M$ and $A$ embeddable in $L$ via Lie monomorphisms,
just regard for example $M\simeq\fr{3}\tr{2}M\quot N^{3}(M)$ and take the quotient of $j_{M}$:
\begin{eqnarray}
\map{\bar j _{M}}{&\fr{3}\tr{2}M\quot N^{3}(M)}{L}\\
&\bar w\longmapsto \bar w
\end{eqnarray}
on account of \pref{compa} is $\bar j_{M}$ one-to-one, in particular its image is again
$\gena{M_{1}}{L}=(\gena{M_{1}}{\fr{3}\ta\lda}+N^{3}(L))\quot N^{3}(L)$.

An embedding of $A$ in $L$ is constructed in the very same way.

\medskip
We show next that
%if $B_{1}\dsu A_{1}$
the embedding of $M$ into $D$ is $\delta_{3}$-strong.

It suffices to prove $\delta_{3}E_{1}\geq\delta_{3}M_{1}$ for each $E_{1}\zsu\ta\lda_{1}$ such that
$E_{1}\nni M_{1}$.%and $E_{1}$ $\ta$-closed.

We first observe $\delta_{3}E_{1}\geq\delta_{3}(E_{1}\cap D_{1})$. On one hand we have $d_{2}E_{1}\geq
d_{2}(E_{1}\cap D_{1})$. On the other hand,
it can be shown that, because $\lda$ is the free amalgam of $\tr{2}M$ and $\tr{2}A$,
it holds $\gena{E_{1}}{\lda}\cap\gena{A_{1}}{\lda}=\gena{E_{1}\cap A_{1}}{\lda}$,
and in particular $\gena{E_{1}}{\fr{3}\ta\lda}\cap\gena{A_{1}}{\fr{3}\ta\lda}=\gena{E_{1}\cap A_{1}}{\fr{3}\ta\lda}$.

Therefore
\begin{multline}\label{ntreonto}
N^{3}(E_{1})=\\
=N^{3}(L)\cap\gena{E_{1}}{\fr{3}\ta D^{\prime}}=
(N^{3}(A)+N^{3}(C))\cap\gena{E_{1}}{\fr{3}\ta D^{\prime}}=\\
=(N^{3}(A)\cap\gena{E_{1}}{\fr{3}\ta D^{\prime}})+N^{3}(C)=\\
=(N^{3}(L)\cap\gena{A_{1}}{\fr{3}\ta D^{\prime}}\cap\gena{E_{1}}{\fr{3}\ta D^{\prime}})+N^{3}(C)=\\
=(N^{3}(L)\cap\gena{A_{1}\cap E_{1}}{\fr{3}\ta D^{\prime}})+N^{3}(C)\inn\\
\inn N^{3}(E_{1}\cap A_{1}+M_{1})\inn N^{3}(E_{1}\cap L_{1})
\end{multline}
Therefore we have to show $\delta_{3}(E_{1}\cap L_{1})\geq\delta_{3}M_{1}$.

As $E_{1}\cap D_{1}=M_{1}+(E_{1}\cap \abu)$ we finish once we show
$\delta_{3}(E_{1}\cap D_{1}\quot C)=\delta_{3}(E_{1}\cap\abu\quot B)$ because
%$E_{1}\cap \abu$ is a $\delta_{2}$ strong subspace of $\abu$ and 
$B\dsu A$ and $\delta_{3}(E_{1}\cap \abu\quot B)\geq 0$.

We have
%$$\delta_{3}(E_{1}\cap D_{1}\quot C)
$$\delta_{3}(E_{1}\cap D_{1})-\delta_{3}(M_{1})=
d_{2}(E_{1}\cap D_{1}\quot M_{1})
-\big(\dfp(N^{3}(E_{1}\cap D_{1}))-\dfp(N^{3}(C))\big).$$
Everything behaves good for $d_{2}$, we have $d_{2}(E_{1}\cap D_{1})-d_{2}(M_{1})=
d_{2}(E_{1}\cap\abu)-d_{2}B_{1}$. (\dots eigentlich stimmt das f`\"ur $\delta_{2}$ aber
man kann approximieren \dots)

To conclude we have to show
$$N^{3}(E_{1}\cap\abu)\quot N^{3}(B)^{\bullet}\simeq
N^{3}(C_{1}+(E_{1}\cap \abu))\quot N^{3}(C)^{\bullet}.$$
Consider the map
\begin{eqnarray*}
N^{3}(E_{1}\cap \abu)\quot N^{3}(B)^{\bullet}&\longrightarrow&N^{3}(E_{1}\cap D_{1})\quot N^{3}(C)^{\bullet}\\
\bar\eta&\longmapsto&\bar\eta\quad\quad\forall\eta\in N^{3}(E_{1}\cap\abu).
\end{eqnarray*}
Soundness is immediate. Moreover our map is injective because
\begin{multline*}
N^{3}(C)^{\bullet}\cap\gena{E_{1}\cap \abu}{\fr{3}\ta D^{\prime}}=\\
=N^{3}(D)\cap\gena{\cbu}{\fr{3}\ta D^{\prime}}\cap\gena{E_{1}\cap \abu}{\fr{3}\ta D^{\prime}}=\\
=N^{3}(D)\cap\gena{\abu\cap C_{1}}{\fr{3}\ta D^{\prime}}=N^{3}(B)^{\bullet}.
\end{multline*}
Here we used $\gena{\cbu}{\fr{3}\ta D^{\prime}}\cap\gena{E_{1}\cap\abu}{\fr{3}\ta D^{\prime}}=
\gena{\cbu\cap\abu}{\fr{3}\ta D^{\prime}}$ because both $\cbu$ and $E_{1}\cap\abu$ are $\delta_{2}$ strong in $D_{1}$ and $\ta$ closed and because $\gena{\cbu}{\ta D^{\prime}}\cap\gena{E_{1}\cap\abu}{\ta D^{\prime}}=\gena{\bbu}{\ta D^{\prime}}$.

To see that this map is onto use the arguments in \pref{ntreonto} again to get
$$N^{3}(E_{1}\cap D_{1})=(
N^{3}(D)\cap\gena{E_{1}\cap A_{1}}{\fr{3}\ta D^{\prime}})+N^{3}(C)^{\bullet}.$$
%On the other hand, since again $\gena{\abu}{\ta D^{\prime}}\cap\gena{E_{1}}{\ta D^{\prime}}=\gena{E_{1}\cap\abu}{\ta D^{\prime}}$ we have
%\begin{multline*}
%N^{3}(E_{1})=N^{3}(D)\cap\gena{E_{1}}{\fr{3}\ta D^{\prime}}=\\
%=(N^{3}(A)^{\bullet}+N^{3}(C)^{\bullet})\cap\gena{E_{1}}{\fr{3}\ta D^{\prime}}=\\
%=(N^{3}(A)^{\bullet}\cap\gena{E_{1}}{\fr{3}\ta D^{\prime}})+N^{3}(C)^{\bullet}=\\
%=(N^{3}(D)\cap\gena{E_{1}\cap A_{1}}{\fr{3}\ta D^{\prime}})+N^{3}(C)^{\bullet}
%\end{multline*}
%and this gives that our map is onto.

We have shown that
$$\dfp(N^{3}(E))-\dfp(N^{3}(C))=\dfp(N^{3}(E_{1}\cap\abu))-\dfp(N^{3}(B)^{\bullet})$$
thus $C$ is $\delta_{3}$-strong embeddable in $D$.

A completely analogous argument proves that $A$ is $\delta_{3}$-strong in $D$.
%\end{itemize}

\smallskip
We show that $D$ has $\sig{3}{2}$.
It suffices to show $\delta_{3}H_{1}\geq2$ for each  $H_{1}\zsu\ta D_{1}$ of finite linear dimension. 

Consider the relative $\ta$-closure $H_{1}^{+}$ of $H_{1}$ in $D$, then it holds
$\gena{H_{1}^{+}}{\tr{2}D}\cap\gena{C_{1}}{\tr{2}D}=\gena{H_{1}^{+}\cap C_{1}}{\tr{2}D}$.

By lemma \ref{}, since $C_{1}\dsu\ta D_{1}$, we have
$$
\delta_{3}(H_{1}^{+}+C_{1})+\delta_{3}(H_{1}^{+}\cap C_{1})
\leq\delta_{3}H_{1}^{+}+\delta_{3}C_{1}\leq\delta_{3}H_{1}^{+}+\delta_{3}(H_{1}^{+}+C_{1}).
$$
Now as $C\in\Klf$, we conclude
$\delta_{3}H_{1}\geq\delta_{3}H_{1}^{+}\geq\delta_{3}(H_{1}^{+}\cap C_{1})\geq2$.


To conclude the proof we have to
show that $\tr{2}D$ actually lies in $\Kl^{2}$ and provide axiom $\sig{3}{2}$ for $D$.
\end{proof}

%
\begin{lem}[Amalgamation Lemma]
Let $A,B$ and $C$ in $\Klz$. If $A\dso B\dsu C$ % and $B\dsu C$ then
%Let $B=\gena{B_{1}}{C}$ and $B\dsu A$ then there exists a $D$ in
%$\Klf$ such that $C\dsu D$ and $A=\gena{A_{1}}{D}$.
then there exists $D\in\Klz$ as well as
Lie monomorphisms $\map{e}{A}{D}$ and $\map{f}{C}{D}$
such that
$A^{e}\dsu D$ and $C^{f}\dsu C$ and $B^{e}=B^{f}$ pointwise.
\end{lem}

\begin{proof}
Without loss of generality we can assume that $B\dsu A$ is a $\Klf$-minimal extension.

We first describe a \emph{na\"{i}ve amalgamation} {\bf (NAM)} that yields an $\nla{3}$-algebra
$D=
A\star_{B}C$ in which $A$ and $C$ embeds over $B$, moreover axiom $\sig{3}{2}$ will be shown to hold in $D$.

Secondly, a discussion on the different type of minimal extensions $A\quot B$
will provide axiom $\sig{3}{3}$ for $D$.

\begin{itemize}
\item[{\bf (NAM)}]
If we denote with $A^{\prime}$, $B^{\prime}$ and $C^{\prime}$, respectively $\tr{2}A$, $\tr{2}B$ and
$\tr{2}C$ then, after the definition of $\Klf$, we have $A^{\prime}\zso B^{\prime}\zsu C^{\prime}$.
We build the free $\nla{2}$-amalgam $D^{\prime}=\fram{A^{\prime}}{B^{\prime}}{C^{\prime}}$ of $A^{\prime}$ and $C^{\prime}$ over $B^{\prime}$. Assume $D^{\prime}$ is in $\Kl^{2}$ then, because
$B^{\prime}\zsu D^{\prime}$,
we $\delta_{2}$-strongly embed $D^{\prime}$ in $\K^{2}$ over $B^{\prime}$,
%we can find a copy of $D^{\prime}$ in 
we keep calling $D^{\prime}$ the isomorphic image in the generic model of $T^{2}$.

Consider $\ta D^{\prime}$ the $\ta$-closure of $D^{\prime}$ in $\K^{2}$,
because both $A^{\prime}$ and $C^{\prime}$ are selfsufficient in $D^{\prime}$
and $D^{\prime}\zsu\ta D^{\prime}$, we have that $A^{\prime}$ and $C^{\prime}$ are both $\delta_{2}$-strong
in $\ta D^{\prime}$. Moreover $\ta$-closure implies
\begin{labeq}{inter}
B^{\prime}=\gena{B_{1}}{\ta D^{\prime}}=\gena{\abu}{\ta D^{\prime}}\cap\gena{\cbu}{\ta D^{\prime}}.
\end{labeq}

Now on account of Lemma \ref{bellemma},
there exist two embeddings $j_{A}$ and $j_{C}$ of $\fr{3}\tr{2}A$ and $\fr{3}\tr{2}C$
into $\fr{3}\ta D^{\prime}$ their images being $\gena{\abu}{\fr{3}\ta D^{\prime}}$ and $\gena{\cbu}
{\fr{3}\ta D^{\prime}}$ respectively. From \pref{inter} and lemma \ref{bellemmino} instead, we get
\begin{labeq}{freeint}
\gena{B_{1}}{\fr{3}\ta D^{\prime}}=\gena{\abu}{\fr{3}\ta D^{\prime}}\cap\gena{\cbu}{\fr{3}\ta D^{\prime}}
\end{labeq}
here $B_{1}$, $\abu$ and $\cbu$ are identified with its images modulo $j_{A}$ and $j_{C}$. 

We set $N^{3}(A)^{\bullet}=j_{A}(N^{3}(A))$ and $N^{3}(C)^{\bullet}=j_{C}(N^{3}(C))$.
And we build
$$D=\fr{3}\ta D^{\prime}\quot(N^{3}(A)^{\bullet}+N^{3}(C)^{\bullet}).$$
As $N^{3}(A)^{\bullet}+N^{3}(C)^{\bullet}$ is an ideal consisting only of weight $3$ elements,
$\fr{3}\tr{2}D=\fr{3}(\tr{2}\fr{3}\ta D^{\prime})%\simeq
=\fr{3}\ta D^{\prime}$, hence we obtain $N^{3}(D)=N^{3}(A)^{\bullet}+N^{3}(C)^{\bullet}$.

From \pref{freeint} we have that
$$N^{3}(B)^{\bullet}:=j_{A}(N^{3}(B))=N^{3}(A)^{\bullet}\cap N^{3}(C)^{\bullet}=j_{C}(N^{3}(B))$$
and the following compatibility equations
\begin{labeq}{compa}
N^{3}(A)^{\bullet}=N^{3}(D)\cap\gena{\abu}{\fr{3}\ta D^{\prime}}
\end{labeq}and
$$
N^{3}(C)^{\bullet}=N^{3}(D)\cap\gena{\cbu}{\fr{3}\ta D^{\prime}}.
$$

So far, are $A$ and $C$ embeddable in $D$ via Lie Algebra monomorphisms,
just regard $A\simeq\fr{3}\tr{2}A\quot N^{3}(A)$ and take the quotient of $j_{A}$:
\begin{eqnarray}
\map{\bar j _{A}}{&\fr{3}\tr{2}A\quot N^{3}(A)}{D}\\
&\bar w\longmapsto \bar w
\end{eqnarray}
on account of \pref{compa} is $\bar j_{A}$ one-to-one, in particular its image is again
$\gena{\abu}{D}=(\gena{\abu}{\fr{3}\ta D^{\prime}}+N^{3}(D))\quot N^{3}(D)$.\quad Do the same for $C$.

\medskip
We show next that if $B_{1}\dsu A_{1}$ then the embedding of $C$ into $D$ is $\delta_{3}$-strong.
We have to prove $\delta_{3}E_{1}\geq\delta_{3}C_{1}$ for each $E_{1}\zsu\ta D_{1}$ such that
$E_{1}\nni\cbu$ and $E_{1}$ $\ta$-closed.

We first observe $\delta_{3}E_{1}\geq\delta_{3} E_{1}\cap D_{1}$. This is because $d_{2}E_{1}\geq
d_{2}E_{1}\cap D_{1}$ and
\begin{multline}\label{ntreonto}
N^{3}(E_{1})=\\
=N^{3}(D)\cap\gena{E_{1}}{\fr{3}\ta D^{\prime}}=
(N^{3}(A)+N^{3}(C))\cap\gena{E_{1}}{\fr{3}\ta D^{\prime}}=\\
=(N^{3}(A)\cap\gena{E_{1}}{\fr{3}\ta D^{\prime}})+N^{3}(C)=\\
=(N^{3}(D)\cap\gena{A_{1}}{\fr{3}\ta D^{\prime}}\cap\gena{E_{1}}{\fr{3}\ta D^{\prime}})+N^{3}(C)=\\
=(N^{3}(D)\cap\gena{A_{1}\cap E_{1}}{\fr{3}\ta D^{\prime}})+N^{3}(C)\inn\\
\inn N^{3}(E_{1}\cap A_{1}+C_{1})\inn N^{3}(E_{1}\cap D_{1})
\end{multline}
Therefore we have to show $\delta_{3}E_{1}\cap D_{1}\geq\delta_{3}C_{1}$.

As $E_{1}\cap D_{1}=C_{1}+(E_{1}\cap \abu)$ we finish once we show
$\delta_{3}(E_{1}\cap D_{1}\quot C)=\delta_{3}(E_{1}\cap\abu\quot B)$ because
%$E_{1}\cap \abu$ is a $\delta_{2}$ strong subspace of $\abu$ and 
$B\dsu A$ and $\delta_{3}(E_{1}\cap \abu\quot B)\geq 0$.

We have
%$$\delta_{3}(E_{1}\cap D_{1}\quot C)
$$\delta_{3}(E_{1}\cap D_{1})-\delta_{3}(C_{1})=
d_{2}(E_{1}\cap D_{1}\quot C_{1})
-\big(\dfp(N^{3}(E_{1}\cap D_{1}))-\dfp(N^{3}(C))\big).$$
Everything behaves good for $d_{2}$, we have $d_{2}(E_{1}\cap D_{1})-d_{2}(C_{1})=
d_{2}(E_{1}\cap\abu)-d_{2}B_{1}$. (\dots eigentlich stimmt das f`\"ur $\delta_{2}$ aber
man kann approximieren \dots)

To conclude we have to show
$$N^{3}(E_{1}\cap\abu)\quot N^{3}(B)^{\bullet}\simeq
N^{3}(C_{1}+(E_{1}\cap \abu))\quot N^{3}(C)^{\bullet}.$$
Consider the map
\begin{eqnarray*}
N^{3}(E_{1}\cap \abu)\quot N^{3}(B)^{\bullet}&\longrightarrow&N^{3}(E_{1}\cap D_{1})\quot N^{3}(C)^{\bullet}\\
\bar\eta&\longmapsto&\bar\eta\quad\quad\forall\eta\in N^{3}(E_{1}\cap\abu).
\end{eqnarray*}
Soundness is immediate. Moreover our map is injective because
\begin{multline*}
N^{3}(C)^{\bullet}\cap\gena{E_{1}\cap \abu}{\fr{3}\ta D^{\prime}}=\\
=N^{3}(D)\cap\gena{\cbu}{\fr{3}\ta D^{\prime}}\cap\gena{E_{1}\cap \abu}{\fr{3}\ta D^{\prime}}=\\
=N^{3}(D)\cap\gena{\abu\cap C_{1}}{\fr{3}\ta D^{\prime}}=N^{3}(B)^{\bullet}.
\end{multline*}
Here we used $\gena{\cbu}{\fr{3}\ta D^{\prime}}\cap\gena{E_{1}\cap\abu}{\fr{3}\ta D^{\prime}}=
\gena{\cbu\cap\abu}{\fr{3}\ta D^{\prime}}$ because both $\cbu$ and $E_{1}\cap\abu$ are $\delta_{2}$ strong in $D_{1}$ and $\ta$ closed and because $\gena{\cbu}{\ta D^{\prime}}\cap\gena{E_{1}\cap\abu}{\ta D^{\prime}}=\gena{\bbu}{\ta D^{\prime}}$.

To see that this map is onto use the arguments in \pref{ntreonto} again to get
$$N^{3}(E_{1}\cap D_{1})=(
N^{3}(D)\cap\gena{E_{1}\cap A_{1}}{\fr{3}\ta D^{\prime}})+N^{3}(C)^{\bullet}.$$
%On the other hand, since again $\gena{\abu}{\ta D^{\prime}}\cap\gena{E_{1}}{\ta D^{\prime}}=\gena{E_{1}\cap\abu}{\ta D^{\prime}}$ we have
%\begin{multline*}
%N^{3}(E_{1})=N^{3}(D)\cap\gena{E_{1}}{\fr{3}\ta D^{\prime}}=\\
%=(N^{3}(A)^{\bullet}+N^{3}(C)^{\bullet})\cap\gena{E_{1}}{\fr{3}\ta D^{\prime}}=\\
%=(N^{3}(A)^{\bullet}\cap\gena{E_{1}}{\fr{3}\ta D^{\prime}})+N^{3}(C)^{\bullet}=\\
%=(N^{3}(D)\cap\gena{E_{1}\cap A_{1}}{\fr{3}\ta D^{\prime}})+N^{3}(C)^{\bullet}
%\end{multline*}
%and this gives that our map is onto.

We have shown that
$$\dfp(N^{3}(E))-\dfp(N^{3}(C))=\dfp(N^{3}(E_{1}\cap\abu))-\dfp(N^{3}(B)^{\bullet})$$
thus $C$ is $\delta_{3}$-strong embeddable in $D$.

A completely analogous argument proves that $A$ is $\delta_{3}$-strong in $D$.
\end{itemize}

\smallskip
We show that $D$ has $\sig{3}{2}$.
It suffices to show $\delta_{3}H_{1}\geq2$ for each  $H_{1}\zsu\ta D_{1}$ of finite linear dimension. 

Consider the relative $\ta$-closure $H_{1}^{+}$ of $H_{1}$ in $D$, then it holds
$\gena{H_{1}^{+}}{\tr{2}D}\cap\gena{C_{1}}{\tr{2}D}=\gena{H_{1}^{+}\cap C_{1}}{\tr{2}D}$.

By lemma \ref{}, since $C_{1}\dsu\ta D_{1}$, we have
$$
\delta_{3}(H_{1}^{+}+C_{1})+\delta_{3}(H_{1}^{+}\cap C_{1})
\leq\delta_{3}H_{1}^{+}+\delta_{3}C_{1}\leq\delta_{3}H_{1}^{+}+\delta_{3}(H_{1}^{+}+C_{1}).
$$
Now as $C\in\Klf$, we conclude
$\delta_{3}H_{1}\geq\delta_{3}H_{1}^{+}\geq\delta_{3}(H_{1}^{+}\cap C_{1})\geq2$.


To conclude the proof we have to
show that $\tr{2}D$ actually lies in $\Kl^{2}$ and provide axiom $\sig{3}{2}$ for $D$.
\end{proof}

%\begin{lem}[Amalgamation Lemma]
Let $A,B$ and $C$ in $\Klf$. If $B\dsu A$ and $B\dsu C$ then
%Let $B=\gena{B_{1}}{C}$ and $B\dsu A$ then there exists a $D$ in
%$\Klf$ such that $C\dsu D$ and $A=\gena{A_{1}}{D}$.
exists $D\in\Klf$ such that $A\dsu D$ and $C\dsu D$ and $B$ is preserved
by the strong embeddings.
%Moreover if $B$ is $\delta_{3}$-strong in $C$ too, then also $A\dsu D$ holds.
\end{lem}
\begin{proof}
Without loss of generality we can assume that $B\dsu A$ is a $\Klf$-minimal extension.

We first construct a candidate $D$ in $\nla{3}$%=A\star_{B}C$
, then
we will prove that $D\in\Klf$ by a discussion of the several minimal extensions $A\quot B$
which can occur. 

If we denote with $A^{\prime}$, $B^{\prime}$ and $C^{\prime}$, respectively $\tr{2}A$, $\tr{2}B$ and
$\tr{2}C$ then, after the definition of $\Klf$, we have $A^{\prime}\zso B^{\prime}\zsu C^{\prime}$.
We build the free $\nla{2}$-amalgam $D^{\prime}=\fram{A^{\prime}}{B^{\prime}}{C^{\prime}}$ of $A^{\prime}$ and $C^{\prime}$ over $B^{\prime}$. Assume $D^{\prime}$ is in $\Kl^{2}$ then, because
$B^{\prime}\zsu D^{\prime}$,
we $\delta_{2}$-strongly embed $D^{\prime}$ in $\K^{2}$ over $B^{\prime}$,
%we can find a copy of $D^{\prime}$ in 
we keep calling $D^{\prime}$ the isomorphic image in the generic model of $T^{2}$.

Consider $\ta D^{\prime}$ the $\ta$-closure of $D^{\prime}$ in $\K^{2}$,
because both $A^{\prime}$ and $C^{\prime}$ are selfsufficient in $D^{\prime}$
and $D^{\prime}\zsu\ta D^{\prime}$, we have that $A^{\prime}$ and $C^{\prime}$ are both $\delta_{2}$-strong
in $\ta D^{\prime}$. Moreover $\ta$-closure implies
\begin{labeq}{inter}
B^{\prime}=\gena{B_{1}}{\ta D^{\prime}}=\gena{\abu}{\ta D^{\prime}}\cap\gena{\cbu}{\ta D^{\prime}}.
\end{labeq}

Now on account of Lemma \ref{bellemma},
there exist two embeddings $j_{A}$ and $j_{C}$ of $\fr{3}\tr{2}A$ and $\fr{3}\tr{2}C$
into $\fr{3}\ta D^{\prime}$ their images being $\gena{\abu}{\fr{3}\ta D^{\prime}}$ and $\gena{\cbu}
{\fr{3}\ta D^{\prime}}$ respectively. From \pref{inter} and lemma \ref{bellemmino} instead, we get
\begin{labeq}{freeint}
\gena{B_{1}}{\fr{3}\ta D^{\prime}}=\gena{\abu}{\fr{3}\ta D^{\prime}}\cap\gena{\cbu}{\fr{3}\ta D^{\prime}}
\end{labeq}
here $B_{1}$, $\abu$ and $\cbu$ are identified with its images modulo $j_{A}$ and $j_{C}$. 

We set $N^{3}(A)^{\bullet}=j_{A}(N^{3}(A))$ and $N^{3}(C)^{\bullet}=j_{C}(N^{3}(C))$.
And we build
$$D=\fr{3}\ta D^{\prime}\quot(N^{3}(A)^{\bullet}+N^{3}(C)^{\bullet}).$$
As $N^{3}(A)^{\bullet}+N^{3}(C)^{\bullet}$ is an ideal consisting only of weight $3$ elements,
$\fr{3}\tr{2}D=\fr{3}(\tr{2}\fr{3}\ta D^{\prime})%\simeq
=\fr{3}\ta D^{\prime}$, hence we obtain $N^{3}(D)=N^{3}(A)^{\bullet}+N^{3}(C)^{\bullet}$.

From \pref{freeint} we have that
$$N^{3}(B)^{\bullet}:=j_{A}(N^{3}(B))=N^{3}(A)^{\bullet}\cap N^{3}(C)^{\bullet}=j_{C}(N^{3}(B))$$
and the following compatibility equations
\begin{labeq}{compa}
N^{3}(A)^{\bullet}=N^{3}(D)\cap\gena{\abu}{\fr{3}\ta D^{\prime}}
\end{labeq}and
$$
N^{3}(C)^{\bullet}=N^{3}(D)\cap\gena{\cbu}{\fr{3}\ta D^{\prime}}.
$$

So far, are $A$ and $C$ embeddable in $D$ via Lie Algebra monomorphisms,
just regard $A\simeq\fr{3}\tr{2}A\quot N^{3}(A)$ and take the quotient of $j_{A}$:
\begin{eqnarray}
\map{\bar j _{A}}{&\fr{3}\tr{2}A\quot N^{3}(A)}{D}\\
&\bar w\longmapsto \bar w
\end{eqnarray}
on account of \pref{compa} is $\bar j_{A}$ one-to-one, in particular its image is again
$\gena{\abu}{D}=(\gena{\abu}{\fr{3}\ta D^{\prime}}+N^{3}(D))\quot N^{3}(D)$.\quad Do the same for $C$.

\medskip
We show next that if $B_{1}\dsu A_{1}$ then the embedding of $C$ into $D$ is $\delta_{3}$-strong.
We have to prove $\delta_{3}E_{1}\geq\delta_{3}C_{1}$ for each $E_{1}\zsu\ta D_{1}$ such that
$E_{1}\nni\cbu$ and $E_{1}$ $\ta$-closed.

We first observe $\delta_{3}E_{1}\geq\delta_{3} E_{1}\cap D_{1}$. This is because $d_{2}E_{1}\geq
d_{2}E_{1}\cap D_{1}$ and
\begin{multline}\label{ntreonto}
N^{3}(E_{1})=\\
=N^{3}(D)\cap\gena{E_{1}}{\fr{3}\ta D^{\prime}}=
(N^{3}(A)+N^{3}(C))\cap\gena{E_{1}}{\fr{3}\ta D^{\prime}}=\\
=(N^{3}(A)\cap\gena{E_{1}}{\fr{3}\ta D^{\prime}})+N^{3}(C)=\\
=(N^{3}(D)\cap\gena{A_{1}}{\fr{3}\ta D^{\prime}}\cap\gena{E_{1}}{\fr{3}\ta D^{\prime}})+N^{3}(C)=\\
=(N^{3}(D)\cap\gena{A_{1}\cap E_{1}}{\fr{3}\ta D^{\prime}})+N^{3}(C)\inn\\
\inn N^{3}(E_{1}\cap A_{1}+C_{1})\inn N^{3}(E_{1}\cap D_{1})
\end{multline}
Therefore we have to show $\delta_{3}E_{1}\cap D_{1}\geq\delta_{3}C_{1}$.

As $E_{1}\cap D_{1}=C_{1}+(E_{1}\cap \abu)$ we finish once we show
$\delta_{3}(E_{1}\cap D_{1}\quot C)=\delta_{3}(E_{1}\cap\abu\quot B)$ because
%$E_{1}\cap \abu$ is a $\delta_{2}$ strong subspace of $\abu$ and 
$B\dsu A$ and $\delta_{3}(E_{1}\cap \abu\quot B)\geq 0$.

We have
%$$\delta_{3}(E_{1}\cap D_{1}\quot C)
$$\delta_{3}(E_{1}\cap D_{1})-\delta_{3}(C_{1})=
d_{2}(E_{1}\cap D_{1}\quot C_{1})
-\big(\dfp(N^{3}(E_{1}\cap D_{1}))-\dfp(N^{3}(C))\big).$$
Everything behaves good for $d_{2}$, we have $d_{2}(E_{1}\cap D_{1})-d_{2}(C_{1})=
d_{2}(E_{1}\cap\abu)-d_{2}B_{1}$. (\dots eigentlich stimmt das f`\"ur $\delta_{2}$ aber
man kann approximieren \dots)

To conclude we have to show
$$N^{3}(E_{1}\cap\abu)\quot N^{3}(B)^{\bullet}\simeq
N^{3}(C_{1}+(E_{1}\cap \abu))\quot N^{3}(C)^{\bullet}.$$
Consider the map
\begin{eqnarray*}
N^{3}(E_{1}\cap \abu)\quot N^{3}(B)^{\bullet}&\longrightarrow&N^{3}(E_{1}\cap D_{1})\quot N^{3}(C)^{\bullet}\\
\bar\eta&\longmapsto&\bar\eta\quad\quad\forall\eta\in N^{3}(E_{1}\cap\abu).
\end{eqnarray*}
Soundness is immediate. Moreover our map is injective because
\begin{multline*}
N^{3}(C)^{\bullet}\cap\gena{E_{1}\cap \abu}{\fr{3}\ta D^{\prime}}=\\
=N^{3}(D)\cap\gena{\cbu}{\fr{3}\ta D^{\prime}}\cap\gena{E_{1}\cap \abu}{\fr{3}\ta D^{\prime}}=\\
=N^{3}(D)\cap\gena{\abu\cap C_{1}}{\fr{3}\ta D^{\prime}}=N^{3}(B)^{\bullet}.
\end{multline*}
Here we used $\gena{\cbu}{\fr{3}\ta D^{\prime}}\cap\gena{E_{1}\cap\abu}{\fr{3}\ta D^{\prime}}=
\gena{\cbu\cap\abu}{\fr{3}\ta D^{\prime}}$ because both $\cbu$ and $E_{1}\cap\abu$ are $\delta_{2}$ strong in $D_{1}$ and $\ta$ closed and because $\gena{\cbu}{\ta D^{\prime}}\cap\gena{E_{1}\cap\abu}{\ta D^{\prime}}=\gena{\bbu}{\ta D^{\prime}}$.

To see that this map is onto use the arguments in \pref{ntreonto} again to get
$$N^{3}(E_{1}\cap D_{1})=(
N^{3}(D)\cap\gena{E_{1}\cap A_{1}}{\fr{3}\ta D^{\prime}})+N^{3}(C)^{\bullet}.$$
%On the other hand, since again $\gena{\abu}{\ta D^{\prime}}\cap\gena{E_{1}}{\ta D^{\prime}}=\gena{E_{1}\cap\abu}{\ta D^{\prime}}$ we have
%\begin{multline*}
%N^{3}(E_{1})=N^{3}(D)\cap\gena{E_{1}}{\fr{3}\ta D^{\prime}}=\\
%=(N^{3}(A)^{\bullet}+N^{3}(C)^{\bullet})\cap\gena{E_{1}}{\fr{3}\ta D^{\prime}}=\\
%=(N^{3}(A)^{\bullet}\cap\gena{E_{1}}{\fr{3}\ta D^{\prime}})+N^{3}(C)^{\bullet}=\\
%=(N^{3}(D)\cap\gena{E_{1}\cap A_{1}}{\fr{3}\ta D^{\prime}})+N^{3}(C)^{\bullet}
%\end{multline*}
%and this gives that our map is onto.

We have shown that
$$\dfp(N^{3}(E))-\dfp(N^{3}(C))=\dfp(N^{3}(E_{1}\cap\abu))-\dfp(N^{3}(B)^{\bullet})$$
thus $C$ is $\delta_{3}$-strong embeddable in $D$.

A completely analogous argument proves that $A$ is $\delta_{3}$-strong in $D$.
%,
%provided $B$ is a $\delta_{3}$-strong substructure in $C$.

\smallskip
We show that $D$ has $\sig{3}{2}$.

It suffices to show $\delta_{3}H_{1}\geq2$ for each  $A_{1}\zsu\ta D_{1}$ of finite linear dimension. 

Consider the relative $\ta$-closure $H_{1}^{+}$ of $H_{1}$ in $D$, then it holds
$\gena{}{}\cap\gena{}{}=\gena{}{}$.

By lemma \ref{} and lemma \ref{} we have
$$
\delta_{3}(H_{1}^{+}+C_{1})+\delta_{3}(H_{1}^{+}\cap C_{1})
\leq\delta_{3}H_{1}^{+}+\delta_{3}C_{1}\leq\delta_{3}H_{1}^{+}+\delta_{3}(H_{1}^{+}+C_{1}).
$$
Now since $C_{1}\dsu\ta D_{1}$ and $C\in\Klf$, we conclude
$\delta_{3}H_{1}\geq\delta_{3}H_{1}^{+}\geq\delta_{3}(H_{1}^{+}\cap C_{1})\geq2$.


To conclude the proof we have to
show that $\tr{2}D$ actually lies in $\Kl^{2}$ and provide axiom $\sig{3}{2}$ for $D$.
\end{proof}

%------------------------------------------------------------------------------
%We distinguish the following cases:
%\begin{itemize}
%\item[-]$A\quot B$ is free minimal
%\item[-]existence of $3$-divisors
%\item[-]pr\"aalgebraic extension.
%\end{itemize}
%The $\delta_{3}$-frei case is Ok. Axiom $\sig{3}{3}$ is evidently satisfied.
%The $3$-divisor is auch Ok: the only case in which $\sig{3}{3}$ could fail in $D$, is in presence of
%a $3$-divisor $e\in A_{1}$ which is paired by an element $c$ of $C$.
%$A\quot B$ prealgebraic and no $3$-divisor. It follows no $2$-divisors. Amalgam
%in $\Kl^{2}$ is possible \dots
%\end{proof}
%------------------------------------------------------------------------------

We list now the axiom system for $T^{3}$ and enunciate the main theorem. An $\mathcal{L}$-structure
$M$ is a model for $T^{3}$ if
\begin{itemize}
\punto{$\sig{3}{1}$}$M$ is in $\nla{3}$ and ${%\tr{2}
M\upharpoonright}_{\mathcal{L}^{2}}\sat T^{2}$ (\emph{vielleicht reichen wenige Axiome davon})
\item[$\sig{3}{2}$]$(\forall \bar x,\,P_{1}(\bar x))(\delta_{2}\gen{\bar x}-\dfp N^{3}(\gen{\bar x})\geq2$)
\punto{$\sig{3}{3}$}$(\forall z,\,P_{2}(z))(\forall x,y,\,P_{1}(x)\wedge P_{1}(y))([z,x]=[z,y]\,\rightarrow
\textsl{``$x$ lin.{}depends on $y$''})$
\punto{$\sig{3}{4}$}$(\forall y,\,P_{3}(y))(\forall z,\,P_{2}(z))(\exists x)([z,x]=y)$
\punto{$\sig{3}{5}$}For each extension $A\quot B$ in $\Klz$ such that $\delta_{3}A\quot B=0$,
if $B\inn M$ then there exist a $B$-isomorphic copy of $A$ in $M$.
\end{itemize}
\begin{teo}\label{richmodel}
Let $K$ be a structure in $\Kl^{3}$, then $K$ is rich if and only if $K$ is an $\omega$-saturated model
of $T^{3}$.
\end{teo}

In order to prove the theorem we 
show first that the behaviour of elements
in $\Kl^{3}$ in terms of predimensions, can be entirely described by $\Klz$-algebras.

%Consider a structure $A\in\Klz$, because $\tr{2}A\in\Kl^{2}_\textsf{0}$ and $\K^{2}$ is $\Kl^{2}$-universal, we can identify $A_{1}$ with
%a $\delta_{2}$-strong subspace of $\K^{2}_{1}$, so that $\tr{2}A=\gena{A_{1}}{\K^{2}}$.
%If we consider the $\delta_{2}$-strong embedding of
%$\tr{2}A$ into $\ta(\tr{2}A)=\gena{\ta A_{1}}{\K^{2}}$, we obtain an embedding of $\fr{3}\tr{2}A$ into $\fr{3}\ta(\tr{2}A)$. So we can define
%\begin{labeq}{starring}
%A^{*}=\fr{3}\ta(\tr{2}A)\quot N^{3}(A).
%\end{labeq}
%As $\ta\tr{2}A$ is $\delta_{2}$-strong in $\K^{2}$ and $d_{2}A_{1}=d_{2}\ta A_{1}$ it is not difficult to prove that $A^{*}$ belongs to $\Klf$. Moreover $A^{*}$ share the same predimensions of $A$.
%\begin{lem}\label{star}
%Let $A,B\in\Klz$ and $M\in\Kl^{3}$ then the following holds
%\begin{itemize}
%\punto{i}$A^{*}$ belongs to $\Klf$, $\delta_{3}A^{*}=\delta_{3}A$ and $d_{2}A^{*}=d_{2}A$
%\punto{ii}If $A\dsu B$ then $A^{*}\dsu B^{*}$
%%\punto{iii}If $A\dsu M$, then $A^{*}\dsu M$.
%\end{itemize}
%\end{lem}
%Now c

Consider  $M\in\Klf$ we say that $H_{1}\inn M_{1}$ is a \emph{core} of $M$ if $\gena{H_{1}}{M}\in\Klz$,
$H_{1}\zsu M_{1}$, $d_{2}H_{1}=d_{2}M_{1}$ and $N^{3}(M)\inn\gena{H_{1}}{\fr{3}\tr{2}M}$.
It is immediate to see that if $H_{1}$ is a core of $M$, then for each $H_{1}\inn V_{1}\inn M_{1}$, then
$\delta_{3}M_{1}=\delta_{3}V_{1}$.

Moreover we have
\begin{lem}\label{chistocore}
Let $H\in\Klf$, $H\inn M$ for some $M\in\Kl^{3}$ and $H_{1}^{\prime}$ be a core of $H$.
Then $H\dsu M$ if and only if $H^{\prime}$ is strong in $M$.
\end{lem}

There is a natural way to extract cores: consider a structure $H\in\Klf$, because $\delta_{3}H\geq2$ the space $N^{3}(H)$ must
be finitely generated in $\fr{3}\tr{2}H$. So call $S_{1}$ the minimal subspace of $H_{1}$,
such that $N^{3}(H)\inn\gena{S_{1}}{\fr{3}\tr{2}H}$, now as $d_{2}H_{1}$ is finite, let $B_{1}$
be a finite $d_{2}$-basis of $H_{1}$. Now take $H^{c}_{1}$ the selfsufficient closure of $S_{1}+B_{1}$
in $H_{1}$ and call $H^{c}$ the subalgebra $\gena{H^{c}_{1}}{H}$. As $\gena{H^{c}_{1}}{\K^{2}}\zsu\K^{2}$,
we can easily see that $H^{c}\in\Klz$. %We call $H^{c}$ is the \emph{standard core}

\begin{dfn}
A structure $K$ in $\Kl^{3}$ is \emph{poorly rich} if for each $A,B\in\Klz$ such that $B\dsu A$ and $B\dsu K$,
there exist a $\delta_{3}$-strong embedding of $A$ in $K$ over $B$.
\end{dfn}
%\begin{prop}\label{richnotrich}
%A structure $K\in\Kl^{3}$ is rich exactly when it is poorly rich.
%\end{prop}
%\begin{proof}
%Assume first $K$ is rich and $A\quot B$ is an extension of algebras in $\Klz$ such that $B\dsu M$.
%Consider $A^{*}\nni A$ as constructed in \pref{starring} being careful of choosing a $\delta_{2}$-strong embedding $B_{1}\inn A_{1}\hookrightarrow \K^{2}_{1}$ such that $\gena{\ta B_{1}}{\tr{2}K}\zsu\gena{\ta A_{1}}{\K^{2}}$. As $d_{2}\ta(B_{1})=d_{2}B_{1}$ and
%$B\dsu K$, we have $N^{3}(\ta(B_{1}))=N^{3}(B_{1})$, then apply lemma \ref{star}.(ii) to
%$B^{*}\simeq\fr{3}\gena{\ta B_{1}}{\tr{2}K}\quot N^{3}(B)=\gena{\ta B_{1}}{K}$. This yields $\gena{\ta B_{1}}{K}\dsu A^{*}$. Now conclude via richness that there exists a $\delta_{3}$-strong embedding of $A^{*}$ in $K$ over $\gena{\ta B_{1}}{K}$
%and in particular, a $B$-isomorphic image of $A$ in $K$, which is strong in $K$ because it is a core for $A^{*}$.
%Now if $A_{1}\quot B_{1}$ is free minimal, that is $\delta_{3}(A\quot B)>0$.
%Then we claim $d_{2}A=d_{2}B+1$ and $A_{1}\quot B_{1}$ is also $\delta_{2}$ minimal.

%We must however have $d_{2}A>d_{2}B$, assume $d_{2}A>d_{2}B+1$, then there exist an element $a\in A_{1}$,
%such that $A_{1}\supsetneq\ssc_{2}(B_{1},a)\supsetneq B_{1}$. Now from minimality it follows
%$\delta_{3}\ssc_{2}(B_{1},a)>\delta_{3}A>\delta_{3}B$. But for each $a\in A_{1}$ we have $\delta_{3}\ssc_{2}
%(B_{1},a)\leq\delta_{3}B+1$. And this cannot happen.
%In particular it follows $\delta_{3}A=\delta_{3}B+1$ and $\dfp(N^{3}(A))=\dfp(N^{3}(B))$.

%Now if there exists $A_{1}\supsetneq H_{1}\supsetneq B_{1}$ with $A_{1}\zso H_{1}$, since
%$N^{3}(A)=N^{3}(B)$, then $\delta_{3}H$ is either equal to $\delta_{3}A$ or to $\delta_{3}B$. In both cases
%that is against minimality. This proves minimality of $A_{1}$ over $B_{1}$ with respect to $\delta_{2}$,
%therefore $A_{1}=\gen{B_{1},a}$ for some $a$ in $A_{1}$ free from $B_{1}$.
%\end{proof}

\medskip
{\bf Soweit, gelten noch manche schwierigkeiten wobei ich noch nicht zu recht gekommen bin!
Deswegen werde ich diese in einem kleinen (Di)lemma isolieren, das nicht so schwer zu beweisen
scheint.}
\begin{lem}\label{dilemma}
Assume $K\in\bar\Kl^{3}$ and $A\quot B$ a prealgebraic minimal extension of $\Klz$, with $B \inn K$
as a subalgebra. then there exists a $\Klf$ extension $A^{\prime}\quot B^{\prime}$ such that
$A\inn A^{\prime}$ and $B\inn B^{\prime}$ and $B^{\prime}\inn K$.

Of course we require the inclusion $B^{\prime}\dsu A^{\prime}$ to preserve $B_{1}\inn A_{1}$.
\end{lem}

We are now able to prove theorem \ref{richmodel}, indeed we obtain our result as a corollary of
the following
\begin{prop}
For an $\La^{3}$-structure $M$ the following are equivalent:
\begin{itemize}
\item[{\rm(i)}]$K$ is an $\omega$-saturated model of $T^{3}$
\item[{\rm(ii)}]$K$ is an $\omega$-saturated, poorly rich $\Kl^{3}$-algebra
\item[{\rm(iii)}]$K$ is a rich structure of $\Kl^{3}$
\end{itemize}
\end{prop}
\begin{proof}
We prove (i)$\rightarrow$(ii)$\rightarrow$(iii)$\rightarrow$(i).
%\begin{proofof}{Theorem \ref{richmodel}}
%After proposition \ref{richnotrich} it is sufficient to prove\\
%{(a)} a rich structure $K\in\Kl^{3}$ is an $\omega$-saturated model of $T^{3}$ and\\
%{(b)} an $\omega$-saturated model $M$ of $T^{3}$ is a poorly rich algebra of $\Kl^{3}$.

Let $K$ model $T^{3}$ and be $\omega$-saturated. Suppose $A\quot B$ is 
a strong extension in $\Klz$, which is minimal prealgebraic and with $B\dsu M$.
On account of axiom $\sig{3}{5}$, in particular follows that we can realise $A$ in $K$ over $B$.
Now, because $\delta_{3}(A\quot B)=0$, the image of $A$ will be also strong in $K$.

If $A\quot B$ is free minimal, we can show that
also $\tr{2}A\quot\tr{2}B$ is free minimal, then it must be of the form $A_{1}=\gen{B_{1},a}$ for some
$a$ which is $d_{2}$-free from $B_{1}$.

Now because $K$ is saturated and \lqq$x$ $\delta_{2}$-free from $B_{1}$ and $\delta_{3}$-free from
$B_{1}$\rqq  is (a part of) a type over the finite set $B_{1}$, we can realise $a$ in $K$ over $B_{1}$.

So far $K$ is an $\nla{3}$ structure which has $\sig{3}{2}$ and $\sig{3}{3}$ and which
is poorly rich. To see that $K$ belongs to $\Kl^{3}$ we note that $\tr{2}M$ is 
definably isomorphic to the reduct structure $M_{\upharpoonright\La^{2}}$ which is
still $\omega$-saturated and a model of $T^{2}$, then conclude $\tr{2}M\simeq\K^{2}$.

%<<<<<<<<<<<<< 2 implica 3 >>>>>>>>>>>>>>>>>>>
\medskip  
Let $K$ be a poorly rich $\Kl^{3}$-structure and $L\quot M$ be a minimal $\Klf$-extension with $M\dsu K$.

Isolate a core $M_{1}^{c}$ of $M$ then put $L_{1}^{c}\colon=\ssc_{2}^{L}(M_{1}^{c},B_{1},S_{1})$ where
$B_{1}$ is a $d_{2}$-basis of $L_{1}$ over $M_{1}$ and $N^{3}(L)\inn\gena{M^{c}_{1},S_{1}}{\fr{3}\tr{2}L}$.
Then $L_{1}^{c}$ is a core for $L$ and $M_{1}^{c}$ can be chosen so that $M_{1}^{c}=L_{1}^{c}\cap M_{1}$.

We prove next that $M_{1}+L_{1}^{c}\zsu L_{1}$ with respect to the predimension on  $\tr{2}L$ induced by
$\K^{2}$. On this purpose it suffices to show $U_{1}+L_{1}^{c}\zsu L_{1}$ for each $\delta_{2}$-strong
finitely generated subspace $U_{1}$ of $M_{1}$ which contain $M_{1}^{c}$. Let $U_{1}$ be such a space
and $C_{1}\nni U_{1}+L_{1}^{c}$ then $\delta_{2}C_{1}-\delta_{2}(U_{1}+L_{1}^{c})\geq\delta_{2}C_{1}-
\delta_{2}U_{1}-\delta_{2}L_{1}^{c}+\delta_{2}(U_{1}\cap L_{1}^{c})\geq\delta_{2}M_{1}^{c}-\delta_{2}U_{1}=0$
because $L_{1}^{c}\zsu L_{1}$, $U_{1}\cap L_{1}^{c}=M_{1}^{c}$ and $\delta_{2}M_{1}^{c}=\delta_{2}U_{1}$.

It follows $\gena{\ta(M_{1}+L_{1}^{c})}{L}\supsetneq M$ lies in $\Klf$ and share the same $\delta_{3}$ of
$L$. Then by minimality
$$L=\gena{\ta(M_{1}+L_{1}^{c})}{L}=\fr{3}\gena{\ta(M_{1}+L_{1}^{c})}{\tr{2}L}\quot M^{3}(L_{1}^{c}).$$
In other words
$L$ is generated
over $M$ by $L_{1}^{c}$ up to the operator $\ta\ssc_{2}$.

Put $M^{c}=\gena{M^{c}_{1}}{M}$ and $L^{c}=\gena{L^{c}_{1}}{L}$,
lemma \ref{chistocore} implies $M^{c}\dsu L^{c}$ and $M^{c}\dsu K$.
Therefore, as $K$ is poorly rich, there exists a $V_{1}\dsu K_{1}$ such that
$\gena{V_{1}}{K}\simeq_{M^{c}}L^{c}$. But then $\gena{V_{1}}{K}\simeq_{M}L^{c}$ also holds.

We discuss now two cases: assume first $\delta_{3}(L\quot M)=0$. Because $M\dsu K$ and $\delta_{3}(V_{1}\quot M_{1})=0$ it follows 
%Now because $\gena{V_{1}}{K}$ is a core for $\gena{M_{1}+V_{1}}{K}$, it follows
that $\gena{M_{1}+V_{1}}{K}\dsu K$ and as $d_{2}(M_{1}+V_{1})=
d_{2}\ta(M_{1}+V_{1})$, we have  $N^{3}(\ta(M_{1}+V_{1}))=N^{3}(M_{1}+V_{1})=N^{3}(V_{1})$.

We conclude noting
$$\gena{\ta(M_{1}+V_{1})}{K}=\fr{3}\gena{\ta(M_{1}+V_{1})}{\tr{2}K}\quot N^{3}(V_{1})\simeq_{M}L.$$

To sort the case $\delta_{3}(L\quot M)>0$ we use analogous arguments and saturation o f$K$.

%<<<<<<<<< 3 implica 1 >>>>>>>>>
\medskip
Assume now $K$ is a rich algebra of $\Kl^{3}$.
%-algebra is an $\mathcal{L}^{3}$-structure and,
%because $K$ is %$\K^{3}$-universal and $\dsu$-homogeneous, that $\
%rich, then $\tr{2}K$ is a rich structure in $\Kl^{2}$ and in particular a model of $T^{2}$.
Then $K$ is an $\mathcal{L}^{3}$-structure and we get axioms $\sig{3}{2}$ and $\sig{3}{3}$ at once.

%Axiom 
%To prove , instead, just consider an arbitrary finite set
%$X$ in $K_{1}$, then, because $K$ has property $\sig{3}{2} ^{\prime}$, we have $\delta_{2}\gen{X}-\dfp N^{3}(\gen{X})\geq
%d_{2}X-\dfp N^{3}(\gen{X})\geq\delta_{3}\ssc_{2}\gen{X}\geq2$.
Consider now a $\Klz$-extension $A\quot B$ such that $\delta_{3}(A\quot B)=0$ and $B$ is a subalgebra
of $K$, $B=\gena{B_{1}}{K}$.

%Now construct $B^{*}$ and $A^{*}$ as in \pref{starring}. Because $K$ is $\Klf$-homogeneous and $\Klf$-universal,
%we can assume that $K\nni B^{*}\nni B$ while on the other side we have $B^{*}\dsu A^{*}$ by lemma \ref{star}.
Now find $A^{\prime}$ and $B^{\prime}$ as in lemma \ref{dilemma} and 
put $F=\ssc_{3}B^{\prime}\dsu K$. As $F\nni B^{\prime}\dsu A^{\prime}$ we can build, via asymmetric amalgamation, an algebra $F^{\prime}\in\Klf$ such
that $F\dsu F^{\prime}$ and $A\inn F^{\prime}$. Now by richness, we strongly embed $F^{\prime}$ in $K$ over $F$. We obtain in particular an embedding of $A$ in $K$ over $B$.
This proves $K\sat\sig{3}{5}$.

Axiom $\sig{3}{4}$ follows from (poor) richness and will be proved separately. It is indeed a consequence
of axioms $\sig{3}{i}$ for $i=1,2,3,5$.

As we have seen above, any $\omega$-saturated model $M^{\prime}$ of $T^{3}$ is back and forth
equivalent to a rich structure, therefore $K$ is $\omega$-saturated too.
It follows that also the reduct of $K$ to $\La^{2}$ is $\omega$-saturated.
 
As $\tr{2}M$ is definably isomorphic to
$M_{\upharpoonright\La^{2}}$, we can prove that $\tr{2}M$ is rich with respect to the class $\Kl^{2}$ and 
in particular a model of $T^{2}$.
We conclude that $M\sat\sig{3}{1}$ and the proof as well.
\end{proof}
%\subsection{Proofs of the Embedding Lemmas}
%\subsection{Crocodile}
Abusing our previous notation we denote again by $\map{\gam{B}{A}}{\fl{B}}{\fl{A}}$ the canonical map of
the free-lift of any extension $A\nni B$ of $\nla{2}$-algebras.
The theorem above is based on the following lemma.
\begin{lem}\label{crocolemma}
Let $A$ be an extension of a finite $\nla{2}$-algebra $C$ with %$\dfp(A_{1}/ C_{1})=1$ and
$A_{1}=\gen{C_{1},a}$ for some $a$ in $A_{1}$ linearly independent over $C_{1}$.

Assume $\delta_{2}(a/ C)\leq0$ and $\left(\tpl{\psi}{n}\right)$ is a basis
for $\rd(A)$ over $\rd(C)$, where $\psi_{i}=[c_{i},a]-w_{i}$ for linearly independent $\tpl{c}{n}$ in
$C_{1}$ and $w_{i}$ in $\exs C_{1}$, then $\dfp(\ker(\gam{C}{A}))\leq\dfp\rd(\gen{\tpl{c}{n}})$.
\end{lem}
%\begin{figure}[hbt]
%\centering\includegraphics{kroko}
%%\caption{That's what happens to nasty relations}
%\end{figure}
\begin{proof}
We first claim that a basis for $\rd(A/ C)$ like in the statement of the lemma
can always be found. For assume $A_{1}=\geno{a,C_{1}}$, if $(\psi_{i})_{i=1}^{n}$ is a basis of
$\rd(a/C)$ then, by bilinearity of the Lie product we can assume\footnote{
express each $\psi_{i}$ in basic monomials with respect to any basis of $A_{1}$, which completes $\{a\}$ and collect all terms which contain $a$.}
each $\psi_{i}$ to be a sum $[c_{i},a]+w_{i}$,
where $c_{i}$ is in $C_{1}$ and $w_{i}\in\exs C_{1}$.

The linear independence of the $\psi$'s over $\exs C_{1}$ yields the linear independence of $c_{1},\dots,c_{n}$.

We now  arrange a basis $\mathcal{B}$
of $A_{1}$ as follows: $\mathcal{B}=\{c_{1}>\dots >c_{n}>\mathcal{C}>a\}$, where $\mathcal{C}$ is some ordering of a base completion of
$c_{1},\dots,c_{m}$ to $C_{1}$. %It follows the set $(w_{i})$ is also independent.

\medskip
Recall that $\ker(\gam{C}{A})$ is $(\fla{3}{C_{1}}\cap\J(A))/\J(C)$ and take an homogeneous element
$\Psi$ in $\fla{3}{C_{1}}\cap\J(A)$ of weight $3$.

Since $\J(A)=[\rd(A),A_{1}]=\sum_{i=1}^{n}[\psi_{i},A_{1}]+[\rd(C),a]+\J(C)$ we may
assume $\Psi$ is a finite homogeneous sum of weight $3$:
\begin{labeq}{firstPsi}
\Psi=\sum_{u\in\mathcal{B}\non\{a\},\,i=1}^{n}\!\!\lambda_{i,u}[\psi_{i},u]+\sum_{i=1}^{n}\theta_{i}[\psi_{i},a]+[\nu,a]
\end{labeq}
for $u\in\mathcal{B}$, $\lambda_{u,i},\theta_{i}\in\Fp$ and some $\nu\in\rd(C)$.

We proceed with the same arguments of Proposition \ref{bellemma} concerning basic and {\em pre}basic commutators
of weight $3$.

We claim first, that terms $[\psi_{i},a]$ do not actually appear in the sum above.
Consider the unique expression for $\Psi\in\fla{3}{C_{1}}$ as sum of basic monomials
over $\{c_{i},\mathcal{C}|i=1,\dots,n\}$. These are chosen according to the linear order on $\mathcal{B}$. %with respect to the base $\mathcal{B}$ obtained from \pref{firstPsi}.

On the other hand, from each $[\psi_{i},a]$ we have $[c_{i},a,a]+[w_{i},a]$ and (Engel) basic monomials like
$[c_{i},a,a]$ cannot be cleared up from the sum \pref{firstPsi} -- applying Jacobi identities -- by any other summand.
But this would contrast the fact that $\Psi$ belongs to $\fla{3}{C_{1}}$.

Hence $\Psi=\sum_{i,u}\lambda_{i,u}[\psi_{i},u]+[\nu,a]$. Furthermore we affirm that each
base element $u$ above must belong to the $c_{i}$'s. For assume instead some $u$ is in $\mathcal{C}$,
and $[\psi_{i},u]=[c_{i},a,u]+[w_{i},u]$ is a non trivial summand in \pref{firstPsi}.
The basic monomial $[c_{i},a,u]$ -- which cannot appear in the expression over $C_{1}$ for $\Psi$ --
%cannot be canceled by terms like $[w_{k},v]$. It follows,
forces the term $[\nu,a]$ to contain $[c_{i},u,a]$ as a summand. This implies {\em both} basic terms $[c_{i},a,u]$ and $[u,a,c_{i}]$
are to be found in the sum of the $\lambda_{i,u}[\psi_{i},u]$'s in \pref{firstPsi}, which is impossible if $u$ differs from all
$c_{i}$'s.

%Then $\Psi$ consists of $\sum_{r,s}[\psi With. 
With totally similar arguments, now follows that $\nu$ also actually
belongs to $\exs\genp{\tpl{c}{m}}$ and hence to $\rd(\genp{\tpl{c}{m}})$. 

\medskip
To conclude,  let $k$ be $\dfp(\ker(\gam{C}{A}))$ and $\{\overline{\Psi}^{t}\}_{t<k}$ a basis of such kernel, where
$$
\Psi^{t}=\sum_{%\substack{
r,s=1 %\\r\neq s}
}^{m}\lambda_{r,s}^{t}[\psi_{r},c_{s}]+[\nu^{t},a]\in\fla{3}{C_{1}}\cap\J(A)
$$
for $\nu^{t}$ in $\rd(c_{1},\dots,c_{m})$. % and $b_{s}\in\mathcal{B}\non\{a\}$.

After the due simplifications, each element $\Psi^{t}$ above reduces to
\begin{labeq}{kelements}
\Psi^{t}=\sum_{r,s=1}^{m}\lambda_{r,s}^{t}[w_{r},c_{s}].
\end{labeq}



We prove the Lemma by showing linear independence of the $\nu^{t}{}^{'s}$.
So assume there are $(\theta_{t})_{t<k}\inn\Fp$, such that $\sum_{t}\theta_{t}\nu^{t}=\triv$.

It follows $\sum_{t}\theta_{t}\Psi^{t}=\sum_{r,s,t}\lambda_{r,s}^{t}\theta_{t}[\psi_{r},c_{s}]=
\sum_{r,s=1}^{m}(\sum_{t=1}^{k}\theta_{t}
\lambda_{r,s}^{t})[\psi_{r},c_{s}]$.

But this yields that  the sum
$$\sum_{r,s=1}^{m}(\sum_{t=1}^{k}\theta_{t}
\lambda_{r,s}^{t})[c_{r},a,c_{s}]$$
belongs to $\fla{3}{C_{1}}$.
This is impossible -- all $[c_{r},a,c_{s}]$ are basic monomials which are linearly independent over $\fla{3}{C_{1}}$ -- unless
$\sum_{t=1}^{k}\theta_{t}
\lambda_{r,s}^{t}$ is trivial for all choices of different $r,s$.

But this gives, then $\sum\theta_{t}\Psi^{t}%=\overline{\sum\theta_{t}\Psi^{t}}
=\triv$ and hence $\theta^{t}$ has to be trivial for all $t<k$. It follows $k\leq\dfp(\rd(\genp{\tpl{c}{m}}))$ as desired.
\end{proof}

\bigskip
In the sequel we denote by $\kerg{B}{A}$ the kernel of $\map{\gam{B}{A}}{\fl B}{\fl A}$ for $A\nni B\in\nla{2}$
and we set $\dkerg{B}{A}:=\dfp(\kerg{B}{A})$.

For any two extensions $A\nni B\nni C$,
since $\gam{C}{A}=\gam{C}{B}\gam{B}{A}$,
$\gam{B}{C}$ maps $K_{B}^{A}$ into $\kerg{C}{A}$ with kernel $\kerg{B}{C}$. In particular we have
\begin{labeq}{addkerg}
\dkerg{C}{A}\leq\dkerg{C}{B}+\dkerg{B}{A}.
\end{labeq}


\begin{proofof}{Theorem \ref{crocotheorem}}
We prove the statement by induction on $l=\dfp(A_{1}/ B_{1})$. For $l=0$
there is nothing to prove.

For $l=1$ remark that $B$ is {\em not} strong in $A$ and
apply Lemma \ref{crocolemma} to the finite extension $A\nni B$. This gives $\dkerg{B}{A}\leq\dfp(\rd(c_{1},\dots,c_{n}))$ where
$c_{1},\dots, c_{n}$ are linearly independent elements of $B_{1}$ and $n=\dfp(\rd(A/B))>1$.
Now since $A\sat\sig{2}{2}$ then $\dfp(\rd(c_{1},\dots,c_{n}))<n$ and we have $\dkerg{B}{A}\leq n-1=-\delta_{2}(A/B)$.

\medskip
Assume $l\geq2$. We divide the proof into different cases:\\[+0.5mm]\noindent
{\bf Case 1:}\quad{\sl There exists a proper subspace $H_{1}$ of $A_{1}$ such that $B_{1}\subsetneq H_{1}\subsetneq A_{1}$ and
$\delta_{2}(H)\leq\delta_{2}(B)$.}

\smallskip
The properties of the self-sufficient closure imply $\delta_{2}(H)>\delta_{2}(A)$ and $A=\ssc^{A}(H)$. Take such an $H$ which is minimal with respect to inclusion and with minimal $\delta_{2}(H)$.\\[+1mm]\noindent
{\bf Case 1.{}1:}\quad$\delta_{2}(H)=\delta_{2}(B)$.

\smallskip
By the choice of $H$, $B\zsu[2]{}H$, hence $\dkerg{B}{H}=0$ and $\dkerg{B}{A}\leq\dkerg{H}{A}$.
By induction now $\dkerg{H}{A}\leq-\delta_{2}(A/H)=-\delta_{2}(A/B)$.
And the assertion follows.\\[+0.8mm]\noindent
{\bf Case 1.{}2:}\quad$\delta_{2}(H)<\delta_{2}(B)$.

\smallskip
In this case we have $H=\ssc^{H}(B)$ and $A=\ssc^{A}(H)$. By applying the inductive hypothesis we obtain
$\dkerg{B}{A}\leq\dkerg{B}{H}+\dkerg{H}{A}\leq-\delta_{2}(H/B)-\delta_{2}(A/H)=-\delta_{2}(A/B)$.\\[+2mm]\noindent
{\bf Case 2:}\quad{\sl There is no such $H_{1}$ like in Case 1. That is for all $B_{1}\subsetneq H_{1}\subsetneq A_{1}$ we have
$\delta_{2}(B)<\delta_{2}(H)$.}
 
\smallskip
Take a subspace $C_{1}\nni B_{1}$ with codimension $1$ in $A_{1}$, such that $A_{1}=\gen{C_{1},a}$ for some
$a$ in $A_{1}$. We have $B\zsu[2]{}C$ and hence $\gam{B}{C}$ is mono.

We proceed like in Lemma \ref{crocolemma} to find a basis
\begin{labeq}{basipsi}\psi_{i}=[c_{i},a]+w_{i}\quad\,i=1,\dots,n\end{labeq}
of $\rd(A)$ over $\rd(C)$ where $w_{i}\in\exs C_{1}$ and the set $(\tpl{c}{n})\inn C_{1}$ is
linearly independent. Also $n>1$ since $\delta(A/C)<0$.

Moreover, any element $\Psi$ of $\kerg{C}{A}$ is the image modulo $\J(C)$ of a sum
\begin{labeq}{kca}
\Psi=\sum_{%\substack{
i,j=1 %\\r\neq s}
}^{n}\lambda_{i,j}[\psi_{i},c_{j}]+[\nu,a]=\sum_{i,j=1}^{n}\lambda_{i,j}[w_{i},c_{j}]
\end{labeq}
for some $\nu$ in $\rd(c_{1},\dots,c_{n})$ (cfr.~\pref{kelements} above).\\[+2mm]\noindent
{\bf Case 2.{}1:}\quad{\sl $C_{1}$ is generated by  $B_{1}$ and the $\tpl{c}{n}$.}

\medskip
For a suitable choice of $m$ independent elements $b_{1},\dots,b_{m}$ of $B_{1}$ and $n-m=:h$ 
elements $\tpl{a}{h}$ of $C_{1}$ independent over $B_{1}$, we may assume
that $c_{i}=b_{i}$ for $i=1,\dots,m$ and that $c_{m+i}=a_{i}$ for $i=1,\dots,h$. 

We arrange and order a basis of $A_{1}$ by taking
$$\{\mathcal{B}>b_{1}>\dots >b_{m}>a_{1}>\dots>a_{h}>a\}$$
where $\mathcal{B}$ is a completion of $\{b_{i}\mid i=1,\dots,m\}$ to a basis for $B_{1}$
and $C_{1}=\genp{B_{1}, a_{j}\mid j=1,\dots,h}$.

Observe also that we may assume $m\geq 1$, for otherwise by comparing the expression in \pref{kca}, we would
have $\fla{3}{B_{1}}\cap\J(A)=\triv$ and hence
$$\kerg{B}{A}\simeq\gam{B}{C}(\kerg{B}{A})=\kerg{C}{A}\cap\gam{B}{C}(\fl{B})\simeq\frac{(\fla{3}{B_{1}}\cap\J(A))+\J(C)}{\J(C)}=\triv$$
and the result would trivially follow.

\medskip
If $k$ denotes the dimension of $\rd(\genp{b_{i},a_{j}})$ and we set  $k_{b}=\dfp(\rd(\genp{b_{i}}))$, then we have
$k-k_{b}\leq\dfp(\rd(C/B))$ and as $\dfp(\rd(a/C))=n=m+h$,
\begin{multline*}
-\delta(A/B)=\dfp(\rd(a/C))+\dfp(\rd(C/B))-(h+1)\geq\\
\geq m+h+k-k_{b}-h-1\geq m-1-k_{b}+k
\end{multline*}
Now since $B$ has $\sig{2}{2}$, $m-1-k_{b}\geq0$ and hence $k\leq-\delta(A/B)$.

Since $B\zsu[2]{}C$ we have $\dkerg{B}{A}\leq\dkerg{C}{A}$ while
by Lemma \ref{crocolemma} we have $\dkerg{C}{A}\leq k$ and hence
$\dkerg{B}{A}\leq-\delta(A/B)$ follows.%
%\smallskip
%Now we  want to understand a basis of our kernel $K^{A}_{B}$ of the natural map
%between $\fl B$ and $\fl A$. This is $K^{A}_{B}=(\fla{3}{B_{1}}\cap\J(A))/\J(B)$.
%Since $\J(A)=[\rd(A),A_{1}]=[\rd(a/ B),A_{1}]+[\rd(B),a]$ our \emph{Crocodile Lemma}
%tells us we can find a basis $(\Psi^{\alpha})_{\alpha=1}^{k}$ of $K_{B}^{A}$, which has the following shape:
%$$
%\Psi^{\alpha}=\sum_{\substack{r,s=1\\r\neq s}}^{m}\lambda_{r,s}^{\alpha}[\psi_{r},b_{s}]+[\nu^{\alpha},a]
%$$
%for $\nu^{1},\dots,\nu^{k}\in \rd(b_{1},\dots,b_{m})$.
%Now if we want to claim linear independence of the $(\nu^{\alpha})^{'s}$, then we assume
%for some some $\mu_{\alpha}$ in the field, that $\sum_{\alpha=1}^{k}\mu_{\alpha}\nu^{\alpha}=\triv$.
%But this yields that $\sum_{\alpha=1}^{k}\mu_{\alpha}\Psi^{\alpha}=\sum_{\alpha=1}^{k}\mu_{\alpha}(
%\sum_{r,s=1}^{m}\lambda_{r,s}^{\alpha}[\psi_{r},b_{s}])$ and therefore the sum
%$$\sum_{r,s=1}^{m}(\sum_{\alpha=1}^{k}\mu_{\alpha}
%\lambda_{r,s}^{\alpha})[b_{r},a,b_{s}]$$ belongs to $\fla{3}{B_{1}}$.
%This is impossible unless $\sum_{\alpha=1}^{k}\mu_{\alpha}
%\lambda_{r,s}^{\alpha}$ is trivial for all choices of different $r,s$.
%\emph{Does this imply also that all $\mu^{\alpha}$ are zero??? I don't know
%what else could be calculated here. And I am sorry  not to see the point.} 
%
%\smallskip
%Statement of crocodile ($k_{B}^{A}:=\dfp(K^{A}_{B})\leq\dfp \rd(b_{1},\dots b_{m})$) is not in peril anyway, for I can replace stepwise
%each $\Psi^{\alpha}$ by a $\widetilde{\Psi}^{\alpha}$, in such a way that the resulting $\widetilde{\nu}^{\alpha\,'s}$ are independent and such that the $\widetilde{\Psi}^{\alpha\,'s}$ span the whole $K_{B}^{A}$.
%This has the only disadvantage of mixing up the terms $[\psi_{r},b_{s}]$ in our presentation of $\Psi^{\alpha}$,
%which is of no harm hier, but could be a problem in \emph{crocodile two} below.
%
%\bigskip
%\subsection*{remarks on the final case}
%In the second crocodile, our extension $A/ B$ can be presented as $A=\gen{B_{1},c_{1},\dots,c_{n},
%a_{1},\dots,a_{s},a}$ for independent $b_{1},\dots b_{m}$ in $B$ and independent
%$c_{1},\dots,c_{n},a_{1},\dots,a_{s},a$ over $B_{1}$,
%such that, if $C$ denotes
%$\gen{B_{1},c_{1},\dots,c_{n},a_{1},\dots,a_{s}}$, then $\rd(a/ C)$ is generated by
%$$
%\psi_{i}=[b_{i},a]+w_{i}\quad i=1,\dots, m\quad w_{i}\in\exs C_{1}
%$$
%$$
%\phi_{j}=[b_{j},a]+y_{j}\quad j=1,\dots, n\quad y_{j}\in\exs C_{1}.
%$$
%If now we set $k=\dfp \rd(b_{1},\dots b_{m},c_{1},\dots, c_{n})$ and $k_{b}=\dfp \rd(b_{1},\dots b_{m})$
%and  call $C^{\prime}=\gen{B_{1},c_{1},\dots,c_{n}}$.
%We get
%$k-k_{b}=\dfp\left(\rd(b_{1},\dots b_{m},c_{1},\dots, c_{n})/ \rd(B)\cap \rd(b_{1},\dots b_{m},c_{1},\dots, c_{n})\right)\leq\dfp \rd(C^{\prime}/ B)$.
%If we argue like in \emph{Crocodile one}, then we get $\dfp(K_{C}^{A})\leq k$ and by the axioms
%$k_{b}\leq m-1$.
%\smallskip
%Now thanks to some clever arguments (\dots) we can write
%\begin{multline*}
%\delta_{2}(A)
%=\delta_{2}(B)+n+s+1-\dfp \rd(A/ B)=\\
%=\delta_{2}(B)+n+s+1-(n+m)-\dfp \rd(C/ C^{\prime})-\dfp \rd(C^{\prime}/ B)\leq\\
%\leq\delta_{2}(B)+s-k_{b}-\dfp \rd(C/ C^{\prime})-(k-k_{b})=\\
%=\delta_{2}(B)-k+s-\dfp \rd(C/ C^{\prime})=\delta_{2}(B)-k+\delta_{2}(C/ C^{\prime}).
%\end{multline*}
%Well this extra term $\delta_{2}(C/ C^{\prime})$ is not that exotic but it would be
%really annoying if it is positive. Till now I couldn't go much further than this. The funny
%part of this way of proceeding is that we do not use any inductive step.
\\[+0.5mm]\noindent
{\bf Case 2.2:}\quad{\sl $C_{1}$ is not generated by the $c_{i}$'s over $B_{1}$ only.}

\medskip
In this case, me may assume $C_{1}$ has an ordered basis
$$\mathcal{C}=\{\mathcal{B}>b_{1}>\dots>b_{m}>a_{1}>\dots>a_{h}>e_{1}>\dots>e_{r}\}$$
where $r\geq1$, the set $\{b_{i},a_{j}\mid i=1,\dots,m,\,j=1,\dots,h\}$ play the role of $\{c_{i}\mid i=1,\dots,n\}$ as before, and $\mathcal{B}$ completes $\{b_{1},\dots,b_{m}\}$ to a basis of $B_{1}$. Also let $a$,
which completes $\mathcal{C}$ to a basis of $A_{1}$, be smaller than any other element.

\medskip
As previously observed, being $\kerg{B}{A}\simeq\gam{B}{C}(\kerg{B}{A})=(\fla{3}{B_{1}}\cap\J(A))+\J(C)/\J(C)$,
we take into account the representative $\Psi$ in $\fla{3}{B_{1}}\cap\J(A)$ for some
arbitrary element $\overline{\Psi}$ in $\gam{B}{C}(\kerg{B}{A})$.

For all $j$, if we set $\hat\psi_{j}=\sum_{i}\lambda_{i,j}\psi_{i}$
and $\hat w_{j}=\sum_{i}\lambda_{i,j}w_{i}$, expression \pref{kca} for $\Psi$ becomes
$$\Psi=\sum_{j}[\hat\psi_{j},c_{j}]+[\nu,a]=\sum_{j}[\hat w_{j},c_{j}].$$

%On the other side,
%we can replace %each term $\lambda_{ij}w_{i}$ from \pref{kca}
%each $w_{i}$ with a sum of basic monomials
%$\sum_{\alpha}s^{\alpha}_{i}[x^{i,\alpha},y^{i,\alpha}]$ with $x^{i,\alpha}>y^{i,\alpha}$ in $\mathcal{C}$. We obtain in particular
%\begin{labeq}{krokobasic}
%\Psi=\sum_{\alpha,i,j}\lambda_{i,j}s^{\alpha}_{i}[x^{i,\alpha},y^{i,\alpha},c_{j}].
%\end{labeq}
On the other side,
we can replace %each term $\lambda_{ij}w_{i}$ from \pref{kca}
each $\hat w_{j}$ with a sum of basic monomials
$\sum_{\alpha}s^{\alpha}_{j}[x^{j,\alpha},y^{j,\alpha}]$ with $x^{j,\alpha}>y^{j,\alpha}$ in $\mathcal{C}$. We obtain in particular
\begin{labeq}{krokobasic}
\Psi=\sum_{\alpha,j}s^{\alpha}_{j}[x^{j,\alpha},y^{j,\alpha},c_{j}].
\end{labeq}
which has to be compared with the unique expression of $\Psi$ in basic commutators
over $\mathcal{B}\cup\{b_{i}\mid i=1,\dots,m\}$.

\smallskip
Consider the subspace
$C^{\prime}=\genp{B_{1},a_{1},\dots,a_{h},e_{1},\dots,e_{r-1},a}$, we claim that $\hat w_{j}$ belongs to $\exs C_{1}^{\prime}$ for all $j$.
If this is not the case, then a term $[x,e_{r}]$ appears {\em with
a nontrivial coefficient} from $\Fp$, in the sum presenting $\hat w_{j}$ for some $j$.
This implies that the weight $3$ commutator $[x,e_{r},c_{j}]$ appears in \pref{krokobasic} with a non-zero coefficient.
Also remark that $[x,e_{r},c_{j}]$ is basic, because $x>e_{r}<c_{j}$.

Since $e_{r}$ does not appear among the $c_{i}$'s, this basic commutator can in no way arise\footnote{
by means of Jacobi identities. Cfr. the proof of Proposition \ref{bellemma}.} from -- nor be eliminated by -- a prebasic monomial, that is,
by a term $[u,v,e_{r}]$ in \pref{krokobasic} with $u>v>e_{r}$. Since $\Psi$ is in $\fla{3}{B_{1}}$, the claim above follows.
%all terms $[x,e_{r},c_{j}]$, for different $x$'s, must cancel each other exclusively within the sum
%$\sum_{\alpha,i}\lambda_{i,j}s^{\alpha}_{i}[x^{i,\alpha},y^{i,\alpha},c_{j}]=%f^{\alpha}_{ij^{*}}[x_{i}^{\alpha},y_{i}^{\alpha},c_{j^{*}}]=
%%\sum_{i}\lambda_{ij^{*}}[w_{i},c_{j^{*}}]=
%[\sum_{i}\lambda_{ij}w_{i},c_{j}]$.

Now if $\hat w_{j}$ is in $\exs C^{\prime}_{1}$, then in particular $\hat\psi_{j}$ is in $\rd(C^{\prime})$ and
$[\hat\psi_{j},c_{j}]$ belongs to $\J(C^{\prime})$ for all $j$.
Since $\nu$ belongs to $\rd(\genp{c_{1},\dots,c_{n}})$, then $\Psi$ is in $\J(C^{\prime})$.

On the other hand, as we are in ``Case 2'' we have $B\zsu[2]{} C^{\prime}$ and hence $\fla{3}{B_{1}}\cap\J(C^{\prime})=\J(B)$,
but then $\Psi$ belongs to $\J(B)\inn\J(C)$, and $\overline{\Psi}$ is trivial.
The statement of the theorem %$\dkerg{B}{A}\leq-\delta(A/B)$,
is (trivially) true in this case as well.
\end{proofof}

\bigskip
%Now let $\kappa_{3}$ be a non-negative integer which is not yet specified and, symmetrically to
In parallel with axiom $\sig{2}{2}$, now define
for $M$ in $\nla{3}$ a denumerable set of $\Lan{3}$-sentences $\sig{3}{2}$ which express:
\begin{itemize}
\punto{$\sig{3}{2}$}for any finite $A_{1}\inn M_{1}$, $\delta_{3}(A)\geq0$\footnote{the weak bound $\geq0$
will have to be replaced with some stronger property, similar to that required by $\sig{2}{2}$ from Chapter \ref{due}.}
\end{itemize}

A natural candidate for a suitable amalgamation class among $\nla{3}$-algebras with $\sig{3}{2}$
has to be found within -- possibly a subset of -- the family
$$\Klt{3}=\{M\in\nla{3}\mid M_{*}\in\Klt{2},M\sat\sig{3}{2}\}.$$

\begin{rem}\label{finalrem}
It is the same whether we check $\sig{3}{2}$ of an $\nla{3}$-algebra $M$ with
$\delta_{3}$ or with $\ded^{M}$.

In fact Corollary \ref{deltaded} ensures $\delta_{3}(A)\geq\ded^{M}(A)$ for any finite $A_{1}\inn M_{1}$,
while on the other hand, $\ded^{M}(A)\geq\ded^{M}(\ssc(A))=\delta_{3}(\ssc(A))$. In other words
$M$ belongs to $\Klt{3}$ exactly if $M_{*}\in\Klt{2}$ and $\ded^{M}(A)\geq0$ for all finite $A_{1}\inn M_{1}$.
\end{rem}

\medskip
In order to study (AP) within $\Klt{3}$, we first find a setting in which the free amalgamation \pref{amalgatre} inherits
property $\sig{3}{2}$ of its constituents.

Assume $N,A,B$ are $\nla{3}$-algebras in $\Klt{3}$ and suppose $N\nni B\dsu A$.
If we take the amalgam $M$ like in \pref{amalgatre} we have $N\dsu M \nni A$.

We also assume that the free $\nla{2}$-amalgam $M_{*}=\am{N_{*}}{B_{*}}{A_{*}}$ is in $\Klt{2}$, which
is not such a serious restriction. Indeed a variant to the class defined above could rely on  $\nla{3}$-algebras $M$
such that $M_{*}$ is a {\em self-sufficient and algebraically closed} subalgebra of $\K$, where $\K\in\Klt{2}$ is the Fra\"iss\'e limit of
the class $\Kl{2}$ defined in the previous chapter \ref{due} (cfr.\,Proposition \ref{amalsigma2} and Lemma \ref{acldiv}).

\begin{lem}\label{finallemma}
Assume $M\in \nla{3}$ is the amalgam above, for $N\nni B\dsu{}A$ in $\Klt{3}$ with $M_{*}=\am{N_{*}}{B_{*}}{A_{*}}\in\Klt{2}$. Assume $E_{1}$ is a finite
subspace of $M_{1}$,
let $C_{1}$ denote $N_{1}+E_{1}$ and set as usual $C$ to be $\gena{C_{1}}{M}$.
Suppose $C_{*}$ is the $\nla{2}$-free amalgam of $N_{*}$ and $E_{*}$ over $\gena{N_{1}\cap E_{1}}{C_{*}}$,
then $\delta_{3}(E)\geq0$.
\end{lem}
\begin{proof}
First notice $\delta_{3}(E)\geq\ded^{C}(E)$ by Lemma \ref{deltaded}.
%we may assume $E$ is $\delta_{2}$-strong in $M$,
%for $\ded^{M}(E)\geq\ded^{M}(\ssc(E))=\delta_{3}(\ssc(E))$ by Lemma \ref{sameded}.

Now with the above assumptions, Lemma \ref{moduliftlem} and \pref{dedim} yield
\begin{labeq}{rtmod}
\rt_{C}(E_{1})\cap\rt_{C}(N_{1})=\rt_{C}(E_{1}\cap N_{1}).
\end{labeq}

In the same way submodularity \pref{submod} on page \pageref{submod} was obtained for $\nla{2}$, now \pref{rtmod} implies %$\delta_{3}(E)=
$\ded^{C}(E)\geq\ded^{C}(E/N)+\ded^{C}(E_{1}\cap N_{1})$.

Since $N$ is $\delta_{2}$-strong in $C$, by Lemma \ref{sameded} and Remark \ref{finalrem}
we have %$E_{1}\cap N_{1}\zsu[2]{}N_{1}\zsu[2]{}C_{1}$ 
$\ded^{C}(E_{1}\cap N_{1})=\ded^{N}(E_{1}\cap N_{1})\geq0$ for $N\in\Klt{3}$.%\i=\delta_{3}(E_{1}\cap N_{1})$.

We have to prove $\ded^{C}(E/N)\geq0$. To achieve this we will show
$\ded^{C}(E/N)\geq\ded^{M}(E/N)$ and use the fact $N\dsu{}M$.

Notice that by the definition of $C$ and since $N\zsu[2]{}M$ we have $d_{2}^{C}(E/N)=\delta_{2}(E/N)$ and by Definition \ref{ded}
$\rt_{C}(E/N)=\rt(C)/\rt(N)$.

As $\ssc(C_{1})$ is finite over $N_{1}$ and $\rt_{M}(C_{1})=\gam{C}{M}(\rt(C))$, we have $\ded^{C}(E/N)-\ded^{M}(E/N)=
-\delta_{2}(\ssc^{M}(C)/C)-\dfp(\kerg{C}{M})$, where as above $\kerg{C}{M}=\kerg{C}{\,\ssc(C)}$ is the kernel of $\gam{C}{M}$.

Now %as $\ssc^{M}(C_{1})$ is a finite extension of $C_{1}$,
by the finite character of $\ssc^{M}$ described in Proposition \ref{fincharssc},
Theorem \ref{crocotheorem} applies with minor changes to the situation above and yields $\dfp(\kerg{C}{M})\leq
-\delta_{2}(\ssc^{M}(C)/C)$.
\end{proof}
Notice that the subalgebra $C_{*}$ is indeed a free amalgam of $N_{*}$ and $\gena{C_{1}\cap A_{1}}{C_{*}}$, but
the proof actually needs the stronger assumptions stated above.

\medskip
Suppose $M$ is the above $\nla{3}$-amalgam of $N$
and $A$ over $B$.
As a last remark to try solving the amalgamation issue inside $\Klt{3}$ we can address to the following problem.
\begin{rem*}
$M$ amalgamates $N$ and $A$ over $B$, for $A$, $B$ and $N$ in $\Klt{3}$ as above.
Assume $M_{*}=\am{N_{*}}{B_{*}}{A_{*}}$ is in $\Klt{2}$ and for each finite $E_{1}$ of $M_{1}$,
there is a subspace $\widetilde E_{1}\nni E_{1}$ with the features:
\begin{itemize}
\item[-]$d_{2}(\widetilde E)=d_{2}(E)$
\item[-]$N_{*}+\widetilde{E}_{*}$ is the free amalgam of $N_{*}$ and $\widetilde{E}_{*}$ over
$\gena{N_{1}\cap\widetilde{E}_{1}}{N_{*}+\widetilde{E}_{*}}$.
\end{itemize}
Then $M$ lay in $\Klt{3}$.
\end{rem*}
This confirms, it could be useful to work with algebraically closed underlying $\nla{2}$-structures, in the sense of Lemma \ref{acldiv}.

\bibliography{Biblio}
\bibliographystyle{amsalpha} %{alphadidiEN} %
\end{document}