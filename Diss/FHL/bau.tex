Recall that, any object $M$ of $\nla{2}$ is associated a {\em presentation}
$$R\linto \fla{2}{M_{1}}\lonto M$$
and we write $M=\gen{M_{1}\mid R}$,
where $R$ is an homogeneous ideal of the free nil-$2$ Lie algebra $\fla{2}{M_{1}}$, of total weight weight $2$. %$(\fla{2}{M_{1}})_{2}$,
That is, $R$ is a $\Fp$-vector subspace of $(\fla{2}{M_{1}})_{2}$.

We let the homogeneous subspace $(\fla{2}{M_{1}})_{2}$ coincide with the exterior square of the $\Fp$-vector space $M_{1}$ and hence
$\fla{2}{M_{1}}\simeq M_{1}\oplus\exs M_{1}$ (see \cite[\S I.1]{ser}).

Hence if $M$ is given by the presentation\footnote{The couple $\left(M_{1},\rd(M_{1})\right)$ informally represents the kind of structures
utilised in \cite{bad}.}
above, we denote $R$ by $\rd(M)$ %or ambiguously $\rd(M_{1})$,
and we have $$M\simeq M_{1}\oplus\frac{\exs M_{1}}{\rd(M)}.$$

\smallskip
If $M$ is an object of $\nla{2}$, an $\nla{2}$-{\em subalgebra} is by definition, a subalgebra $H$ of $M$,
which is generated by a $\Fp$-subspace $H_{1}$ of $M_{1}$. We write in this case $H=\gena{H_{1}}{M}$.
Conversely for any subspace $H_{1}$ of $M_{1}$, we adopt the convention to denote
by $H$ the $\nla{2}$-subalgebra $\gena{H_{1}}{M}$. % generated by $H_{1}$ in $M$.
By {\em subalgebras} we will exclusively mean $\nla{2}$-subalgebras in the future.

For $\nla{2}$-subalgebras $A$ and $B$ of $M$,
with abuse of the common meaning we {\em denote} by $A+B$ the subalgebra $\gena{A_{1}+B_{1}}{M}$.
%and let $A\sqcap B$ denotes $\gena{A_{1}\cap B_{1}}{M}$.
As $\Fp$-vector spaces, {\em finitely generated} $\nla{2}$-algebras are {\em finite}.

\smallskip
For $M\in\nla{2}$ and a subspace $H_{1}$ of $M_{1}$ we consider $\exs H_{1}$ as a natural subspace of $\exs M_{1}$.
To any such $H_{1}$ or equivalently, to any $\nla{2}$-subalgebra $H=\gena{H_{1}}{M}$ of $M$ we set
\begin{labeq}{erredi}
\rd_{M}(H_{1})=\rd_{M}(H):=\rd(M)\cap\exs{H_{1}}.
\end{labeq}
If the ambient structure $M$ is clear from the context, we simply  write $\rd(H_{1})$ or $\rd(H)$.
%Observe that if $H$ is $\gena{H_{1}}{M}$, then
In any case we have\footnote{Here below instead, $+$ indicates the ordinary sum between a subalgebra and an {\em ideal}. In the sequel
this will be almost never the case.}
$$H\simeq(\fla{2}{H_{1}}+\rd(M))\quot\rd(M)\simeq\fla{2}{H_{1}}\quot \rd_{M}(H).$$

\bigskip
We now introduce an integer valued %\emph{predimension}
function $\delta$ with entries on the finite $\Fp$-subspaces of $M_{1}$, which measures
in terms of $\Fp$-dimension, {\sl how much a finitely generated structure differs from a free one.}
The term {\em deficiency} is also motivated by section \ref{schur}, observe in this case $H_{2}(M)=\rd(M)$.

\begin{dfn}\label{deficienzwei}
Assume an algebra $M$ of $\nla{2}$ has been fixed.
For a finite $\nla{2}$-subalgebra $A$ of $M$ set
\begin{labeq}{deltadue}
\delta(A)=\dfp(A_{1})-\dfp\left(\rd_{M}(A)\right).
\end{labeq}
We call $\delta(A)$ the {\em deficiency} of the subalgebra $A=\gena{A_{1}}{M}$ of $M$.
\end{dfn}
Observe that if an algebra $M$ is fixed, then $\delta(A)$ depends -- by \pref{erredi} --
only on the subspace $A_{1}$ of $M_{1}$. In fact we will write indifferently $\delta(A_{1})$ or $\delta(A)$
for the deficiency of $A$.

Also, $\delta(A)$ is an invariant of the isomorphism type of the structure $A$ and $\delta(A)=
\dfp(A_{1})$ implies $A\simeq\fla{2}{A_{1}}$.

For arbitrary $\nla{2}$-subalgebras $H$ of $M$, and  finite $C_{1}$ (over $H_{1}$), we introduce a {\em relative} deficiency\footnote{with values in $\Z\cup\{-\infty\}$.}
by means of
\[
\delta(C\quot H)=\dim_{\Fp}(C_{1}/H_{1})-\dfp(\rd_{M}(C/H))
\]
provided we define $\rd_{M}(C/H)$ to be the quotient space $\rd(H_{1}+C_{1})\quot \rd(H_{1})$.
We also allow expressions $\delta(C_{1}/H)$ and $\delta(C_{1}/H_{1})$ to denote the above.

For finite $A$ and $B$, we have of course $\delta(A\quot B)=\delta(A+B)-\delta(B)$, while
for finite {\em sets} $\mathcal{U}$ or {\em tuples} in $M_{1}$ and arbitrary $H$, % _{1}$ of $M_{1}$,
we set $\delta(\mathcal{U}/H)=\delta(\genp{H_{1},\mathcal{U}}/H)$ and $\delta(\bar a/H)=\delta(\genp{H,\bar a}/H)$.

\smallskip
Now let $M$ be an $\nla{2}$-algebra, by virtue of Fact \ref{ubc} we have
\begin{rem}\label{rem:exsmod}
For all $H_{1}$ and $K_{1}$ in $M_{1}$
\begin{labeq}{exsmod}
\exs(H_{1}\cap K_{1})=\exs H_{1}\cap\exs K_{1}
\end{labeq}
\end{rem}
\begin{cor}\label{cor:2modker}
Fixed an algebra $M$ of $\nla{2}$, we observe a modular behaviour of the operator
$\rd$ on the subspaces of $M_{1}$, that is for all $H_{1}$ and $K_{1}$,
\begin{labeq}{2modker}
\rd(H_{1}\cap K_{1})=\rd(H_{1})\cap \rd(K_{1})
\end{labeq}
\end{cor}

\smallskip
As a first consequence of the above results, % of modularity in $\rd$
we obtain that $\delta$ is actually a predimension {\em on} $M_{1}$. In fact the
relative deficiency satisfies a stronger version of submodularity.

\begin{lem}\label{2transmogrifer}
Let $H_{1}\nni V_{1}$ and $C_{1}$ be subspaces of $M_{1}$, for $M$ in $\nla{2}$. If $C_{1}$
is finite and $H_{1}\cap C_{1} \inn V_{1}$, %\inn H_{1}$,
then $\delta(C/H)\leq\delta(C/V)$.
\end{lem}
\begin{proof}
On one side, the assumption yields $\dfp(C_{1}\quot H_{1})=\dfp(C_{1}/V_{1})$.

For the negative part of $\delta$, observe that $\rd(C/V)$ embeds into
$\rd(C/H)$. This follows from
$$\rd(V+C)\cap\rd(H)=\rd((V_{1}+C_{1})\cap H_{1})=\rd(V_{1}+(H_{1}\cap C_{1}))=\rd(V).$$
\end{proof}

As an extremal case, we get \emph{submodularity} for $\delta$ on $M_{1}$, that is 
\begin{labeq}{submod}
\delta(C\quot H)\leq\delta(C_{1}\quot C_{1}\cap H_{1})
\end{labeq}
for any $H_{1}$ and finite $C_{1}$.
On finite spaces, if $\cl$ denotes the $\Fp$-linear span in $M_{1}$, this is exactly \pref{summo}.

\smallskip
The next obliged step is to force $\delta$ to be non-negative. With this purpose define the property
\begin{itemize}
\punto{$\sig{2}{2}$}\label{sig22}
\quad for any finite $A_{1}\inn M_{1}$,\;\;$\delta(A)\geq\min(2,\,\dfp(A_{1}))$.
\end{itemize}
As $\delta$ is an invariant of
the isomorphism type of finite $\nla{2}$-algebras, property $\sig{2}{2}$ is first order expressible in
the language $\Lan{2}$ by a denumerable axiom system: just negate the diagrams of those which \emph{do not}
have the desired property.

Some remark about the choice of the number $2$ as a lower bound are to be given.
Of course $1$-generated subalgebras are isomorphic to $\Fp$ in any $\nla{c}$-algebra.

Condition $\sig{2}{2}$ imposes that $2$-generated subalgebras are free. Equivalently,
for any fixed element $a\in M_{1}$ for $M$ with $\sig{2}{2}$, the the kernel of the natural
derivation $ad_{a}\colon x\mapsto[a,x]$ coincides with $\genp{a}$ and,
as a consequence the centre $Z(M)$ is forced to coincide with $M_{2}$. This last condition -- which is equivalent
to require the form $[\,\cdot,\cdot]$ to be non-degenerate -- is therefore weaker than $\sig{2}{2}$.

%By Theorem \ref{faikaha}, this property
This feature reflects to the associated group $G(M)$ (or $\mathscr{G}(M)$) reconstructed from $M$
in Section \ref{algegruppi} (cfr.{\,}Remark \ref{baucond}).
In particular, $\ngb{2}{p}$-groups obtained via $G(\cdot)$ from $\nla{2}$-algebras with $\sig{2}{2}$ all share the property
$G^{\prime}=Z(G)$.

On the other hand, $\sig{2}{2}$ influences the pregeometry on $M_{1}$ associated to $\delta$ (cfr.{ }Corollary \pref{2geom} below).

\smallskip
\begin{rem}\label{preg2}
Assume $M$ has $\sig{2}{2}$, if $\cl$ denotes the $\Fp$-linear closure in $M_{1}$, then
$\delta$ defines a $\cl$-predimension on $M_{1}$ according to definition \ref{clpred}.

Denote by $d^{M}$ or simply $d$,
the dimension function on $M_{1}$ associated to $\delta$
with Lemma \ref{preg} and and by $\cl_{d}^{M}$ or $\cl_{d}$ the resulting closure. For finite $\nla{2}$-subalgebras $A$, we have
\begin{itemize}
\item[-]$d(A)\defeq d(A_{1})=\min(\delta(C)\mid C_{1}\nni A_{1})$ and $d(A)\leq\dfp(A_{1})$
\item[-]$\cl_{d}$ extends $\cl$ and $b\in\cl_{d}(A_{1})$ exactly if $d(A_{1},b)=d(A)$.
\end{itemize}
\end{rem}

\smallskip
In presence of $\sig{2}{2}$ the notion of {\em self-sufficiency} which follows, let us {\em choose} for any given $A_{1}$, a distinguished
minimal space of deficiency $d(A)$ above $A_{1}$.

\begin{dfn}\label{2strong}
Let $H_{1}$ be a subspace of $M_{1}$, for $M\in\nla{2}$.
We call both $H_{1}$ and the $\nla{2}$-sublagebra $H$,
\emph{strong} or \emph{self-sufficient} in $M_{1}$ or $M$ respectively if
for any finite subspace $C_{1}\inn M_{1}$, we have $\delta(C\quot H)\geq0$.
This is written $H_{1}\zsu{}M_{1}$ or $H\zsu{}M$.

For any integer $n<\omega$, we say that $H$ is $n$-strong in $M$, if
$\delta(C_{1}\quot H_{1})\geq0$ holds for all subspaces $C_{1}$ of $M_{1}$
with $\dfp(C_{1}\quot H_{1})\leq n$. We write in this case $H\zsu{n}M$.
We say that an $\nla{2}$-embedding $\phi$ of $H$ into $M$ is ($n$-){\em strong} %or ($n$-){\em self-sufficient}
if $\phi(H)$ is ($n$-) strong in $M$. %(for $n<\omega$).
\end{dfn}
\begin{rem}\label{deltadi}
A finite subspace $A_{1}$ of $M_{1}$ is self-sufficient in $M_{1}$ if and only if $d^{M}(A)=\delta(A)$ and
in general $d(A)\leq\delta(A)$.
\end{rem}

\begin{dfn}
Let $B_{1}$ be a finite subspace of $M_{1}$, define a \emph{self-sufficient closure} of $B_{1}$ in $M_{1}$ to be an
$\inn$-minimal subspace $A_{1}$ of $M_{1}$ containing $B_{1}$ with $\delta(A)=d(B)$.
By Lemma \ref{interstrong} below, the family of strong subspaces of $M_{1}$ is closed under
intersection, as a consequence the notion of self-sufficient closure of a finite space $A_{1}$ depends on
$A$ and $M$ only and is univocally determined as:
$$\ssc^{M}(A_{1})=\ssc(A_{1})\defeq\bigcap\{C_{1}\zsu{}M_{1}\mid C_{1}\, \text{finite and}\, C_{1}\nni A_{1} \}$$
We define the $\nla{2}$-subalgebra $\ssc^{M}\!(A)=\ssc(A)$ of $M$ as $\gena{\ssc(H_{1})}{M}$and we call it
the self-sufficient closure {\em of} $H$ in $M$. For a finite {\em subset} $\mathcal{U}$ of $M_{1}$, we set $\ssc(\mathcal{U})$
to be $\ssc(\genp{\mathcal{U}})$.
\end{dfn}

\smallskip
Note that this definition implies the operator $\ssc$ is actually a closure operator:
it is monotone, and has properties (cl1) and (cl2) of definition $\ref{pregdef}$, moreover
$\ssc(A_{1})\inn\cl_{d}(A_{1})$.

%the intersection above may be considered  to involve finitely many subspaces only.
%In the next section it will be clear how the {\sl ambient structure} $M$ influences the nature of $\ssc$.
%By the fact $d(A_{1})=\delta(\ssc(A_{1}))=d(\ssc(A_{1}))$, it follows now 
%for all $A_{1}$ and by lemma \ref{preg2}
%$d(A_{1})\leq\dfp(A_{1})$. 

\begin{cor}\label{2geom}
For any algebra $M$ with $\sig{2}{2}$, the pregeometry $(M_{1},\cl_{d})$ associated to $\delta$ is actually a
geometry over the $\Fp$-linear closure according to Definition \ref{pregext}.

For a given $M$, the geometry  $\cl_{d}$ is not in general locally-modular.
\end{cor}
\begin{proof}
Axiom $\sig{2}{2}$ implies any two linearly independent couple $a,b$ generates a self-sufficient
subalgebra $\gena{a,b}{M}\simeq\fla{2}{a,b}$ and $d(a,b)=2$. Analogously
$d(\genp{\vac})=d(\triv)=\delta(\triv)=0$ and for any $a\in M_{1}$, $d(\genp{a})=
\delta(a)=1$. It follows $\cl_{d}(\vac)=\triv$ and $\cl_{d}(a)=\genp{a}$.

\smallskip
Consider now the finite $\nla{2}$-algebra $M=\gen{M_{1}\mid\rd(M)}$ whose $M_{1}$ has $\Fp$-base $\{a,b,c,x,y\}$ and such that the relator ideal $\rd(M)$
is spanned in $\exs M_{1}$ by the independent homogeneous elements
$$[a,b]+[x,y],\quad[c,x]+[y,b],\quad[a,y]+[b,c].$$
One checks $M$ has $\sig{2}{2}$ and the subspace $\genp{a,b,c}$ is not self-sufficient in $M$: $\delta(M/a,b,c)=-1$. It follows
$2=d(a,b,c)<\delta(a,b,c)=3$ and this yields
$$d(a,b)+d(b,c)=4>3=d(a,b,c)+d(b).$$
\pref{mod} is not (even locally) satisfied.
\end{proof}

\medskip
Some properties of sef-sufficient spaces will now follow. We assume an algebra $M$ of $\nla{2}$ has been fixed
with $\sig{2}{2}$. All the subspaces and subalgebras considered, lay in $M_{1}$ and $M$ respectively.

\begin{rem}\label{finitedeltabase}
For a self-sufficient $H$ and a finite $A$, we can always find a {\em finite} strong subalgebra $H^{\rm o}$
such that $\delta(A/H)=\delta(A/H^{\rm o})$.

For an arbitrary $H$, one has
$$\delta(A/H)=\inf\left(\delta(A/C)\mid C_{1}\text{ finite and }A_{1}\cap H_{1}\inn C_{1}\zsu{}H_{1}\right).$$
%This also follows by the definition itself of $\delta(A_{1}\quot H_{1})$ above: assume $H$ is self sufficient,
%since $\rd(A_{1}\quot H_{1})$ has finite dimension,
%choose $C_{1}$ in $H_{1}$ such that $\exs (C_{1}+A_{1})$ supports each element of a basis of
%$\rd(H_{1}+A_{1})$ over $\rd(H)$.
\end{rem}
\begin{proof}
For the first part, since $\rd(A/H)$ has to be finite dimensional, pick a finite $\nla{2}$-subalgebra
$H^{\rm o}$ in $H$ with $H^{\rm o}_{1}\nni H_{1}\cap A_{1}$ and such that $\rd(H+A)$ has a basis in $\exs\genp{H^{\rm o}_{1},A_{1}}$ over
$\exs H_{1}$. By Corollary \ref{interstrong} below we can choose $H^{\rm o}$ to be self-sufficient. 

The second part follows by Lemma \ref{2transmogrifer} and the above arguments.
\end{proof}
%As a  consequence of this, self-sufficiency is closed under increasing chains of subspaces: assume $B_{1}^{i}\zsu{} M_{1}$ for all $i<\omega$ is an increasing family of finite subspaces
%whoose union subspace is $B_{1}\inn M_{1}$. Then by the previous remarks, for any finite $A_{1}$, we can prove $\delta(A_{1}\quot B_{1}^{i})\searrow\delta(A_{1}\quot B_{1})$ for $i\rightarrow\infty$.
%%[EXT] -->  \dfp K_{1}\quot V_{1}^{i} is def.ly constant = \dfp K_{1}\quot V_{1} while \rdK_{1}\quot V_{1}^{i} converges increasing to \rdK_{1}\quot V_{1}.
%This gives
%$V_{1}\zsu{} M_{1}$.

The next lemma shows {\em transitivity} of strong embeddings.
\begin{lem}\label{2trans}
If $H$ is $n$-strong in $K$ and $K$
is self-sufficient in $M$, then $H$ is $n$-strong in $M$.

In particular from $H\zsu{} K$ and $K\zsu{} M$, follows $H\zsu{}M$.
\end{lem}
\begin{proof}
Let $C_{1}$ be a finite subspace, both statements of the lemma follow from the
inequality $\delta(C_{1}/H)\geq\delta(C_{1}\cap K_{1}/H)+\delta(C_{1}/K)$.

We have equality for the $\Fp$-linear dimensions and for the negative parts, we observe that
$\rd(C/H)$ maps to $\rd(C/K)$ with kernel
$$\frac{\rd(H+C)\cap\rd(K)}{\rd(H)}=\frac{\rd((H_{1}+C_{1})\cap K_{1})}{\rd(H)}=\frac{\rd(H_{1}+(C_{1}\cap K_{1}))}{\rd(H)}.$$
\end{proof}


Another straightforward application of lemma \ref{2transmogrifer} is the following:
\begin{lem}[Cut Lemma]\label{2cut}
If $H$ is self-sufficient in $K$, then for any subspace $V_{1}$ of $M_{1}$, we have
$H_{1}\cap V_{1}\zsu{} K_{1}\cap V_{1}$.
\end{lem}
\begin{proof}
Observe $\delta(E_{1}/H_{1}\cap V_{1})\geq\delta(E/H)\geq0$ whenever $E_{1}\inn K_{1}\cap V_{1}$, since
$H_{1}\cap V_{1}$ contains $E_{1}\cap H_{1}$.
\end{proof}
\begin{cor}\label{interstrong}
If $H$ and $K$ are self-sufficient,
then the intersection $H_{1}\cap K_{1}$ is also strong in $M_{1}$.
\end{cor}
\begin{proof}
By Lemma \ref{2cut} we have $H_{1}\cap K_{1}\zsu{} K_{1}$.
Then conclude by transitivity of $\zsu{}$ (Lemma \ref{2trans}).
\end{proof}

%We conclude this section with further results involving the predimension $\delta$ and
%its interactions with $d^{M}$.

\begin{lem}\label{finchar}
Let $H_{1}$  be a subspace of $M_{1}$ then
$H$ is strong if and only if for any finite subspace $C_{1}$ of $H_{1}$ there exists a finite subspace
$C_{1}^{\rm o}\inn H_{1}$ containing $C_{1}$, such that $C^{\rm o}\zsu{}M$.
\end{lem}
\begin{proof}
If $H$ is strong, given any finite $C_{1}$ in $H_{1}$, then take $C^{\rm o}$ to be $\ssc(C)$.
Because of Lemma \ref{interstrong} $C^{\rm o}$ is contained in $H$.

\smallskip
For the converse, if $A_{1}$ is finite in $M_{1}$, we want $\delta(A/H)$ to be non negative.
But this follows by the hypothesis applying Remark \ref{finitedeltabase}.
\end{proof}

Given two algebras $N\inn M$ of $\nla{2}$ the self-sufficient
closure of a finite subspace $A_{1}$ of $N_{1}$ computed in $N$ may differ from $\ssc^{M}(A_{1})$.
But as expected we have
\begin{rem}\label{samed2}
Assume $N$ is an $\nla{2}$-subalgebra of $M$. Then $N$ is strong in $M$ if and only if for all subspaces
$V_{1}$ of $N_{1}$ the closures $\ssc^{N}(V)$ and $\ssc^{M}(V)$ %$d^{N}(V_{1})$ and $d^{M}(V_{1})$
coincide.
\end{rem}
\begin{proof}
We may suppose $V$ are {\em finite} subalgebras in the statement and in general $\ssc^{M}(V)\inn\ssc^{N}(V)$ as strongness
is expressible via universal sentences.

The first condition is clearly sufficient. It is necessary by virtue of Lemma \ref{finchar},
since for any finite $V_{1}\inn N_{1}$, we now know $\ssc^{N}(V)=\ssc^{M}(V)$ is {\em inside} $N$, but strong {\em in} $M$.
\end{proof}
We might have stated Remark \ref{samed2} in terms of $\cl_{d}$-dimensions: $N\zsu{}M\iff d^{N}(V_{1})=d^{M}(V_{1})$
for any subspace $V_{1}\inn N_{1}$.

\begin{lem}\label{samedelta2}
Assume $H\zsu{} M$ and $\delta(A/H)=0$ for some finite subspace $A_{1}$ of $M_{1}$, then $H+A$ is %$\delta$-
self-sufficient in $M$ as well.

Moreover if an element $a$ of $M_{1}$ is $\cl_{d}$-independent of $H_{1}$, i.{}e.\,$d^{M}(a/H)=1$, then
$\gena{H_{1},a}{M}$ is strong in $M$.
\end{lem}

\begin{proof}
Consider a finite subspace $E_{1}$ of $M_{1}$, then the first statement follows by computing
$$\delta(E/H+A)=\delta(E+A/H)-\delta(A/H).$$

For the second one, note that any finite subspace of $\genp{H_{1},a}$ is
contained in some $\genp{A_{1},a}$ where $A_{1}$ is a finite strong subspace of $H_{1}$.
Since $d=d^{M}$ is a dimension, $1=d(a/H_{1})\leq d(a/A_{1})\leq1$ implies
$d(A_{1})+1=d(A_{1},a)$.

We conclude by Lemma \ref{finchar} showing that $\genp{A_{1},a}\zsu{}M_{1}$.
We have indeed, since $A\zsu{}M$, $\delta(A_{1},a)\leq\delta(A)+1=d(A_{1})+1=d(A_{1},a)$.
This yields $\delta(A_{1},a)=d(A_{1},a)$.
\end{proof}

\bigskip
For an arbitrary space $H_{1}$ we
define the {\em self-sufficient closure} $\ssc^{M}(H_{1})$ of $H_{1}$ as the subspace of $M_{1}$ generated by the
self-sufficient closures of all the finite parts of $H_{1}$. This space is strong on account of Lemma \ref{finchar}.
As before, by $\ssc(H)$ we mean $\gena{\ssc(H_{1})}{M}$. This is the minimal strong $\nla{2}$-subalgebra of $M$
containig $H$.

\smallskip
We adopt for the sequel the following notation: for any subspace$H_{1}$ and tuple $\bar a$ of $M_{1}$, we write
$\ssc(H_{1},\bar a)$ for $\ssc(\genp{H_{1},\bar a})$ and $\ssc(H,\bar a)$ for $\gena{\ssc(H_{1},\bar a)}{M}$.
On the other hand, by default $d^{M}$ reads indifferently sets, subspaces or tuples of $M_{1}$.

\begin{prop}\label{fincharssc}
Assume $H_{1}$ is a strong subspace of $M_{1}$ and $\bar a$ is a finite tuple in $M_{1}$, then
\begin{itemize}
\punto{i}$d(\bar a\quot H_{1})\leq\delta(\bar a\quot H_{1})$ %for any finite tuple $\bar c\inn M_{1}$
\punto{ii}$\ssc(H_{1},\bar a)$ is a {\em finite} extension of $H_{1}$,
\punto{iii}$d(\bar a\quot H_{1})=\delta(\ssc(H_{1},\bar a)\quot H_{1})=\min(\delta(A_{1}\quot H_{1})\mid A_{1}\nni\bar a)$
\punto{iv}$d(\bar a\quot H_{1})=\delta(\bar a\quot H_{1})$ iff $\genp{H_{1},\bar a}\zsu{}M_{1}$.
\punto{v}There exists a finite $H_{1}^{\rm o}\zsu{}H_{1}$ such that $\ssc(H,\bar a)=H+\ssc(H^{\rm o},\bar a)$, 
$H_{1}\cap\ssc(H_{1}^{\rm o},\bar a)=H_{1}^{\rm o}$ and that %$\delta(\ssc(H_{1}^{\rm o},\bar a)\quot H^{\rm o}_{1})
$d(\bar a/ H)=d(\bar a/ H^{\rm o})$.
\end{itemize}
%Moreover 
\end{prop}
\begin{proof}
(i). By (fin) and \pref{dimadd} of Section \ref{qdim} and Remark \ref{finitedeltabase} above, we can find a finite subspace $H^{\rm o}_{1}\zsu{}H_{1}$
with $H^{\rm o}_{1}\nni H_{1}\cap\genp{\bar a}$, such that
$$d(\bar a\quot H_{1})=d(\bar a\quot H_{1}^{\rm o})=d(H^{\rm o}_{1},\bar a)-d(H^{\rm o}_{1})$$
and that
$$\delta(\bar a\quot H_{1})=\delta(\bar a\quot H_{1}^{\rm o})=\delta(H^{\rm o}_{1},\bar a)-\delta(H^{\rm o}_{1}).$$
Since $H^{\rm o}\zsu{}M$, the statement follows immediately by the relation between $\delta$ and $d$ for
finite subspaces of $M_{1}$.

\smallskip
(ii). Since $\delta(A/H)$ is non-negative for all finite subspace $A_{1}$ in $M_{1}$, take
a finite subspace $A_{1}$ containing $\bar a$ with a minimal value of $\delta(A/H)$.
It follows that for an arbitrary finite $C_{1}$ one has
$$\delta(C/H+A)=\delta(C+A/H)-\delta(A/H)\geq0.$$
This means $H+A$ is self-sufficient in $M$ and hence contains $\ssc(H,\bar a)$.

As a consequence,
%the generalised self-sufficient closure
%preserves the properties of her finite analogous, that is,
the second equality in (iii) holds: $\delta(\ssc(H_{1},\bar a)/H)=\min(\delta(A/H)\mid\bar a\inn A_{1}\inn M_{1},A_{1}\text{ finite})$.

\smallskip
(iii). Take a finite tuple $\bar b$ of $M_{1}$ linear independent over $H_{1}$,
such that $\genp{H_{1},\bar b}=\ssc(H_{1},\bar a)$.
Since $\bar b\inn\ssc(H_{1},\bar a)\inn\cl_{d}(\bar a\quot H_{1})$,
we have $d(\bar b\quot H_{1})=d(\bar a\quot H_{1})$.

As $\genp{H_{1},\bar b}$ is self-sufficient, we can find a finite strong subalgebra $H^{\rm o}$ of $H$,
such that $\genp{H_{1}^{\rm o},\bar b}\zsu{}M_{1}$ with $H_{1}^{\rm o}\nni H_{1}\cap\genp{\bar a}$ and
$d(\bar b/H^{\rm o})=d(\bar b/H)$.

Now by (i) and Lemma \ref{2transmogrifer} we obtain
$d(\bar b/H)\leq\delta(\bar b/H)\leq\delta(\bar b/H^{\rm o})=d(\bar b/H^{\rm o})$ and hence
$\delta(\ssc(H_{1},\bar a)/H)=\delta(\bar b/H)=d(\bar a/H)$. 

\smallskip
(iv). Follows from (iii). 

\smallskip
(v). Let $H^{\rm o}$ and $\bar b$ like in (iii).\,above, since $\ssc(H^{\rm o}_{1},\bar a)\inn\genp{H^{\rm o}_{1},\bar b}$,
we have
$$H^{\rm o}_{1}=\genp{H^{\rm o}_{1},\bar b}\cap H_{1}\nni\ssc(H^{\rm o}_{1},\bar a)\cap H_{1}\nni H^{\rm o}_{1}$$
that is $H^{\rm o}_{1}=\ssc(H^{\rm o}_{1},\bar a)\cap H_{1}$.

On the other hand, by applying submodularity \pref{submod} and (iii) above, we get
$$\delta(\bar b\quot H)\leq\delta(\ssc(H^{\rm o}_{1},\bar a)/H_{1})
\leq\delta(\ssc(H^{\rm o}_{1},\bar a)/H_{1}^{\rm o})=d(\bar b\quot H^{\rm o})=d(\bar b\quot H).$$
Thus by (iv), since  $\delta(\ssc(H^{\rm o}_{1},\bar a)/H)=d(\bar a/H)=d(\ssc(H^{\rm o}_{1},\bar a)/H)$, we have
$H+\ssc(H^{\rm o},\bar a)\zsu{}M$.
It follows $\genp{H_{1},\bar b}\inn H_{1}+\ssc(H^{\rm o}_{1},\bar a)$ and hence $\ssc(H_{1},\bar a)=H_{1}+\ssc(H_{1}^{\rm o},\bar a)$.
Moreover this yields also $\genp{H_{1}^{\rm o},\bar b}=\ssc(H^{\rm o}_{1},\bar a)$.
\end{proof}
From the last proposition it follows, for $H$ and $\bar a$ as above, that $\ssc(H_{1},\bar a)$ is
the intersection of all strong subspaces of $M_{1}$ containing $\genp{H_{1},\bar a}$. % and finite over $H_{1}$.
\begin{rem}\label{cielle2}
Let $\mathcal{B}$ be any set of $M_{1}$, then
%$$\cl_{d}(\mathcal{B})=\gen{\,\bigcup\{C_{1}\,\text{finite}\mid\delta(C_{1}\quot\ssc(\mathcal{B}))=0\}}.$$
$\cl_{d}(\mathcal{B})$ is the subspace of $M_{1}$ generated by all finite $C_{1}\inn M_{1}$ such that $\delta(C_{1}/
\ssc(\mathcal{B}))=0$.

In particular $\ssc(\mathcal{B})\inn\cl_{d}(\mathcal{B})$ for all sets $\mathcal{B}$.
\end{rem}
%\begin{proof}
%By definition $\delta(\ssc(\mathcal{B},a/\mathcal{B})=0$ for all $a\in\cl_{d}(\mathcal{B})$.
%
%On the contrary
%\end{proof}
