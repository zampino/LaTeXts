%\chapter{Amalgamation within $\nla{n}$}
%\section{Free Functors}\label{frefu}
Consider the functor $\map{\tr{n}}{\nla{n+1}}{\nla{n}}$ defined by quotienting out the last component:
$\tr{n}A=A\quot A_{n+1}$, note that here $A_{n+1}=A^{n+1}$, in particular $A_{n+1}$ is an ideal.
%We interpret $\tr{n}$ locally, as the natural projection onto $A_{n+1}$.

If $A\in\nla{n}$ we can assume $A=\fla{n}{X_1}\quot R$, where $X_1$ is a free basis for the linear space $A_1$.
%Now consider $R_i=R\cap(\fla{n}{X_1})_i$ as a vector subspace of $(\fla{n+1}{A_1})_i$, so define $\jei{A}$
%%=\genid{R_2,\dots,R_n}{\fla{n+1}{A_1}}$.
%as the ideal of $\fla{n+1}{A_1}$ generated by $R_2$,\,\dots,\,$R_n$
(by our assumpion on $\nla{n}$, we have $R_{1}=(0)$).

Consider the %canonical embedding
inclusion $i\colon\fla{n}{X_{1}}%\lesssim
\inn\fla{n+1}{X_{1}}$ and
let $\jei{A}=\genid{i(R)}{\fla{n+1}{X_{1}}}$.
%Id est $$
%\jei{A}=\genid{R_2,\dots,R_n}{\fla{n+1}{X_1}}. 
%$$

We define a mapping
\begin{eqnarray*}
%\map{\fr{n+1}}{&\!\!\nla{n}&}{\nla{n+1}}\\ %, such that $
&\!\!A&\longmapsto\fr{n+1}A=\fla{n+1}{X_1}\quot\jei{A}%$, for each $
\quad\forall A\in\nla{n}.
\end{eqnarray*}
Of course if $A\in\nla{n}$ then $\tr{n}(\fr{n+1}A)=A.$

If $M\in\nla{n}$ and %$\fr{n+1}M$ is the $n+1$-th nilpotent Lie Algebra, which is the \emph{most free} among algebras
%$M$ up to the first $n$ component
%In fact for each
$N\in\nla{n+1}$ is such that $\tr{n}N=M$ there exists a morphism of Lie Algebras $\map{\pi}{\fr{n+1}M}{N}$ such that the following diagram commutes:
%\labeq
\begin{labeq}{communo}
\xymatrix{
{\fr{n+1}M}\ar[dr]_{\tr{n}}\ar[rr]^{\pi}&&N\ar[dl]^{\tr{n}}\\
&M&}
\end{labeq}
To see this assume $\map{\chi}{\fla{n+1}{X_1}}{N}$ is a \emph{presentation} of $N$, $X_1$ a base of $N_1$ and $M=\tr{n}N$, we can obtain a presentation $\chi^M$ of $M$ from $\fla{n}{X_1}$ such that
%$i\chi\tr{n}=\chi^{M}$ and
%the scheme: assume $M=\gena{X_1}{M}$ where $N=\gena{X_1}{N}$.
$$\xymatrix{&\fla{n+1}{X_1}%\ar[dl]_{\tr{n}}
\ar[dr]^{\chi}&\\\fla{n}{X_1}\ar[ur]^{i}\ar[dr]^{\chi^M}&&N\ar[dl]_{\tr{n}}\\&M&}$$
commutes, so far we have
\begin{labeq}{communoemezzo}
%\xymatrix{&\fla{n+1}{X_1}\ar[dl]_{\tr{n}}\ar[dr]^{\chi}&\\\fla{n}{X_1}\ar[dr]^{\chi^M}&&N\ar[dl]_{\tr{n}}\\&M&
\xymatrix{&\fla{n+1}{X_1}\ar[dl]_{\pi_{\jei{M}}}\ar[dr]^{\chi}&\\\fr{n+1}{M}\ar[rr]^{\pi}&&N}
\end{labeq}
%where $\chi$ is a \emph{presentation} for the algebra $N$.
%and $\chi^M$ is the \emph{presentation} of $M$ induced by $\chi$.

%If $I=\ker\chi$ then $\jei{M}\inn I$. % (considering a \emph{presentation} of $M$ from $\fla{n}{X_1}$, induced by $\chi$).
In the actual setting $\jei{M}=\genid{i(\ker(\chi^{M}))}{\fla{n+1}{X_{1}}}\inn\ker\chi$, therefore the map $\pi$ quotient of $\chi$ modulo $\jei{M}$
%Take $\phi$ as the quotient map of $\chi$ modulo $\jei{M}$.
satisfies the property in \pref{communo} as we wanted.
Moreover $\pi$ is surgective and $\ker\pi\inn(\fr{n+1}M)_{n+1}$.
%________________________________________________________________________________________
\section{Embedding Lemmas ($n=3$)}\label{emblem}
Consider $M\in\nla{2}$, then $M=M_1\oplus\left(\exs M_1\quot N\right)$.
Pick a subalgebra $H=\gen{H_1}$ with $H_1\sa M_1$.

By our definition in \dots it comes out
$H=H_1\oplus\exs H_1\quot N(H)=H_1\oplus(\exs H_1+N)\quot N$, where $N(H)=\exs H_1\cap N$.

If we consider the subalgebra $\gena{H_1}{\fr{3}{M}}$, then of course $\tr{2}\gena{H_1}{\fr{3}{M}}=H$ so diagram
\pref{communo} bekome:
\begin{labeq}{commdue}
\xymatrix{
{\fr{3}H}\ar[dr]_{\tr{2}}\ar[rr]^{\pi}&&\gena{H_1}{\fr{3}{M}}\ar[dl]^{\tr{2}}\\
&H&}
\end{labeq}

Next lemma shows operator $\fr{3}$ behaves good with respect to subalgebras, provided the substructure
is selfsufficient at level 1.

\label{bellemma}
\begin{lem}
$M$ is an element of $\nla{2}$.

Let $H=\gena{H_{1}}{L}$ be a subalgebra of $M$, where $H_1$ is \emph{selfsufficient} in $M_1$,
then the map $\pi$ of diagramm {\rm\pref{commdue}} is a Lie Algebra monomorphism.

Insbesondere einbettet $\fr{3}H$ in $\fr{3}M$.
\end{lem}
\begin{proof}
Let $X_1$ be a base for $M_1$ and $\xoh\inn X_1$ be a base for $H_1$.
As in diagramm \pref{communoemezzo} we have:
\begin{labeq}{lollipop}
%$$
\xymatrix{&\fla{3}{\xoh}\ar[dl]_{\pi_{\jei{H}}}\ar[dr]^{\chi}\ar@{-}[r]^{i}&\fla{3}{X_1}\ar[dr]^{\pi_{\jei{M}}}\\
\fr{3}{H}\ar[rr]^{\pi}&&\gena{H_1}{\fr{3}{M}}\ar@{-}[r]^{i}&\fr{3}{M}}
%\ar@{^{(}->}[r]
%{\chi}_{|_{\xoh}}
%$$
\end{labeq}

here $\chi\colon u\longmapsto u^{\pi_{\jei{M}}}$ for each $u\in\xoh$,
therefore $\ker\chi=\jei{M}\cap\fla{3}{\xoh}$.
Now construction of $\pi$ in \ref{frefu} gives
\begin{labeq}{nucleo}
\ker\pi=(\jei{M}\cap\fla{3}{\xoh})\quot\jei{H}.
\end{labeq}
Assuming $\pi$ not injective, there exists a $w\in(\jei{M}\cap\fla{3}{\xoh})\setminus\jei{H}$.
As $(\fla{3}{X_1})_1=M_1$ and $(\fla{3}{\xoh})_1=H_1$, we have
\begin{eqnarray*}
\jei{M}=\genid{N}{\fla{3}{X_1}}&=&\gen{\mu,\,[\nu,z]\mid\mu,\nu\in N,\,z\in M_1}{}\\
\jei{H}&=&\gen{\mu,\,[\nu,z]\mid\mu,\nu\in N(H),\,z\in H_1}{}
\end{eqnarray*}
%$\jei{M}=\genid{N}{\fla{3}{X_1}}=\gen{\nu,\,[\nu,z]\mid\nu\in N,\,z\in M_1}{}$ and $\jei{H}=\gen{\nu,\,[\nu,z]\mid%\nu\in N(H),\,z\in H_1}{}$.

According to our remarks in \ref{frefu}, $w$ must have weight $3$, therefore
\begin{labeq}{summa}
%\ker\pi\ni\bar
w=\sum_{\al}[\nu_{\al},z_{\al}].
\end{labeq}
for some $\nu_{\al}\in N$ and $z_{\al}\in M_1$. From the structure of $\ker\pi$, $w$ must be also a sum of commutators of weight $3$ in the elements $\xoh$.

\smallskip
We first bring the sum \pref{summa} to an expression of the form
\begin{labeq}{mustiola}
w=\sum_{i=1}^{m_{1}}[\nuu{i}, u_i]+\sum_{j=1}^{n_{1}}[\nuw{j}, w_j]
\end{labeq}
where the set ${\{u_i\}}_{i}$ is linearly independent in $H_1$, while $\{w_j\}_{j}$ is an independent set over $H_1$ of $M_{1}$.

This is possible once the following inductive step is adopted: assume $u_1,\dots, u_{m_{0}}\inn H_{1}$ appear as a first independent segment of the $z_{\al}$'s in \pref{summa}, followed by $w_1,\dots, w_t$ in $M_1$,
independent over $H_1$, where $t$ is a maximal number for such elements.
%, these elements are extracted out of $(z_{\al})$.

If a suitable independence condition fails for the next term $z$, %$w_{t+1}$,
then $z=\sum_{j=1}^{t}\theta_j{w_j}+\sum_{i=1}^{m_{0}}{\gamma_i}{u_i}+h$, where $h\in H_1\setminus\spn(u_1,\dots u_{m_{0}})$
and some of the terms $\theta_{j}$, $\gamma_{i}$ or $h$ may be trivial.

Using bilinearity of the Lie multiplication we get$$
[\nu,z]=\sum_{i=1}^{m_{0}}[\gamma_i\nu, u_i]+[\nu,h]+\sum_{j=1}^{t}[\theta_{j}\nu, w_j].
$$

Now if we put $u_{{m_{0}}+1}=h$ and $\nuu{{m_{0}}+1}=\nu$, \pref{summa} becomes
\begin{multline*}
w=\sum_{i=1}^{m_{0}}[\nuu{i},u_i]+\sum_{j=1}^{t}[\nuw{j},w_j]+[\nu,z]+\summ{\beta}{[\nu_{\beta},}{z_{\beta}]}=\\
=\sum_{i=1}^{m_{0}}[\nuu{i}+\gamma_i\nu,u_i]+[\nuu{m_{0}+1},u_{m_{0}+1}]+\sum_{j=1}^{t}[\nuw{j}+\theta_j
\nu,w_j]+\summ{\beta}{[\nu_{\beta},}{z_{\beta}]}.
%$$ %
\end{multline*}
After renaming $\nuu{i}$ and $\nuw{j}$, we obtain
$$
w=\sum_{i=1}^{m_{0}+1}[\nuu{i},u_i]+\sum_{j=1}^{t}[\nuw{j},w_j]+\summ{\beta}{[\nu_{\beta},}{z_{\beta}]}.
$$
Iterating this process we get to an expression of type \pref{mustiola}.

\smallskip
Now we claim that sum \pref{mustiola} can be transformed to obtain
\begin{labeq}{sangennaro}
w=\sum_{i=1}^m[\nuu{i}, u_i]+\sum_{j=1}^n[\nux{j}, x_j]+\sum_{k=1}^r[\lay{k},y_k]
\end{labeq}
where again ${\{u_i\}}_{i}$ is independent in $H_1$, ${\{x_j, y_k\}}_{j,k}$ is independent over $H_1$,
the set ${\{\nuu{i},\nux{j}\}}_{i,j}$ is in $N$, independent over
$\exs H_1$ and ${\{\lay{k}\}}_{k}$ lays in $N(H)$ as an independent set.

Working modulo $\jei{H}$, we assume none of the terms $\nuu{i}$ in \pref{mustiola} lays in $\exs H_1$,
moreover the $\nuu{}$'s can assumed to be independent over $\exs H_1$.

\smallskip
We first transform part $\sum_{j=1}^{n_{1}}[\nuw{j}, w_j]$ of \pref{mustiola}.
%$$
%\sum_{j=1}^{n_{1}}[\nuv{j}, v_j]\;\text{of \pref{mustiola} in}\;
%\sum_{j=1}^n[\nux{j}, x_j]+\sum_{k=1}^r[\lay{k},y_k]
%$$ \tojoris{with suitable properties}.
Assume
%\pref{mustiola} has been trasformed, in such a way that a first segment of the sum
%has the desired independence properties:$$
we have
$$
\sum_{j=1}^{n_{1}}[\nuw{j}, w_j]=\sum_{j=1}^{n_0}[\nux{j}, x_j]+\sum_{k=1}^{r_0}[\lay{k},y_k]+[\nu,z]+\sum_{\beta}[\nu_{\beta},z_{\beta}]
$$
where ${\{\nux{j}\}}_{j}$ is a maximal set of $\nuw{}$'s which are independent over $\exs H_{1}$ and
the $\lay{}$'s are independent inside $N(H)$.  

If now $\{\nux{1},\dots,\nux{n_0},\nu\}$ is not independent over $\exs H_1$,
%and $(x_{j},y_{k},z,z_{\beta})$ is independent over $H_1$.
then ${\nu}=\summ{j}{\gamma_j}{\nux{j}}+
\summ{k}{\mu_k}{\lay{k}}+h$, with $h\in N(H)$, $h$ indepent over ${\{\lay{k}\}}_{k=1}^{r_0}$. %and  $[\nu,z]=$.

Now put $\lay{r_0+1}=h$ and $y_{r_0+1}=z$, distribute $\nu$ by bilinearity, and obtain
$$
%=\sum_{i=1}^{m_0}[\nuu{i}, u_i]+
\sum_{j=1}^{n_{1}}[\nuw{j}, w_j]=\sum_{j=1}^{n_0}[\nux{j}, x_j+\gamma_jz]+\sum_{k=1}^{r_0}[\lay{k},y_k+\mu_kz]
+[\lay{r_0+1},y_{r_0+1}]+\sum_{\beta}[\nu_{\beta},z_{\beta}].
$$
Note that the set ${\{x_j+\gamma_jz,\, y_k+\mu_kz,\, y_{r_{0}+1},\,z_{\beta}\}}_{j,k,\beta}$
is independent over $H_1$.
Changing names to the second entries in the Lie bracket we have
$$
\sum_{j=1}^{n_{1}}[\nuw{j}, w_j]=\sum_{j=1}^{n_0}[\nux{j}, x_j]+\sum_{k=1}^{r_{0}+1}[\lay{k},y_k]
+\sum_{\beta}[\nu_{\beta},z_{\beta}].
$$
Repeated application of this procedure will lead to
$$
\sum_{j=1}^{n_{1}}[\nuw{j}, w_j]=\sum_{j=1}^{n}[\nux{j}, x_j]+\sum_{k=1}^{r}[\lay{k},y_k]
$$
with ${\{x_{j},y_{k}\}}_{j,k}$ independent over $H_{1}$, ${\{\nux{j}\}}_{j}$ independent over $\exs H_{1}$ and
${\{\lay{k}\}}_{k}$ independent inside $N(H)$.

To get \pref{sangennaro} we have to modify the set ${\{\nuu{i},\nux{j}\}}_{i,j}$ to get an independent
(over $\exs H_{1}$) one; this would possibly reduce the length $m_{1}$ of the first segment of \pref{mustiola}.
Assume we have $\nuu{t+1},\dots,\nuu{m},\nux{1},\dots,\nux{n}$ independent over $\exs H_{1}$ for some
$t<m$, but
$\nuu{t},\dots,\nuu{m},\nux{1},\dots,\nux{n}$ is not.

This imply $\nuu{t}=\summ{i>t}{\theta_{i}}{\nuu{i}}+\sum_{j=1}^{n}\gamma_{j}\nux{j}+h$ where
$h\in N(H)$ and some of the $\theta_{i}$ or $\gamma_{j}$ may be trivial.

Now as $[h,u_{t}]$ lays in $\jei{H}$ we have
$$
[\nuu{t},u_{t}]=\sum_{i>t}[\nuu{i},\theta_{i}u_{t}]+\sum_{j=1}^{n}[\nux{j},\gamma_{j}u_{t}]
$$
and
\begin{multline*}
w=\sum_{i=1}^{m_{1}}[\nuu{i}, u_i]+\sum_{j=1}^n[\nux{j}, x_j]+\sum_{k=1}^r[\lay{k},y_k]=\\
=\sum_{\substack{i=1\\ i\neq t}}^{m_{1}}
[\nuu{i},u_{i}+\theta_{i}u_{t}]+\sum_{j=1}^{n}[\nux{j},x_{j}+\gamma_{j}u_{t}]+\sum_{k=1}^r[\lay{k},y_k].
\end{multline*}
In this last sum $\theta_{i}=0$ for each $i<t$ and the elements $u_{i}+\theta_{i}u_{t}$ are still independent
in $H_{1}$,
moreover one checks that the set ${\{x_{j}+\gamma_{j}u_{t},y_{k}\}}_{j,k}$ is still independent over $H_{1}$.

Repeating this last argument and renaming the terms of weight $1$, we reach an expression with properties required for sum \pref{sangennaro}.
%iterating this process we get to a form of type \pref{santarita}.

\medskip
Now choose an ordering for a base $X_1$ of $M_1$ according to the following rule:$$
X_1=\{\bu>\hu>X>Y>\vu\}
$$
with $\bu=(u_i)$, $\hu$ a completion of $\bu$ to a basis of $H_1$, so that $\xoh=\bu\hu$.
Moreover $X=(x_j)$, $Y=(y_k)$ and $\vu$ is a base-completion of $\bu\hu X Y$ to $M_1$.
Each subpart of $X_{1}$ is arbitrarily ordered.

Now write elements $\nuu{i}$, $\nux{j}$ and $\lay{k}$ as an $\Fp$-linear combination
of \emph{weight 2}-basic monomials with respect to the basis $X_1$, according to the ordering chosen above.
As a result, the sum \pref{santarita} appears as a linear combination
\begin{labeq}{santarita}
\summ{\al}{\theta_{\al}}{[a_{\al},b_{\al},z_{\al}]}
\end{labeq}
of \emph{weight 3}-simple commutators of the form $\trec{a_{\al}}{b_{\al}}{z_{\al}}=\trec{a}{b}{z}=[[a,b],z]$,
where $a,b\in X_1$, $[a,b]$ is basic with $a>b$ and $z$ necessarily lays in $\bu X Y$.

If in addition $z\geq b$, the term $\trec{a}{b}{z}$ is a basic monomial, while, on the contrary, if $a>b>z$ the
commutator $\trec{a}{b}{z}$ will be called a \emph{prebasic monomial}.

Every prebasic monomial $\trec{a}{b}{z}$ can be transformed in the sum of two basic commutators, that's worked out applying Jacobi identity, namely
\begin{labeq}{prebi}
\trec{a}{b}{z}=\trec{a}{z}{b}-\trec{b}{z}{a}.
\end{labeq}

As an element of $\fla{3}{\xoh}\inn\fla{3}{X_1}$, $w$ admits also a unique expression $B^H$ via basic monomials over $\xoh=\bu\hu$ of
weight $3$. We briefly write
\begin{labeq}{basipre}
B^H=w=B+pB=B+B_*
\end{labeq}
where $B$, $pB$ are sums of respectively basic, prebasic commutators over $X_1$ representing $\pref{santarita}$
and $B_*$ is a sum of basic monomials over $X_{1}$ arising from $pB$ by means of substitutions \pref{prebi}.

Here, by abuse, $B$, $pB$ and $B_{*}$ denote also the sets of basic commutators contained in the
corresponding sum.

From a comparison of equality $B^{H}=B+B_{*}$ %in $\pref{basipre}$
and by uniqueness condition in Hall theorem, it follows that 
$B^{H}\inn BB_{*}$ and
%in $B+B_*$ will be cancelled all the terms that do not appear in $B^H$.
each addendum in $B+B_{*}$ which is not in $B^{H}$, must be cancelled by the same commutator with
opposite sign (read coefficient), the latter laying again in $BB_{*}$.
In particular all the basic terms containing elements
of $X_1$ which are not in $\xoh$ will be erased from $B+B_*$.

Assume a term $\trec{a}{b}{z}$ appearing in \pref{santarita} as a $B$-element is to be cancelled,
then the same commutator, with opposite sign will be necessarily found in $B_{*}$ and not of course in $B$ again. Here is to be noticed the role of $X_{1}$-elements appearing in the second entries
of the Lie multiplication, this forces sum \pref{santarita} to be
grouped after labels $u_{i}$, $x_{j}$ and $y_{k}$.\label{wahnsinn1}

Analogously, if a basic monomial appears in the sum $B_{*}$, and is not part of $B^{H}$, only but a $B$-term can do the job of canceling it. A basic monomial in $B_{*}$ comes from a unique prebasic commutator of $pB$,
cancellations among $B_{*}$ part only will not take place.\label{wahnsinn2}

\begin{itemize}
\punto{Claim 1}Monomials $\trec{a}{b}{z}$ appearing in expression \pref{santarita} do not contain any element from $\vu$.
\end{itemize}
Assume on the contrary, \pref{santarita} contains a commutator $\trec{a}{v}{z}$ with $v\in\vu$, $z\in\bu X$.
Then necessarily $z\geq v$ and $\trec{a}{v}{z}$ is basic. It follows, monomials (partially) supported on $\vu$
cannot appear in $pB$, and then $B_{*}$-basic terms will not contain $\vu$-elements.
We conclude, there is no hope for  $\trec{a}{v}{z}$ to be cancelled from $B$. Which should be quatsch.
%What is asserted above tells us that there's no hope of cancelling these terms.

In particular, elements $\nuu{i}$, $\nux{j}$ are supported uniquely on $\bu\hu XY$.

%\medskip
\begin{itemize}
\punto{Claim 2}The sum \pref{sangennaro} contains terms $[\nu, z]$ with $z\in XY$.
Concisely $n+r\neq0$. Moreover $m\neq0$.
\end{itemize}
%Clearly $m\neq0$.
Assume $w=\sum_{i=1}^m[\nuu{i}, u_i]$ only. As $\nuu{i}$ are independent over $\exs H_1$, for each $i$, exists a generic term $\trec{a}{b}{u_i}$ with $b\in XY$. It follows such a  term
%$\trec{a}{b}{u_i}$
is basic, as $\bu>X>Y$ and does not belong to $B^{H}$. Now the same
argument used to prove (Claim 1) applies: such terms cannot be cancelled from part $B$ of \pref{basipre}
because there is no counterpart in $B_{*}$.

Finally $m\neq0$, for if in $B+B_*$ every monomial contains an element from $XY$, nothing
remains after cancellation of non-$\xoh$ terms. 

\begin{itemize}
\punto{Claim 3}Elements $\lay{k}$ in \pref{santarita} are supported uniquely on $\bu$.
\end{itemize}
Assume not. Then in the sum appears a term $\trec{a}{b}{y_k}$ with $\{a,b\}\inn\xoh\non\bu$.
As $a>b$ and $\bu>\hu>Y$, it follows necessarily
$b\in\hu$ and $\trec{a}{b}{y_k}$ is prebasic, its transformation in two $B_*$-elements produces basic terms
$\trec{a}{y_k}{b}$ and $\trec{b}{y_k}{a}$, both not in $B^{H}$.
The commutator $\trec{a}{y_k}{b}$ cannot be found in part $B$ of \pref{basipre} and will not be cancelled. Absurd.

\medskip
We prove the lemma contradicting  selfsufficiency of $H_1$ in $M_1$.
Take the space $C=\spn(H_1,X,Y)$ of $M_{1}$ above $H_{1}$, then $\dim_{\Fp}(C)=\dim_{\Fp}(H_1)+n+r$.

We get a contradiction if we find more than $n+r$ \emph{new relations} over $C$, i.e. elements
of $N$ which are supported on $\xoh XY$ and are independent over $\exs H_{1}$.
Natural candidates are the $\nuu{i}$ and $\nux{j}$.

We have actually to check that they are enough. Our axioms on $\delta$ imply $r<m$ as $\delta(\gen{u_1,\dots,u_m})>0$ and $\lay{1},\dots,\lay{r}$ are over $\bu$.

Therefore we have $r+n<m+n$ and $\{\nuu{i},\nux{j}\}$ are $m+n$ indepentent relations over only $r+n$ \emph{new} generators, namely $\{x_{i},y_{j}\}$. In other words we have $\delta(C)<\delta(H_1)$ which is not possible,
as $H_{1}$ is selfsufficient in $M_{1}$.

If $n=0$, we still have $r<m$, then take $C=\spn(H_1,Y)$. This yields $r$ \emph{new} generators but $m$ \emph{new} relations over $C$, namely $\nuu{1},\dots,\nuu{m}$.  

Case $r=0$ simply gives $n<n+m$, then repeat previous arguments with $C=\spn(H_{1},X)$.
\end{proof}
%----------------------------------------------------------------------------------------------------
\begin{lem}\label{bellemmino}
Seien $H,\,K\inn M$ aus $\nla{2}$ mit $H=\gena{H_{1}}{L}$ und $K=\gena{K_{1}}{L}$ f\"ur starke \"Unterraume $H_{1}$ und $K_{1}$ von $M_{1}$


Let now $U$ and $V$ be two selfsufficient substructure of finite linear dimension in $M\in\nla{2}$, such that $U\cap V=\gena{U_{1}\cap
V_{1}}{M}$.
We have % $\fr{3}(U\cap V)=\fr{3}(U)\cap\fr{3}(V)$.
$\gena{U_{1}\cap V_{1}}{\fr{3}M}=\gena{U_{1}}{\fr{3}M}\cap\gena{V_{1}}{\fr{3}M}$.
\end{lem}
\begin{proof}
%%On account of the previous lemma, being $U$ and $V$ strong spaces in $M$, $\fr{3}(U)$ and $\fr{3}(V)$ are
%both subalgebras of $\fr{3}M$, therefore $\fr{3}U\cap\fr{3}V$ is still a subalgebra of $\fr{3}M$.

Let $\xou$ and $\xov$ be bases of $U_{1}$ and of $V_{1}$ respectively, so that $\xou\cap\xov$ is a base for $U_{1}\cap V_{1}$, moreover let $X_{1}$ be a base of $M_{1}$ which contains both $\xou$ and $\xov$.

Let $\bbj=\jei{M}$. Consider the canonical projection $\pj$ of $\fla{3}{X_{1}}$ modulo $\bbj$, and set $\chi=
\res{\pj}{\fla{3}{\xou\cap\xov}}$.

Because $\fla{3}{\xou\cap\xov}=\fla{3}{\xou}\cap\fla{3}{\xov}$, we can check $\chi$ is onto
%$\fr{3}{U}\cap\fr{3}{V}=
$\gena{U_{1}}{\fr{3}M}\cap\gena{V_{1}}{\fr{3}M}$  {\bf(CHECK THIS)}.

As $\tr{2}(\gena{U_{1}}{\fr{3}M}\cap\gena{V_{1}}{\fr{3}M})=U\cap V$, applying the universal property of $\fr{3}$ we obtain an epimorphism $\pi$.

Repeating the arguments that led to diagram \pref{lollipop} we find the following
%\begin{labeq}{lollipop1}\gena{U_{1}}{\fr{3}M}\cap\gena{V_{1}}{\fr{3}M}
$$
\xymatrix{&\fla{3}{\xou\cap\xov}\ar[dl]_{\pi_{\jei{U\cap V}}}\ar[dr]^{\chi}\ar@{-}[r]^{i}&\fla{3}{X_1}\ar[dr]^{\pj}\\
\fr{3}{U\cap V}\ar[rr]^{\pi}&&\gena{U_{1}}{\fr{3}M}\cap\gena{V_{1}}{\fr{3}M}\ar@{-}[r]^{i}&\fr{3}{M}}
%\end{labeq}
$$
and again we get $\ker\chi=\bbj\cap\fla{3}{\xou\cap\xov}$ and
$$
\ker\pi=(\bbj\cap\fla{3}{\xou\cap\xov})\quot\jei{U\cap V}.
$$
On one side we have, by Lemma \ref{bellemma}, $\fr{3}(U\cap V)\simeq\gena{U_{1}\cap V_{1}}{\fr{3}M}$.

With the very same arguments of Lemma \ref{bellemma}, we conclude that if $\ker\pi$ is not trivial, then $U_{1}
\cap V_{1}$ is not selfsufficient in $M_{1}$ which is a contradiction.
%Because $U\cap V=\gena{U_{1}\cap V_{1}}{M}=\gena{\xou\cap\xov}{M}$, from now on we proceed just like in
%lemma \ref{bellemma} and we find that $U\cap V$ is not a selfsufficient space of $M$, which is obviously a
%contradiction.
\end{proof}
%If we consider the images of $\fr{3}U$ and $\fr{3}V$ inside $\fr{3}$, from the previous Lemma we
%have
%$$\gena{U_{1}}{\fr{3}M}\cap\gena{V_{1}}{\fr{3}M}=\gena{U_{1}\cap V_{1}}{\fr{3}M}.$$