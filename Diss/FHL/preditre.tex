For the rest of the chapter we assume a prime $p$ has been fixed, greater than $3$.
We will consider Lie algebras of $\nla{3}$.

\smallskip
For a chosen $M$ in $\nla{3}$ and any $\nla{3}$-subalgebra $A$ of $M$,
the truncation $A_{*}$ of Section \ref{freelift}, is $\nla{2}$-isomorphic to $\gena{A_{1}}{M_{*}}$. These algebras will be identified in the
sequel.

This means, we can apply on $M_{1}$ the ``pregeometric machinery'' introduced in Chapter \ref{due},
associated to $M_{*}$.
With this purpose we rename by $\delta_{2}$, the nil-$2$ deficiency $\delta$ of Definition \ref{deficienzwei} and set, for finite subspaces
$A_{1}$ of $M_{1}$
$$\delta_{2}(A_{1})=\dfp(A_{1})-\rd(A_{*}).$$
Following our previous convention, the same integer will be equal to $\delta_{2}(A)$ for ease of notation.

On the same line, for any $\nla{3}$-extension $M$ of $N$, we will
write $N\zsu[2]{}M$ if $N_{*}$ is self-sufficient in $M_{*}$ according to Definition \ref{2strong}.
The same meaning is attributed to the expression $N_{1}\zsu[2]{}M_{1}$.

Moreover, for a given $M$ in $\nla{3}$, we denote by $d_{2}^{M}$ the
dimension function on $M_{1}$ obtained by the predimension $\delta_{2}$.

\medskip
On the other hand for an $\nla{3}$-algebra $A$, we have from \pref{morphipres} of Section \ref{maxrels} above,
the following {\em lifted} presentation:
\begin{labeq}{liftpres}
\rt(A)\lto\fl{A_{*}}\stackrel{\pam{A}}{\lto}A
\end{labeq}
and in particular
$$A\simeq_{\nla{3}}\frac{\fl{A_{*}}}{\rt(A)}=A_{*}\oplus\frac{(\fl{A_{*}})_{3}}{\rt(A)}.$$

This yields a new integer invariant attached to $\nla{3}$-objects, defined in the following
\begin{dfn}
For a finitely generated $\nla{3}$-algebra $A$ we define the
$\nla{3}$-deficiency as the integer
\begin{labeq}{deltre}
\delta_{3}(A)=\delta_{2}(A)-\dfp\left(\rt(A)\right).
\end{labeq}
%where $\rt(A)$ is the kernel of the canonical map $\map{\pam{A}}{\fl{A_{*}}}{A}$ defined in \pref{morphipres}.
In particular $\delta_{3}(A)$ depends only of the quantifier-free $\Lan{3}$-diagram
of $A$. As a consequence a possible lower bound to $\delta_{3}$ in $M_{1}$ is axiomatisable via $\Lan{3}$-sentences.
\end{dfn}

\smallskip
In the scope of section \ref{schur}, if we compare \pref{deltre} above with \pref{LieDef} on page \pageref{LieDef},
we obtain the same thing, just
differently organised. In fact if we consider \pref{LieSchur} we get
$$\dfp(H_{2}(A,\nla{3}))=\dfp(\rd(A_{*}))+\dfp(\rt(A)).$$

\medskip
This said, in view of a definition of self-sufficiency for $\nla{3}$-extensions, we need a different
notion of predimension, which emulates the local property \pref{erredi} of $\rd$ and ease computations for a future
notion of free $\nla{3}$-amalgam.

The point here is that for an arbitrary $A\inn M$ like above, $\rt(A)$ is not in general a subspace of $\rt(M)$.

\smallskip
Take an $\nla{3}$-inclusion $i\colon H\inn M$, this means as usual $H=\gena{H_{1}}{M}$ for $H_{1}\inn M_{1}$ and consider
the truncation to $\nla{2}$, $i_{*}\colon H_{*}\inn M_{*}$.
We denote by $\gam{H}{M}$ the map described in \pref{gam}
\begin{labeq}{gammalift}
\lmap{\gam{H}{M}:=\frl(i_{*})}{\fl{H_{*}}}{\fl{M_{*}}}.
\end{labeq}
By \pref{liftbild} and \pref{kerfli}, %if $\kerg{H}{M}$ denotes the kernel of $\gam{H}{M}$,
we have
\begin{labeq}{immaker}
\im(\gam{H}{M})=\gena{H_{1}}{\fl{M_{*}}}\quad\text{and}\quad%\kerg{H}{M}
\ker(\gam{H}{M})=\frac{\fla{3}{H_{1}}\cap\J(M_{*})}{\J(H_{*})}
\end{labeq}
while by Proposition \ref{bellemma} we obtain for all $\nla{3}$-subalgebras $H\inn M$ as above
\begin{cor}\label{corembel}
If $H\zsu[2]{}M$ then $\map{\gam{H}{M}}{\fl{H_{*}}}{\fl{M_{*}}}$ is injective.
\end{cor}

We want also, the new presentation obtained in \pref{liftpres} to interact with subalgebras, that is
\begin{lem}\label{gammap}
For an $\nla{3}$-subalgebra $H$ of $M$, we obtain the following commutative diagram with exact rows.
\begin{labeq}{liftpresmorph}
\begin{split}
\xymatrix@!C{
\rt(H)\ar^{\gam{H}{M}}[d]\ar[r]&\fl{H_{*}}\ar^{\gam{H}{M}}[d]\ar^{\pam{H}}[r]&\extracolsep{3cm} H\ar^{i}[d]\\
\rt(M)\ar[r]&\fl{M_{*}}\ar^{\pam{M}}[r]&\extracolsep{3cm}M}
\end{split}
\end{labeq}
In particular $\ker(\gam{H}{M})\inn\rt(H)$.
%\uwave{In general for a given $M$ we may consider the directed system $(\fl{H_{*}},\gam{H}{K}\mid H_{1}\inn K_{1}\inn M_{1})$.}
\end{lem}
\begin{proof}
We show that the rightmost square in \pref{liftpresmorph} commutes. This follows by proposition \ref{morphifreelift}
and the fact $(\gam{H}{M}\pam{M})\ast=\ast\,i_{*}=(\pam{H}i)\ast$ applied to the diagrams
$$\xymatrix@C+7mm{
\fl{H_{*}}\ar^{\ast}[d]\ar^{\gam{H}{M}\pam{M}}@<+3pt>[r]\ar_{\pam{H}i}@<-3pt>[r]&M\ar^{\ast}[d]\\
H_{*}\ar_{i_{*}}[r]&M_{*}}$$
\end{proof}

This allows us to define a more {\em adaptive} deficiency, which depends
of the embedding in the ambient structure $M$.
\begin{dfn}\label{ded}
Let $M$ be an $\nla{3}$-algebra. For any $H_{1}\inn M_{1}$. We set
$$\rt_{M}(H)=\rt_{M}(H_{1})=\gam{H}{M}\left(\rt(H)\right)$$
%as the space of the {\em relative relators} for $H$ in $M$.
and define for finitely generated $H$% the {\em relative deficiency} as
\begin{labeq}{dedef}
\ded^{M}(H)=d^{M}_{2}(H_{1})-\dfp(\rt_{M}(H))
\end{labeq}
and for any $\nla{3}$-subalgebra $N$ of $M$ and finite $H_{1}\inn M_{1}$
\begin{labeq}{reldedef}
\ded^{M}(H/N)=d^{M}_{2}(H/N)-\dfp(\rt_{M}(H/N))
\end{labeq}
where
$\rt_{M}(H/N)$ is the quotient $\Fp$-vector space $%\frac{
\rt_{M}(N+H)/\rt_{M}(N)$.
As before for $N+H$ is meant $\gena{N_{1}+H_{1}}{M}$ and in general any of the expressions above are allowed to carry
indices $_{1}$. In fact as in the nil-$2$ case,
we are searching for a notion of predimension -- and eventually a pregeometry -- which is concerned with sets from the {\em sort}
$M_{1}$. 
\end{dfn}
\begin{rem}\label{vecchialenza}
By \pref{immaker} and \pref{liftpresmorph} we have
\begin{itemize}
\item[1.]for $H_{1}\inn N_{1}\inn M_{1}$ we have
\begin{labeq}{dedim}
\rt_{M}(H)=\rt(M)\cap\im(\gam{H}{M})=\rt(M)\cap\gena{H_{1}}{\fl M}
\end{labeq}
\item[2.]since $\gam{H}{N}\gam{N}{M}=\gam{H}{M}$, $\rt_{N}(H_{1})$ maps {\em onto} $\rt_{M}(H_{1})$ via $\gam{N}{M}$. 
\end{itemize}
\end{rem}
In particular we obtain a form of the local relators, which is a lot similar to $\rd_{M}(H)$ in \pref{erredi} of Chapter \ref{due}.

\medskip
Assume $A$ is a finite $\nla{3}$-subalgebra of $M$ with $A\zsu[2]{}M$, by Corollary \ref{corembel} above follows,
that $\rt(A)\simeq_{\Fp}\rt_{M}(A)$ and hence, as $d_{2}^{M}(A)=\delta_{2}(A)$, we have $\delta_{3}(A)=\ded^{M}(A)$.
In particular, by Lemma \ref{samed2} and Remark \ref{vecchialenza}.(2.)\,we have.
\begin{lem}\label{sameded}
For a given $\nla{3}$-algebra $M$, the integers $\ded^{M}(A)$ and $\delta_{3}(A)$ do coincide on all finitely generated $\nla{3}$-subalgebras $A$ of $M$ when $A_{*}$ is self-sufficient in $M_{*}$ with respect to $\delta_{2}$.

Moreover $\ded^{M}$ coincides with $\ded^{N}$ on the subspaces of $N_{1}$, for all $\delta_{2}$-strong extensions $N\zsu[2]{}M$.
\end{lem}
In Section \ref{classetre} below we actually show that $\delta_{3}$ and $\ded$ are always comparable, in the direction
$\ded^{M}\leq\delta_{3}$ for all $M$.

\smallskip
It is worth to mention here, that for a given $M$ and subspaces $H_{1}$, $K_{1}$ of $M_{1}$, it is not in general the case that
\begin{labeq}{notmodular}
\gena{H_{1}}{\fl M}\cap\gena{K_{1}}{\fl M}\quad\text{and}\quad\gena{H_{1}\cap K_{1}}{\fl M}
\end{labeq}
coincide.

It is also not true in general
that $\rt_{M}(H_{1})\cap\rt_{M}(K)$ equals $\rt_{M}(H_{1}\cap K_{1})$ and the analogous of {\em submodularity} \pref{submod}
of Section \ref{sec:deltadue} for
$\delta_{3}$ and $\ded$ may fail. In fact we have
\begin{rem}\label{nossummo}
$\ded^{M}$ and $\delta_{3}$ are not in general submodular.
\end{rem}
\begin{proof}
Consider the $\nla{3}$-algebra given by the presentation $M=\gen{M_{1}\mid R}$, where $M_{1}$ is the vector space over $\Fp$ with
basis $\{a,b_{1},\dots,b_{4},m_{1},m_{2}\}$ and $R$ is the ideal of $\fla{3}{M_{1}}$ generated by the relators
\begin{gather}
\begin{split}
\rho=[a,b_{1}]-[b_{2},b_{3}]
\end{split}\qquad
\begin{split}
\alpha=&[b_{2},b_{3},b_{4}]-[m_{1},m_{2},m_{2}],\\
\beta=&[b_{2},b_{3},m_{1}]-[b_{4},m_{2},m_{2}].
\end{split}
\end{gather}
In this algebra, $\J(M)$ is the ideal of $\fla{3}{M_{1}}$ generated by $\rho$.

Set now $N_{1}$ the subspace of $M_{1}$ generated by $b_{1},\dots,b_{4},m_{1},m_{2}$ and $E_{1}=\genp{a,b_{1},b_{4},m_{1},
m_{2}}$. We have $\alpha\equiv[a,b_{1},b_{4}]-[m_{1},m_{2},m_{2}]$ and $\beta\equiv[a,b_{1},m_{1}]-[b_{4},m_{2},m_{2}]$ modulo
$\J(M)$, and hence $\bar\alpha\in\rt_{M}(E_{1})\non\rt_{M}(E_{1}\cap N_{1})$.

As $\gen{N_{1},a}=M$, we have $d_{2}^{M}(E/N)=d_{2}^{M}(a/N)=\delta_{2}(a/N)=0$ while $d_{2}^{M}(E_{1}/E_{1}\cap N_{1})=
\delta_{2}(E_{1}/E_{1}\cap N_{1})=1$. This means
$$0=\ded^{M}(E/N)>\ded^{M}(E_{1}/E_{1}\cap N_{1})=-1$$
and, as $E,N\zsu[2]{}M$, $\delta_{3}(E_{1}+N_{1})+\delta_{3}(E_{1}\cap N_{1})>\delta_{3}(E)+\delta_{3}(N)$.
\end{proof}

\medskip
We now define self-sufficiency on extensions of $\nla{3}$-algebras.
\begin{dfn}\label{trestrongness}
We say that an $\nla{3}$-subalgebra $H$ of $M$ is self-sufficient and write $H\zsu[3]{}M$ if
\begin{itemize}
\item[-] $H\zsu[2]{}M$ and
\item[-] $\ded^{M}(E\quot H)\geq0$ for all finite subspaces $E_{1}\inn M_{1}$.
\end{itemize}
By the first condition, and the following Lemma \ref{deltaded}, it is possible to express
$H\dsu M$ in terms of $\delta_{3}$ as well\footnote{
One defines of course $\delta_{3}(E/H)$ as $\delta_{2}(E/H)-\dfp(\rt(H+E)/\rt(H))$.}.
For a finite $H$ -- say -- this property is
actually part of the elementary type of $H$.
\end{dfn}