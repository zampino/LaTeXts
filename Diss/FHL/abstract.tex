\selectlanguage{ngerman}
\xsubsection{\abstractname}
\medskip
\begin{minipage}{.9\textwidth}
\setlength{\parindent}{2ex}
In dieser Arbeit wird das Fra\"iss\'e-Hrushowskis Amalgamationsverfahren in Zusammenhang
%mit nilpotenten Gruppen von endlichem Exponent beziehungsweise
mit nilpotenten graduierten Lie Algebren \"uber einem endlichen K\"orper untersucht.

Die Pr\"adimensionen die in der Konstruktion auftauchen sind mit dem gruppentheoretischen Begriff der {\em Defizienz}
zu vergleichen, welche auf homologische Methoden zur\"uckgef\"uhrt werden kann.

Dar�ber hinaus wird die Magnus-Lazardsche Korrespondenz zwischen den oben genannten Lie Algebren
und nilpotenten Gruppen von Primzahl-Exponenten beschrieben.
Dabei werden solche Gruppen durch die Baker-Haussdorfsche Formel
in den entsprechenden Algebren definierbar interpretiert.

\medskip
Es wird eine $\omega$-stabile Lie Algebra von Nilpotenzklasse 2 und Morleyrang $\omega\cdot2$ erhalten, indem
man eine {\em unkollabierte} Version der von Baudisch konstruierten {\em new uncountably categorical group} betrachtet.
Diese wird genau analysiert. Unter anderem wird die Unabh\"angigkeitsrelation des Nicht-Gabelns durch die
Konfiguration des freien Amalgams charakterisiert.
%Ziel dieser Arbeit ist eine Erweiterung auf h\"oheren Nilpotenzklassen der von Baudisch
%konstruierten nil-2 {\sl } von prim Exponent.

\smallskip
Mittels eines induktiven Ansatzes werden die Grundlagen entwickelt, um neue Pr\"adimensionen f\"ur Lie Algebren der Nilpotenzklassen
gr\"o\ss er als zwei zu schaffen.
%F\"ur den nil-3 Fall geben wir eine Notion einer {\em selbstgen\"ugende} Erweiterung;
%damit wird ein erstes Amalgamationslemma bewiesen.

Dies erweist sich als wesentlich schwieriger als im Fall 2.
Wir konzentrieren uns daher auf die Nilpotenzklasse 3, als Induktionsbasis des oben genannten Prozesses.

In diesem Fall wird die Invariante der Defizienz
auf endlich erzeugte Lie Algebren adaptiert. Erstes Hauptergebnis der
Arbeit ist der Nachweis dass diese Definition zu einem vern\"uftigen Begriff selbst-gen\"ugender
Erweiterungen von Lie Algebren f\"uhrt und
sehr nah einer gew\"unschten Pr\"adimension im Hrushovskischen Sinn ist.

Wir zeigen -- als zweites Hauptergebnis -- ein erstes Amalgamationslemma bez\"uglich
selbst-gen\"ugender Einbettungen.
\end{minipage}
\newpage
\thispagestyle{empty}
\cleardoublepage
\selectlanguage{english}
\xsubsection{\abstractname}
\medskip
\begin{minipage}{.9\textwidth}
\setlength{\parindent}{2ex}
In this work, the so called Fra\"iss\'e-Hrushowski amalgamation is applied to nilpotent graded Lie algebras
over the $p$-elements field with $p$ a prime. We are mainly concerned
with the {\em uncollapsed} version of the original process.

The predimension used in the construction is compared with
the group theoretical notion of {\em deficiency}, arising from group Homology.

We also describe in detail the Magnus-Lazard correspondence, to switch between the aforementioned Lie algebras
and nilpotent groups of prime exponent.
In this context, the Baker-Hausdorff formula allows such groups to be definably interpreted in the corresponding
algebras.

Starting from the structures 
which led to Baudisch' {\em new uncountably categorical group}, %Fra\"iss\'e-%
%Hrushovski amalgamation is first applied to a suitable class of
%graded 2-nilpotent Lie algebras over the finite field $\Fp$, for $p$ prime.
%for a class of nilpotent graded Lie algebras in higher nilpotency classes
%to support an analogous construction.
%We limit ourself to the {\em uncollapsed} phase of the procedure and obtain
we obtain an $\omega$-stable Lie algebra of nilpotency class 2,
as the countable rich Fra\"iss\'e limit of a suitable class of finite algebras over $\Fp$.

We study the theory of this structure in detail: we show its Morley rank is $\omega\cdot2$ and
a complete description of non-forking independence is given, in terms of free amalgams.

\smallskip
In a second part, we develop a new framework for the construction of
deficiency-predimensions among graded Lie algebras of nilpotency class
higher than $2$. This turns out to be considerably harder
than the previous case. The nil-3 case in particular has been extensively treated, as
the starting point of an inductive procedure.

In this nilpotency class, our main results concern a suitable deficiency function, which behaves
for many aspects like a Hrushovski predimension.
A related notion of {\em self-sufficient} extension is given.

We also prove a first amalgamation lemma with respect to self-sufficient embeddings.
\smallskip
\end{minipage}
\newpage
\thispagestyle{empty}
\cleardoublepage