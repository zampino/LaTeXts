\section{Combinatorial Pregeometries}\label{qdim}
If $M$ is a set, $\ps{M}$ denotes its powerset.
To denote unions of sets, juxtaposition will almost everywhere replace the symbol $\cup$ in the sequel, so that
$AB$ will mean $A\cup B$ and $Ab$ will be $A\cup\{b\}$ for all sets $A,B$ and elements $b$, of $M$.
%\end{presection}

%\cbstart
%\begin{dfn}
%A {\em closure operator}\mn{define a pregeometry directly!!!} $\cl$ on the set $M$, is a map
%$\map{\cl}{\ps{M}}{\ps{M}}$ which satisfies the following properties
%\begin{itemize}
%\punto{cl1}$A\inn\cl(A)$ for all $A\in\ps{M}$
%\punto{cl2}$\cl\circ\cl=\cl$
%%\punto{cl3}$\cl A\inn\cl B$ whenever $A\inn B$
%\punto{cl3}$\cl(B)\inn\cl(A)$ whenever $A\nni B$.
%%\cl(A)$ is the union of all $\cl(A^{\prime})$, where the $A^{\prime}$ range over $\fp{M}$
%\end{itemize}
%
%We say that the closure operator $\cl$ on $M$, has the {\em exchange property} if in addition
%\begin{itemize} 
%\punto{ex}For all $a,b\in M$ and all $A\in\ps{M}$, when $a\in\cl(Ab)\non\cl(A)$, then $b\in\cl(Aa)$
%\end{itemize}
%while $\cl$ is said a {\em finitary} closure operator if for all sets $A$ in $M$
%\begin{itemize}
%\punto{fin}$\cl(A)$ is the union of all $\cl(A^{\prime})$, where the $A^{\prime}$ range over the finite parts of $A$.
%\end{itemize}
%\end{dfn}
%\cbend
\begin{dfn}\label{pregdef}
A \emph{pr{\ae}geometry} %(or a {\em combinatorial matroid})
$(M,\cl)$ is a %couple $(M,\cl)$
set $M$ endowed with a closure operator $\map{\cl}{\ps{M}}{\ps{M}}$ on $M$,
which satisfies the {\em Steiniz exchange property}. This means the following properties are required of $\cl$:
\begin{itemize}
\punto{cl1}$A\inn\cl(A)$ for all $A\in\ps{M}$
\punto{cl2}$\cl\circ\cl=\cl$
\punto{fin}$\cl(A)$ is the union of all $\cl(B)$, where the $B$'s range over the finite parts of $A$.
\punto{ex}For all $a,b\in M$ and all $A\in\ps{M}$, when $a\in\cl(Ab)\non\cl(A)$, then $b\in\cl(Aa)$
\end{itemize}

If in addition $\cl(\vac)=\vac$ and $\cl(a):=\cl(\{a\})=\{a\}$ for all singletons $a\in M$, we say
that $(M,\cl)$ is a {\em geometry}. 
\end{dfn}

Note that (fin) alone implies monotonicity %(cl3)
of the closure operator $\cl$. A {\em closed set} of $M$ is defined, as usual, as a fixed point of $\cl$.
%Property (ex) is referred to as the {\em Steiniz exchange property}.
%The approach of J.B. Goode in \cite{JBG} is rather to introduce a (family of) relations between elements
%and finite sets
%``$cis(x,\bar y)$'' of {\em ciscendency} to express ``$x\in\cl(\bar y)$'', satisfying
%the above conditions. That context is actually the most (\dots) to the statement of lemma \ref{} below
%
%Again Goode refers these notions to be first considered by Van der Waerden and Krasner.

\medskip
From a pregeometry $(M,\cl)$ we obtain a geometry $(M_{*},\cl_{*})$ if we define $M_{*}$ as $(M\non\cl(\vac))/_{\sim}$
for $a\sim b\iff\cl(a)=\cl(b)$, and $\cl_{*}(A/_{\sim})$ to be $\cl(A)/_{\sim}$.
This procedure is exactly the way a projective space is obtained out of a vector space:
each line is identified to a point.

\medskip
If $(M,\cl)$ is a pregeometry and $B$ a subset of $M$ we define its {\em localisation at $B$}
as the pregeometry $(M,\cl_{B})$ given by $\cl_{B}(U):=\cl(BU)$ for all subsets $U\inn M$.

\medskip
We say that the subset $A$ of $M$ is {\em independent over} $B$
(or $B$-{\em independent}) if $a\notin\cl_{B}(A\non\{a\})$ for all $a\in A$. 

We say that a subset $C$ of $A\inn M$ is a {\em base for $A$ over $B$}, if it is independent over $B$ and $A\inn\cl_{B}(C)$.
The definition of an independent set or of a base of a set are obtained if we put $B$ above to be the empty set.

By the exchange property, given any set $A$, a maximal $B$-independent subset of $A$ is a base over $B$. Moreover
all bases have the same cardinality\footnote{Exchange property is essentially needed to prove that {\em finite} bases have all the
same size.}, which is defined as the {\em dimension of $A$ over $B$} and denoted
with $\dim(A/B)$. This (ordinal) number satisfies the following additivity property
\begin{labeq}{dimadd}
\dim(AB)=\dim(B)+\dim(A/B)
\end{labeq}
for any sets $A$ and $B$.

We may also say that a set $D$ is independent of $C$ over $B$ if $\dim(D/B)=\dim(D/CB)$.
\begin{dfn}
A pregeometry $(M,\cl)$ is {\em trivial} or {\em disintegrated} if for any sets $A,B\inn M$ we have $\cl(AB)=\cl(A)\cup\cl(B)$.

We say that a pregeometry $(M,\cl)$ is {\em modular} if for all closed sets $A$ and $B$, we have
$\dim(A/B)=\dim(A/A\cap B)$.

A pregeometry $(M,\cl)$ is {\em locally modular} if the above equality is true whenever $\dim(A\cap B)>0$
or equivalently if $(M,\cl_{\{a\}})$ is modular for all $a\in M$.
%\uwave{for some finite set $A$ of $M$}.\mn{\bf check def!!}
\end{dfn}
Remark that a trivial pregeometry is always modular, and that a modular geometry is also
locally modular. Moreover a pregeometry is modular exactly if any closed set $A$ is independent of any closed $B$
over their intersection and also iff the following equality holds on finite-dimensional closed sets $A,B$
\begin{labeq}{mod}
\dim(AB)+\dim(A\cap B)=\dim(A)+dim(B).
\end{labeq}
%It follows that a pregeometry is locally modular exactly when condition \pref{mod}
%is satisfied for all closed $A$ and $B$ with $\dim(A\cap B)>0$.

It is routine to mention the following examples:
\begin{itemize}
\item[-]A vector space $V$ over a field $\cur$ is a pregeometry if we set $\cl(A)=\gen{A}_{\cur}$,
the $\cur$-linear span of a subset $A$ in $V$. This is a non-trivial modular pregeometry.

\item[-]If $\mathbb{A}$ is an affine space with underlying vector space $V$, the affine closure
turns $\mathbb{A}$ into a non-modular, locally modular pregeometry.

\item[-]Algebraic closure in an algebraically closed field (of large enough transcendence degree)
gives rise to a non-locally modular pregeometry.
\end{itemize}

%\subsection{Zil'ber's Structural Conjecture}
%--- MUTE-----Zilber Trichotomy
%We recall that a complete first-order theory is called strongly minimal if Morley rank and degree of its
%monstermodel $\mon$ are both equal to one. This translates to the fact that no definable set
%can be infinite and co-infinite, in any elementary extension.
%
%The esential tool which reproves Morley categoricity results (rephrased in a pregeometric flavour
%by Baldwin and Lachlan in \cite{blsms}) is the fact that, in strongly minimal structures, the
%algebraic closure is a pregeometry. Strongly minimal structures are in particular $\aleph_{1}$-categorical,
%and on the contrary uncountably categorical structures do always ``contain'' strongly minimal sets as
%-- we maight say -- building blocks.
%%Therefore it makes sense to classify a strongly minimal theory
%%according to the pregeometry $(\mon,\acl)$.
%
%In \cite{hruabi} and \cite{jbg} it is recalled that the pregeometries attached to the strongly minimal sets definable in a $\aleph_{1}$-categorical structure, have (after localization) all isomorphic associated geometries. This {\em local} isomorphism type constitutes therefore an
%invariant of such structures.
%
%Zilber's {\em trichotomy conjecture} -- formulated in \cite{} -- essentially assigned
%to an $\aleph_{1}$-categorical theory $T$ one of the following geometries:\mn{Compare/Consult John B. Goode}
%\begin{itemize}
%\punto{1}A disintegrated geometry. No almost strongly minimal group is definable in $T$.
%\punto{2}A non-trival modular geometry of a vector space. A rank-$1$ group is definable in $T$, but no infinite field does.
%\punto{3}A non locally modular geometry. $T$ is not one-based and an infinite field is interpretable
%in $T$.
%\end{itemize}
%
%The conjecture was disproved by Hrushovski in \cite{hruabi} by means of a {\em new strongly minimal set}:
%which has a non-locally modular geometry, but nevertheless does not interpret a field. Some detail
%of this are given in section \ref{abi}.

\subsection{Predimensions and associated Pregeometry Extensions}\label{pregextsec}
We denote by $\fp{M}$ the set of the finite parts of $M$.
\begin{dfn}\label{pregext}
Assume $\cl$ and $\cl_{0}$ are closure operators which both turn $M$ into a pregeometry.
We say that $\cl$ {\em extends} $\cl_{0}$ if for all $A\inn M$ we have $\cl_{0}(A)\inn\cl(A)$.

We say that $(M,\cl)$ is a {\em geometry over} $\cl_{0}$, if $\cl$ extends $\cl_{0}$, if $\cl(\vac)=\cl_{0}(\vac)$
and if $\cl(a)=\cl_{0}(a)$ for all $a\in M$. In the case $\cl_{0}$ is the identical closure ($\cl_{0}(A)=A$, for all $A\inn M$)
$(M,\cl)$ is called a {\em geometry}.
\end{dfn}

If $\cl$ extends $\cl_{0}$ and $\dim$, $\dim_{0}$ denote the associated dimensions, then for each
$A\in\fp{M}$ we have clearly $\dim_{0}(A)\geq\dim(A)$, moreover
$$
\dim(\cl_{0}(A))\leq\dim(\cl(A))=\dim(A)\quad\text{and}\quad\dim(A)\leq\dim(\cl_{0}(A)).
$$
In particular $\dim(A)=\dim(\cl_{0}(A))$, that is, $\dim$ is determined by its value on $\cl_{0}$-closed sets.

\smallskip
Let now $(M,\cl)$ be a pregeometry, %with dimension $d$, we
we denote the set of \emph{finitely generated $\cl$-closed parts} of $M$ by
$${\fp{M}}_{\cl}=\{B\inn M\mid B\,\text{is $\cl$-closed with $\dim(B)$ finite}\}.$$
\begin{dfn}\label{clpred}
We call a map $\map{\delta}{\fp{M}_{\cl}}{\N}$ a {\em predimension over $\cl$} or a \emph{$\cl$-predimension}
on $M$ if the following holds:
\begin{align}
\tag{normalization}\qquad&\delta(\cl(\vac))=0\quad\text{and}\quad\delta(\cl(a))\leq1&\\[+2mm]
\qquad&\delta(\cl(UV))\leq\delta(U)+\delta(V)-\delta(U\cap V)&\label{summo}
%\intertext{for all $U,V\in\fpe{M}$.}
\end{align}
for all $a\in M$ and $U,V\in\fp{M}_{\cl}$. Compared to \pref{mod}, property \pref{summo} above is referred to as
{\em submodularity}.

A {\em predimension} on $M$ is, by definition, a $\cl$-predimension where $\cl$
is the identical closure on $M$.%: $\cl(A)=A$, for all sets $A$. %$\cl=id_{\ps{M}}$.

A predimension $d$ on $M$ which is {\em monotone}, that is $d(B)\leq d(A)$ for all finite $B\inn A$ in $\fp{M}$
%\begin{itemize}
%\punto{monotonicity}$d(B)\leq d(A)$ for all finite subsets $B\inn A$ of $M$
%\end{itemize}
is called a {\em dimension function} on $M$.
\end{dfn}

\medskip
Assume $\delta$ is a $\cl$-predimension on $M$ and set, for all $A$ in $\fp{M}$
%$$\delta^{*}(B)=\min\{\delta(C)\mid C\in\fp{M}_{1},\,C\nni B\}$$
%and successively
%\begin{labeq}{didef}
%d_{2}(A)=\delta^{*}(\cle(A))\quad\text{for all}\,A\in\fp{M}.
%\end{labeq}
%(It can also be defined:
\begin{labeq}{ddidelta}
d(A):=\min(\delta(C)\mid C\in\fp{M}_{\cl},\:C\nni A).
\end{labeq}
With the above definition we still have $d(\vac)=0$ and $d(a)\leq1$ for all singleton $a$.
Moreover for finite  $A,B$ in $M$ let us choose $\cl$-closed oversets $A^{\prime}\nni A$ and $B^{\prime}\nni B$
%with minimal $\delta$, that is
with $d(A)=\delta(A^{\prime})$ and $d(B)=\delta(B^{\prime})$.
Since closed sets are closed under intersection, we have
$$d(AB)+d(A\cap B)\leq\delta(\cl(A^{\prime}B^{\prime}))+\delta(A^{\prime}\cap B^{\prime})\leq d(A)+d(B),$$
that is $d$ is a dimension function on $M$ after Definition \ref{clpred} above, and
is called {\em the} dimension function {\em associated to} $\delta$.
Also note that, in the definition of $d$, is crucial to require $\delta$ to be non-negative.

The next lemma shows that $d$ is actually
the dimension associated to a prescribed pregeometry.
\begin{lem}\label{preg}
Assume $d$ is the dimension function associated to a $\cl$-predimension $\delta$ on
the set $M$ via \pref{ddidelta}.

For any $A\in\fp{M}$ define $\cl_{d}(A)$ to be the set of all $b$ of $M$ such that $$d(Ab)=d(A).$$

If we define $\cl_{d}$ on arbitrary sets in the natural way, that is by putting $\cl_{d}(A)=\bigcup\{\cl_{d}(F)\mid F\in\fp{A}\}$,
then $(M,\cl_{d})$ is a pregeometry which extends $(M,\cl)$ and has dimension $d$.
\end{lem}
\begin{proof}
It is enough to show properties (cl2) and (ex) holds for $\cl_{d}$ over finite sets, while (cl1) is clear.

For (cl2) assume %then that each element $b$ of
that a finite set $B$ is contained in %in the closure of a finite set
$\cl_{d}(A)$ for some finite set $A\inn M$. That is $d(Ab)=d(A)$ for all $b$ in $B$.
By induction on the cardinality of $B$, using submodularity \pref{summo}, it follows $d(BA)=d(A)$.

We need to show that if an element $a$ is in $\cl_{d}(B)$ then it is in $\cl_{d}(A)$.
Applying submodularity we have $d(aBA)\leq d(aB)+d(BA)-d(B)=d(BA)$. 

We can conclude $d(aA)\leq d(aBA)\leq d(A)$ as desired. This gives $\cl_{d}(A)\nni \cl_{d}(\cl_{d}(A))$ and (cl2) follows.

\smallskip
To obtain the exchange property (ex), observe first that, as as a dimension-function, $d$ satisfies $d(Ab)\leq d(A)+1$, for any finite $A\inn M$.

Assume $a\in\cl_{d}(A\,b)\non\cl_{d}(A)$, this means $d(A)<d(aA)\leq
d(aAb)=d(Ab)\leq d(A)+1$. This forces $b$ to be in the closure of $Aa$.

\smallskip
That $\cl_{d}$ extends $\cl$ is readily seen, as $d(Ab)=d(A)$ for all $b\in\cl(A)$ by definition \pref{ddidelta}, for all $A\in\fp{M}$.

And it is trivial to verify that $d(S)$ %=\dim(S)$
is the $\cl_{d}$-dimension of $S$, for every set $S$ in $M$.
%It remains to check that $d$ is actually the dimension of $\cl_{d}$. We have namely to
%check that, for any finite $S$ in $M$, $d(S)$ coincide with the number of elements in a basis of $S$. Let $\{s_{1},\,\dots,\,s_{n}\}$ be such a basis. If $n=1$ then since $\{s_{1}\}$ is independent $d(s_{1})=1$, assume it is true of $n-1$.
\end{proof}

%We also remark that since the axioms for a predimension are universal (modulo $\cl$), the following 
%
%Assume $\delta$ is a $\cl$-predimension on the set $M$ and $d$
%is the dimension function associated to $\delta$.
%Assume $\cl^{\prime}$ is the pregeometry  on $M$ extending $\cl$ derived from $d$.
%
%If now $N$ is a $\cl$-closed subset of $M$, then the restriction of $\cl^{\prime}$ to $N$, $\res{\cl^{\prime}}{\ps{N}}$
%coincide with the pregeometry obtained by the dimension function on $N$ associated to $\res{\delta}{\fp{N}_{1}}$.