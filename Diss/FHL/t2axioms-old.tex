%\begin{presection}
%The axiom-schema we propose here is the natural and obligatory one in a setting
%in which strongness is given by predimension and in which the class supports
%a nice amalgamation. This is also the line of \cite{jbg}, for an extensive account of
%various versions of amalgamation constructions one can see \cite{wag}.
%
%It has to be pointed out that this approach of approximating richness, by means of axioms
%$\sig{2}{3}$ below, works because of the {\em asymmetric} amalgam (lemma \ref{asymalg}).
%\end{presection}
In this last section, we axiomatise the $\Lan{2}$-theory of the Fra\"iss\'e limit $\K$ of $(\Kl{2},\zsu{})$. We prove
it is $\omega$-stable and calculate its Morley rank.

\medskip
Nilpotency of class $2$ can be expressed universally in $\Lan{2}$,
in terms of {\em simple} commutators, by requiring $[x,y,z]=[[x,y],z]=\triv$ for all $x,y,z$.
Altough the language $\Lan{2}$ can naturally express the grading on each $\nla{2}$-algebra, by means of $M=P_{1}(M)+P_{2}(M)$,
$P_{1}(M)\cap P_{2}(M)=\triv$
and $(\forall xy)P_{2}([x,y])$, in general there is no first-order bound to the length of polynomials in weight $2$, which could express
$\gena{P_{1}(M)}{}=M$.

In the axiom system chosen below for $\K$, this is sorted out in the strongest way possible: we require
for each weight $2$ element to be exactly the Lie bracket of two elements (from $P_{1}$). This bound may be compared with
a corollary of Zilber's indecomposability theorem, in groups $G$ of finite Morley rank for which
an integer $n$ exists, $g\in G^{\prime}$ then $g$ is the product of at most $n$ commutators $[x_{i},y_{i}]$. 

As a consequence, it will be true that elementary $\Lan{2}$-extensions are $\nla{2}$-extensions.

\smallskip
If $M$ be an $\Lan{2}$ structure, we define the theory
$T_{2}$ by means of the following denumerable first order schema of axioms for $M$:
\begin{itemize}
\punto{$\sig{2}{1}$}$M$ is a graded nil-$2$ algebra
\punto{$\sig{2}{2}$}For any finite subspace $H_{1}$ of $M_{1}$ $\delta(H)\geq\min(\dfp(H_{1}),\,2)$. 
\punto{$\sig{2}{3}$}for any $n<\omega$ and any finite strong extension
$A\nni B$ of $\Kl{2}$-algebras, 
if $B$ is $(\dfp(A_{1}\quot B_{1})+n)$-selfsufficient in $M$,
then there exists an isomorphic copy of $A$ in $M$ over $B$, which is $n$-self-sufficient.
\end{itemize}
We can first observe that axioms $\sig{2}{2}$ and $\sig{3}{3}$ imply that a model of $T_{2}$ cannot be finite.

\begin{teo}
%The rich %countable structures in 
An $\La_{2}$-structure $K$ is a rich algebra of $\Kl{2}$
%are exactly the countable $\omega$-saturated models of $T^{2}$.
if and only if $K$ is an $\omega$-saturated model of $T^{2}$.
\end{teo}
\begin{proof}
We start proving that a rich algebra $K$ of $\Klt{2}$ is also a model of $T^{2}$ which is henceforth consistent: the
Fra\"iss\'e limit $\K$ of $(\Kl{2},\zsu{})$ exhibits a countable model. The second part of the proof shows that an $\omega$-saturated model of $T^{2}$ is a rich $\Klt{2}$-algebra, now since rich %countable\mn{be careful!}
structures are $\La_{\infty,\omega}$-equivalent, %\mn{by lemma \ref{bafo} below!!}
as lemma \ref{bafo} below explains,
it follows that rich $\Klt{2}$-structures are $\omega$-saturated.

\smallskip
So let $K$ be such a rich algebra in $\Klt{2}$, axioms $\sig{2}{1}$ and $\sig{2}{2}$ are
satisfied automatically.

Now assume $A$ is a finite strong extension in $\nla{2}$ of a finite subalgebra
$B$ of ${K}$ and
$B$ is $(\dfp(A_{1}\quot B_{1})+n)$-selfsufficient in ${K}$. Take a finite strong subalgebra $\widetilde{B}$ of ${K}$ containing $B$ (the selfsufficient closure of $B$ for instance).
We use the asymmetric amalgamation lemma \ref{asymalgadue} to obtain a strong extension $\widetilde{A}$ in $\Kl{2}$ of $\widetilde{B}$,
such that  $A$ is an $n$-selfsufficient substructure of $\widetilde{A}$.

Now since ${K}$ is rich, $\widetilde{A}$ strongly embeds into ${K}$ over $\widetilde{B}$.
As a consequence of transitivity (lemma \ref{2trans}), $A$ embeds into ${K}$
$n$-selfsufficiently over $B$.

\medskip
For the reverse implication suppose $M$ is a countable $\omega$-saturated model of $T^{2}$ and $A\quot B$
a finite strong extension of $\Kl{2}$-algebras, where $B$ is a finite strong
substructure of $M$. We may assume, without loss
of generality, that $A\quot B$ is a minimal extension; otherwise we decompose it
in a chain of minimal strong sections which we embed stepwise in $M$.

Assume first $A\quot B$ is a free extension and $B$ is a finite strong substructure in $M$,
by lemma \ref{samedelta2} we are done if we find $a\in M_{1}$ which is $\cl_{2}$-independent of $B_{1}$, as $d_{2}(a\quot B_{1})=1$ implies that $\gena{B_{1},a}{M}$ is strong in $M$

If we can prove that $d_{2}^{M}(M_{1})$ is infinite, the desired condition will follow.
To do so, denote by $\fla{2,n}{\bar x}$ the quantifier-free diagram of
the free nil-$2$ Lie algebra $\fla{2}{\bar x}$ generated by an $n$-element tuple $\tpl{x}{n}$.
Thus if $\bar a$ is an $n$-tuple of $M_{1}$, $M\sat\fla{2,n}{\bar a}$ means $\gena{\tpl{a}{n}}{M}\simeq\fla{2}{\bar a}$.

We show, with an inductive argument, that we can {\em strongly} embed $\fla{2,n}{\bar x}$ in $M$ for any $n<\omega$.
Axiom $\sig{2}{2}$ ensure  that for any independent pair $m_{1},m_{2}$ of $M_{1}$,
$\gena{m_{1},m_{2}}{M}$ is a selfsufficient subalgebra of $M$ isomorphic to
the free nilpotent algebra $\fla{2}{m_{1},m_{2}}$; this will be our inductive base.

Assume now $M\sat\fla{2,n}{\bar b}$ and $\gena{\bar b}{M}$ is strong in $M$. Consider
the collection $\Phi^{n+1}(x,\bar b)$ of all formulae\mn{better $\Phi^{n+1}(x,\bar b)\wedge
\Sigma(\bar b)$ to say $\bar b$ str?}
$\phi^{n+1}_{k}(x,\bar b)$ for $k<\omega$, where an $\La^{2}$-structure $L$
satisfies $\phi^{l}_{k}(\bar y)$ in $\bar m \inn L_{1}$ exactly when $L\sat\fla{2,l}{\bar m}$
and $\gena{\bar m}{L}$ is $k$-strong in $L$.

Now a finite portion of $\Phi^{n+1}$ is implied by a single formula $\phi^{n+1}_{k}(x,\bar b)$ with a sufficiently
large $k$. %, here $k$ may also be assumed to be bigger than. Now use axiom
Now since $\gena{\bar b}{M}$ is strong, we have $M\sat\phi^{n}_{k+1}(\bar b)$ and hence by $\sig{2}{3}$ there exists
$a$ in $M_{1}$ such that $M\sat\phi^{n+1}_{k}(a,\bar b)$.

We showed $\Phi^{n+1}(x,\bar b)$ is consistent with $T^{2}_{\bar b}$ and hence realized in $M_{1}$
by $\omega$-saturation for some $m\in M_{1}$. It follows $\gena{m,\bar b}{M}$ is
selfsufficient and hence $d_{2}^{M}(m,\bar b)=\delta(m,\bar b)=n+1$.
By induction, $M$ has infinite $d_{2}$-dimension.
\medskip

Assume now $A\quot B$ is a minimal strong extension with $\delta(A\quot B)=0$ (hence
algebraic or prealgebraic). Since $B$ is strong in $M$, (anyone among) axioms $\sig{2}{3}$ ensure
the existence of an isomorphic copy $A^{\prime}$ of $A$ in $M$ over $B$. Because
of $\delta(A^{\prime}\quot B)=0$, we have that $A^{\prime}$ is selfsufficient in $M$ as
well. This proves that $M$ is a rich Lie algebra in $\Kl{2}$.
\end{proof}

The proof of the theorem also shows that if $M$ is a $\kappa$-saturated model of $T_{2}$,
then its $\cl_{2}$ dimension $d_{2}(M)$ is not smaller than $\kappa$.


Note that the Fra�ss� limit of $\Kl{2}$ that we constructed in \ref{} is ``the'' countable saturated model of $T^{2}$.
\begin{lem}\label{alg-ex-cls}
Define a sentence $\sig{2}{4}$ by the property
\begin{itemize}
\punto{$\sig{2}{4}$} for all $y\in M$ with $P_{2}(y)$ and all $\triv\neq z\in M_{1}$ there is $x\in M_{1}$ such that $[z,x]=w$
\end{itemize}
then $T^{2,\omega}\models\sig{2}{4}$.
\end{lem}
\begin{proof}
Pick an element $w\in M_{2}$ and $m\in M_{1}$ for some $\Lan{2}$-structure $M$
which is a model of $T^{2}$.
Now since $M=\gena{M_{1}}{M}$, there exists a finite subspace
$B_{1}$ of $M_{1}$ with $m\in B_{1}$ and $w\in B_{2}$. We may
clearly assume that $B$ is self-sufficient in $M$.

If there exists $c$ in $B_{1}$ such that $[c,m]=w$ we are done, so we assume
there is no such $c$ in $B_{1}$. Define an extension $A$ of $B$ as follows: set first $A_{1}:=B_{1}\oplus\gen{a}$ and then $\rd(A):=\rd(B)\oplus\gen{[a,m]-\beta}$, where
$\beta$ is an element of $\exs B_{1}$ which represent $w$ modulo
$\rd(B)$. Hence $[a,m]-\beta$ is an element of $\exs A_{1}$.

Since in $\delta(A_{1}\quot B_{1})=0$, $B$ is self-sufficient in $A$. We show
next that $A$ is in $\Kl{2}$. Let $E_{1}$ be a finite subspace of $A_{1}$,
then $E_{1}$ has dimension at most $1$ over $E_{1}\cap B_{1}$. Thus in a nontrivial
case, there exists $b\in B_{1}$ such that $E_{1}=\gen{a+b,E_{1}\cap B_{1}}$.

Since $E_{1}\cap B_{1}\zsu{}E_{1}$ by lemma \ref{2cut}, and since
$\delta(E_{1})=\delta(E_{1}\cap B_{1})+\delta(a+b\quot E_{1}\cap B_{1})$,
if $\dfp(E_{1}\cap B_{1})\geq2$ we have $\delta(E_{1})\geq2$.

The other only case to be considered is when $\dfp(E_{1})=2$ and
$E_{1}=\gen{a+b,u}$ for $b,u$ in $B_{1}$. If $\rd(E)\neq\triv$, then in
$\exs A_{1}$ we have the equality $[a+b,u]=[a,m]-\beta+\eta$ for
some $\eta$ in $\rd(B)$. This translates into
$[a,u-m]=[u,b]-\beta+\eta\in\exs B_{1}$. If we take any $\Fp$-basis $(b_{i}\mid i<n)$ of $B_{1}$
for some $n<\omega$, then the set $([a,b_{i}]\mid i<n)$ is a basis for
$\exs A_{1}$ over $\exs B_{1}$. This yields that $u=m$ and that
$[u,b]-\beta$ belongs to $\rd(B)$. Thus the element $-b$ of $B_{1}$ solves the desired equation $[-b,m]=w$ in $B$, contradicting our assumption.

\smallskip
We have shown that $A\quot B$ is a strong extension of $\Kl{2}$-algebras.
Now since $B\zsu{}M$, any of the axioms of $\sig{2}{3}$ implies that $A$ is
%take an $\omega$-saturated elementary extension $N$ of $M$, as a rich
%algebra of $\nla{2}$, $N$ embeds $A$ over $B$. In particular
realised in $M$ over $B$. In particular 
there is $a^{\prime}$ in $M_{1}$ with $[a^{\prime},m]=w$ as desired.
\end{proof}

\begin{lem}\label{strongelm}
Elementary $\nla{2}$-extensions are strong.
%and conversely\mn{make any sense?}, strong model extensions are elementary.
\end{lem}
\begin{proof}
Assume $N$ elementarily extends a structure $M$ of $\nla{2}$.

Observe first that $M$ is then a $\nla{2}$-subalgebra of $N$. If $M$ is not strong in $N$, $\delta(A/M)<0$ for
some finite subspace $A_{1}$ of $N_{1}$. By section \ref{deltadue} there is a finite $C_{1}$ strong
in $M_{1}$ such that $\delta(A/M)=\delta(A/C)$. But then, since now $\delta(A/C)$ is expressible through
a formula -- the $\La$-diagram of $C+A$ over $C$,
for some finite subspace $A^{\prime}_{1}$ of $M_{1}$ we have $\delta(A^{\prime}_{1}\quot C_{1})$
contradicting self-sufficiency of $C$ in $M$.
\end{proof}

For the rest of the section\mn{notational convention goes at the section-beginning}, if $M\in\Klt{2}$ and $\bar a$ is a tuple of $M_{1}$, $\delta(\bar a\quot B_{1})$
stands for $\delta(\gen{B_{1},\bar a}\quot B_{1})$ and
with $\ssc^{M}(B_{1},\bar a)$ we denote the self-sufficient closure in ${M}_{1}$ of the subspace $\gen{B_{1},\bar a}$
of ${M}_{1}$, for all subspaces $B_{1}$.

\begin{prop}\label{bafo}
Assume $M$ and $M^{\prime}$ are two models of $T^{2}$. Let $\bar a$ and $\bar a^{\prime}$
be tuples of $M_{1}$ and $M^{\prime}_{1}$ respectively.

Then $\mathrm{tp}(\bar a)=\mathrm{tp}(\bar a^{\prime})$ %\mn{\"aqv:$\gena{a}{M}\equiv \gena{a^{\prime}}{M^{\prime}}$?}
if and only if the selfsufficient closure $\gena{\ssc_{2}(\bar a)}{M}$ is $\nla{2}$-isomorphic to
$\gena{\ssc_{2}(\bar a^{\prime})}{M^{\prime}}$
via a Lie isomorphism mapping $\bar a$ onto $\bar a^{\prime}$.
\end{prop}
\begin{proof}
If we assume $\mathrm{tp}(\bar a)=\mathrm{tp}({\bar a}^{\prime})$, then
we have $d_{2}^{M}(\bar a)=d_{2}^{M^{\prime}}({\bar a}^{\prime})$ and we can
find a finite subspace $A_{1}$ of $M_{1}$ containing $\bar a$ and
isomorphic to $\ssc^{M^{\prime}}({\bar a}^{\prime})$. This yields
$A_{1}=\ssc^{A}(\bar a)$. Now since $\delta(A)=d_{2}^{M}(\bar a ^{\prime})=d_{2}^{M}(\bar a)\leq d_{2}^{M}(A_{1})$,
$A$ is strong in $M$, it follows $A_{1}=\ssc^{M}(\bar a)$ by lemma \ref{samed2}.

\medskip
For the other direction
we may assume that $M$ and $M^{\prime}$ are $\omega$-saturated,
since by Lemma \ref{strongelm} the self-sufficient closure of a subspace of $M_{1}$
will remain the same if computed in any elementary (saturated) extension of $M$.

Assume tuples $\bar b\inn M_{1}$ and $\bar b ^{\prime}\inn M_{1}^{\prime}$ generate isomorphic strong subalgebras
in $M$ and $M^{\prime}$ respectively, we show that $\bar b$ and $\bar b ^{\prime}$
can be matched up by an Ehrenfeucht-Fra\"iss\'e game of lenght $\omega$.
This implies that an isomorphism between $\gena{\bar b}{M}$ and $\gena{\bar b ^{\prime}}{M^{\prime}}$ preserves $\La_{\infty,\omega}$-formulas, hence $\mathrm{tp}(\bar b)=\mathrm{tp}(\bar b ^{\prime})$.

Assume one player chooses an element $m$ of $M$ -- say -- outside $\gena{\bar b}{M}$. Then the other player
first adds a linear independent tuple $\bar c$ over $\bar b$ such that
$m\in\gena{\bar b,\bar c}{M}$ and such that $\gen{\bar b,\bar c}\zsu{}M_{1}$.
Since $\gen{\bar b^{\prime}}$ is strong embeddable into $\gen{\bar b,\bar c}$ and $M^{\prime}$ is a rich $\Klt{2}$-structure, one can {\em respond} with a tuple $\bar c^{\prime}$ of $M_{1}^{\prime}$
with $\gena{\bar b,\bar c}{M}\simeq
\gena{\bar b^{\prime},\bar c^{\prime}}{M}$ and $\gen{\bar b^{\prime},\bar c^{\prime}}\zsu{}M_{1}^{\prime}$.
We can play $\omega$ rounds in this way, back-and-forth between $M$ and $M^{\prime}$.
%Without
%loss of generality we may assume both $\bar a ^{i}$ to be contained in $K^{i}_{1}$,
%for assume for instance, $w=w_{1}+w_{2}$ is in $\bar a ^{1}$ for $w_{2}\neq\triv$.
%If $w_{2}$ is an homogeneous sum of Lie products from $K^{1}_{1}\cap\gena{\bar a ^{1}}{K^{1}}$ then replace $w$ with $w_{1}$ (and do the same for the corresponding
%element of $\bar a ^{2}$), otherwise pick $m$ in $K_{1}^{1}$ with $[m,e^{1}]=w_{2}$ for
%some $e^{1}\in K^{1}_{1}\cap\gena{\bar a ^{1}}{K^{1}}$ (this space may be also
%assumed to be non trivial). We have $\bar a ^{1} m$ is still strong in $K^{1}$.\mn{\bf need strong for non $\nla{2}$-subalg?}
%-------MAYBE THIS?-------------
%We say that a tuple $\bar a$ in $M\in\Kl{2}$ is a {\em strong tuple}\mn{{?}see ``constructions'' in AddColl} if there exists
%a strong finite subspace $H_{1}$ in $M_{1}$ such that $\bar a\inn \gena{H_{1}}{M}$
%and such that the set of elements from $H_{1}$ which are not in $\bar a$ are $d_{2}$-independent
%over $M_{1}\cap\bar a$.
\end{proof}

Proposition \ref{bafo} allows a {\em converse} statement of lemma \ref{strongelm}:
\begin{rem*}
Any extension $L$ of a model $M$ of $T_{2}$ in which $M$ is self-sufficient, is elementary.
\end{rem*}


Since $\triv$ is self-sufficient in every model, by the lemma above
we obtain that the theory $T^{2}$ is complete. Also observe, that in general
any two algebras $H\zsu{}M$ and $H^{\prime}\zsu{}M^{\prime}$ which are self-sufficient in models $M$ and $M^{\prime}$ of $T_{2}$
do have the same elementary type if and only if they are isomorphic.

\bigskip
For the rest of the chapter, we assume a large saturated model $\mathbb{M}$ has been fixed, the {\em monster} model of $T_{2}$.
Any model $M$ of $T_{2}$ is an elementary substructure of $\mathbb{M}$ with $\card{M}<\mathbb{M}$
and in particular an $\nla{2}$-subalgebra of $\mathbb{M}$.
It follows by lemma \ref{strongelm}, that $M$ is strong in $\mathbb{M}$ and in particular, by lemma
\ref{samed2} $d_{2}^{M}=d_{2}^{\mathbb{M}}$ on $M_{1}$ for any model $M$.

\medskip
As an immediate corollary of the previous proposition
\begin{rem}\label{indtypes}
For any strong $H$ in $\mathbb{M}$, any $a,a^{\prime}$ in $\mathbb{M}_{1}$ are
$\cl_{2}$-independent of $H$ -- that is $d_{2}(a/H_{1})=d_{2}(a^{\prime}/H_{1})=1$ -- exactly if
$\tp{a}{H}=\tp{a^{\prime}}{H}$.
\end{rem}
\begin{proof}
If $a$ is $\cl_{2}$-independent of $H_{1}$, then $a$ is linearly independent of $H_{1}$ and
there is no {\em link} between $a$ and $H$, i.{}e. $\rd(H_{1},a)=\rd(H_{1})$. Therefore, under
the above assumptions, $\gena{H_{1},a}{\mathbb{M}}\simeq_{H}\gena{H_{1},a^{\prime}}{\mathbb{M}}$ follows.
Moreover by Lemma \ref{samedelta2} both substructures are self-sufficient, hence the previous proposition yields
the desired statement.
\end{proof}

On the other hand by Proposition \ref{samedelta2} and \ref{bafo} we obtain
\begin{cor}\label{isola}
Let $B$ be a finite
strong subalgebra of %$M_{1}$ for some %$\omega$-saturated
a model $M$ of $T^{2}$.

Assume $\bar a$ is a tuple
in $M_{1}$ such that $d_{2}(\bar a\quot B_{1})=0$.
Let the $\Lan{2}_{B}$-formula $\Delta(\bar x,\bar{y})$
describe the quantifier-free diagram of $\ssc(B,\bar a)$ in such a way that
for any tuple $\bar c$ of $M_{1}$, for $M$ to satisfy $\Delta(\bar a,\bar c)$ means that
$\gena{B_{1},\bar a,\bar c}{M}\simeq\ssc(B,\bar a)$. %$\gena{B_{1},a,\bar c}{M}\simeq\gena{\ssc(B,a)}{M}$.
Then the formula $\exists\bar y\Delta(\bar x,\bar y)$ isolates $\tp{\bar a}{B_{1}}$.
\end{cor}


\smallskip
We will now prove that our theory $T^{2}$ is totally transcendental. %we follow the proof line of Wagner in \cite{Wa}.
\begin{prop}\label{omegastab}
$T^{2}$ is $\omega$-stable.
\end{prop}
\begin{proof}
Since $\mathbb{M}=\gena{\mathbb{M}_{1}}{\mathbb{M}}$,
it is sufficient to count types $\tp{\bar m}{H}$ for tuples $\bar m$ in $\mathbb{M}_{1}$
and countable sets $H\inn\mathbb{M}$. Moreover
without loss of generality we might assume that $H=\gena{H_{1}}{\mathbb{M}}$ is
a self-sufficient subalgebra of $\mathbb{M}$.\mn{Can we really made this assumption??}

The type of $\bar m$ over $H$ is fully determined by the quantifier-free type
of $\gena{\ssc(H_{1},\bar m)}{\mathbb{M}}$. By lemma \ref{fincharssc} we have
$\ssc(H_{1},\bar m)=\geno{H_{1},\bar a}$ for a finite tuple $\bar a$ of $\mathbb{M}_{1}$, linearly independent over $H_{1}$.
Moreover -- still by Lemma \ref{fincharssc} -- we can find a finite $B_{1}\zsu{}H_{1}$ with $A_{1}=\geno{B_{1},\bar a}$
such that $\delta(A_{1}\quot B_{1})=\delta(A_{1}\quot H_{1})$ and $A_{1}\cap H_{1}=B_{1}$.

By lemma \ref{freecomp} $H$ and $A$ are in free composition over $B$, that is
$$\gena{\ssc(H_{1},\bar m)}{\mathbb{M}}=H+A\simeq\am{H}{B}{A}.$$

Since the isomorphism type of the free amalgam is fully determined by its components,
the type of $H+A$ is determined by $\tp{A_{1}}{B_{1}}$ and by $\tp{B_{1}}{H_{1}}$ -- that is
by the choice of $B_{1}$ into $H_{1}$.

\smallskip
Since we have a countable saturated model, namely the Fra�ss� limit %$K$
of $(\Kl{2},\zsu{})$,
the theory $T^{2}$ is small, this gives only countably many choices for $\tp{A_{1}}{B_{1}}$.
Altogether we have $\aleph_{\sss 0}\cdot\card{H_{1}}^{<\aleph_{\sss 0}}=\aleph_{\sss 0}$ possibilities for $\tp{\bar a}{H}$
and in particular for $\tp{\bar m}{H}$.
\end{proof}

\begin{rem}\label{tildarich}
Any $\omega$-saturated model $M$ of $T_{2}$, satisfies a stronger version of richness over $\Klt{2}$.
That is for {\em any} self-sufficient $\nla{2}$-subalgebra $N$ of $M$, if $H$ is a finite strong extension of $N$ in $\Klt{2}$,
then $M$ embeds $H$ self-sufficiently over $N$.
\end{rem}
\begin{proof}
Split $H_{1}/N_{1}$ into two strong sections $H_{1}/K_{1}$ and $K_{1}/N_{1}$ (cfr. Definition \ref{mindecomp}),
such that $\delta(H/K)=0$ and $d(H/N)=d(K/N)=\dfp(K_{1}/N_{1})$.

Now by saturation of $M$, iterating Remark \ref{indtypes} above, we first find a strong $\nla{2}$-subalgebra $\widetilde{K}$ of $M$
with $N\inn\widetilde{K}$ and $\widetilde{K}\simeq_{N}K$.

Secondly we consider the embedding $\widetilde{K}\into H$ and find -- by Proposition \ref{fincharssc} -- a finite $K^{\rm o}\zsu{}\widetilde{K}$ such that $H\simeq\am{H^{\rm o}}{K^{\rm o}}{\widetilde{K}}$ for a suitable finite $H^{\rm o}\inn H$ such that $H=K+H^{\rm o}$.

Conclude with richness of $M$, by strongly embedding $H^{\rm o}$ into $M$ over $K^{\rm o}$. Now since $\delta(H^{\rm o}/\widetilde{K})
=\delta(H^{\rm o}/K^{\rm o})=0$, we obtained the desired strong embedding of $H$ into $M$.
\end{proof}

\bigskip
The next paragraphs are devoted to describe the algebraic closure of sets
of $\mathbb{M}_{1}$.

First observe that axioms $\sig{2}{2}$ imply that $\aut(\mathbb{M})$ is $2$-transitive on $\mathbb{M}_{1}$ as a group
of $\Fp$-linear automorphisms, this is to say,
transitive on the set of ordered pairs of linearly independent elements from $\mathbb{M}$.
In particular  $\acl(\triv)=\triv$, and $\acl(a,b)=\gena{a,b}{\mathbb{M}}$ for {\em any} pair of elements. \sout{$T_{2}$ is not $\aleph_{0}$-categorical.}

Now take a subspace $C_{1}$ of $\mathbb{M}_{1}$, since by saturation, $d_{2}(\mathbb{M}_{1}/C)$ is infinite, lemma \ref{bafo} implies that for any , $\acl(C_{1})\cap\mathbb{M}_{1}$ is contained in $\cl_{2}(C_{1})$.

It is also quite straightforward to see that $\ssc(C)$ is contained in
the algebraic closure of $C$:
it is sufficient to consider a finite subalgebra $A$ of $C$ which
is not strong in and let $A^{\prime}$ be its self-sufficient closure,
if $A^{\prime}$ has infinitely many conjugates in $\mathbb{M}$ over $A$, then
we can find a strong subalgebra $A^{\prime\prime}$ such that
$A\inn A^{\prime}\cap A^{\prime\prime}\subsetneq A^{\prime}$. As strongness is invariant and closed
under intersections, this contradicts minimality of self-sufficient closure.
With Lemma \ref{finchar}, we may also conclude that $\acl(C_{1})\cap\mathbb{M}_{1}$ is self-sufficient.
One has then
$$ssc(C_{1})\zsu{}\acl(C_{1})\cap\mathbb{M}_{1}\zsu{}\cl_{2}(C_{1}).$$

We also have
\begin{labeq}{acluno}
\acl(C_{1})=\gena{\acl(C_{1})\cap\mathbb{M}_{1}}{\mathbb{M}}
\end{labeq}
for if
$C_{1}$ is not trivial, for any element $m=m_{1}+m_{2}$ of $\acl(C_{1})$, property $\sig{2}{4}$ implies
$m=m_{1}+[h,x]$ for some $h\in C_{1}$ and some $x$ in $\mathbb{M}_{1}$.

Now $m_{1}\in\acl(C_{1})\cap\mathbb{M}_{1}$ and $x$ is algebraic over $m_{1},h_{1}$, by axiom
$\sig{2}{2}$.


\medskip\cbstart
As opposed to amalgamation constructions in relational languages, here the
self-sufficient closure {\em does not equal} algebraic closure. On the other hand
in our theory $T^{2}$ the algebraic closure does not coincide with the
geometric closure $\cl_{2}$, as actually happens in the collapsed frame.
\cbend

\smallskip
We can actually do even better, and fully characterise $\acl(C_{1})$
for a given $C_{1}\inn\mathbb{M}_{1}$, in terms of {\em divisor} elements.
Recall that, by Definition \ref{}, an element $x\in\mathbb{M}_{1}$ is a divisor of $C$ if $\delta(x/C)=0$\mn{or $\leq$?} or equivalently
$[x,h]\in C_{2}$ for some $h\in C_{1}$. By axiom $\sig{2}{2}$ then, a divisor $x$ of $C$ is
{\em algebraic} over $C$.

\smallskip
We call a self-sufficient subalgebra $C$ of $\mathbb{M}$ {\em divisibly closed} if there is no divisor of $C$ in $\mathbb{M}_{1}$,
which is linearly independent over $C_{1}$.
By the remark above, $\acl(C_{1})$ is divisibly closed. Moreover if $U$ and $V$ are divisibly closed $\nla{2}$-algebras 
then $W=\gena{U_{1}\cap V_{1}}{\mathbb{M}}$ is also divisibly closed, for %as a consequence of axiom $\sig{2}{2}$. This is for,
if $\delta(x/U_{1}\cap V_{1})=0$ then
%there is $u\in C_{1}\cap C_{1}$ such that $[x,u]\in U_{2}$, hence there are 
$\delta(x/U)=\delta(x/V)=0$ and $x\in U_{1}\cap V_{1}$.

Since meet-closed classes give rise to closure operators, we may define


Let $\mathscr{D}_{C_{1}}$ denote the set of all subspaces $H_{1}$ containing $C_{1}$, which generate divisibly closed self-sufficient algebras in $\mathbb{M}$, %containing $C$,
then we set
$$\div(C_{1})=\bigcap\mathscr{D}_{C_{1}}$$
and consistently to our terminology $\div(C)=\gena{\div(C_{1})}{\mathbb{M}}$.

\begin{lem}\label{acldiv}
For any subspace $C_{1}$ of $\mathbb{M}_{1}$ we have
$$\acl(C_{1})%\cap\mathbb{M}_{1}
=\div(\ssc(C))$$
\end{lem}
\begin{proof}
%We can of course reduce ourself to the case in which $C$
Since $\ssc(C_{1})\inn\acl(C_{1})$, we may actually assume $C$ to be self-sufficient.

Moreover, as $\acl(C_{1})=\gena{\acl(C_{1})\cap\mathbb{M}_{1}}{\mathbb{M}}$ is divisibly closed,
it is enough to show that $\div(C_{1})$ contains $\acl(C_{1})\cap\mathbb{M}_{1}$.

Assume an element  $a$ of $\mathbb{M}_{1}$ is in $\acl(C_{1})$, let $A_{1}$ denote $\ssc(C_{1},a)$ and $B_{1}$ denote $\ssc(C_{1},a)\cap\div(C_{1})$.
Suppose $b$ does not belong to $\div(C_{1})$, then we have a non-trivial finite strong section $A_{1}/B_{1}$ of $\mathbb{M}$
such that $\delta(A/B)=0$.

If $A=A^{1}$, $A^{2}$, \dots, $A^{n}, \dots$ are distinct $B$-isomorphic copies of $A$ for $n<\omega$, set $\circledast^{0}_{B}A=B$ and $\circledast^{1}_{B}A$ as $A^{1}$. For all $1\leq n<\omega$, also define inductively
$\circledast^{n}_{B}A=\am{(\circledast^{n-1}_{B}A)}{B}{A^{n}}$.

%$$\circledast^{n}_{B}A=\am{\left(\dots\am{(\am{A^{1}}{B}{A^{2}})}{B}{A^{3}}\dots\right)}{B}{A^{n}}.$$
Observe that for all $n$, $\circledast^{n}_{B}A$ is a structure of $\Klt{2}$, this follows (inductively) by lemma \ref{amalsigma2}:
since $B$ is divisibly closed in $A^{n}$, there is no algebraic element in $A_{1}$ to prevent the free amalgam of $\circledast^{n-1}_{B}A$
and $A^{n}$ over $B$ from satisfying property $\sig{2}{2}$.

Now richness of $\mathbb{M}$ with
respect to finite $\Klt{2}$-extensions, implies that for all $n<\omega$, we can
strongly embed $\circledast^{n}_{B}A$ inductively into $\mathbb{M}$ over $\circledast^{n-1}_{B}A$.

We obtain -- by transitivity of strong embeddings -- arbitrarily many distinct copies $A^{i}$ of $A$ over $B$, self-sufficient in $\mathbb{M}$.
This yields, in particular, that $A$ has infinitely many $C$-conjugates, contradicting algebraicity over $C$.
\end{proof}
\begin{rem}\label{aclssc}
Let $A$ finitely extend $B$ in $\mathbb{M}$. 
Assume $$B=B^{0}
\zsu{}B^{1}\zsu{}\dots\zsu{}B^{n}=A$$
is a minimal decomposition
of $A$ over $B$. Then $\acl(B_{1})\cap A_{1}=B^{k}_{1}$, for some $1\leq k\leq n$
such that $B^{i+1}\nni B^{i}$ is a minimal algebraic extension for all $i=1,\dots,k$
and $k$ is maximal with respect to this property.
\end{rem}
\begin{proof}
As observed before, $\acl(B_{1})$ is self-sufficient in $\mathbb{M}$,
hence $\acl(B_{1})\cap A_{1}\zsu{}A_{1}$. By minimality of the extensions $B^{i}\nni B^{i-1}$
then $\acl(B_{1})\cap A_{1}=B^{k}$ for some $k$. The rest of the claim follows by
the previous proposition.
\end{proof}