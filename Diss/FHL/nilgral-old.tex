%\begin{presection}
%Among the first to establish such connection, we can cite Magnus \cite{mag} and Lazard \cite{laz}.
%We essentially follow \cite{khk} for the basic notions involving nilpotent groups, and \cite{ser} or \cite{bah} for
%what concerns Lie algebras.
%\end{presection}
%
%\medskip
We collect in this section some facts and notations from group theory and Lie algebras. We give a picture
of the {\em Magnus-Lazard} correspondence between groups and Lie rings.

\medskip
%We recall that the {\em lower central series} of the group $G$ is the normal series $%\boldsymbol{\gamma}(G)=
%(\gamma_{i}(G))_{1\leq i<\omega}$ of $G$ where $\gamma_{1}(G)$ denotes $G$ and $\gamma_{n+1}(G)=[\gamma_{n}(G),G]$.
%
%This is the most rapidly descending central chain which can be found in $G$, that is, whenever $(H_{i})_{i<\omega}$,
%is a decreasing chain of subgroups of $G$ with $H_{1}=G$ and $[H_{i},G]\inn G_{i+1}$ for all $i<\omega$, then $\gamma_{i}(G)\inn H_{i}$ for all $i<\omega$.
%
%We {\em define} a group $G$ to be {\em nilpotent of class} (at most) $c$ if
%$\gamma_{c+1}(G)=1$ for some {\em natural number} $c<\omega$.
%
We refer to the (group) word $\gamma_{k}(x_{1},\dots,x_{k})$ as the {\em left-normed} or {\em simple} group commutator of length $k$
$$[x_{1},\dots,x_{k}]=[\dots[x_{1},x_{2}],x_{3}],\dots],x_{k-1}],x_{k}]$$
where $\gamma_{1}(x)=x$ and $\gamma_{2}(x_{1},x_{2})$ is $[x_{1},x_{2}]=x_{1}^{-1}x_{1}^{x_{2}}=x_{1}^{-1}x_{2}^{-1}x_{1}x_{2}$.
We will not formally distinguish between group commutators and Lie brackets in the sequel.

For any group $G$, with $\gamma_{k}(G)$ we denote the verbal subgroup of $G$ determined by $\gamma_{k}$,
that is $\gen{[g_{1},\dots,g_{k}]\mid g_{i}\in G}$. The subgroups $G^{k}\defeq\gamma_{k}(G)$ for $k<\omega$ constitute the
{\em lower central series} of $G$, the most rapidly descending central series of $G$. For this and the subsequent notions
we refer to $\cite{khk,rob}$.

We dually define the {\em upper central sequence} $(\zeta_{k}(G))_{k<\omega}$, where the subgroup
$\zeta_{k}(G)$ is defined as the set $\{g\mid [h_{1},\dots,h_{k},g]=1,\forall h_{i}\in G\}$.

%Moreover
%if $S$ is a system of generators for $G$, then the $k^{th}$-term of the lower central series is
%the normal closure of all $\gamma_{k}$
%\begin{labeq}{commugen}
%\gamma_{k}(G)=\gen{\gamma_{k}(s_{1},\dots,s_{k})\mid s_{i}\in S}^{G}
%\end{labeq}

We denote by $\ngb{c}{}$ the variety defined by the word $\gamma_{c+1}$: all groups $G$
with $\gamma_{c+1}(G)=1$. This is the class of all groups of {\em nilpotency class} (at most) $c$.
We have $\ngb{c}{}\inn\ngb{c+1}{}$ for all $c<\omega$, and we may call for short {\em nil-$c$} groups, the objects of $\ngb{c}{}$.

If $p$ is a prime, by $\ngb{c}{p}$ we denote the variety defined by the words $\gamma_{c+1}(\bar x)$ and $x^{p}$, this is the
class of all nilpotent groups of class $c$ and of bounded exponent $p$. 

\smallskip
The lower central series, is in particular a {\em Lazard series}, that is a decreasing chain $(H^{n})_{n<\omega}$ of
subgroups in $G$, with
$H^{1}=G$ and $[H^{i},H^{j}]\sg H^{i+j}$ for all $i,j<\omega$. The properties we are going to state for
the lower central series, also hold for Lazard series in general. Any Lazard series is a central series.

%For simplicity $G^{n}$ will denote $\gamma_{n}(G)$ for any group $G$ in what follows. %,$(G^{n})$ is a Lazard series.
%If we define $\wg(g)\in\N\cup\{\infty\}$ to be the
%supremum of the set $\{n<\omega\mid g\in G^{n}\}$\footnote{we may assume that the series $(G^{n}$) is {\sl s�parante (fr.)}
%that is $\cap_{n<\infty} G^{n}=1$. In the definition of $L^{G}$ below,
%we have $L^{G}=L^{\overline{G}}$ if $\overline{G}$ is $G/{\cap_{n<\omega}G^{n}}$}. for all group elements $g$ of $G$, then we obtain an {\em integral filtration} on $G$ in the sense of \cite{ser}. We set by definition $G^{\infty}$ to be
%$\bigcap_{n<\infty}G^{n}$.
%
\medskip
The series $G=G^{1}\og G^{2}\og\dots\og G^{k}\og\dots$ allow the construction of a Lie ring from any given group, this
will be discussed below, following \cite[\S3.2]{khk} or the first chapter of \cite{laz}.
\begin{rem}
Let $x,y,z$ be elements of a group $G$ and set $\alpha\leq\omega$ %=\wg(x)+\wg(y)+\wg(z)+1$,
to be highest ordinal of $\omega\cup\{\omega\}$, such that any element among $x,y$ or $z$ belongs to $G^{\alpha}$. Here
by $G^{\omega}$ is meant $\bigcap\{G^{k}\mid k<\omega\}$.
 
Then the following, well known, identities
are true in $G$:
\begin{align}
[x,y]=[y,x]&^{-1}\label{commin}\\
[xy,z]=[x,z][y,z]&\mod G^{\alpha+1}\label{commbil}\\
[x,y,z][y,z,x][z,x,y]=1&\mod G^{\alpha+1}\tag{Witt's identity}
\end{align}
\end{rem}

Now set for all $i<\omega$
\begin{equation}
%{L^{G}}_{\!i}=
\gr_{i}G:=G^{i}/G^{i+1}\quad\text{and define}\quad\gr(G)
:=\bigoplus_{0\leq i<\omega}\!%{L^{G}}_{\!i}
\gr_{i}G\tag{$\gr$}
\end{equation}
where $\gr_{0}G\defeq\triv$.

With this definition and properties \pref{commin} and \pref{commbil}, $\gr(G)$ has a natural ring structure $(\gr(G),+,[\,\,,\,],\triv)$ where
the sum is the componentwise quotient group operation (written additively) and the product is induced by the group commutator:
$$[\boldsymbol{u},\boldsymbol{v}]=\sum_{k}(\sum_{i+j=k}[\bar u_{i},\bar v_{j}])\quad\text{for $\boldsymbol{u}=(u_{i})$ and
$\boldsymbol{v}=(v_{i})$}.$$

Witt's identity and equation \pref{commin} above turn $\gr(G)$ into a non-associative,
anti-commutative ($[a,b]=-[b,a]$ for all $a,b\in L$) ring in which the {\em Jacobi identity}
\begin{labeq}{jacob}
J(a,b,c)\defeq[[a,b],c]+[[b,c],a]+[[c,a],b]=0
\end{labeq}
holds for all $a,b,c$ in $L$. That is $\gr G$ is a {\em Lie ring} (a $\Z$-Lie algebra) according to the
next definition. Notice that $\gr(G)\simeq\gr(G/G^{\omega})$.

\begin{dfn*}
If $\cur$ is a commutative unitary ring, a {\em Lie algebra $L$ over $\cur$} or
a Lie $\cur$-algebra is a $\cur$-module
endowed with a $~k$-bilinear map $[\,\,,\,]$ which factorises through $\exs L$, that is $[a,a]=0$
for all $a$ in $L$, and such that the Jacobi identity $J(a,b,c)=\triv$ is satisfied for all $a,b,c$ in $L$.
\end{dfn*}

For a subset $S$ of a Lie $\cur$-algebra $L$ we denote by $\gen{S}$ or $\gena{S}{L}$ the subalgebra
generated by $S$ in $L$ while $\gen{S}_{\cur}$ denotes the $\cur$-submodule of $L$ generated by $S$.
The product $[S,T]$ of subsets $S$ and $T$ of $L$ is $\genp{[s,t]\mid s\in S,t\in T}$,
while the ideal generated in $L$ by $S$ is denoted by $\genid{S}{L}$. This is -- by means of anti-commutativity and repeated applications
of the Jacobi identity -- also $\genp{S,[S,L],[S,L,L],\dots}$.

Exactly like for groups, we define the terms of the lower central series $\gamma_{n}(L)$ of $L$ recursively as $[\gamma_{n-1}(L),L]$ for
all $n<\omega$ where $\gamma_{1}(L)=L$. These builds a decreasing chain of ideals of $L$ and
$\gamma_{n}(L)=\gen{\gamma_{n}(s_{1},\dots,s_{n})\mid s_{i}\in L}_{\cur}$. The definition
of upper central series is again exactly the same defined for groups.

We say that $L$ is {\em nilpotent of class} (at most) $c$, if $\gamma_{c+1}(L)=\triv$.

If a Lie $\cur$-algebra $L$ is generated by a set $S$, then an inductive argument on Jacobi identities shows, that $L$ is generated
as a $\cur$-module, by all the {\em simple monomials} with entries in $S$, that is by left-normed
commutators $[s_{1},\dots,s_{k}]$ of {\em weight} or {\em lenght} $k$, for all $k<\omega$. In particular $\gamma_{n}(L)$ is the ideal generated by all simple monomials
%$[s_{1},\dots,s_{n}]$
of length $n$ in elements of $S$.
%%\footnote{for the $S$-weight to be well defined one should also
%%require for all $s\in S$, $s\notin\gen{S\non\{s\}}$.}
%%
%%We recursively define {\em monomials of $S$-weight $n$} as follows. Put $\wg_{S}(\triv)=0$ and
%%$\wg_{S}(s)=1$ for all $s\in S$. Assume all monomials $m$ of $S$-weight less than $n$ are defined and identified
%%by writing $\wg_{S}(m)=k$ for some $k<n$. Then an element $m$ of $L$ is of $S$-weight $n$ -- short $\wg_{S}(m)=n$ -- if
%%$w=[a,b]$ for $a,b$ monomials of $S$-weight less than $n$ such that $\wg_{S}(a)+\wg_{S}(b)=n$.
%%
%%We denote by $L_{S,i}$, the submodule
%$\gen{a\in L\mid\wg_{S}(a)=i}_{\cur}$ and call it the {\em homogeneous submodule of $S$-weight $i$}.
%%\end{dfn}
%Remark that in general $L_{S,i}\cap\sum_{j\neq i}L_{S,j}$ may not be trivial and $\wg_{S}$ may not be a function,
%this is however the case for free Lie algebras and for the object of the class $\nla{c}$  below.
%We call an ideal $R$ of $L$ {\em homogeneous} (in $S$) if $R=\sum_{i}(R\cap L_{S,i})$.
%\begin{rem}
%For a generating system $S$ of $L$ we have $L=\sum_{i<\omega}L_{S,i}$ and $\gamma_{n}(L)=\sum_{i\geq n}L_{S,i}$.
%\end{rem}


\begin{rem}\label{gradigroup}
For all $G$ in $\ngb{c}{p}$, the Lie ring $\gr(G)$ carries a natural
Lie $\Fp$-algebra structure which is nilpotent of class $c$.

Moreover $\gr(G)$ is generated as a ring, by the {\em abelianised} group $G_{ab}$: the quotient
of $G$ modulo $G^{\prime}=\gamma_{2}(G)$. This follows by the natural
surjective morphism of $\otimes_{\Z}^{n}G_{ab}$ onto $\gr_{i}G$ (see \cite[\S2]{khk}).

In particular $\gr(G)$ is a {\em graded algebra}: it is the direct sum of its {\em homogeneous submodules} $\gr_{i}G$
{\em of weight $i$ in $G_{ab}$}, for all $i<\omega$ and these are such that $[\gr_{i}G,\gr_{k}G]\inn\gr_{i+k}G$.
\end{rem}




\subsubsection*{Free Algebras and basic commutators}
For the following definitions we follow \cite{ser} and \cite{bah}.

\smallskip
For any set $X$, the {\em free magma} $(\mathcal{M}(X),\cdot\,)$ is -- roughly speaking --
the image of $X$ under the {\em free functor} $\mathcal{M}$ from {\em sets} to the category
%the free object in the {\em category}
of all structures which interpret a signature consisting of
%constants $X$ and
a binary operation only.

%inductively defined as the disjoint union over $n<\omega$, of all the
%$$\mathcal{M}_{n}(X):=\amalg_{p+q=n}\mathcal{M}_{p}(X)\times\mathcal{M}_{q}(X)$$
%where $\mathcal{M}_{1}(X)=X$. In this way we obtain the natural binary operation
%\begin{align*}
%\mathcal{M}(X)\times\mathcal{M}(X)&\longmapsto\mathcal{M}(X)\\
%u,v&\longmapsto(u,v)=:u\cdot v
%\end{align*}

The elements of $\mathcal{M}(X)$ which are referred to as {\em non-associative words over $X$}
are the disjoint union of $\omega$ subsets $\mathcal{M}_{n}(X)$, each one collecting the {\em words of weight} or {\em length} $n$, for $n\geq1$ (cfr.\,\cite[\S2]{bbk}). We have $\mathcal{M}_{1}(X)=X$ and the product $\cdot$ maps $\mathcal{M}_{i}(X)\times\mathcal{M}_{k}(X)$
into $\mathcal{M}_{i+k}(X)$.

\smallskip
Define the {\em free $\cur$-algebra on $X$} as the free $\cur$-module $\mathcal{F}(X,\cur)$ with
basis $\mathcal{M}(X)$ and with a $\cur$-bilinear multiplication $\cdot$ extended from the product on $\mathcal{M}(X)$,
which in particular makes $(\mathcal{F},\cdot,\triv)$ a non-associative ring without unit, such that
$(ta)\cdot b=t(a\cdot b)=a\cdot tb$, for all $t\in\cur$ and all $a,b$ from $\mathcal{F}$.

If $\mathcal{F}=\mathcal{F}(X,\cur)$ then $\mathcal{M}(X)$ induces a natural grading on $\mathcal{F}$
given by $\mathcal{F}=\bigoplus_{n<\omega}\mathcal{F}_{n}$ for $\mathcal{F}_{n}:=\gen{\mathcal{M}_{n}(X)}_{\cur}$.
Each element $a$ of $\mathcal{F}$ is henceforth expressible in a unique manner as a finite sum $\sum a_{n}$, where
$a_{n}\in\mathcal{F}_{n}$ are called the {\em homogeneous components} of $a$.

\smallskip
Let now $\mathcal{A}$ and $\mathcal{B}$ be the ideals of $\mathcal{F}$ respectively generated by the
sets $\{(u\cdot v)\cdot w-u\cdot(v\cdot w)\mid u,v,w\in\mathcal{F}\}$ and $\{u\cdot u, J(u,v,w)\mid u,v,w\in\mathcal{F}\}$
where $J$ is the homogeneous term associated to the Jacobi identity \pref{jacob}.
We define
$$A^{\rm o}(X,\cur)=\mathcal{F}(X,\cur)/\mathcal{A}\quad\text{and}\quad L(X,\cur)=\mathcal{F}(X,\cur)/\mathcal{B}$$
as respectively the {\em free associative} and the {\em free Lie} algebra {\em on $X$ over $\cur$}.

In \cite[2.1]{bah} is proved, that both $\mathcal{A}$ and $\mathcal{B}$ are {\em homogeneous ideals},
that means, $\mathcal{A}=\sum_{i}\mathcal{A}\cap\mathcal{F}_{i}$ and $\mathcal{B}=\sum_{i}\mathcal{B}\cap\mathcal{F}_{i}$.
It follows $A^{\sss 0}(X,\cur)$ and $L(X,\cur)$ inherit from $\mathcal{F}$ the grading:
\begin{labeq}{gradal}
A^{\sss 0}(X,\cur)=\bigoplus_{i\geq1}\mathcal{F}_{i}/\mathcal{A}\cap\mathcal{F}_{i}\quad\text{and}\quad
L(X,\cur)=\bigoplus_{i\geq1}\mathcal{F}_{i}/\mathcal{A}\cap\mathcal{F}_{i}.
\end{labeq}

We denote  by $A_{i}=A_{i}(X,\cur)$ and $L_{i}=L_{i}(X,\cur)$ %, for $1\leq i<\omega$
the $\cur$-submodules in the grading above, and call them {\em homogeneous} submodules of weight $i$.

Each element of $A_{i}$ or $L_{i}$ is generated by the {\em homogeneous monomials} of weight $i$,
that is, of images in $A^{\rm o}$ and $L$ of words from $\mathcal{M}_{i}(X)$.

It is customary to speak of {\em degree} $i$ for the homogeneous elements of $A_{i}(X,\cur)$, rather than of weight.
\begin{rem}\label{asslie}
Any Lie $\cur$-algebra $M$ is the image of a free Lie algebra $L(X,\cur)$ for some $X$.
Moreover any associative
algebra $A$ is endowed with a product $[a,b]=ab-ba$ which turns $(A,+,[\,\,,\,])$ into a Lie algebra.
As a consequence, there exists a natural Lie $\cur$-algebra homomorphism $\epsilon$ of $L(X,\cur)$ onto the Lie subalgebra
of $A(X,\cur)$ generated by $X$ which extends $x\mto x$ for $x\in X$.
\end{rem}

\medskip
Whether for the $\cur$-module $A(X,\cur)$ a $\cur$-basis is easy to find, for $L(X,\cur)$ instead, we recur
to the so called basic monomials, also called {\em Hall's Families}. In the groups context the very same definition applies to {\em basic
commutators}.
\begin{dfn}[Basic Monomials]\label{basicommutators}
We inductively construct a linearly ordered set of Lie monomials $\mathscr{B}=\bigcup_{n\geq1}\mathscr{B}_{n}$,
where each $\mathscr{B}_{n}\inn L_{n}(X,\cur)$ will be called, by definition, the set of {\em basic} monomials of weight $n$.

Let $\mathscr{B}_{1}$ coincide with some linear order on $X$.

Assume a set of {\em basic monomials} $\mathscr{B}_{<n}$ {\em of weight less than $n$} has been defined and totally
ordered, by choosing a linear ordering for each $\mathscr{B}_{k}$ and following the rule: if $a>b$, then necessarily
the weight of $a$ is grater than weight of $b$, for all $a,b\in\mathscr{B}_{<n}$.

Now for any pair of monomials $u,v$ in $\mathscr{B}_{<n}$ whose weights summed give $n$, % equals the sum of their %$\wgt(u)+\wgt(v)=n$
the product $[u,v]$ is a {\em basic monomials of weight} $n$ and lays in $\mathscr{B}_{n}$ if both of the
following conditions are satisfied:
\begin{itemize}
\item[-]$u>v$,
\item[-]if $u=[z,w]$ for $z,w\in\mathscr{B}_{<n}$, then $w\leq v$.
\end{itemize}
\end{dfn}

The following theorem is referred to as {\em Hall's Basis Theorem}.\footnote{Although ambiguous, the name doesn't harm
the fatherhood of both Philip and Marshall.

To the former, the so called {\em collecting process} (see \cite{mhall}) is attributed, from which Definition \ref{basicommutators} arise.
It is an algorithm to stepwise transform a group word into an ordered expression of basic commutators \cite{phall}.

In \cite{mhalll}, the latter author describes a collecting process in the context of Lie rings but claims the same result to hold
for Lie algebras over any field.

The proof in \cite{bah} -- attributed to A.{}I.\,Shirshov -- holds for arbitrary commutative rings.}
\begin{teo}[{\cite{mhalll},\cite[Theorem 2.2.1]{bah}}]\label{hallstheor} 
Let $\mathscr{B}$ a set of basis commutators on $X$ in $L=L(X,\cur)$. Then $L$ is a free $\cur$-module with
basis $\mathscr{B}$.

In particular each homogeneous submodule $L_{n}(X,\cur)$ is free, with basis $\mathscr{B}_{n}$ for all $n\geq1$. 
\end{teo}
For a clear account of the Group theoretical analogous around Hall's Theorem and the {\em collecting process}
we refer to \cite{khk}.

As a corollary to the above theorem in \cite{bah} we find
\begin{fact}\label{ellea}
The canonical Lie morphism $\epsilon$ of Remark \ref{asslie}
mapping $L(X,\cur)$ into $A(X,\cur)$ is a Lie algebra embedding.
\end{fact}
The above $\epsilon$, coincide with the canonical map of a Lie algebra $L$ in its
{\em universal envelope} $U(L)$ (see also \cite{ser}), this is always an embedding provided
$L$ is a free $\cur$-module.

The following fact will also be needed.
\begin{fact}[{\cite[Lemma 2.3.3]{bah}}]\label{basisgen}
Let $L$ be the free Lie algebra $L(X,\cur)$ over the set $X$. If $\mathcal{B}$ is a basis
for the free $\cur$-submodule $\gen{X}_{\cur}=L_{1}(X,\cur)=\mathcal{F}_{1}(X,\cur)$ of $L$, then $L(\mathcal{B},\cur)=L$.
\end{fact}

\bigskip
We introduce below the class of nilpotent Algebras, for which a notion of Hrushowski predimension is constructed. 
We show that these structures isolate exactly those Lie algebras which arise from $\ngb{c}{p}$-groups as images under the functor
$\gr$.

\begin{dfn}\label{lcp}
For a prime number $p$ greater than $c$, we define by $\nla{c}$ the class of all $c$-nilpotent (graded)
Lie algebras $M$ over the $p$-element field $\Fp$, which satisfy the following properties:
\begin{enumerate}
\item there are $\Fp$-subspaces $M_{i}$ for $1\leq i\leq c$  with $M=M_{1}\oplus\dots\oplus M_{c}$,
\item$[M_{i},M_{j}]\inn M_{i+j}$ for all $i,j$
\item\label{genero}$M=\gen{M_{1}}$
\end{enumerate}
\end{dfn}
\begin{rem}
The whole grading $M_{i}$ depends indeed only on the {\em choice} for the space $M_{1}$: by property $3.$
above each subspace $M_{i}$ is the $\Fp$-subspace of $M$ generated by (simple) monomials of weight $i$ in elements from $M_{1}$.
\end{rem}

\smallskip
Lie subalgebras of an $\nla{c}$-algebra $M$ are not always again $\nla{c}$-objects,
we hence {\em define} an $\nla{c}$-subalgebra $H$ of $M$, if $H=\gena{H_{1}}{M}$ for some $\Fp$-subspace
$H_{1}$ of $M_{1}$.

By an $\nla{c}$-morphisms we mean a graded homomorphism of Lie algebras $\map{\phi}{M}{N}$, that is
%$\phi(M_{1})\inn N_{1}$. Definition \ref{lcp} implies $\phi$ is a {\em graded morphism}, that is
$\phi(M_{i})\inn N_{i}$ for all $i$.

\begin{rem*}
As observed above, we get a correspondence
\begin{labeq}{grafu}
\map{\gr}{\ngb{c}{p}}{\nla{c}}
\end{labeq}
under which the abelianised group $G_{\rm ab}$ of $G$ corresponds to $\gr(G)_{1}$ of Definition \ref{lcp}.
Under this correspondence some morphisms are lost: $\gr$ is neither a full, nor a faithful functor. In the
next sections, it will be shown that $\gr$ is onto of the respective objects.
\end{rem*}

For a first-order treatment of the category $\nla{c}$ we choose the signature $\Lan{c}$ which consists of
the ring symbols $\triv$, $+$ and $[\,\,,\,]$, \uwave{of the scalar functions from $\Fp$}\mn{\bf Redundant (?), nach Ziegler} and of predicates $P_{i}$ which are interpreted by the grading ($M^{P_{i}}=M_{i}$).
Notice that $\nla{c}$ is not an elementary $\Lan{c}$-class. Property \ref{genero}.\,in Definition \ref{lcp} cannot be expressed
unless a bound to the length of sums in $M$ is given.

\medskip
By the previous discussion, for any set $X$, the {\em free $c$-nilpotent Lie $\Fp$-algebra over $X$}, which we {\em define} as
$$\fla{c}{X}:=L(X,\Fp)/\gamma_{c+1}(L(X,\Fp))$$ is an object of $\nla{c}$.
This is for $L(X,\Fp)$ is a graded Lie algebra and $\gamma_{c+1}(L(X,\Fp))$ is an homogeneous ideal which is equal to $\sum_{i>c}L_{i}(X,\Fp)$.
Similarly, for any object $M$ of $\nla{c}$, $\gamma_{i}(M)=\sum_{j\geq i}M_{j}$.

As a corollary to Hall's Theorem above we get (cfr.\,\cite[Corollary 2.7.3]{khk})
\begin{fact}\label{ubc}
For any given set $\mathscr{B}$ of basic monomials over $X$, % in $L(X,\Fp)$,
denote by $\mathscr{B}_{\!{\sss\leq} c}$
the set of elements in $\mathscr{B}$ of weight not greater than $c$. Then $\mathscr{B}_{\!{\sss \leq} c}$ is
a $\Fp$-basis of $\fla{c}{X}$ and in particular
every element $w$ of $\fla{c}{X}$ admits a unique expression
\begin{gather*}\label{BC}\tag{\sf BC}
\left.\begin{split}
w=\sum s_{\alpha}{\bf b}_{\alpha}\quad\text{with ${\bf b}_{\alpha}$ in $\mathscr{B}_{\!{\sss \leq} c}$}
\end{split}\right.
\end{gather*}
\end{fact}
\begin{dfn}\label{supp}
If $\mathscr{B}_{\leq c}$ and $\fla{c}{X}$ are as above, we define the {\em support} of an element $w\in\fla{c}{X}$,
as the minimal subset $\supp(w)$ of $X$, for which each basic monomial $\boldsymbol{b}_{\alpha}$ in the sum \pref{BC}
which appears with a nontrivial $s_{\alpha}$, is built from elements of $\supp(w)$ according to Definition
\ref{basicommutators}. We may specify the set over which $\mathscr{B}$ is constructed, by writing $\supp_{X}(w)$. 

If a basic monomial $\boldsymbol{b}\in\mathscr{B}$, has support in a subset $Y$ of $X$, we refer to
$\boldsymbol{b}$ also as a monomial {\em over $Y$}, or shortly, as a basic $Y$-monomial.
\end{dfn}

\begin{rem}
With Fact \ref{basisgen}, the functor $L^{c}(\,\cdot\,)$ may be seen as a {\em free} functor of %universal property in
$\Fp$-vector spaces into $\nla{c}$-algebras, adjoint to the predicate $P_{1}$ of $\Lan{c}$:
for any $\Fp$-vector space $V$ and all $\nla{c}$-algebras $M$, we have -- with the obvious maps -- a bijection
%, to any $\Fp$-linear map $f$ of $M_{1}$ into $N_{1}$
%corresponds a unique $\nla{c}$-morphism $\phi$ of $\fla{c}{M_{1}}$ into $N$, such that $f=\iota\phi$ where
%$\iota$ denotes the $\Fp$-linear embedding fo $M_{1}$ inside $\fla{c}{M_{1}}$.
%
%The correspondence $f\mapsto\phi$ gives rise to a \uwave{natural} adjunction, witnessed by an
%isomorphism of $\Fp$-modules
\begin{labeq}{freeadjoint}
\Hom_{\Fp}(V,P_{1}(M))\to\Hom_{\nla{c}}(\fla{c}{V},M).
\end{labeq}
\end{rem}

\smallskip
In particular any object $M$ of $\nla{c}$ is the quotient of $\fla{c}{M_{1}}$ modulo an {\em homogeneous ideal}
$R=\sum_{i\leq c}R\cap L_{i}(M_{1})$ with $R\cap L_{1}(M_{1})=R\cap M_{1}=\triv$.

Since the subspace $M_{1}$ is intrinsic to the structure $M$, the choice of the relators ideal $R$ may be regarded
as canonically associated to $M$. By an {\em $\nla{c}$-presentation} of $M$ we denote both the expression
$M=\gen{M_{1}\mid R}$ as well as the associated exact sequence
\begin{labeq}{pres}
R\linto\fla{c}{M_{1}}\stackrel{\epsilon}{\lonto}M
\end{labeq}

On the other hand, to any homogeneous ideal $R$ of $L=\fla{c}{X}$, for some set $X$, the quotient $L/R$ is an object of $\nla{c}$.

We say that $M$ is finitely generated if $M_{1}$ has finite $\Fp$-dimension, hence exactly if $M_{1}$ is finite.
Note that in the category $\nla{c}$ the notion of {\em finitely presented} (that is $M_{1}$ and $R$ are
finite dimensional) coincide with being finitely generated. The same holds in general for nilpotent groups.

\medskip
We may also remark, that morphisms among objects of $\nla{c}$ aren't richer than those among their generating $\Fp$-vector spaces
$P_{1}$.

%Let $M$ and $N$ be presented from $\fla{c}{M_{1}}$ and $\fla{c}{N_{1}}$
%by means of epimorphisms $\epsilon_{M}$ and $\epsilon_{N}$ respectively as in \pref{pres}.
%
%Assume an $\nla{c}$-morphism $\phi$ of $M$ into $N$ is given.
%
%If $\phi_{1}$ denotes the restriction of $\phi$ to the subspace $M_{1}$ with image in $N_{1}$, then
%we may assume that $\map{\phi_{1}}{M_{1}}{\fla{c}{N_{1}}}$. Now apply \pref{freeadjoint} to $\phi_{1}$
%to get a morphism $\map{\widehat\phi}{\fla{c}{M_{1}}}{\fla{c}{N_{1}}}$. It is clear that the following holds.

\begin{lem}\label{commufreeno}
For any $M$ and $N$ in $\nla{c}$,
%once two presentations $\gen{M_{1}\mid R_{M}}$ and $\gen{N_{1}\mid R_{N}}$ as above are chosen,
to any $\nla{c}$-morphism $\phi$ of $M$ to $N$, there is a unique $\widehat\phi\in\Hom_{\nla{c}}(\fla{c}{M_{1}},\fla{c}{N_{1}})$
%As $\phi$ and $\phi_{1}\epsilon_{N}$ coincide on $M_{1}$ and $\phi_{1}=\iota_{M}\widehat\phi$,
which makes the square below commute
\begin{labeq}{commufresco}
\begin{split}
\xymatrix{
\fla{c}{M_1}\ar[r]^{\widehat\phi}\ar[d]^{\epsilon_{M}}&\fla{c}{N_{1}}\ar[d]^{\epsilon_{N}}\\
M\ar[r]^{\phi}&N
}
\end{split}
\end{labeq}
\end{lem}

\subsection*{Relations between the $\ngb{c}{p}$-free  group and the free $\nla{c}$-algebra}
%We will describe two ways by which is possible to recover a group from a given Lie algebra,
%the first method, which is more group theoretical furnishes a functor $\mathscr{G}$ of $\nla{c}$
%to $\ngb{c}{p}$ %``{inverse}'' to $\gr$
%such that for all groups $G$ in $\ngb{c}{}$ gives $\mathscr{G}(\gr G)=G$ and $\gr(\mathscr{G}(L))=L$ for all
%nilpotent Lie rings $L$. 
%
%The second process uses more {\em topological} Campbell Hausdorff formula to
%equip a Lie ring with a a group structure, which in the nilpotent case allows a model theoretical interpretation
%of groups in Lie rings.

Before we prove that \pref{grafu} is onto $\ngb{c}{p}$-groups, we first establish a correspondence between the free objects in the
classes $\nla{c}$ and $\ngb{c}{p}$.

\smallskip
Let $A^{\sss 0}(X)$ be the free associative algebra $A^{\sss 0}(X,\Fp)$ over $\Fp$ defined above. We add a
multiplicative unit -- and hence {\em elements of zero degree} -- by defining
$A(X)$ to be $\Fp\oplus A^{\sss 0}(X)$ and extending addition and multiplication in the natural way.
$A(X)$ inherits the grading \pref{gradal} and we set $A_{0}(X)=\Fp$.


Let $A^{c}(X)$ be the quotient algebra of $A(X)$ modulo the ideal $\sum_{i>c}A_{i}(X)$,
that is the {\em free associative nilpotent} algebra with unit over $X$ of class $c$, also
$$A^{c}(X)\simeq\Fp\oplus A_{1}(X)\oplus\cdots\oplus A_{c}(X).$$

If $\epsilon$ is the canonical embedding of Fact \ref{ellea}, since we have $\epsilon(\gamma_{k}(L(X)))=
\epsilon(L(X))\cap\sum_{i\geq k}A_{i}(X)$ for all $1\leq k$.
This means $\epsilon$
factorises to a Lie monomorphism
of the free nilpotent Lie algebra $L^{c}(X)$ into $A^{c}(X)$.
%We therefore identify,
%$L^{c}(X)$ with the $c$-nilpotent, Lie subalgebra of $A^{c}(X)$ generated by $X$.

Let now $F_{\sss p}(X)$ denote the free group of exponent $p$ on the set $X$, then $F_{\sss p}^{c}(X):=F_{\sss p}(X)/\gamma_{c+1}(F_{\sss p}(X))$
is the free group
in $\ngb{c}{p}$ over the set $X$.

Now for all $x$ in $X$, since in $A^{c}(X)$, $(1+x)^{p}=1+px=1$, one can extend
the map $X\ni x\mapsto1+x$ to a group homomorphism $\phi$ of $F_{\sss p}(X)$ onto the subgroup $\gen{1+x\mid x\in X}$
of the group of units $A^{c}(X)^{*}$ (the multiplicative inverse of some $1+a$ being $1-a+a^{2}-a^{3}+\cdots$).

\smallskip
We introduce the following notation, for all words $w$ of $F_{\sss p}(X)$
$$\phi\colon w\longmapsto 1+\lambda(w)+W$$
where $\lambda(w)$ is the %$n^{\text{th}}$
homogeneous component $\phi(w)_{n}\in A_{n}(X)$ of $\phi(w)$ of minimal degree $0<n\leq c$,
to be preceded by zero components only. That is if
$$\phi(w)=1+\phi(w)_{1}+\cdots+\phi(w)_{c}\quad\text{with $\phi(w)_{i}$ in $A_{i}(X)$}$$
then $\phi(w)_{1}=\dots=\phi(w)_{n-1}=\triv$. $W$ is the sum of the
remaining components of higher degree and $n$ is called the {\em weight} of $w$.
%Set $\wg(w)=i$ if $\lambda(w)\in A^{c}_{i}(X)$.

On this theme, one should also consult \cite[IV.6]{ser}, \cite{mag37} and \cite{mag}.

\begin{fact}[{\cite{witt},\cite{mag},\cite[Lemma 11.2.2]{mhall}}]\label{lambda}
For any word $w$ of $F_{\sss p}(X)$, denote its weight by $\wgt(w)$. It holds then
\begin{itemize}
\item[-]$\lambda(gh)$ is $\lambda(g)$ or $\lambda(h)$ according to which among $g$ and $h$ has lower weight.
If $g$ and $h$ do have the same weight, and if $\lambda(g)+\lambda(h)\neq\triv$, then $\lambda(gh)=\lambda(g)+\lambda(h)$.
In any case $\lambda(gh)\geq\max(\lambda(g),\lambda(h))$.
\item[-]$\lambda(g^{-1})=-\lambda(g)$.
\item[-]If $[\lambda(g),\lambda(h)]\neq\triv$, then $\lambda([g,h])=[\lambda(g),\lambda(h)]$.
%The weight of  $[g,h]$ is not smaller than $\wg(g)+\wg(h)$.
$\wgt([g,h])\geq\wgt(g)+\wgt(h)$ and if $[\lambda(g),\lambda(h)]=\triv$, then $\wgt(g)=\wgt(h)$ and $\lambda(g^{h})=\lambda(g)$.
\end{itemize}
Here above, the $[\lambda(g),\lambda(h)]$ denotes the Lie product $\lambda(g)\lambda(h)-\lambda(h)\lambda(g)$
in the associative algebra $A^{c}(X)$.
\end{fact}

By the fact above follows that $\phi(\gamma_{c+1}(F_{\sss p}(X)))=1$ and hence we may factorise $\phi$ to a group
homomorphism
$$\map{\phi}{F_{\sss p}^{c}(X)}{A^{c}(X)^{*}}.$$
Moreover, we still have a well defined map $\lambda\colon\bar w\mapsto\lambda(w)$ for all $\bar w=\gamma_{c+1}(F_{\sss p}(X))w$ in
$F_{\sss p}^{c}(X)$.

\smallskip
Now for $1\leq n\leq c+1$ %for $1\leq n\leq c+1$
%let ${\mathfrak m}_{n}$ be the ideal $\sum_{i\geq n}A^{c}_{i}(X)$. And
set $H^{n}$ to be the set of all $g$ in $F^{c}_{\sss p}(X)$ with $\wgt(g)\geq n$. Then $H^{1}=F^{c}_{\sss p}(X)$ and
%$\lambda(g)\in\sum_{i\geq n} A_{i}(X)$.
Fact \ref{lambda} above implies $H^{n}$ are subgroups of $F^{c}_{\sss p}(X)$ with $[H^{i},H^{j}]\inn H^{i+j}$ for all $i+j\leq c+1$,
%and $H^{c}=1$,
that is, if we set $H^{i}=\{1\}$ for all $i>c+1$, then $(H^{i})_{i<\omega}$ is a Lazar series, as defined above. %of length $c$.
This is in particular a central series and hence $\gamma_{i}(F^{c}_{p}(X))\inn H^{i}$.

If we assume $p>c$ then the same arguments used in a theorem of \cite{ser} which is proved in characteristic $0$,
applies to our means:
\begin{fact}[{\cite[Theorem 6.3]{ser}}]\label{serreth} If $p>c$, then in the above notations we have
$$H^{i}=\gamma_{i}(F^{c}_{p}(X))$$
for all $i$.
\end{fact}
\uwave{The assumption on $p$ is needed, to prevent that commutators of length $p$ to be trivial}. See on this purpose \cite[\S3.3]{khk}. 


It follows in particular $\ker(\phi)=H^{c+1}=\gamma_{c+1}(F^{c}_{p})=1$.
%Let $\mathcal{H}$ be the Lie ring obtained from the Lazard series $(H^{i})$ as in $(\gr)$ above.
%We have in particular a canonical map of $\gr(F^{c}_{p}(X))$ into $\mathcal{H}$ and the map $\phi$ above
%is a monomorphism.

\smallskip
Now by means of $\lambda$ with facts \ref{lambda} and \ref{serreth} we obtain for all $i$, an $\Fp$-linear monomorphism
$\lambda_{i}$ of the $\Fp$-vector space $H^{i}/H^{i+1}$ into
$A_{i}(X)$, given by $\lambda_{i}\colon\bar w\mapsto\lambda(w)$. This
gives rise to a Lie homomorphism $\tilde\lambda$ of $\gr F_{\sss p}^{c}(X)$ into the Lie subring of $A^{c}(X)$ generated by $X$.

Now since $\gr F_{\sss p}^{c}(X)$ is generated by the image of $X$ modulo $H^{2}$ and is a nilpotent
Lie algebra of class $c$, there is a Lie epimorphism $\eta$
of $L^{c}(X)$ onto $\gr F_{\sss p}^{c}(X)$. Since the image of $X$ under the composition
$$L^{c}(X)\stackrel{\eta}{\longrightarrow}\gr F_{\sss p}^{c}(X)\stackrel{\tilde\lambda}{\longrightarrow}A^{c}(X)$$
coincide with that of $\map{\epsilon}{L^{c}(X)}{A^{c}(X)}$ above, one obtains $\eta\tilde\lambda=\epsilon$ and hence
the following
\begin{fact}\label{lambdiso}
$\gr F_{\sss p}^{c}(X)$ is $\nla{c}$-isomorphic as a Lie $\Fp$-algebra with $L^{c}(X)$, and maps via $\tilde\lambda$ isomorphically onto the Lie subring of $A^{c}(X)$ generated by $X$.

Moreover $\tilde\lambda$ maps $H^{i}/H^{i+1}$ $\Fp$-isomorphically over $L_{i}(X)$.
\end{fact}


\subsubsection{Retrieving groups from $\nla{c}$-algebras}\label{algegruppi}\mn{\bf to be shortened a lot}

Now we are able to prove the desired correspondence between $\nla{c}$ and $\ngb{c}{p}$ as follows.
Let $M$ be a Lie algebra of $\nla{c}$. As observed above $M$ is isomorphic to the quotient $\fla{c}{X}/R$,
where $X$ is an $\Fp$-basis of $M_{1}$ and $R$ is an homogeneous ideal of $\fla{c}{X}$.

The idea is associate $M$ to a group $G$ of $\ngb{c}{p}$, which is a quotient of $F^{c}_{p}(X)$:
we need to find a suitable normal group.

Define %$N$ to be the subset of $F_{\sss p}^{c}(X)$, which consists of all $w$ with $\lambda(w)\in R$,
$$N=\left\{w\in F_{ p}^{c}(X)\mid\lambda(w)\in R\right\}$$
then by Fact \ref{lambda}, as $\lambda(hg^{-1})$ equals $\lambda(h)$ or $-\lambda(g)$ or
again $\lambda(h)-\lambda(g)$, $N$ is a subgroup of $F_{p}^{c}(X)$.

Moreover, the same fact implies that for all $g$ in $N$ and all $x$ in $X$, %$\lambda(g^{x})=
either $\lambda(g^{x})=\lambda(g)$ or $\lambda([g,x])=[\lambda(g),x]$ is in the ideal $R$. This yields that $N$ is a normal subgroup of $F_{\sss p}^{c}(X)$. Hence the quotient $F_{\sss p}^{c}(X)/N$ which we denote by $\mathscr{G}(M)$, is a group in the variety $\ngb{c}{p}$. We omit to show that $\mathscr{G}(M)$ actually does not depend of the particular presentation $\gen{X\mid R}$ of $M$. 

\medskip
We can now prove the following
\begin{teo}
Let $p$ be a prime number greater than $c$.
There  is a functor $\mathscr{G}$ of the category $\nla{c}$
into the category of $\ngb{c}{p}$-groups such that $\gr\,\mathscr{G}(M)\simeq M$, for all $M$ in $\nla{c}$,
which is defined through $M\mto F_{\sss p}^{c}(X)/N$.

\cbstart
For a fixed $M$ in $\nla{c}$, then $\nla{c}$-subalgebras (ideals) of $M$ correspond
to subgroups (normal subgroups) of $\mathscr{G}(M)$.
Also $\mathscr{G}$ let the terms of the lower central series of $M$
correspond to those of $\mathscr{G}(M)$ and preserves the
exact nilpotency class of $M$.
\cbend

%\begin{itemize}
%\punto{i}$\gr\,\mathscr{G}(M)\simeq M$, for all $M$ in $\nla{c}$ and 
%\punto{ii}\sout{$\mathscr{G}(\gr G)\simeq G$, for all groups $G$ of $\ngb{c}{p}$.}
%\end{itemize}
\end{teo}
\begin{proof}
For (i) let $M=\gen{X\mid R}$, where $X$ is an $\Fp$-basis of $M_{1}$ and $G=\mathscr{G}(M)$. Hence $G$ is the quotient $F/N$ of $F=F_{\sss p}^{c}(X)$
modulo the normal subgroup $N$ defined above. Also $M=L^{c}(X)/R$ and since $R$ is a homogeneous
ideal, then for all $n$, $M_{n}\stackrel[]{}{\simeq}L_{n}(X)/R_{n}$ as $\Fp$-vector spaces.

It is immediate to verify that, if $F^{n}$ denotes $\gamma_{n}(F)$ for all $n\leq c$, then
\begin{labeq}{gammapres}
\gamma_{n}(G)/\gamma_{n+1}(G)\simeq F^{n}/F^{n+1}(F^{n}\cap N).
\end{labeq}
On the other hand, Fact \ref{lambdiso} and the definition of $N$ imply that the $\Fp$-isomorphism $\map{\lambda_{n}}
{F^{n}/F^{n+1}}{L_{n}(X)}$ maps $F^{n+1}(F^{n}\cap N)/F^{n+1}$ exactly onto $R_{n}$.
It follows $M_{n}$ is isomorphic to $\gamma_{n}(G)/\gamma_{n+1}(G)$ as an $\Fp$-vector space, for all $n$. Moreover,
since the Lie algebra structure on $\gr G$ is given by the quotient of $\gr F\simeq L^{c}(X)$ modulo
the ideal $\sum_{n} F^{n+1}(F^{n}\cap N)/F^{n+1}$, then $\gr G$ is $\nla{c}$-isomorphic to $M$.

\smallskip
The remaining statements directly descend from the definition of $\mathscr{G}$.
%\medskip
%To get (ii), present a group $G$ of $\ngb{c}{p}$ by means of the $\ngb{c}{p}$-free group $F=F_{\sss p}^{c}(X)$ modulo some normal
%subgroup $N$, for some set of generators $X$. Then if $M$ denotes $\gr G\in\nla{c}$, $M_{1}\simeq F_{ab}$ is a $\Fp$-vector space with basis
%$X$.
%subgroup
\end{proof}
\subsubsection{The Baker-Hausdorff Formula}
We mention here that a filtration $({\mathfrak g}_{i})_{i<\omega}$ of a $\cur$-algebra ${\mathfrak g}$ is a decreasing series of ideals
${\mathfrak g}_{i}$ with ${\mathfrak g}_{i}\cdot{\mathfrak g}_{j}\inn{\mathfrak g}_{i+j}$. 

In fact the lower central series $(\gamma_{k}(L))_{k}$ of a Lie algebra $L$ constitutes an example of (central) filtration.
We say that ${\mathfrak g}$ is {\em separated} with respect to the filtration $({\mathfrak g}_{i})$ if $\bigcap{\mathfrak g}_{i}=\triv$.
A separating filtration induces an Hausdorff (T$_{2}$) topology on ${\mathfrak g}$. We refer to \cite{bbk,laz} for
these notions.

\medskip
Consider the Magnuss algebra (\cite[\S5.1]{bbk}) $\widehat{A}=\widehat{A}(X,\Q)$
over the rationals.
It is the topological completion of the free associative unitary $\Q$-algebra
$$A=A(X,\Q)=\Q\cdot\!1\oplus A^{\rm o}(X,\Q)$$\mn{notation!!} with respect to the
topology induced by the natural filtration.

Elements of $\widehat{A}$ are noncommutative formal power series in the {\em indeterminates} $X$ and coefficients in $\Q$:
$$a=\sum_{i<\omega}a_{i}\qquad\text{for $a_{i}\in A_{i}(X,\Q)$, $a_{0}\in\Fp$}.$$

As $L(X,\Q)$ is identified with the Lie subalgebra of $A(X,\Q)$ generated by $X$
with the product $[a,b]=ab-ba$, we define $\widehat{L}$ as the formal series $\sum_{i\geq1}b_{i}$ of $\widehat{A}$
with each homogeneous component $b_{i}$ belonging to $L_{i}(X,\Q)$.

If ${\mathfrak m}$ denotes the ideal $A^{\rm o}=\sum_{i\geq1}A_{i}(X)$, then $1+{\mathfrak m}$ is a group with
the product of $A$ and we have continuos maps (cfr.\cite[1,IV.7]{ser})
\begin{align*}
\lmap{\exp}{{\mathfrak m} &}{ 1+{\mathfrak m} }&\lmap{\log}{1+{\mathfrak m}&}{{\mathfrak m}}\\
a&\longmapsto\sum_{i\geq0}\frac{a^{i}}{i!}&1+b&\longmapsto\sum_{i\geq1}(-1)^{i+1}\frac{b^{i}}{i}
\end{align*}
with the usual properties $\log(\exp a)=a$ and $\exp(\log(1+b))=1+b$ and also
if $[a,b]=\triv$ then $\exp(a+b)=exp(a)\exp(b)$. Also $\widehat{L}\inn{\mathfrak m}$ of course.

\begin{fact}[{\cite[1.IV.7]{ser}\cite[Theorem 6.1,1]{bah}}]\label{expelle}
In the above notations,
$\exp(\widehat{L})$ is a multiplicative subgroup of $1+{\mathfrak m}$.

Moreover\mn{bring out} if $\epsilon$ denotes the homomorphism of the free group $F(X)$
on $X$ into $1+{\mathfrak m}$ which extends $x\mapsto\exp(x)$ for all $x$ in $X$,
then $\epsilon\log$ is a goup monomorphism of $F(X)$ into $(\widehat{L},\circ)$
where $\circ$ is a group law on $\widehat{L}$ given by
$$\xi\circ\eta=\log(\exp(\xi)\exp(\eta))$$
for all $\xi,\eta\in\widehat{L}$.
\end{fact}

\begin{teo}[{\cite[Proposition 6.2.1]{bah}},{\cite[Proposition \S5.4]{bbk}}]\label{faikaha}
Let now $X$ be the set $\{x,y\}$ the element of $\widehat{A}(x,y,\Q)$, then we have
\begin{labeq}{hausdseries}
x\circ y=:\mathsf{H}(x,y)=\sum_{i=1}^{\infty}t_{i}\mathsf{h}_{i}(x,y)
\end{labeq}
and $\mathsf{h}_{i}(x,y)$ is a homogeneous term in $L_{i}(\{x,y\},\Z)$ of total weight $i$ in $x$ and $y$ and $t_{i}\in\Q$.

For any complete, separated, filtered Lie Algebra ${\mathfrak g}$ over a characteristic zero field $\cur$ the map
\begin{align}\label{hausdgroup}
\lmap{\circ}{&{\mathfrak g}\times{\mathfrak g}}{{\mathfrak g}}\\
%\intertext{defined by}
&(a,b)\longmapsto\mathsf{H}(a,b)\tag*{}
\end{align}
induces a {\em group structure} on ${\mathfrak g}$ compatible with the topology induced by the filtration and such that
\begin{itemize}
\item[-]$m\circ\triv=m=\triv\circ m$
\item[-]$m\circ-m=\triv=-m\circ m$
\item[-]the $n${\sl -th} power $a^{n}$ in $\circ$ of any element $a$ of ${\mathfrak g}$ is $n\cdot a$ for all $n\in\Z$
\item[-]the group commutator $[l,m]_{\circ}$ built from the group operation $\circ$
coincides with the Lie product $[l,m]$ in ${\mathfrak g}$
\item[-]the subgroups of the lower and upper central series of the group $({\mathfrak g},\circ,\triv)$ coincide with
the ideals of the lower and upper central series in the Lie algebra ${\mathfrak g}$.
\end{itemize}
\end{teo}
For an explicit calculation of the terms $s_{i}\mathsf{h}_{i}(x,y)$ in \pref{hausdseries} one may see
\cite[1.IV.8]{ser}. We also find in \cite{bbk,bah,ser}, that a first segment of $\xi\circ\eta$ is given by
$$\xi\circ\eta=\xi+\eta+\frac{1}{2}[\xi,\eta]-\frac{1}{12}([\xi,\eta,\eta]+[\eta,\xi,\xi])+\cdots$$

\medskip
Now the crucial fact which allows us to apply the above machinery to our $\Fp$-algebras in $\nla{c}$,
is the following observation.
\begin{fact}[\cite{mag,laz}]
Let $\Q_{c}$ denote the subring of $\Q$ which consists of all quotients $r/s$ for coprime $r,s$ such that
if a prime $q$ divides $s$, then $q\leq c$.
In \pref{hausdgroup} above we have $t_{i}\in \Q_{i}$ for all $i<\omega$.
\end{fact}
Now we may observe that, since $p>c$ as a $\Fp$-vector space any object $M$ in $\nla{c}$ carries
a $\Q_{c}$-algebra structure, simply letting $r/s\cdot m$=$\bar r\bar{s}^{-1}m$ where
$\bar r$ and $\bar s$ denote $r$ and $s$ modulo $p$.

As observed in \cite{bbk}, to a {\em finite} central filtration
%in an algebra ${\mathfrak g}$ over $\cur$ as above, then ${\mathfrak g}$ 
automatically correspond a complete and separated (discrete) topology. This is the case for nilpotent algebras,
in particular Theorem \ref{faikaha} yields (cfr.{} \cite[Theorem II,4.2]{laz}):
\begin{cor}
We obtain a functorial correspondence
\begin{align}
\lmap{G}{\nla{c}&}{\ngb{c}{p}}\label{haugrp}\\
M&\longmapsto G(M)=(M,\circ,\triv)\notag
\end{align}
By Theorem \ref{faikaha} and by the definition of $\nla{c}$, for each such algebra $M$,
since $M=\gen{M_{1}}$ we obtain $\gr(G(M))=M$. Moreover for any $\nla{c}$-extension $M\nni N$,
the corresponding groups $G=G(M)$ and $H=G(N)$ satisfy $\gamma_{k}(H)=\gamma_{k}(G)\cap H$.
In particular $\gr(H)$ is an $\nla{c}$-subalgebra of $\gr(G)$.
\end{cor}

\medskip\cbstart
In a model theoretical setting
\begin{rem}
For any $\nla{c}$-algebra $M$, the group $G(M)$ is {\em definably interpretable} in the $\Lan{c}$-structure $M$.
\end{rem}
\cbend