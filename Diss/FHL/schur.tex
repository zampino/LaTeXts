In this section we see how the second homology of a finitely presented group or Lie algebra, together with the first
lower central section, entirely captures the relevant informations expressible in terms of generators and relations.

The objects we illustrate below present strong similarities with those defined in the second and especially the third
chapter. The last were developed independently from appealing to homology.

We refer to the book \cite{hilsta} for the basic facts concerning group homology.

\medskip
A {\em finitely presented} group $G$ is $(n,r)$-presented, if it admits a presentation
with $n$ generators and $r$ relators. The {\em deficiency} of $G$ is defined as
$$\mathrm{def}(G)=\max(n-r\mid G\,\,\text{is $(n, r)$-presented}).$$

%\cbstart
%In a nilpotent setting\mn{maybe just as footnote}, the \sout{{\em finitely presentation} property} -- which is closed under extensions -- is
%equivalent to being finitely generated. On the other hand in a Fra\"iss\'e construction, the starting structures
%are always finitely generated.
%\cbend

It is possible to estimate the deficiency of a finitely presented group $G$ in terms of the {\em Schur multiplicator}
$H_{2}(G)=H_{2}(G,\Z)$, the second homology group of $G$ with integer coefficients.
The following result, in \cite[14.1.5]{rob}, is attributed to Philip Hall.

If $A$ is a finitely generated abelian group, we denote by $\mathit{rk}(A)$ the rank of $A$ and $d(A)$ the minimal number of
elements required to generate $A$. %For any group $G_{ab}$ is $G/G^{\prime}$.
\begin{fact}
If $G$ is a finitely presented group, then $H_{2}(G)$ is finitely generated. Moreover
\begin{labeq}{defi}
\mathrm{def}(G)\leq\mathit{rk}(G_{ab})-d(H_{2}(G)).
\end{labeq}
\end{fact}

{\em Hopf's formula} (\cite{hopf}) expresses $H_{2}(G)$ in terms of any presentation of the group $G$:
%very close to our definitions in Chapter \ref{tre}:
\begin{labeq}{hopf}
H_{2}(G)=\frac{F^{\prime}\!\cap R}{[F,R\,]}
\end{labeq}
provided $R\to F\to G$ presents $G$.

Hopf's formula was later recognised independently by Stallings (\cite{stall}) and Stammbach (\cite{stam},\cite[\S8]{hilsta}) to stem from the {\em 5-term} Homology sequence 
\begin{labeq}{5term}
H_{2}(E)\to H_{2}(Q)\to N/[E,N]\to E_{ab}\to Q_{ab}\to\triv
\end{labeq}
associated to any short exact sequence $N\to E\to Q$, by applying \pref{5term}
to a presentation $R\to F\to G$ of the group $G$. Now \pref{hopf} follows by the fact $H_{2}(F)=\triv$.

\medskip
%The 5-term sequence above finds a corresponding sequence when we specialise to
%a group variety $\mathcal{V}$ (cfr.{}\cite[]{stahom}). In particular for a subvariety  of ${\mathfrak B}_{p}$, the groups
%with exponent $p$ we can use  homology with coefficients in $\Z/p\Z$ and obtain.
We can specialise -- with Stammbach's \cite[\S III]{stahom} -- the 5-term sequence above to a group variety $\mathcal{V}$, obtaining
a notion of schur multiplier $H_{2}(G;\mathcal{V},B)$ relative to $\mathcal{V}$ for any $G$-module $B$ and group $G\in\mathcal{V}$.
With this technique, an Hopf formula in terms of $\mathcal{V}$-presentations is achieved. %(cfr.{}\cite[\S III,(2.10)]{stahom}).
On the other hand Stallings (\cite[Theorem 2.1]{stall}) points out how group homology with coefficients in $\Z/p\Z$,
is connected to a {\em $p$-exponent modification\footnote{
the group words $\gamma_{k}(\bar x)$, which define the lower central series are replaced by $\gamma_{k}(\bar x)y^{p}$. In a group
of exponent $p$ we reobtain the old series.}} of the lower central series. 

Inspired by the results cited above, similar features concerning finitely presented
$\ngb{c}{p}$-groups may be derived. Note that in general a finitely generated nilpotent group is also
finitely presented, being this property closed under extensions of groups.
\begin{lem}\label{pdeficienza}
%Let $\mathcal{V}$ be an exponent $p$ variety of groups.
We say that $G\in\ngb{c}{p}$ is $(n,r)$-{\em presented in} $\ngb{c}{p}$
if it admits a presentation by the $n$-generated $\ngb{c}{p}$-free group $F$ modulo a normal subgroup $R$, which
is the normal closure of $r$ elements of $F$.

If ${\rm def}_{\ngb{c}{p}}(G)=max(n-r\mid G\text{ is $(n,r)$-presented in $\ngb{c}{p}$})$, then this number exists finite
and we have
\begin{labeq}{vardefi}
{\rm def}_{\ngb{c}{p}}(G)\leq\dfp(G_{ab})-\dfp(H_{2}(G;\ngb{c}{p}))
\end{labeq}
where $H_{2}(G;\ngb{c}{p})$ is {\em defined} as the %leftmost
kernel of the natural map $\phi$ in the exact sequence of $\Fp$-vector spaces
\begin{labeq}{varschur}
%\triv\to H_{2}^{\ngb{c}{p}}(G,\Fp)\to
R/[F,R]\stackrel{\phi\:}{\lto}F_{\sl a{}b}\to F/RF^{\prime}\to\triv
%\defeq\ker\left(    \right)
\end{labeq}
where $R\to F\to G$ is {\em any finite} $\ngb{c}{p}$-presentation of $G$.
\end{lem}
\begin{rem*}
By \pref{varschur} we have
$$H_{2}(G;\ngb{c}{p})=\frac{F^{\prime}\!\cap R}{[F,R\,]}$$
and by \cite[\S III.1,2]{stahom} this group does not depend of the chosen $\ngb{c}{p}$-presentation.
\end{rem*}
\begin{proofof}{Lemma \ref{pdeficienza}}
Assume the group $G$ is $(n,r)$-presented in $\ngb{c}{p}$ by $F$ modulo $R$.

Since $F$ is the $n$-generated $\ngb{c}{p}$-free group, we have $\dfp(F_{ab})=n$.
%Moreover $\dfp(R/[F,R])\leq r$, $G_{ab}$ is $F/RF^{\prime}$ and all terms of \pref{varschur} are $\Fp$-vectorspaces.
%\pref{vardefi} easily follows with some dimension calculus
Exactness in \pref{varschur} now yields $n-r\leq\dfp(F_{\sl ab})-\dfp(R/[F,R])=
\dfp(G_{ab})-\dfp(H_{2}(G;\ngb{c}{p}))$.
\end{proofof}
We list some facts to underline the strength of these concepts.
\begin{fact*}[{\cite[Theorem 6.5]{stall}}]
Let $G$ be a $\ngb{c}{p}$-group with $H_{2}(G;\ngb{c}{p})=\triv$ and $(x_{i})_{i\in I}$ a set of elements in $G$ whose images in
$G_{ab}$ are $\Fp$-linearly independent. Then the $x_{i}'s$ generate a $\ngb{c}{p}$-free subgroup of $G$.
\end{fact*}

\begin{fact*}[\cite{stall,stam}]
Let $\phi$ be a group homomorphism of $G$ in $K$, if $\phi$
induces an isomorphism of $G_{ab}$ to $K_{ab}$ and an epimorphism
$\phi_{*}$ of $H_{2}(G)$ onto $H_{2}(K)$, then $\phi$ induces
isomorphisms of $G\quot\gamma_{i}(G)$ to $K\quot\gamma_{i}(K)$ for all
$i<\omega$.

In particular if $G$ and $K$ are nilpotent, they are isomorphic.
\end{fact*}

A finitely presented group is called {\em efficient} if equality holds in \pref{defi} and $\mathcal{V}$-efficient
if the same equality holds for the corresponding $\mathcal{V}$-deficiency.
\begin{fact*}[{\cite[Theorem 6.5]{stahom}}]
Let $G$ be a group in $\mathcal{V}$, given by a finite $\mathcal{V}$-presentation.
Then there exists an efficient group $K\in\mathcal{V}$ and a surjective homomorphism
$\map{f}{K}{G}$ which induces an isomorphism $\map{f_{i}}{K/\gamma_{i}(K)}{G/\gamma_{i}(G)}$ for every $i\geq1$.

In particular $\ngb{c}{p}$-groups are $\ngb{c}{p}$-efficient: equality in \pref{vardefi} holds!
\end{fact*}

\medskip
The objects and facts reported above apply, in the very same fashion, to Lie algebras. One may check \cite{staknu} or
\cite[\S VII]{hilsta}.

In particular for a presentation $\mathfrak{r}\to\mathfrak{f}\to\mathfrak{g}$, the second integral homology group of ${\mathfrak g}$
is given by $$H_{2}(\mathfrak{g})=\mathfrak{f}^{\prime}\cap\mathfrak{r}\quot[\mathfrak{f},\mathfrak{r}]$$

As before, we find the analogous notion related to our special class of Lie algebras $\nla{c}$ over $\Fp$.
In particular for $M=\gen{M_{1}\mid R}$ in 
%Note that if ${\mathfrak g}$ belongs to our graded category
$\nla{c}$, as $R$ is contained in the {\em commutator algebra} $(\fla{c}{M_{1}})^{\prime}=\gamma_{2}(\fla{c}{M_{1}})$,
we find
\begin{labeq}{LieSchur}
H_{2}(M,\nla{c})=\frac{R}{[R,L]}
\end{labeq}
where $L$ denotes $\fla{c}{M_{1}}$.

Of course we can define -- as in Lemma \ref{pdeficienza} -- the corresponding
notion ${\rm def}_{\nla{c}}$ of $\nla{c}$-deficiency for finitely generated algebras $M$.
We may as well speak of {\em efficient $\nla{c}$-algebras} and in particular, we have 
\begin{labeq}{LieDef}
{\rm def}_{\nla{c}}(M)=\dfp(M_{1})-\dfp(H_{2}(M,\nla{c})).
\end{labeq}
In our case recall that the ideal $R$ is homogeneous and $R=R_{2}\oplus\dots\oplus R_{c}$.
Roughly speaking the group $R/[L,R]$ mods out for all $i\leq c$, the relators $R_{i}$ of
weight $i$ of the redundant terms: the elements of $R_{i}$ which arise as brackets $[r^{\prime},x_{1},\dots,x_{i-k}]$,
for relators of {\em lower weight} $r^{\prime}\in R_{k}$.

In section \ref{freelift} of Chapter \ref{tre} we will encounter exactly this {\em shifting} phenomenon.

\medskip
It is worth to note that the above notions interact with free products with amalgamated subgroup. This is connected with
the Mayer-Vietoris sequence (cfr.{\,}\cite[\S II.6]{stahom}).