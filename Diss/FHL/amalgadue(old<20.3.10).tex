Let $M$, $K$ and $H$ be algebras of $\nla{2}$.
Assume $H$ is both a subalgebra\mn{define amlgs for embeddings?} of $M$ and of $K$  in
the sense of $\nla{2}$, that is $\gena{H_{1}}{M}=H=\gena{H_{1}}{K}=M\cap K$.
We have then %that $H_{2}=H\cap M_{2}$\mn{to be placed before in the text} and
$M_{1}\cap K_{1}=H_{1}$.

Consider the $\Fp$-vector space amalgam $L_{1}=
M_{1}\oplus_{H_{1}}K_{1}$, recall $L_{1}=M_{1}\oplus_{H_{1}}K_{1}$, which by definition
is $M_{1}\oplus K_{1}\quot\Delta(H_1)$ where $\Delta(H_{1})=\{(h,-h)\mid h\in H_{1}\}$,
and define
$$\am{M}{H}{K}=L_{1}\oplus\,\exs L_{1}\,\quot(\rd(M)+\rd(K))=\fla{2}{L_{1}}\quot\rd(M)+\rd(H)$$\mn{invert $H$ with $K$
in all this section}
the {\em free amalgam of $M$ and $K$ over $H$}.

First of all we remark that, as $\fla{2}{L_{1}}$ is  $2$-nilpotent, an easy {\em weight} argument shows $\rd(M)+\rd(K)$ is
actually an {\em ideal}. % of $\fla{2}{L_{1}}$.
Then we show that $M$ and $K$ embeds into $L$ via $\nla{2}$-morphisms: as $\fla{2}{M_{1}}$ is a subalgebra 
of $\fla{2}{L_{1}}$ we have to show $\rd(L)\cap\fla{2}{M_{1}}=\rd(M)$ and the analogous condition for $K$.
But the latter holds, since $(\rd(M)+\rd(K))\cap\fla{2}{M_{1}}=(\rd(M)+\rd(K))\cap\exs M_{1}=\rd(M)+(\rd(K)\cap\exs M_{1})
=\rd(M)+(\rd(K)\cap\exs H_{1})=\rd(M)+\rd(M)+\rd(H)$.

%Clearly $=\am{M}{H}{K}$, 
Moreover $\gena{M_{1}}{\am{M}{H}{K}}\cap\gena{K_{1}}{\am{M}{H}{K}}=H$ holds.

\smallskip
This is a {\em free product with amalgamation} in the category $\nla{2}$, that is an algebra
satisfying the following diagram:
$$\textsl{xy-picture}$$

%\begin{vertbar}
\cbstart %[0.05cm]
\mn{redundant, all this?}%
We may first define
by $M \ast K$ the {\em free product} in $\nla{2}$ of $M$ and $K$, by definition the
Lie algebra given by the $\nla{2}$-presentation $\gen{M_{1}\oplus K_{1}\mid R_{M},R_{K}}$, then the quotient
$$M\ast_{H} K:=M\ast K\quot\geni{\iota_{M}(h)-\iota_{K}(h)\mid h\in H}$$
where $\iota_{M}$ and $\iota_{K}$ are the canonical inclusion $\nla{3}$-morphisms of $H$ into $M$ and $K$ respectively.
Then show $M\ast_{H} K\simeq\am{M}{H}{K}$\dots

A key observation here is to define  $\Delta(H)={(\!(}\iota_{M}(h)-\iota_{K}(h)\mid h\in H\mathbf{)\!)}$, and note
that this is actually, as an ideal, generated by $\Delta(H_{1})$, i.{}e. $\Delta(H)=\geni{\Delta(H_{1})}=
\geni{\iota_{M}(h)-\iota_{K}(h)\mid h\in H_{1}}$. \

This yields
$(M\ast_{H} K)_{1}\simeq M_{1}\oplus K_{1}\quot\Delta(H_{1})=M_{1}\oplus_{H_{1}} K_{1}$.
So we can find, on one side a morphism $\alpha$ of $M\ast K$ onto $\am{M}{H}{K}$ (by categorical universal property)
which factorises modulo $\Delta(H)$ to $\bar\alpha$. On the other hand
a morphism $\beta$ of $\fla{2}{\vam{M_{1}}{H_{1}}{K_{1}}}$ onto $M\ast_{H} K$, whose quotient $\bar\beta$ modulo $R_{M}+R_{K}$ is such that $\bar\alpha \bar\beta=id$
and $\bar\beta \bar\alpha=id$, which means $\am{M}{H}{K}\simeq M\ast_{H} K$.
\cbend
%\end{vertbar}

%And consider the fact $\fla{3}{M_{1}\cup K_{1}}\simeq\fla{3}{M_{1}\times K_{1}}$.
%As a consequence of this fact, 

\bigskip
Assume $D=\am{M}{H}{K}$, for $M$ and $K$ in $\nla{2}$ which share a common $\nla{2}$-subalgebra $H$.
Since we are up to prove that axiom $\sig{2}{2}$ is inherited by amalgams provided
the {\sl side} structures have it, we need an accurate description of the relators $N^{2}(E)=\rd(D)\cap\exs E_{1}$
for an arbitrary subspace $E_{1}\inn D_{1}$.

To accomplish this we first have do find a basis of $E_{1}$ in a form which suits our needs.

\begin{lem}\label{basee}
Let $E_{1}$ be a finite subspace of $D_{1}$. Assume there exists $n<\omega$ and subsets $\mathcal{U}=(u_{i})$ and $\mathcal{V}=(v_{i})$ for $i=1,\dots,n$ of $M_{1}$ and $K_{1}$ respectively,
such that $(u_{i}+v_{i})$ is a basis of $E_{1}$ over $E_{1}\cap M_{1}+E_{1}\cap K_{1}$.

%[EXT:] This yields that the set $(u_{i}+v_{i})$ is linearly
%independent over $E_{1}\cap M_{1}+E_{1}\cap K_{1}+H_{1}$ as well.
%[EXT:] \lambda\cdot u+v \in E_{1}\cap M_{1}+E_{1}\cap K_{1}+H_{1}
% implies, since it is in E_{1} too, that it is actually in E_{1}\cap M_{1}+
%E_{1}\cap K_{1}+(E_{1}\cap H_{1})=E_{1}\cap M_{1}+E_{1}\cap K_{1} !]
Then the set $\mathcal{UV}$ is linearly independent over $E_{1}\cap M_{1}+E_{1}\cap K_{1}+H_{1}$ in $D_{1}$.
\end{lem}
\begin{proof}\mn{expand}
We follow an inductive argument over $n<\omega$. Assume the assertion holds for $1\leq k\leq n-1$ and
%to consider \uwave{the first step of} an
let $\mathcal{U}$ and $\mathcal{V}$ be as mentioned in the statement of the lemma
if we set $\hat{\mathcal{U}}=\{u_{i}\mid i<n\}$ and $\hat{\mathcal{V}}=\{v_{i}\mid i<n\}$, then
%set $\hat{E}_{1}:=\gen{E_{1}\cap M_{1}+E_{1}\cap K_{1},u_{i}+v_{i}\mid i<n}$ so that
$\hat{\mathcal{U}}\hat{\mathcal{V}}$ is linearly independent over $E_{1}\cap M_{1}+E_{1}\cap K_{1}+H_{1}$.

Set $\tilde{E}_{1}:=\gen{E_{1}\cap M_{1}+E_{1}\cap K_{1},\hat{\mathcal{U}},\hat{\mathcal{V}},u_{n}+v_{n}}$ and notice
%so that $\tilde{E}_{1}\cap M_{1}+\tilde{E}_{1}\cap K_{1}=$\\
that $u_{n}+v_{n}$ generates $\tilde{E}_{1}$ over $\tilde{E}_{1}\cap M_{1}+\tilde{E}_{1}\cap K_{1}$ hence
$\{u_{n},v_{n}\}$ is linearly independent over $\tilde{E}_{1}\cap M_{1}+\tilde{E}_{1}\cap K_{1}+H_{1}=
\gen{E_{1}\cap M_{1}+E_{1}\cap K_{1}+H_{1},\hat{\mathcal{U}},\hat{\mathcal{V}}}$ and as a consequence
the set $\mathcal{UV}$ is independent over $E_{1}\cap M_{1}+E_{1}\cap K_{1}+H_{1}$ in $D_{1}$.

\medskip
It is then sufficient to prove the assertion for the case $n=1$.
Let then $E_{1}$ be generated by a sum $u+v$ over
$E_{1}\cap M_{1}+E_{1}\cap K_{1}$, hence $u+v$ is not
in $E_{1}\cap M_{1}+E_{1}\cap K_{1}+H_{1}$.%otherwise
%$u+v\in E_{1}\cap M_{1}+E_{1}\cap K_{1}+(E_{1}\cap H_{1})=E_{1}\cap M_{1}+E_{1}\cap K_{1}$.

If now $\lambda u + \mu v \in E_{1}\cap M_{1}+E_{1}\cap K_{1}+H_{1}$ for some $\lambda$ and $\mu$ in $\Fp$ and,
say, $\lambda\neq0$ we have then $%\ambda(u+v)+  $\lambda\neq\mu$
(\mu-\lambda)v\in K_{1}\cap (E_{1} +H_{1})=H_{1}+(E_{1}\cap K_{1})$ and thus $\lambda(u_{n}+v_{n})\in E_{1}\cap M_{1}+E_{1}\cap K_{1}+H_{1}$ which
is a contradiction.
\end{proof}

In this way, if a basis of $E_{1}$ is given in the form
$$\mathcal{E}^{m}\mathcal{E}^{h}\mathcal{E}^{k}(u_{i}+v_{i}:i=1,\dots,n) \quad\quad{\bf(\flat)}$$
described above, we can always complete the $\Fp$-independent set $\mathcal{E}^{m}\mathcal{E}^{h}\mathcal{E}^{k}
\mathcal{U}\mathcal{V}$ to a basis $\mathcal{D}$ of $D_{1}$ as follows: begin by finding a basis $\mathcal{E}^{h}$ of $E_{1}\cap H_{1}$, this is to be extended with $\mathcal{E}^{m}$ and $\mathcal{E}^{k}$ to a basis of $E_{1}\cap M_{1}$ and $E_{1}\cap K_{1}$ respectively. In this way $\mathcal{E}^{m}\mathcal{E}^{h}\mathcal{E}^{k}$ forms a basis of
$E_{1}\cap M_{1}+E_{1}\cap K_{1}$.
Let $\mathcal{U}$ and $\mathcal{V}$ be as above, according to lemma \ref{basee}, we can first complete $\mathcal{E}^{h}$ to a basis $\mathcal{D}^{h}$ of $H_{1}$, and then 
find  $\mathcal{D}^{m}$ containing $\mathcal{E}^{m}\mathcal{U}$ such that $\mathcal{D}^{h}\mathcal{D}^{m}$ is a basis of $M_{1}$.
On the other hand we can complete  $\mathcal{D}^{h}$ to a basis $\mathcal{D}^{h}\mathcal{D}^{k}$ of $K_{1}$ with $\mathcal{D}^{k}\nni
\mathcal{E}^{k}\mathcal{V}$.
We call $\mathcal{D}=\mathcal{D}^{m}\mathcal{D}^{h}\mathcal{D}^{k}$
a {\em compatible basis} of $D_{1}$ \emph{with} {\bf $(\flat)$}. %$E_{1}$.

This way of extending bases for $D_{1}$ leads to the following description of $N^{2}(E)$.
\begin{lem}\label{reldue}
Let $E=\gena{E_{1}}{D}$ be a finite subalgebra of the free amalgam $D=\am{M}{H}{K}$ for $M\nni H\inn K$ algebras
in $\nla{2}$.

Let  $\mathcal{E}^{m}\mathcal{E}^{h}\mathcal{E}^{k}(u_{i}+v_{i}:i=1,\dots,n)$ be a basis of $E_{1}$ as described above, and chose a linear order of it in such a way that
$\mathcal{E}^{h}>\mathcal{E}^{m}>\mathcal{U}>\mathcal{E}^{k}>\mathcal{V}$.
%(each set in the sequence is ordered arbitrarily).

Furthermore consider an ordered compatible basis $\mathcal{D}=\mathcal{D}^{h}>\mathcal{D}^{m}>\mathcal{D}^{k}$ of $D_{1}$ extending the order chosen above\mn{adjust indices here}.

Then each element $\Phi$ of $N^{2}(E)$ has the form $\Phi^{M}+\Phi^{K}+\phi^{U}+\phi^{V}$ where
\begin{itemize}
\item[]$\Phi^{M}$ and $\Phi^{K}$ are linear combination of $\mathcal{D}$-basic commutators with entries from $\mathcal{E}^{h}\mathcal{E}^{m}$ and
from $\mathcal{E}^{h}\mathcal{E}^{k}$ respectively
\item[]$\phi^{U}$ are linear combinations of $\mathcal{D}$-basic commutators $[h,u_{i}]$ where $h$ belongs to $\mathcal{E}^{h}$ and
$u_{i}$ is in $\mathcal{U}$
\item[]$\phi^{V}$ is obtained by replacing each entry $u_{i}$ in $\phi^{U}$ by the corresponding $v_{i}$ from $\mathcal{V}$.
\end{itemize}
Finally there exists $\eta$ in $\exs H_{1}$ such that $\Phi^{M}+\phi^{U}+\eta$ belongs to $N^{2}(M)$ and $\Phi^{K}+\phi^{V}-\eta$ to $N^{2}(K)$.
\end{lem}

\begin{proof}
Let $\Phi$ be an element of $N^{2}(D)\cap\exs E_{1}=(N^{2}(M)+N^{2}(K))\cap\exs E_{1}$, hence there exist $\nu^{M}$ in
$N^{2}(M)$ and $\nu^{K}\in N^{2}(K)$ such that $\Phi=\nu^{M}+\nu^{K}$.

On one hand, write $\Phi\in\exs M_{1}+\exs K_{1}\inn\exs D_{1}$ as a linear combination $\Phi_{\mathcal{D}}$ of $\mathcal{D}$-basic commutators, these being constructed according to the ordering of $\mathcal{D}$ as chosen above. Each basic monomial $[b_{1},b_{2}]$ with $b_{1}>b_{2}$ occurring in this expression carries both
entries $b_{1},\,b_{2}$ either from $\mathcal{D}^{h}\mathcal{D}^{m}$ or $\mathcal{D}^{h}\mathcal{D}^{k}$.

On the other hand, consider the ordered basis $\mathcal{E}=\mathcal{E}^{h}>\mathcal{E}^{m}>\mathcal{E}^{k}>
(u_{i}+v_{i}\mid i=1,\dots,n)$  of $E_{1}$, each set ordered according to the ordering in $\mathcal{D}$. Write $\Phi$ as a linear combination $\Phi_{\mathcal{E}}$ of $\mathcal{E}$-basic commutators.
%We have $\Phi_{\mathcal{D}}=\Phi=\Phi_{\mathcal{E}}$.
By linearity, each $\mathcal{E}$-basic monomial in $\Phi_{\mathcal{E}}$ involving entries $u_{i}+v_{i}$ expands into a sum of 
basic monomials over the basis $\mathcal{D}$,
one just have to transpose the order of the entries and accordingly change the sign of terms which
are not basic. 
%of the form $[b,u_{i}+v_{i}]$ for $b$ in $\mathcal{E}$ equals
%indeed $\mathcal{D}$-basic terms $[b,u_{i}]+[b,v_{i}]$ (with transposed entries and
%inverse sign, if $b$ happens to be greater than $u_{i}$ or $v_{i}$).
We conclude that $\Phi_{\mathcal{E}}$ is actually also a linear combination $\Phi^{\prime}$ in $\mathcal{D}$-basic monomials.

Now comparing expressions $\Phi_{\mathcal{D}}=\Phi=\Phi^{\prime}$, by Hall's basis theorem
\pref{}, we realise that terms in $\Phi_{\mathcal{E}}$ of the kind
$[b,u_{i}+v_{i}]$ with $b\in\mathcal{E}^{m}\mathcal{E}^{k}$ or $b=u_{j}+v_{j}$ for some $j$, are not allowed. Following the same
argument, basic monomials $[m,k]$ with $m\in\mathcal{E}^{m}$ and $k\in\mathcal{E}^{k}$ are excluded from $\Phi_{\mathcal{E}}$
as well.

We can conclude $\Phi_{\mathcal{E}}$ consists of the sum $\Phi^{m}+\Phi^{k}+\phi^{u}+\phi^{v}$ described in the statement of the lemma.

Finally compare expressions $\Phi_{\mathcal{E}}=\nu^{m}+\nu^{k}$ and set
$\eta:=\nu^{m}-  \Phi^{m} -\phi^{u}=\Phi^{m} +\phi^{u}-\nu^{k}     \in\exs M_{1}\cap\exs K_{1}=\exs H_{1}$.
\end{proof}

We prove next, that self-sufficient extensions of $\nla{2}$, are {\em lifted} by amalgams. This
is the next lemma:
\begin{lem}\label{asymam2}
In the above notation, $H\zsu{} K$ implies $M\zsu{} \am{M}{H}{K}$.
\end{lem}
\begin{proof}\mn{we give two argmts cfr. strategy nil-$3$ case}\\
{\sl (Argument I)}\quad Assume $E_{1}\inn L_{1}$ is a finite subspace.

Find a basis 
$\mathcal{E}^{m}\mathcal{E}^{h}\mathcal{E}^{k}(u_{i}+v_{i}:i=1,\dots,n)$ like {\bf ($\flat$)} for $E_{1}$.

We have to show $\delta_{2}(E\quot M)\geq0$.

By definition $\delta_{2}(E_{1}\quot M_{1})=\dfp(E_{1}\quot M_{1})-\dfp({\sf R_{2}}(E\quot M))$
And still by definition ${\sf R_{2}}(E\quot M)$ equals ${\sf R_{2}}(M+E)\quot{\sf R_{2}}(M)$.

We are going to build an $\Fp$-isomorphism of ${\sf R_{2}}(M_{1}+E_{1})\quot{\sf R_{2}}(M_{1})$ into
${\sf R_{2}}(H_{1}+E_{1}\cap K_{1}+\gen{\mathcal{V}})\quot{\sf R_{2}}(H_{1})$ as follows.
Since $(M_{1}+E_{1})\cap K_{1}=H_{1}+E_{1}\cap K_{1}+\gen{\mathcal{V}}$, any $\phi$ in
${\sf R_{2}}(M_{1}+E_{1})$ decomposes, by lemma \pref{} into $\phi^{M}+\phi^{K}$ where $\phi^{M}\in\exs M_{1}$, $\phi^{K}\in\exs (H_{1}+E_{1}\cap K_{1}+\gen{\mathcal{V}})$ and there exists $I\in\exs H_{1}$ such that $I+\phi^{K}\in {\sf R_{2}}(K)\cap\exs(H_{1}+E_{1}\cap K_{1}+\gen{\mathcal{V}})$.

Now map $\overline{\phi^{M}+\phi^{K}} $ to $\overline{I+\phi^{K}}$. This is independent of the choice
of $I$ as two such $I$'s differ by an element of ${\sf R_{2}}(K)\cap\exs H_{1}$.

The map is obviously onto and mono, since for $I+\phi^{K}$ to be in ${\sf R_{2}}(H)$ means
$\phi^{M}-I+I+\phi^{K}$ is in ${\sf R_{2}}(M)$.

\smallskip
On the other hand $\dfp(E_{1}\quot M_{1})=\dfp(E_{1}\cap K_{1}+\gen{\mathcal{V}}\quot H_{1})$ and
thus $\delta_{2}(E_{1}\quot M_{1})=\delta_{2}(E_{1}\cap K_{1}+\gen{\mathcal{V}}\quot H_{1})\geq0$
as desired. 

\bigskip
{\sl (Argument II)}\quad Consider a subspace $D_{1}\nni M_{1}$ of $L_{1}$ with finite $\dfp(D_{1}\quot M_{1})$. By the nature of $H_{1}\oplus_{H_{1}}K_{1}$
we have that $D_{1}=M_{1}+(D_{1}\cap K_{1})$. As a result
$D_{1}\quot M_{1}\simeq D_{1}\cap K_{1}\quot H_{1}$.

On the other hand, since $N^{2}(K)=N^{2}(\am{M}{H}{K})\cap\exs K_{1}$, we have $N^{2}(D)=(N^{2}(M)+N^{2}(K))\cap\exs D_{1}=N^{2}(M)+N^{2}(D_{1}\cap K_{1})$.
Now it is easy to check that $N^{2}(D_{1}\quot M_{1})=N^{2}(
D_{1}\cap K_{1}\quot H_{1})$.

Conclude $\delta_{2}(D_{1}\quot M_{1})=\delta_{2}(D_{1}\cap K_{1}\quot H_{1})\geq0
\mn{prove the same for $d_{2}$, needed for $3$-amalgam }$.
\end{proof}

The last equality characterise\mn{check it} the free amalgam in the following
sense. Assume that $M=\gena{M_{1}}{L}$, $K=\gena{K_{1}}{L}$ and
$H=\gena{M_{1}\cap K_{1}}{L}$ are subalgebras of some $\nla{2}$-structure $L$ and that $K_{1}$ is finite over $H_{1}$, then $\gena{M_{1}+K_{1}}{L}$ is isomorphic to $\am{M}{H}{K}$ iff
$\delta_{2}(K\quot H)=\delta_{2}(K\quot M)$. This follows from the fact
$\dfp({\sf R_{2}}(K\quot H))=\dfp({\sf R_{2}}(K\quot M))$ implies ${\sf R_{2}}(M+K)=
{\sf R_{2}}(M)+{\sf R_{2}}(K)$.
In this very case
we say that $M$ and $K$ are in free composition over $H$.

\bigskip
By the very proof of the previous lemma, the same result still holds
if strong embeddings are replaced by $k$-strong\mn{define them!!!} ones
\begin{cor}
$H\zsu{k} K$ implies $M\zsu{k} \am{M}{H}{K}$, for all $k<\omega$.
\end{cor}

\medskip
The following lemma is a key tool in the proof of amalgamation
in the nil-$2$ class. % relies on pretty different arguments than those in [Ba96].
\begin{lem}\label{amalsigma2}\mn{needs minimal ext.s}
Assume $M\zso{k}H\zsu{} K$ have all property $\sig{2}{2}$ for some $k$
greater or equal than the linear dimension of $K_{1}\quot H_{1}$.

Assume also that $K\quot H$ is minimal and if $K\quot H$ is minimal algebraic, then it
%if algebraic, has no solutions
is not realised in $M_{1}$ over $H_{1}$,\mn{needs finer statement\\cfr.``main theorem''}
then $\am{M}{H}{K}$ also satisfies $\sig{2}{2}$.
\end{lem}
\begin{proof}
Let $E_{1}$ be an arbitrary finite subspace of $M_{1}\oplus_{H_{1}}K_{1}$ and fix a 
basis $\mathcal{E}=\mathcal{E}^{h}\mathcal{E}^{m}\mathcal{E}^{k}(
u_{i}+v_{i}\mid i=1,\dots n)$ of $E_{1}$ for suitable
$u_{i}$'s in $M_{1}$ and $v_{i}$'s in $K_{1}$ as described in lemma \pref{basee}.
We have to show $\delta_{2}(E)\geq\min(2,\dfp(E_{1}))$.

First use subadditivity of our predimension to observe that $\delta_{2}(E_{1})\geq\delta_{2}(E_{1}\quot M_{1})+\delta_{2}(E_{1}\cap M_{1})$.
Since $M$ is a self-sufficient subalgebra of the amalgam $\am{M}{H}{K}$, if $\dfp(E_{1}\cap M_{1})\geq2$ we are done
%\mn{fit case dim(E)=2 and ints =1 here?}
as $M$ satisfies $\sig{2}{2}$, we might then assume $\dfp(E_{1}\cap M_{1})<2$.

\smallskip
If $E_{1}\cap M_{1}=\triv$,
then %$E_{1}=\gen{E_{1}\cap K_{1}, and we find a basis $\mathcal{E}^{}$
$\mathcal{E}=\mathcal{E}^{k}(u_{i}+v_{i}\mid i=1,\dots, n)$ and by lemma
\pref{reldue} we have $N^{2}(E_{1})=%\mn{check! Here and below}
N^{2}(E_{1}\cap K_{1})$.
It follows $\delta_{2}(E_{1})=\delta_{2}(E_{1}\cap K_{1})+n$ and
this yields $\delta_{2}(E_{1})\geq\min(2,\dfp(E_{1}))$ since $E_{1}\cap K_{1}$ does
and $\dfp(E_{1})=\dfp(E_{1}\cap K_{1})+n$.

\smallskip
Assume $E_{1}\cap M_{1}$ %\mn{too many cases considered?}
has dimension $1$. If $E_{1}\cap H_{1}=\triv$ then by lemma \pref{reldue} again, $N^{2}(E)=N^{2}(E_{1}\cap
K_{1})$ because $\mathcal{E}=\{m\}\mathcal{E}^{k}(u_{i}+v_{i}\mid i=1,\dots n)$
and we can conclude as above: this time $\delta_{2}(E)=\delta_{2}(E_{1}\cap K_{1})+n+1$.

\medskip
Assume eventually $E_{1}\cap M_{1}=E_{1}\cap H_{1}=\gen{h}$ has dimension $1$, so that
$E_{1}=\gen{E_{1}\cap K_{1},u_{i}+v_{i}\mid i=1,\dots,n}$ and $\mathcal{E}=\{h\}\mathcal{E}^{k}(u_{i}+v_{i}\mid i=1,\dots n)$.

If $E_{1}\cap K_{1}$ is $\gen{h}$ as well ($\mathcal{E}^{k}=\vac$), then
$E_{1}=\gen{h,u_{i}+v_{i}\mid i=1,\dots,n}$. If we assume that 
$N^{2}(E)$ is nontrivial -- the converse would give us the desired conclusion -- then a nonzero element
$\phi$ of $N^{2}(E)$ is equal to a sum $\sum_{i=1}^{n}s_{i}[h,u_{i}+v_{i}]$ for some scalars $s_{i}$ and, for a suitable
$\beta$ in $\exs H_{1}$ we have by lemma \pref{reldue} that $[h,\sum_{i=1}^{n}s_{i}u_{i}]+\beta$ lies
in $N^{2}(M)$ and $[h,\sum_{i=1}^{n}s_{i}v_{i}]-\beta$ in $N^{2}(K)$. If we now set $v:=\sum_{i=1}^{n}s_{i}v_{i}$
which lays in $K_{1}$ over $H_{1}$, then we get $\delta_{2}(v\quot H)=0$. Since $K\quot H$ is a minimal
extension, then $K_{1}=\gen{H_{1}, v}$ and in this case the defining relation\mn{no! use $1$-slfscy and ``realisation'' instead} $[h,v]-\beta$ has a solution
in $M_{1}$ by means of $-\sum_{i=1}^{n}s_{i}u_{i}$ but his option is not allowed by the hypothesis.

\smallskip
For the very last case we may assume $\dfp(E_{1}\cap K_{1})>1$ and %$\rd(E)=\rd(E_{1}\cap K_{1}0$ we are done since %\dots, otherwise assume 
$\rd(E_{1}\quot E_{1}\cap K_{1})$
is non-trivial.

We appeal once again to lemma \pref{reldue} to conclude that any $\phi$ in $N^{2}(E)$ equals
a sum $\phi^{u}+\phi^{K}+\phi^{v}$, where $\phi^{u}=\sum_{i=1}^{n} s_{i}[h,u_{i}]$, $\phi^{v}=\sum_{i=1}^{n}s_{i}[h,v_{i}]$
for $s_{i}\in\Fp$ and $\phi^{K}$ in $\exs E_{1}\cap K_{1}$. Moreover to any such $\phi$, an element $\beta$ of
$\exs H_{1}$ is given, such that $\phi^{u}+\beta$ is in $N^{2}(M)$ and $\phi^{K}+\phi^{v}-\beta$ is in $N^{2}(K)$.

Notice that the map $\phi\mapsto\phi^{u}+\beta$ can be factorised to a well defined linear morphism
of $N^{2}(E_{1}\quot E_{1}\cap K_{1})$ to $N^{2}(\gen{u_{i}\mid i=1,\dots,n}\quot H_{1})$. If
now $\phi^{u}+\beta$ is in $N^{2}(H_{1})$ then $\phi$ belongs to $N^{2}(E)\cap\exs K_{1}=N^{2}(E_{1}\cap K_{1})$,
and therefore $N^{2}(E_{1}\quot E_{1}\cap K_{1})$ embeds into $N^{2}(\gen{u_{i}\mid i=1,\dots,n}\quot H_{1})$.

We have then $\delta_{2}(E_{1})=\delta_{2}(E_{1}\cap K_{1})+n-\dfp(N^{2}(E_{1}\quot E_{1}\cap K_{1}))\geq
\delta_{2}(E_{1}\cap K_{1})+\delta_{2}(\gen{u_{i}\mid i=1,\dots,n}\quot H_{1})$ and 
$\delta_{2}(\gen{u_{i}\mid i=1,\dots,n}\quot H_{1})\geq0$ since $H$ is $k$-self-sufficient in $M$ and
$n\leq\dfp(K_{1}\quot H_{1})\leq k$. Thus the considered amalgam has $\sig{2}{2}$ and the
proof is completed.
\end{proof}


Define the $\La_{2}$\mn{define languages!} elementary
class $\Klt{2}=\left\{
M\in\nla{2}\mid M\sat\sig{2}{2}\right\}$ and
$\Kl{2}$ to be the finitely generated structures of $\Klt{2}$
%$\{M\in\Klt{2}\mid\dfp M_{1}\,\text{is finite}\}$
\begin{lem}[Asymmetric Amalgam]\label{asymalgadue}\mn{add some pictures}
Let $M$, $B$ and $A$ be $\nla{2}$-algebras with $\sig{2}{2}$ such that $B$ is strong in $A$ and $n+\dfp(A_{1}\quot B_{1})$-strong in $M$. There exists $L$ in $\nla{2}$, with $M\zsu{}L\zso{n}A$, which amalgamates
$M$ and $A$ over $B$.
\end{lem}
\begin{proof}\mn{rewrite!}\cbstart
Let $M\zso{} B\zsu{} A$ be $\Kl{2}$-algebras.
Lets disassemble $A$ into a finite chain of nested subalgebras,
$$B=A^{0}\zsu{}A^{1}\zsu{}\,\cdots\,\zsu{} A^{n}=A$$
such that $A_{1}^{i}\zsu{}A_{1}^{i+1}$ and $A^{i+1}$ is minimal over $A^{i}$. Denote
$\iota_{i}$ the natural strong inclusion of $A^{i}$ into $A^{i-1}$.

Set $M^{0}=M$ and for all $i\leq n$ %assume an algebra $M^{i-1}$ has been defined
%with properties \dots then
define
$$
M^{i}:=
\begin{cases}
\am{M^{i-1}}{A^{i-1}}{A^{i}}&\text{if $A^{i}$ is not realized in $M^{i-1}$ over $A^{i-1}$}\\
M^{i-1}&\text{if $A^{i}$ is realized in $M^{i-1}$ over $A^{i-1}$}
\end{cases}
$$
thus at each step $M^{i}$ satisfies $\sig{2}{2}$ by lemma \ref{amalsigma2}.

If one sets $\alpha_{0}$ to be the embedding of $A^{0}$ into $M^{0}$,
for each $i\geq0$ we have strong $\nla{2}$-monomorphisms
$\emb{\alpha_{i}}{A^{i}}{M^{i}}$ and $\emb{\mu_{i}}{M^{i}}{M^{i+1}}$
such that $\alpha_{i}\mu_{i}=\iota_{i}\alpha_{i+1}$.

Now define $\iota$ the embedding of $B$ into $A$ and set 
$\mu:=\mu_{0}\cdot\mu_{1}\cdot\,\cdots\,\cdot\mu_{n-1}$
and $\alpha:=\alpha_{n}$. We claim that $M^{n}$ amalgamates
%\mn{define/show its a <<strong amalgam>> by shifting amalgams}
$M$ and $A$ over $B$, because if we consider $\emb{\mu}{M}{M^{n}}$ and $\emb{\alpha}{A}{M^{n}}$ which
are strong embeddings by transitivity of selfsufficiency,
then we have $\alpha_{0}\mu=\iota\alpha$ as the following shows
\begin{align*}
\alpha_{0}\mu=\alpha_{0}\mu_{0}\mu_{1}\cdot\,\cdots\,\cdot\mu_{n-1}&=\\
=\iota_{0}\alpha_{1}\mu_{1}\cdot\,\cdots\,\cdot\mu_{n-1}&=\\
=\iota_{0}\iota_{1}\alpha_{2}\mu_{2}\cdot\,\cdots\,\cdot\mu_{n-1}&=\\
=\iota_{0}\iota_{1}\cdot\,\cdots\,\cdot\iota_{n-1}\alpha_{n-1}&=\iota\alpha
\end{align*}
\end{proof}\cbend
\begin{cor}\label{amalgadue}
$\Kl{2}$ has the amalgamation property (AP) with respect to
$\delta_{2}$-strong embeddings.
\end{cor}
\begin{changebar}\mn{to be put nice}
As $\sig{2}{2}$ is a universal axiom, $\Klt{2}$ has the so called {\sl hereditary property} (HP):
any finitely generated substructure $B$ of an algebra $A$ of $\Klt{2}$ is isomorphic to
an object of the same class. Moreover, since the trivial algebra $\triv$ is strong
in any other, amalgamation applies to pairs of algebras in $\Kl{2}$ with
trivial intersection. This implies $\Klt{2}$ has the {\sl joint embedding property} (JEP).

Altogether posses $\Klt{2}$ properties (HP), (JEP) and (AP),
which allows the construction of a countable Fra\"iss\'e Limit $\K^{2}$ of $\Klt{2}$
with age exactly $\Kl{2}$.

$\K^{2}$ lays in $\Klt{2}$ and is $\Kl{2}$-homogeneous and universal with respect to
self-sufficient embeddings. It follows $\K^{2}$ is \emph{rich},
in the sense of the following definition\dots
that is for any $\delta_{2}$-strong extension $A\quot B$ of
algebras in $\Kl{2}$, if $B\zsu{}\K^{2}$ then $A$ can be
strongly embedded into $\K^{2}$ over $B$.
\end{changebar}


